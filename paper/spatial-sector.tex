\documentclass[12pt,preprint]{elsarticle}
% \documentclass[12pt,1p]{elsarticle}
% \documentclass[11pt]{article}

\journal{Joule}
\widowpenalty10000
\clubpenalty10000
\abstracttitle{Summary}

% A typical research article will be
% 4000 words of text with
% 3–5 figures and
% 30 references.

% open acess fee: 7500 Euros

\bibliographystyle{elsarticle-num}
\biboptions{numbers,sort&compress,super}

% format hacks
\usepackage{libertine}
\usepackage{libertinust1math}

% unused
% \renewcommand*\familydefault{\sfdefault} % biolinum
%\usepackage[pass]{geometry}
% \usepackage[latin1]{inputenc}
% \usepackage{amsfonts}
% \usepackage{amssymb}

\usepackage{geometry}
% \geometry{
% top=30mm,
% bottom=35mm,
% }

\usepackage{amsmath}
\usepackage{bbold}
\usepackage{graphicx}
\usepackage{eurosym}
\usepackage{mathtools}
\usepackage{url}
\usepackage{booktabs}
\usepackage{epstopdf}
\usepackage{xfrac}
\usepackage{tabularx}
\usepackage[version=4]{mhchem}
\usepackage{bm}
\usepackage[colorlinks]{hyperref}
\usepackage[nameinlink,sort&compress,capitalise,noabbrev]{cleveref}
\usepackage[leftcaption,raggedright]{sidecap}
\usepackage{subcaption}
\usepackage{blindtext}
\usepackage[parfill]{parskip}
% \usepackage[toc,title,page]{appendix}


\usepackage[prependcaption,textsize=scriptsize]{todonotes}
% \usepackage[prependcaption,textsize=scriptsize,disable]{todonotes}

\usepackage{siunitx}
\sisetup{
	range-units = single,
	per-mode = symbol
}
\DeclareSIUnit\year{a}
\DeclareSIUnit{\tco}{t_{\ce{CO2}}}
\DeclareSIUnit{\sieuro}{\mbox{\euro}}
\DeclareSIUnit{\twh}{{\tera\watt\hour}}
\DeclareSIUnit{\mwh}{{\mega\watt\hour}}
\DeclareSIUnit{\kwh}{{\kilo\watt\hour}}

\newcommand{\co}{\ce{CO2}~}

\usepackage{longtable}
\usepackage{multirow}
\usepackage{threeparttable}
\usepackage{pdflscape}

\usepackage[export]{adjustbox}

\usepackage[resetlabels,labeled]{multibib}
\newcites{S}{Supplementary References}
\bibliographystyleS{elsarticle-num}

\graphicspath{{../results/graphics/}, {../../../playgrounds/pr/pypsa-eur-sec/results/}}

\newcommand{\onwrun}{20211218-181-onw}
\newcommand{\gasrun}{20211218-181-gas}
\newcommand{\decrun}{20211218-181-decentral}
\newcommand{\lvrun}{20211218-181-lv}
\newcommand{\hyrun}{20211218-181-h2}
\newcommand{\gasrunscen}{20211218-181-gas/elec_s_181_lv1.0__Co2L0-3H-T-H-B-I-A-solar+p3-linemaxext10_2030}
\newcommand{\runelec}{20211218-181-h2/elec_s_181_lvopt__Co2L0-3H-T-H-B-I-A-solar+p3-linemaxext10-noH2network_2030}
\newcommand{\runhy}{20211218-181-lv/elec_s_181_lv1.0__Co2L0-3H-T-H-B-I-A-solar+p3-linemaxext10_2030}

% \usepackage[
% 	type={CC},
% 	modifier={by},
% 	version={4.0},
% ]{doclicense}
%
% \newcommand{\topic}[1]{\todo[color=green!40,noline]{#1}}

\newcommand{\abs}[1]{\left|#1\right|}
\newcommand{\norm}[1]{\left\lVert#1\right\rVert}
\newcommand{\set}[1]{\left\{#1\right\}}
\DeclareMathOperator*{\argmin}{\arg\!\min}
\def\cT{\mathcal{T}}
\newcommand{\R}{\mathbb{R}}
\newcommand{\B}{\mathbb{B}}
\newcommand{\N}{\mathbb{N}}
% \def\cL{\mathcal{L}}
% \def\cN{\mathcal{N}}
% \def\cT{\mathcal{T}}
% \def\cM{\mathcal{M}}
% \def\cA{\mathcal{A}}
% \def\ba{{\bm{\alpha}}}
% \def\card{\text{card}\,}
% \def\x{\boldsymbol{\mathsf{x}}}

%\def\co{CO${}_2$}
\def\el{${}_{\textrm{el}}$}
\def\th{${}_{\textrm{th}}$}
\def\hy{${}_{\textrm{H}_2}$}
\def\deg{${}^\circ$}

% \urlstyle{sf}

\usepackage{xcolor}
\usepackage{framed}
\definecolor{shadecolor}{rgb}{.95,.95,.95}

\usepackage{tocloft}
\cftsetindents{section}{1em}{2.75em}

\begin{document}


\begin{frontmatter}

	\title{Benefits of a Hydrogen Network in Europe}

	\author[tubaddress]{Fabian Neumann\corref{correspondingauthor}}
	\ead{f.neumann@tu-berlin.de}
	\author[tubaddress]{Elisabeth Zeyen}
	\author[aarhus,aarhus2]{Marta Victoria}
	\author[tubaddress]{Tom Brown}
	% \cortext[correspondingauthor]{}
	\address[tubaddress]{Department of Digital Transformation in Energy Systems, Institute of Energy Technology, Technische Universität Berlin, Fakultät III, Einsteinufer 25 (TA 8), 10587 Berlin, Germany}
	\address[aarhus]{Department of Mechanical and Production Engineering, Aarhus University, Inge Lehmanns Gade 10, 8000 Aarhus, Denmark}
	\address[aarhus2]{iCLIMATE Interdisciplinary Centre for Climate Change, Aarhus University}

	\begin{abstract}
		Electricity transmission expansion has suffered many delays in Europe in recent
decades, despite its significance for integrating renewable electricity. A
hydrogen network reusing the existing gas network could not only help to supply
demand for low-emission fuels, but could also balance variations in wind and
solar energy across the continent and thus avoid power grid expansion. We pursue
this idea by varying the allowed expansion of electricity and hydrogen grids in
net-zero \co scenarios for a sector-coupled European energy system. With 181
regions and 3-hourly time series, we capture transmission bottlenecks, the
variability of demand and renewable supply, and potentials for retrofitting gas
pipelines and developing geological hydrogen storage. We find that a hydrogen
network connecting regions with low-cost and abundant renewable potentials to
demand centers, electrofuel production and cavern storage sites reduces system
costs by up to \maxhybenefitabs~bn\euro/a (\maxhybenefitrel\%). Between 64\% and
69\% of this network could reuse natural gas pipelines. While the expansion of
both networks together can achieve the largest cost savings of
\gridbenefitrel\%, the expansion of neither appears as essential in a net-zero
system as long as higher costs can be accepted and flexibility options are
enabled to manage grid bottlenecks.


	\end{abstract}

	\begin{keyword}
		hydrogen backbone, sector-coupling, retrofitting, energy systems, transmission expansion, Europe, climate-neutral, renewables
	\end{keyword}

	% \begin{graphicalabstract}
	% \end{graphicalabstract}

\end{frontmatter}



\section*{Highlights}

\begin{itemize}
	\item examines the benefit of a hydrogen network and gas pipeline retrofitting in net-zero \co scenarios for Europe with high shares of renewables
	\item uses open energy system model PyPSA-Eur-Sec with 181 regions, 3-hourly resolution for a year and all energy sectors (electricity, buildings, transport, industry, agriculture) represented
	\item hydrogen network reduces system costs by up to 6\%, with highest benefits when power grid expansion restricted
	\item between 58-66\% of hydrogen backbone uses retrofitted gas network pipelines
	\item cost benefit of electricity grid expansion is higher than of hydrogen network (6\% versus 8\%), but both together reduce costs by up to 12\%
\end{itemize}

Words: 5386
(excl. abstract, captions, \nameref{sec:methods}, \nameref{sec:si})

% \addcontentsline{toc}{section}{Context \& Scale}
% \begin{shaded}
% \vspace{-0.5cm}
% \section*{Context \& Scale}
% 
% TODO 1000 characters in 1-2 paragraphs

Many combinations of infrastructure could make Europe climate-neutral by
mid-century. But not all solutions meet the same level of acceptance. For
example, power transmission reinforcements experience many delays, despite their
value for integrating renewable electricity. A hydrogen network which can reuse
gas pipelines could offer a substitute for moving cheap but remote renewable
energy across the continent to where demand is.

We study such trade-offs between building new transmission lines and developing
a hydrogen network in the European energy system with all sectors represented
and net-zero \co emissions. We find that a hydrogen network is consistently
beneficial infrastructure and that a large part could repurpose unused gas
pipelines. Energy transport as electrons and molecules offer complementary
strengths, achieving highest cost savings together. But neither is truly
essential for a low-cost system. This means that there are many ways to achieve
net-zero emissions in Europe that are affordable, giving policymakers different
options to choose from.

% \end{shaded}

\newpage
\section*{Introduction}
\label{sec:intro}
\addcontentsline{toc}{section}{\nameref{sec:intro}}

\topic{public acceptance problems / importance of power grid}

There are many different combinations of infrastructure that would allow Europe
to reach net-zero greenhouse gas emissions by 2050. However, not all
technologies meet the same level of acceptance among the public. The last few
decades have seen public resistance to new and existing nuclear power plants,
projects with carbon capture and sequestration (CCS), onshore wind power plants,
and overhead transmission lines. The lack of public acceptance can both delay
the deployment of a technology and even stop its deployment altogether. This may
make it harder to reach greenhouse gas reduction targets in time or cause rising
costs through the substitution with other technologies. In particular,
electricity transmission network expansion has suffered many delays in Europe in
recent decades, despite its importance for integrating large amounts of
renewable electricity such that all energy sectors can be decarbonised.

\topic{hydrogen: an important energy carrier}

Hydrogen will likely become a pivotal energy carrier in such a climate-neutral
energy system. Hydrogen is needed in the industry to produce ammonia for
fertilisers and direct reduced iron for steelmaking. It is also a critical
feedstock to produce synthetic methane and liquid hydrocarbons for use as
aviation fuel and as a precursor to high-value chemicals. Further hydrogen use
extends to heavy-duty land transport, shipping as liquified hydrogen, and backup
heat and power supply through re-electrification in stationary fuel cells.

\topic{idea of European hydrogen backbone as replacement?}

The limited social acceptance for electricity grid reinforcement and the
advancing role of hydrogen raises the question of whether a new network of
hydrogen pipelines could offer a replacement for balancing variable renewable
electricity generation and moving energy across the continent. Such a vision for
a \textit{European Hydrogen Backbone (EHB)} has recently been expressed by
Europe's gas industry in a series of reports
\cite{gasforclimateEuropeanHydrogen2020,gasforclimateEuropeanHydrogen2021,gasforclimateExtendingEuropean2021,gasforclimateEuropeanHydrogen2022}.
It would offer an alternative route to connecting remote regions with abundant
and cost-effective wind and solar potentials to densely-populated and
industry-heavy regions with high demand but limited supply options.

\topic{large retrofitting potential of gas network / higher acceptance?}

Since Europe has a sizeable existing gas transmission network that is set to
become increasingly expendable as the system transitions towards climate
neutrality, the option to repurpose parts of the network, to transport hydrogen
instead, may reinforce this idea. This is because retrofitting gas pipelines
would significantly lower the development costs compared to building new
hydrogen pipelines. Moreover, repurposed and new pipelines may also meet higher
levels of acceptance among the local populations than transmission lines. Unlike
transmission towers, pipelines are less visible because they usually run below
or near the ground. Particularly where gas pipelines already exist, the
perceivable impact would be minimal.

\topic{shortcomings in literature/EHB}

However, few studies have evaluated the benefit of a hydrogen network in Europe
so far. The EHB reports do not include an assessment based on the
co-optimisation of energy system components
\cite{gasforclimateEuropeanHydrogen2020,gasforclimateEuropeanHydrogen2021,gasforclimateExtendingEuropean2021,gasforclimateEuropeanHydrogen2022}.
Other sector-coupling studies have not included hydrogen networks at all
\cite{brownSynergiesSector2018}, or when they do, model Europe only at
country-level resolution
\cite{europeancommission.directorategeneralforenergy.METISStudy2021,victoriaSpeedTechnological2021},
have a country-specific focus with limited geographical scope
\cite{gilsInteractionHydrogen2021}, or neglect some energy sectors or non-energy
demands that involve hydrogen
\cite{gilsInteractionHydrogen2021,Caglayan2019,caglayanRobustDesign2021}. None
of the studies have explored the interplay between hydrogen network expansion
and electricity grid reinforcements. Neither have the potentials for lower
development costs through pipeline retrofitting been taken into account so far.


\topic{in this paper}

This paper examines the economic benefit of a hydrogen backbone in scenarios for
a European energy system with net-zero carbon dioxide emissions and high shares
of renewable electricity production. We analyse four main scenarios to
investigate if a hydrogen network can compensate for a potential lack of power
grid expansion. These vary on whether or not electricity and hydrogen grids can
be expanded, including potentials for gas pipeline retrofitting. Evaluation
criteria are the total system cost, the composition and spatial distribution of
technologies and transmission infrastructure in the system.

% As a supplementary sensitivity analysis, we also evaluate the
% impact of restricted onshore wind potentials.

\topic{mini model description and novelty}

For our analysis, we use an open-source capacity expansion model of the European
energy system, PyPSA-Eur-Sec, which, in contrast to previous studies
\cite{henningComprehensiveModel2014,mathiesenSmartEnergy2015,connollySmartEnergy2016,lofflerDesigningModel2017,blancoPotentialHydrogen2018,brownSynergiesSector2018,in-depth_2018,victoria2020},
combines a fully sector-coupled approach with a high spatio-temporal resolution
so that it can capture the transmission bottlenecks that constrain the
cost-effective integration of variable renewable energy. The model co-optimises
the investment and operation of generation, storage, conversion and transmission
infrastructures in a single linear optimisation problem, covering 181 regions
and a 3-hourly time resolution for a full year. It incorporates spatially
distributed demands of the electricity, industry, buildings, agriculture and
transport sectors, including shipping and aviation as well as non-energy
feedstock demands in the chemicals industry. Primary energy supply comes from
wind, solar, biomass, hydro, and limited amounts of fossil oil and gas. The
energy flows between the system's energy carriers are modelled by various
technologies, including heat pumps, CHPs, thermal storage, electric vehicles,
batteries, power-to-X processes, fuel cells, and geological potentials of
underground hydrogen storage. Moreover, data on electricity and gas transmission
infrastructure is included to determine grid expansion needs and retrofitting
potentials. Finally, the model also features detailed management of carbon flows
between capture, usage, sequestration and the atmosphere. More details on the
model are presented in \nameref{sec:methods} and \nameref{sec:si}.

% challenges to the model compared to electricity, heating and transport are
% strongly peaked
% - heating is strongly seasonal but also with synoptic variations
% - transport has strong daily periodicity There are difficult periods in winter
%   with
% - low wind and solar (high prices)
% - high space heating demand
% - low air temperatures, which are bad for air-sourced heat pump performance
% - less smart solution: backup gas boilers
% - smart solution: building retrofitting, TES in district heating, CHP

\topic{scenario assumptions}

All investigations are conducted with a constraint that carbon dioxide emissions
into the atmosphere balance out to zero over the year. The model can sequester
up to 200 Mt\co per year, allowing it to sequester industry process emissions,
such as calcination in cement manufacture, that have a fossil origin but limits
the use of negative emission technologies compared to other works
\cite{blancoPotentialHydrogen2018}. In our scenarios, we do not
consider energy imports into Europe and apply technology assumptions for
the year 2030 \cite{DEA}.

% Higher sequestration rates would be
% favoured in the model. For comparison, the JRC-EU-TIMES model sees optimal
% sequestration levels up to 1.4 Gt\co per year for a 95\% reduction of \co
% emissions in the EU compared to 1990 levels .

% general:
% \cite{
%     mckennaScenicnessAssessment2021,
%     krummModellingSocial2022,
%     weinandImpactPublic2021,
%     weinandExploringTrilemma,
%     trondleTradeOffsGeographic2020,
%     sasseDistributionalTradeoffs2019,
%     sasseRegionalImpacts2020,
%     ludererImpactDeclining2021,
%     victoria2020,
%     victoriaSpeedTechnological2021,
%     lombardiPolicyDecision2020,
%     tsiropoulosNetzeroEmissions2020,
%     europeancommission.directorategeneralforenergy.METISStudy2021,
%     tafarteQuantifyingTrade,
%     lehmannManagingSpatial}

% PV cost
% \cite{
%     jaxa-rozenSourcesUncertainty2021,
%     victoriaSolarPhotovoltaics2021,
%     xiaoPlummetingCosts2021}



\begin{figure}
    \centering
    \makebox[\textwidth][c]{
    \includegraphics[width=1.3\textwidth]{balance}
    } \caption{ Energy and carbon dioxide balances across all scenarios. Energy
    consumption includes final energy and non-energy demands by carrier as well
    as conversion losses in electrofuel synthesis. The ambient heat retrieved by
    heat pumps is counted as energy supply. A breakdown of final energy and
    non-energy demands by sector is shown in
    \cref{fig:demand-by-sector-carrier}. For technologies with carbon capture
    option (CC), the carbon dioxide balance shows residual uncaptured emissions.
    }
    \label{fig:balance}
\end{figure}

\begin{figure}
    \centering
    \begin{subfigure}[t]{\textwidth}
        \centering
        \caption{cost reductions induced by hydrogen and power grid expansion}
        \includegraphics[width=0.85\textwidth]{sensitivity-h2-new.pdf}
        \label{fig:sensitivity-h2-a}
    \end{subfigure}
    \begin{subfigure}[t]{\textwidth}
        \centering
        \caption{system cost difference to full hydrogen and power grid expansion scenario}
        \includegraphics[width=.85\textwidth]{diff-internal-cost.pdf}
        \label{fig:sensitivity-h2-b}
    \end{subfigure}
    \caption{Benefits of electricity and hydrogen network infrastructure.
    \cref{fig:sensitivity-h2-a} compares four scenarios with and without
    expansion of a hydrogen network (left to right) and the electricity grid
    (top to bottom). Each bar depicts the total system cost of one scenario
    alongside its cost composition. Arrows between the bars indicate absolute
    and relative cost increases as network infrastructures are successively
    restricted. \cref{fig:sensitivity-h2-b} shows how the model reacts to grid
    expansion restrictions relative to the least-cost solution with full hydrogen and power grid expansion.}
    \label{fig:sensitivity-h2}
\end{figure}


\section*{Energy and carbon balances are dominated by wind and solar supply, electricity and heat demand, emissions of hydrocarbons, and removal from biomass}
\label{sec:balances}
\addcontentsline{toc}{section}{\nameref{sec:balances}}

First of all, we underline with the energy balance in \cref{fig:balance} the
central role of wind and solar electricity supply in all scenarios.
Hydroelectricity, biomass and the recovery of ambient heat through heat pumps
further support the energy supply, whereas fossil oil and gas only play a
marginal role as options to offset their emissions are limited by the assumed
sequestration potentials. Electricity demand for industrial processes,
electrified transport and the residential sector, alongside heat for hot water
provision, space heating and industrial processes, dominate the energy
consumption. For the most part, hydrogen is only an intermediate product between
electricity and another energy carrier. There are only few direct applications
of hydrogen, for instance, in the industry sector for producing steel with
hydrogen-based direct reduced iron rather than coke-fired integrated steelworks,
as well as for heavy-duty land transport. Most hydrogen is used to produce
derivatives like Fischer-Tropsch fuels, methane and methanol, which are used for
dense aviation and shipping fuels and as feedstock for high-value chemicals.
These liquid hydrocarbons are the third largest consumer in the energy balance.
In terms of energy, the reconversion of hydrogen to electricity only assumes a
secondary role. Conversion losses are also shown and are most pronounced for
electrolysis, as we assume that waste heat from other fuel synthesis plants can
be injected into the local district heating networks.

The atmospheric \co balance in \cref{fig:balance} shows that liquid hydrocarbons
in shipping, aviation and the decay or incineration of plastics constitute the
major uncaptured carbon dioxide emissions in the system. Some additional \co is
emitted through using methane (natural gas, biogas or synthetic) in gas boilers
and CHP plants in the heating sector during the challenging cold winter periods
with low renewable energy supply. Industrial process emissions are largely
captured such that, owing to imperfect capture rates, only residual emissions
are released into the atmosphere. Most carbon dioxide removal is achieved
through biomass technologies. For instance, biogenic \co is captured in biomass
CHP plants or industrial low-temperature heat applications. Direct air capture
takes a much smaller supplementary part. Of the \co handled by the system for
the synthesis of electrofuels and long-term sequestration, the largest share is
of biogenic origin (62\%) followed by fossil \co (25\%); direct air capture has
the smallest share with (13\%). The broad availability of captured \co from
industrial processes and biofuel combustion is advantageous for the system, as
it lowers the cost of fuel synthesis by evading costly and energy-intensive
direct air capture.

\section*{Cost benefit of hydrogen network is consistent, and strongest without power grid expansion}
\label{sec:h2}
\addcontentsline{toc}{section}{\nameref{sec:h2}}

In \cref{fig:sensitivity-h2}, we first compare the total energy system costs and
their composition between the four main scenarios, which vary in whether or not
the power grid can be expanded beyond today's levels and if a new hydrogen
network based on new and retrofitted pipelines can be built. Across all
scenarios, the total costs are dominated by investments in wind and solar
capacities, power-to-heat applications (primarily heat pumps), and in
electrolysers and further electrofuel synthesis plants (for transport fuels and
as a feedstock for the chemicals industry). System costs vary between
\minsystemcost~and \maxsystemcost~bn\euro/a, depending on available network
expansion options.

Overall, we find that energy system costs are not overly affected by restrictions on
the development of electricity or hydrogen transmission infrastructure. The
realisable cost savings are small compared to total costs, and systems
without grid expansion appear as equally feasible alternatives. The
combined net benefit of hydrogen and electricity grid expansion is
\gridbenefitabs~bn\euro/a; a system without either would be around
\gridbenefitrel\% more expensive. This limited cost increase can be attributed
to the high level of synthetic fuel production for industry, transport, and
backup electricity and heating applications. The option for a flexible operation
of conversion plants, inexpensive energy storage and low-cost energy transport as
hydrocarbons between regions offer sufficient leeway to manage electricity and
hydrogen transport restrictions effectively (see \nameref{sec:es}).

The total net benefit of power grid expansion is between
\minacbenefitabs-\maxacbenefitabs~bn\euro/a
(\minacbenefitrel-\maxacbenefitrel\%) compared to costs for transmission
reinforcements between \minaccost-\maxaccost~bn\euro/a. System costs decrease
despite the increasing investments in electricity transmission infrastructure.
The benefit is strongest if no hydrogen network can compensate for the lack of
electricity grid capacity to transport energy over long distances.
\cref{sec:si:lv} presents additional intermediate system cost changes between a
doubling of power grid capacity and no grid expansion. Electricity grid
reinforcement enables renewable resources with higher capacity factors to be
integrated from further away, resulting in lower capacity needs for solar and
wind. The grid also allows renewable variations to be smoothed in space and
facilitates the integration of offshore wind, resulting in lower hydrogen demand
for balancing power and heat and less hydrogen infrastructure (electrolysis,
cavern storage, re-conversion, pipelines). Restrictions on power grid expansion
entail more local production from solar photovoltaics and increased hydrogen
production.

The presence of a new hydrogen network can reduce system costs by up to
\maxhybenefitrel\%. The net benefit between
\minhybenefitabs-\maxhybenefitabs~bn\euro/a
(\minhybenefitrel-\maxhybenefitrel\%) largely exceeds the cost of the hydrogen
network, which costs between \minhycost-\maxhycost~bn\euro/a. Its system cost
benefit is strongest when the electricity grid is not expanded. However, even
with high levels of power grid expansion, the hydrogen network is still
beneficial infrastructure. The hydrogen network offers an alternative for bulk
energy transport from the windiest and sunniest regions in Europe's periphery to
low-cost geological storage sites and the industrial clusters in Central Europe
with high energy demand but less attractive and more constrained renewable
potentials (see \nameref{sec:energy-moved}).

Although power grid reinforcements provide higher cost reductions, hydrogen and
electricity networks are strongest together. Around
\hyoftotalbenefit\% of the combined cost benefit of transmission infrastructure
can be achieved solely with a new hydrogen network. In contrast,
\acoftotalbenefit\% of the combined cost benefit can be reached by just
reinforcing the electricity transmission system. Compared to the combined net
benefit of \gridbenefitabs~bn\euro/a, the individual benefits sum up to a value
that is only \additivebenefitrel\% higher (\maxacbenefitabs{} + \maxhybenefitabs{} =
\additivebenefitabs{} bn\euro/a). Thus, offered cost reductions are mainly
additive.

This also means that a hydrogen network cannot substitute perfectly for power
grid reinforcements. It can only partially compensate for the lack of grid
expansion, yielding only \benefithyofac\% of the cost reductions achieved
by electricity grid expansion. Instead, energy transport as electrons and
molecules seem to offer complementary strengths. From a system-level
perspective, network expansion will only lead to small cost reductions.
A system built exclusively around hydrogen network expansion is just \acvshycost\%
more expensive than an alternative system that only allows electricity grid
expansion.

\section*{Common design features across four scenarios of European climate neutrality}
\label{sec:es}
\addcontentsline{toc}{section}{\nameref{sec:es}}

Across all scenarios, we see \SIrange{\minoffwind}{\maxoffwind}{\giga\watt}
offshore wind, \SIrange{\minonwind}{\maxonwind}{\giga\watt} onshore wind, and
\SIrange{\minsolar}{\maxsolar}{\giga\watt} solar photovoltaics
(\cref{fig:si:capacities}). The wide range of solar capacities is due to an
increased localisation of electricity generation through solar photovoltaics
when the expansion of transmission infrastructure is limited. As network
expansion options are constrained, we see demand for local daily storage with
batteries almost quadrupling (from 73 to 272 GW with an energy-to-power ratio of
6 hours) and doubling for synoptic and seasonal storage with hydrogen and
thermal storage (from 73 to 141 TWh, see \cref{fig:si:capacities}). For all
scenarios, the capacities of photovoltaics split on average into
\meanrooftopshare\% rooftop PV and \meanutilityshare\% utility-scale PV. The
offshore share of wind generation capacities varies between \minoffshoreshare\%
and \maxoffshoreshare\% and is highest when networks can be fully expanded.

The spatial distribution of investments per scenario is shown in \cref{fig:tsc}.
While solar capacities are found throughout Europe, especially in the South,
onshore and offshore wind capacities are mostly found in the North Sea region
and the British Isles. When allowed, new electricity transmission capacity is
built where they help the integration of remote wind production and the
transport to inland demand centres (see \nameref{sec:h2}). Consequently, most
grid expansion is seen in and between Northwestern and Central Europe. Battery
storage pairs with solar generation in Southern Euope, particularly when power
grid reinforcement is limited. Besides their wider use overall, battery
deployment also progresses northbound in this case.

Furthermore, electrolyser capacities for power-to-hydrogen see a massive
scale-up to between \SIrange{\minelectrolysis}{\maxelectrolysis}{\giga\watt}
depending on the scenario. The capacities are lowest when the electricity grid
can be expanded. In this case, their locations correlate strongly with wind and
solar capacities (Pearson correlation coefficient $R^2=0.64$ for each,
\cref{fig:tsc}). If no hydrogen or electricity transmission expansion is
allowed, the electrolysis correlates more strongly with wind ($R^2=0.74$) than
solar ($R^2=0.46$). The build-out of hydrogen production facilities is
accompanied by a network of pipelines and hydrogen underground storage in Europe
to help balance generation from renewables in time and space.

In space, a new pipeline network transports hydrogen from preferred production
sites to the rest of Europe, where hydrogen is consumed by industry (for
ammonia, high-value chemicals and steel production), aviation and shipping, as
well as fuel cell CHPs for combined power and heat backup. Varying in magnitude
per scenario, we see major net flows of hydrogen from Great Britain to the
Benelux Union, Germany and Norway, from Northern Germany to the South, and from
the East of Spain to Southern France. The favoured network topology strongly
depends on the potentials for cheap renewable electricity. If onshore wind
potentials were restricted, e.g. due to limited social acceptance in Northern
Europe, the network infrastructure would be tailored to deliver larger amounts
of solar-based hydrogen from Southern Europe to Central Europe. We discuss this
supplementary sensitivity analysis in \cref{sec:si:onw,sec:si:onw-compromise}.

% Compared to net flows in the electricity network,
% which also balances renewable generation back and forth as weather systems pass
% the continent, the hydrogen network more distinctly targets energy transport
% over long distances (see also \nameref{sec:imbalance} and
% \cref{fig:si:flow-ac,fig:h2-network}).

The development of a hydrogen network is driven by the fact that (i) industry
demand for hydrogen is located in areas with less attractive renewable
potentials, (ii) the best wind and solar potentials are located in the periphery
of Europe, (iii) bottlenecks in the electricity transmission network give
impetus to alternative energy transport options, and (iv) moving hydrogen from
production sites to where the geological conditions allow for cheap underground
storage is significantly more cost-effective than local storage in steel tanks.
Another subsidiary location factor for hydrogen network infrastructure is linked
to the siting of electrofuel production. Because we assume that waste heat from
these processes can be recovered for district heating networks, urban areas with
attractive renewable potentials nearby may be preferred sites for fuel synthesis
to which the hydrogen would need to be transferred. Since we assume that liquid
hydrocarbons can be moved freely in the model, the spatial distribution of their
demands is not a siting factor that is considered. Neither is the location of
carbon dioxide sources and sinks.

The flexible operation of electrolysers further supports the system integration
of variable renewables in time. Hydrogen production leverages periods with
exceptionally high wind speeds across Europe by running the electrolysis with
average utilisation rates between \mincfelectrolysis\% and \maxcfelectrolysis\%
(see \cref{fig:output-ts-1,fig:output-ts-3}). The produced hydrogen is buffered
in salt caverns which then allows for higher full load hours of fuel synthesis
processes. For Fischer-Tropsch and methanolisation plants, we see average
utilisation rates between \mincfFT\% and \maxcfFT\% which aligns with the higher
upfront investment costs of these processes. Their operation is mostly
interrupted in winter periods with low wind speeds and low ambient temperatures
to give way to backup heat and power supply options (see
\cref{fig:output-ts-1,fig:output-ts-3}). By exploiting periods of peak
generation and curbing production in periods of scarcity, large amounts of
variable renewable power generation that serves the systems' abundant synthetic
fuel demands can be incorporated into the system cost-effectively. This
ultimately leads to little curtailment of renewables between 2\% and 3\%
(\cref{fig:si:curtailment}) and low levels of firm capacity. In relation to a
peak electricity consumption of 2626~GW\el, we observe OCGT and CHP plant
capacities between 106 and 218~GW\el, most of which are gas CHP plants. The
lowest values were attained when additional power transmission could be built.

Hydrogen storage is required to benefit from temporal balancing through flexible
electrolyser operation. We find cost-optimal storage capacities between
\SIrange{\hydrogenstorageacyhyy}{\hydrogenstorageacnhyy}{\twh} with a hydrogen
network and \SIrange{\hydrogenstorageacnhyn}{\hydrogenstorageacyhyn}{\twh}
without a hydrogen network while featuring similar filling level patterns
throughout the year (\cref{fig:si:soc}). Almost all hydrogen is stored in salt
caverns, exploiting vast geological potentials across Europe mostly in Northern
Ireland, England and Denmark. We observe no storage in steel tanks unless
neither a hydrogen nor the electricity network can be expanded. In this case, we
see up to 1~TWh of steel tank capacity, which represents 5\% of the total
hydrogen storage capacity. If the options for network development are
restricted, more hydrogen storage is built to balance renewables in time rather
than space.

Together with the supporting infrastructure, the production of large amounts of
hydrogen (\hydrogenproduction~TWh/a) offers versatile use cases. Most of the
hydrogen is used to produce methanol and Fischer-Tropsch fuels for organic
chemicals and transport fuels in aviation and shipping
(\ptlhydrogenusage~TWh/a), of which \ptlwasteheat~TWh/a is useable  in the form
of waste heat for district heating networks. A total of
\hydrogentransportdemand~TWh/a is used in land transport, while the industry
sector consumes \hydrogenindustrydemand~TWh/a, excluding the consumption of hydrogen for
industry feedstocks (e.g.~high-value chemicals). Around \hydrogenlosses~TWh/a of
hydrogen is lost during synthetic fuel production. If the electricity grid
expansion is restricted, but hydrogen can be transported, even more hydrogen is
produced to be re-electrified in fuel cells during critical phases of system
operation (\hydrogenfuelcell~TWh$_{\ce{H2}}$). According to \cref{fig:tsc},
these fuel cells would mostly be built inland in Central Europe. In all scenarios with
network expansion, no synthetic methane for process
heat in some industrial applications and as a heating backup for power-to-heat
units is produced. This is because the model prefers to use the full potential
for biogas (\biogas~TWh/a) and limited amounts of fossil gas (\fossilgas~TWh/a),
which are offset by sequestering biogenic carbon dioxide, over synthetic
production.

Only when neither hydrogen nor power network expansion were allowed, we see
methanation (\ce{H2}-to-\ce{CH4}, \SI{\hydrogenmethanation}{\twh}) and steam
methane reforming with carbon capture (\ce{CH4}-to-\ce{H2},
\SI{\bluehydrogen}{\twh}). In this case, despite the associated conversion
losses, synthetic methane is used as a transport medium to use the existing gas
network to counteract the restricted transport options otherwise. Apart from
imperfect capture rates of 90\% that requires supplementing some CO$_2$, this
process creates a carbon cycle provided that the \co is returned to the
methanation sites with an appropriate \co transport infrastructure. Direct air
capture (\SIrange{\mindac}{\maxdac}{\mega\tco\per\year}) was made use of in all
scenarios, supplementing carbon available from biogenic or fossil sources, while
remaining within set sequestration limits. For a comprehensive overview of
energy and carbon flows in each scenario see
\cref{fig:si:sankey,fig:si:carbon-sankey}.

\begin{figure}
    \centering
    \vspace{-2cm}
    \makebox[\textwidth][c]{
        \begin{subfigure}[t]{0.5\textwidth}
            \centering
            \caption{with power grid reinforcement, with hydrogen network}
            \includegraphics[width=\textwidth, trim=0cm .3cm 7cm 0cm, clip]{\hyrun/maps/elec_s_181_lvopt__Co2L0-3H-T-H-B-I-A-solar+p3-linemaxext10-costs-all_2050.pdf}
            \label{fig:tsc:w-el-w-h2}
        \end{subfigure}
        \begin{subfigure}[t]{0.5\textwidth}
            \centering
            \caption{with power grid reinforcement, without hydrogen network}
            \includegraphics[width=\textwidth, trim=0cm .3cm 7cm 0cm, clip]{\hyrun/maps/elec_s_181_lvopt__Co2L0-3H-T-H-B-I-A-solar+p3-linemaxext10-noH2network-costs-all_2050.pdf}
            \label{fig:tsc:w-el-wo-h2}
        \end{subfigure}
    } \makebox[\textwidth][c]{
        \begin{subfigure}[t]{0.5\textwidth}
            \centering
            \caption{without power grid reinforcement, with hydrogen network}
            \includegraphics[width=\textwidth, trim=0cm .3cm 7cm 0cm, clip]{\hyrun/maps/elec_s_181_lv1.0__Co2L0-3H-T-H-B-I-A-solar+p3-linemaxext10-costs-all_2050.pdf}
            \label{fig:tsc:wo-el-w-h2}
        \end{subfigure}
        \begin{subfigure}[t]{0.5\textwidth}
            \centering
            \caption{without power grid reinforcement, without hydrogen network}
            \includegraphics[width=\textwidth, trim=0cm .3cm 7cm 0cm, clip]{\hyrun/maps/elec_s_181_lv1.0__Co2L0-3H-T-H-B-I-A-solar+p3-linemaxext10-noH2network-costs-all_2050.pdf}
            \label{fig:tsc:wo-el-wo-h2}
        \end{subfigure}
    } \vspace{-.5cm} \makebox[1\textwidth][c]{
        \centering
    \includegraphics[width=1\textwidth]{color_legend}
    } \caption{ Regional distribution of system costs and electricity grid
    expansion for scenarios with and without electricity or hydrogen network
    expansion. The pie charts depict the annualised system cost alongside the
    shares of the various technologies for each region. The line widths depict
    the level of added grid capacity between two regions, which was capped at 10
    GW.}
    \label{fig:tsc}
\end{figure}


\section*{Hydrogen network takes over role of bulk energy transport}
\label{sec:energy-moved}
\addcontentsline{toc}{section}{\nameref{sec:energy-moved}}

\begin{figure}
    \centering
    % \makebox[\textwidth][c]{
        \begin{subfigure}[t]{0.49\textwidth}
            \centering
            \caption{transmission capacity built}
            \includegraphics[width=\textwidth]{twkm-main}
            \label{fig:network-stats:twkm}
        \end{subfigure}
        \begin{subfigure}[t]{0.49\textwidth}
            \centering
            \caption{energy volume transported}
            \includegraphics[width=\textwidth]{ewhkm-main}
            \label{fig:network-stats:ewhkm}
        \end{subfigure}
    % }
    \caption{Transmission capacity built and energy volume transported for
        various network expansion scenarios. For the hydrogen network, a
        distinction between retrofitted and new pipelines is made. For the
        electricity network, a distinction is made between existing and added
        capacity or how much energy is moved via HVAC or HVDC power lines. Both
        measures weight capacity (TW) or energy (EWh) by the length (km) of the
        network connection.}
    \label{fig:network-stats}
\end{figure}

%  shows statistics on the total electricity and hydrogen
% transmission capacity built as well as how much energy is moved through the
% respective networks, while distinguishing between retrofitted and new
% capacities.

Depending on the level of power grid expansion, between 204 and
307~TWkm of hydrogen pipelines are built
(\cref{fig:network-stats:twkm}). The higher value is obtained when the hydrogen
network partially offsets the lack of electricity grid reinforcement. On the
other hand, restricting hydrogen expansion only has a small effect on
cost-optimal levels of power grid expansion. The length-weighted power grid
capacity is more than doubled in the least-cost scenario; without a hydrogen
network, the cost-optimal power grid capacity is \twkmhigher\% higher.

When both hydrogen and electricity grid expansion is allowed, the hydrogen
network transports approximately half the amount of energy transmitted via the
electricity network (\cref{fig:network-stats:ewhkm}). This is striking because
the hydrogen network capacity is little more than a quarter that of the power
grid (\cref{fig:network-stats:twkm}). In consequence, the utilisation rate of
\utilisationHy\% of the hydrogen network is much higher than the
\utilisationAC\% of the electricity grid (\cref{fig:si:grid-utilisation}). One
plausible explanation for this observation is that the buffering of produced
hydrogen in cavern storage allows more coordinated bulk energy tranport in
hydrogen networks, whereas the power grid directly balances the variability of
renewable electricity supply and is subject to linearised power flow physics
(Kirchhoff's circuit laws).

When electricity grid expansion is restricted, the hydrogen network plays a
dominant role in transporting energy around Europe. In this case, around twice
as much energy is moved in the hydrogen network (\ewhkmhydrogen~EWhkm) than in
the electricity network (\ewhkmelectricity~EWhkm). Between only power grid
expansion and only hydrogen network expansion, the difference in the total
volume of energy transported is only \ewhkmdiff\%.

\section*{New hydrogen network can leverage repurposed natural gas pipelines}
\label{sec:repurposed}
\addcontentsline{toc}{section}{\nameref{sec:repurposed}}

\begin{figure}
    \centering
    \makebox[\textwidth][c]{
        \begin{subfigure}[t]{0.65\textwidth}
            \centering
            \caption{hydrogen infrastructure with power grid reinforcement}
            \includegraphics[width=\textwidth]{\hyrun/maps/elec_s_181_lvopt__Co2L0-3H-T-H-B-I-A-solar+p3-linemaxext10-h2_network_2050.pdf}
            \label{fig:h2-network:w-el}
        \end{subfigure}
        \begin{subfigure}[t]{0.65\textwidth}
            \centering
            \caption{hydrogen infrastructure without power grid reinforcement}
            \includegraphics[width=\textwidth]{\hyrun/maps/elec_s_181_lv1.0__Co2L0-3H-T-H-B-I-A-solar+p3-linemaxext10-h2_network_2050.pdf}
            \label{fig:h2-network:wo-el}
        \end{subfigure}
    }
    \makebox[\textwidth][c]{
    \begin{subfigure}[t]{0.65\textwidth}
        \centering
        \caption{hydrogen flows with power grid reinforcement}
        \includegraphics[width=\textwidth]{\hyrun/elec_s_181_lvopt__Co2L0-3H-T-H-B-I-A-solar+p3-linemaxext10_2050/H2-flow-map-backbone.pdf}
    \end{subfigure}
    \begin{subfigure}[t]{0.65\textwidth}
        \centering
        \caption{hydrogen flows without power grid reinforcement}
        \includegraphics[width=\textwidth]{\hyrun/elec_s_181_lv1.0__Co2L0-3H-T-H-B-I-A-solar+p3-linemaxext10_2050/H2-flow-map-backbone.pdf}
    \end{subfigure}
    } \caption{Optimised hydrogen network, storage, reconversion and production
    sites with and without electricity grid reinforcement. The size of the
    circles depicts the electrolysis and fuel cell capacities in the respective
    region. The line widths depict the optimised hydrogen pipeline capacities.
    The darker shade depicts the share of capacity built from retrofitted gas
    pipelines. The coloring of the regions indicates installed hydrogen storage
    capacities. The second row shows net flow of hydrogen in the network and the
    respective energy balance. Flows larger than 2 TWh are shown with
    arrow sizes proportional to net flow volume.}
    \label{fig:h2-network}
\end{figure}

With our assumptions, developing electricity transmission lines is approximately
60\% more expensive than building new hydrogen pipelines. We assume costs for a
new hydrogen pipeline of 250~\euro/MW/km, whereas, for a new high-voltage
transmission line, we assume 400~\euro/MW/km (see~\cref{sec:si:costs}). Despite
higher costs, we observe that electricity grid reinforcements are preferred over
hydrogen pipelines. Part of the reason may be that electricity is more versatile
in our scenarios with high levels of direct electrification. If hydrogen has to
be produced and then re-electrified, the efficiency losses mean additional
generation capacity would be needed to compensate. This makes energy transport
in form of hydrogen less competitive. However, hydrogen pipelines are
particularly attractive where the end-use is hydrogen-based.

The appeal of a hydrogen network is further spurred when existing natural gas
pipelines are available for retrofitting. Repurposing costs just around half
that of building a new hydrogen pipeline (117 versus 250 \euro/MW/km;
see~\cref{sec:si:costs}). For the capacity retrofit we include costs for
required compressor substitutions and assume that for every unit of gas pipeline
decommissioned, 60\% of its capacity becomes available for hydrogen transport.
In consequence, even detours of the hydrogen network topology may be
cost-effective if, through rerouting, more repurposing potentials can be tapped.

As \cref{fig:h2-network} illustrates, the optimised hydrogen network topology is
built around supporting flows into the industry and population centres of
Central Europe. We see strong pipeline connections in Northwestern Europe to
integrate wind-based hydrogen hubs as well as connections for the transport of
solar hydrogen hubs from Spai, Italy and Greece. Individual pipeline connections
between regions have optimised capacities up to 30 GW. Of the total hydrogen
network volume, between \minretroshare\% and \maxretroshare\% consists of
repurposed gas pipelines. The share is highest when the electricity grid is not
permitted to be reinforced. Up to a quarter of the existing natural gas network
is retrofitted to transport hydrogen instead, leaving large capacities that are
used neither for hydrogen nor methane transport. In our scenarios, 29-42\% of
retrofittable gas pipelines fully exhaust their conversion potential to
hydrogen. The most notable corridors for gas pipeline retrofitting are located
offshore across the North Sea and the English Channel and in Great Britain,
Germany, Austria, Switzerland, Northern France and Italy. The most prominent new
hydrogen pipelines are built in the British Isles particularly to connect
Ireland, Northern France, the Netherlands, and in Spain and Portugal. The
sizeable existing natural gas transmission capacities in Southern Italy and
Eastern Europe are largely not repurposed for hydrogen transport in this
self-sufficient scenario for Europe.

However, this picture would change if clean energy import options were
considered. Since most hydrogen is used to produce synthetic fuels and ammonia,
if these were imported, much of the hydrogen demand would fall away, thereby
also reducing the extent hydrogen infrastructure. We present this case in
\cref{sec:si:sensitivity-imports}. Moreover, direct hydrogen imports into Europe
could alter cost-effective network topologies as new import locations need to be
connected rather than domestic production sites. For instance, the networks role
might change from distributing energy from North Sea hydrogen hubs to
integrating inbound pipelines from North-Africa with increased network
capacities in Southern Europe.\cite{wetzelGreenEnergy2022}

\section*{Regional imbalance of supply and demand is reinforced by transmission}
\label{sec:imbalance}
\addcontentsline{toc}{section}{\nameref{sec:imbalance}}

\begin{figure}
    \centering
    \makebox[\textwidth][c]{
        \begin{subfigure}[t]{0.65\textwidth}
            \centering
            \caption{with power grid reinforcement, with hydrogen network}
            \includegraphics[width=\textwidth]{\hyrun/elec_s_181_lvopt__Co2L0-3H-T-H-B-I-A-solar+p3-linemaxext10_2050/import-export-total-200.pdf}
            \label{fig:io:w-el-w-h2}
        \end{subfigure}
        \begin{subfigure}[t]{0.65\textwidth}
            \centering
            \caption{with power grid reinforcement, without hydrogen network}
            \includegraphics[width=\textwidth]{\hyrun/elec_s_181_lvopt__Co2L0-3H-T-H-B-I-A-solar+p3-linemaxext10-noH2network_2050/import-export-total-200.pdf}
            \label{fig:io:w-el-wo-h2}
        \end{subfigure}
    } \makebox[\textwidth][c]{
        \begin{subfigure}[t]{0.65\textwidth}
            \centering
            \caption{without power grid reinforcement, with hydrogen network}
            \includegraphics[width=\textwidth]{\hyrun/elec_s_181_lv1.0__Co2L0-3H-T-H-B-I-A-solar+p3-linemaxext10_2050/import-export-total-200.pdf}
            \label{fig:io:wo-el-w-h2}
        \end{subfigure}
        \begin{subfigure}[t]{0.65\textwidth}
            \centering
            \caption{without power grid reinforcement, without hydrogen network}
            \includegraphics[width=\textwidth]{\hyrun/elec_s_181_lv1.0__Co2L0-3H-T-H-B-I-A-solar+p3-linemaxext10-noH2network_2050/import-export-total-200.pdf}
            \label{fig:io:wo-el-wo-h2}
        \end{subfigure}
    } \caption{Total energy balances for scenarios with and without electricity
    or hydrogen network expansion for the 181 model regions, revealing regions
    with net energy surpluses and deficits. The Lorenz curves on the upper left
    of each map depict the regional imbalances of electricity, hydrogen, methane
    and liquid hydrocarbon supply relative to demand. Methane and liquid
    hydrocarbon supply can be of fossil, biogenic or synthetic origin. If the
    annual sums of supply and demand are equal in each region, the Lorenz curve
    resides on the identiy line. But the more imbalanced the regional supply is
    relative to demand, the further the curve dents into the bottom right corner
    of the graph.}
    \label{fig:io}
\end{figure}

In line with previously shown capacity expansion plans, energy surplus is found
largely in the windy coastal and sunny Southern regions that supply the inland
regions of Europe, which have high demands but less attractive renewable
potentials (\cref{fig:io}). The net energy surplus of individual regions amounts
to up to 260 TWh. Examples are Danish offshore wind power exports and large
wind-based production sites for synthetic fuels in Ireland. For Denmark, this
surplus is more than twice as high as its final energy demand, resulting in the
situation that three quarters of Denmark's energy production is exported. Net
deficits of single regions can have similarly high values, close to 200 TWh.
Examples are, in particular, the industrial cluster between Rotterdam and the
Ruhr valley as well as other European metropolises.

Energy transport infrastructure fuels the uneven regional distribution of supply
relative to demand. This is illustrated by the Lorenz curves in
\cref{fig:io} for different energy carriers. The
Lorenz curves plot the carrier's cumulative share of supply versus the
cumulative share of demand, sorted by the ratio of supply and demand in
ascending order. If the annual sums of supply and demand are equal in each
region, the Lorenz curve resides on the identity line. However, the more unequal
the regional supply is relative to demand, the further the curves dent into the
graph's bottom right corner.

For the least-cost scenario, \cref{fig:io:w-el-w-h2} highlights that hydrogen
supply is slightly more regionally imbalanced relative to demand than
electricity supply. Roughly 60\% of the hydrogen demand is consumed in regions
that produce less than 11\% of the total hydrogen supply. Conversely, 40\% of
the hydrogen supply is produced in regions that consume less than 12\% of total
hydrogen demand. Naturally, reduced electricity grid expansion causes more
evenly distributed electricity supply
(\cref{fig:io:wo-el-w-h2,fig:io:wo-el-wo-h2}). If hydrogen transport is
restricted (\cref{fig:io:w-el-wo-h2,fig:io:wo-el-wo-h2}), the production of
liquid hydrocarbons is increased in regions with attractive renewable potentials
because they can be transported at low cost.

%In this case, 70\% of the demand for
% liquid hydrocarbons is consumed in regions that produce less than 1\% of the
% total supply. With full network expansion, 70\% of demand is consumed in regions
% that produce 16\% of the total supply.




\section*{Discussion}
\label{sec:discussion}
\addcontentsline{toc}{section}{\nameref{sec:discussion}}

To put our results into a broader perspective, for the discussion we compare our
results to related literature and proposals presented in the European Hydrogen
Backbone reports by the gas industry. This is followed by a short discussion of
the public acceptance for hydrogen infrastructure and a critical apprasal of our
assumption of a self-sufficient Europe without energy imports and other
limitations of the study.

\subsection*{Comparison to Related Literature}

Compared to the net-zero scenarios from the European Commission released in 2018
\cite{in-depth_2018}, we see much larger renewable electricity generation, reaching beyond
8781~TWh/a. This represents approximately a tripling
of today's electricity generation (compared to at most a 145\% increase by 2050
in \cite{in-depth_2018}), with roughly one third going to regular electricity
demand, one third going to newly-electrified sectors in heating, transport and
industry, and one third going to hydrogen production (dominated by
Fischer-Tropsch fuels). The major difference is caused by lower electrification
rates in other sectors in \cite{in-depth_2018}, the higher biomass potentials in
\cite{in-depth_2018}, and the fact that \cite{in-depth_2018} relies on imports
of fossil oil for non-energy uses such as plastics and does not count them
towards net emissions like we do.

In \cite{brownSynergiesSector2018} an optimal grid expansion brought a benefit
of \euro~64~billion per year compared to the case with no transmission between
European countries, which is higher than the \euro~47~billion per year benefit
found here. There are at least four causes for this difference: the model here
has higher resolution (181 versus 30 nodes) which allows better placement of
wind at good sites; here we start from today's grid, which already has some
international transmission; in \cite{brownSynergiesSector2018} there was no
hydrogen pipeline network and no underground hydrogen storage (just steel
tanks); and finally we have higher demand for hydrogen from industry and
synthetic fuels, which provides a large flexible load that helps to integrate
wind and solar.

Caglayan et al.~\cite{Caglayan2019} also consider European decarbonisation
scenarios with both electricity transmission and new hydrogen pipelines, but at
a lower spatial resolution (96 nodes). A similar pattern of hydrogen pipeline
expansion towards the British Isles and North Sea is seen, but lower overall
hydrogen capacities (258~GW compared to more than 1000~GW in our scenarios)
because industry, shipping, aviation and non-electrified heating are not
included. While \cite{Caglayan2019} see 130~TWh of cavern storage, our scenarios
are in a lower range between 32 and 66~TWh.

\subsection*{Comparison to the European Hydrogen Backbone}

compare annualised investment costs and volume to EHB

Cost of hydrogen network
- our models \euro 5-8 bn/a
- EHB: \euro 4-8 bn/a

\begin{tabular}{lrr}
    \toprule
     & Repurposed & New \\
     Scenario& [TWkm] & [TWkm] \\
    \midrule
    European Hydrogen Backbone & 208 & 101 \\
    PyPSA-Eur-Sec (with grid expansion) & 199 & 143 \\
    PyPSA-Eur-Sec (no grid expansion) & 277 & 145 \\
    \bottomrule
  \end{tabular}

EHB (June 2021)
- made for hydrogen demand between 2100 and 2700 TWh for 2050 compared to 2437 TWh in our scenarios
- however, large re-electrification (650 TWh) which is only seen with restricted electricity transmission (200~TWh H2)
- considers imports
- 60\% repurposed pipelines
- levelised transport cost (0.11-0.21 \euro/kg/1000km or 3.3-6.3 \euro/MWh/1000km)

EHB (April 2022)
- 80-143 bn\euro + 1.6-3.2 bn\euro/a approx. 7.5-14 bn\euro/a with 50 year lifetime and 7\% discount rate
- without Balkan

\cite{gasforclimateExtendingEuropean2021}

\subsection*{Public acceptance for hydrogen infrastructure}

One of the biggest changes seen is the built-out of hydrogen infastructure: huge new
electrolyzer capacities, underground storage in salt caverns as well
as a new hydrogen pipeline network. It is not fully clear that a new
hydrogen network will have any higher public acceptance than the power
grid. However, if existing natural gas pipelines can be reused for hydrogen,
or if at least the pipeline routes can be used, and are always discreetly buried underground, the disturbance to
the public is minimised.

\subsection*{Energy imports and European self-sufficiency}

limitation no imports

in these scenarios, Europe is largely self-sufficient (fossil gas imports allowed)

very uneven infrastructure distribution

with imports
- strong point sources
- higher role of hydrogen network

future: increase self-sufficiency constraints of individual regions

\subsection*{Further limitations of the study}

Broad ranges of options with similar costs. The flatness of the total system
costs as we vary grid expansion and onshore wind potentials is a general feature
of energy system models: there are many directions in the feasible space where
we can change the system composition with only a small change in total system
costs. This flatness can be explored systematically using techniques similar to
Modelling to Generate Alternatives (MGA), and was investigated for an
electricity-only version of this model in \cite{Neumann2019}.


\section*{Conclusion}
\label{sec:conclusion}
\addcontentsline{toc}{section}{\nameref{sec:conclusion}}

In this work, we have investigated the potential role of a hydrogen network in
net-zero \co scenarios for Europe with high shares of renewables. The analysis
was performed using the open sector-coupled energy system model PyPSA-Eur-Sec
featuring high spatio-temporal coverage of all energy sectors (electricity,
buildings, transport, agriculture and industry across 181 regions and 3-hourly
resolution for a year). With this level of spatial, temporal, technological and
sectoral resolution, it is possible to represent grid bottlenecks as well as the
variability and regional distribution of demand and renewable supply. Thereby,
the system's infrastructure needs regarding generation, storage, transmission
and conversion can be assessed. This includes in particular trade-offs between
electricity grid reinforcement, which has limited public support, and developing
a hydrogen network, for which increasingly unused gas pipelines can be
repurposed.

Besides large-scale renewables expansion, the build-out of hydrogen
infrastructure is one of the biggest changes seen in our scenarios of the future
European energy system. Huge new electrolyser capacities enter the system and
operate flexibly to aid renewables integration. Furthermore, underground storage
in salt caverns is developed for seasonal balancing and a new continent-spanning
hydrogen pipeline network is built to connect cheap supply and storage
potentials with demand centres. This new hydrogen network is found to be
supported by considerable amounts of gas pipeline retrofitting: between
\minretroshare\% and \maxretroshare\% of the network uses repurposed pipelines.

Our analysis reveals that a hydrogen network can reduce energy system costs by
up to \maxhybenefitrel\%. Cost reductions are shown to be highest when the
expansion of the power grid is restricted. However, hydrogen networks can only
partially substitute for grid expansion. We found that in fact both ways of
transporting energy and balancing renewable generation complement each other and
achieve the highest cost savings of up to \gridbenefitrel\% together. At the
same time, these findings also support the interpretation that neither
electricity nor hydrogen network expansion are essential for achieving a
cost-effective system design if such a cost premium can be accepted to achieve
alternative goals.

In conclusion, there appear to be many infrastructure trade-offs regarding how
energy is transported across Europe with limited impacts on total system cost
such that policymakers could choose from a wide range of near-optimal compromise
energy system designs with equally low cost but possibly higher acceptance.

% cross-sectoral approaches are important to reduce CO2 emissions
% cost-effectively and for flexibility

\section*{Experimental Procedures}
\label{sec:methods}
\addcontentsline{toc}{section}{\nameref{sec:methods}}

overnight scenario, disregarding pathway

weather year 2013
- Detailed transmission grid representation in EU-wide model
- Detailed representation of demand sectors: Industry, buildings, transport
- so that the variability of demand and variable renewable supply can be represented, and so that existing grid bottlenecks are visible.
In this study, we used the historical year 2013 for weather-dependent inputs and applied a net-zero
emission constraint

The model uses linear optimisation to minimise total annual operational and
investment costs subject to technical and physical constraints, assuming perfect
competition and perfect foresight over one year of 3-hourly operation. Apart
from existing electricity transmission and hydroelectric facilities, no other
existing assets are assumed (so-called `greenfield optimisation'), so that the
model represents an ideal steady state.  Cost assumptions are taken, where
possible, from predictions for the year 2030 by the Danish Energy Agency
\cite{dea2019}. The model is implemented in the free software framework Python
for Power System Analysis (PyPSA) \cite{brownPyPSAPython2018}.

In this section the main assumptions used in the model PyPSA-Eur-Sec are
presented. PyPSA-Eur-Sec builds upon the model from \cite{brown2018}, which
covered electricity, heating in buildings and ground transport in Europe with
one node per country. PyPSA-Eur-Sec adds biomass on the supply side, and
industry, aviation and shipping on the demand side. Unavoidable process
emissions, as well as the need for feedstocks for the chemicals industry and
dense hydrocarbon fuels for aviation, necessitate careful management of the
carbon cycle, including carbon capture from industry, biomass combustion and
directly from the air.

Figure gives an overview of the supply, transmission,
storage and demand sectors implemented in the European sector-coupled model
PyPSA-Eur-Sec. Generator capacities (for onshore wind, offshore wind, solar
photovoltaic (PV), biomass and natural gas), storage capacities (for batteries,
hydrogen, methane, liquid hydrocarbons, carbon dioxide and hot water tanks),
heating capacities (for heat pumps, resistive heaters, gas boilers, combined
heat and power (CHP) plants and solar thermal collector units), carbon capture
(from industry, CHP plants and directly from the air), energy converters
(electrolyzers, methanation, Fischer-Tropsch) and transmission capacities for
electricity and hydrogen are all subject to optimisation, as well as the
operational dispatch of each unit in each hour. Demand curves for the different
sectors, the ratio of district heating to decentralised heating, the number of
electric vehicles, methane storage and hydroelectricity capacities (for
reservoir and run-of-river generators and pumped hydro storage) are exogenous to
the model.

The European transmission network model is based on the open model PyPSA-Eur
presented in \cite{horschPyPSAEurOpen2018}. The state of the network in 2018 is plotted. The full European transmission network is clustered
down to 181 representative nodes based on the methodology used in
\cite{Hoersch2017}, thereby preserving the most important transmission corridors
that cause bottlenecks. The linearised optimal power flow uses a cycle-based
formulation from \cite{horschLinearOptimal2018} that significantly improves computational
performance; as transmission lines are expanded, impedances are updated
iteratively until convergence is achieved.

The sector-coupling model is based on the open model PyPSA-Eur-Sec from
\cite{brownSynergiesSector2018} which added to the electricity-only model of
\cite{schlachtbergerBenefitsCooperation2017} both land transport as well as space and water heating
in the residential and commercial sectors. In the model presented in this
contribution, biomass, industry demand (separately for sectors including iron
and steel, concrete, chemicals) and transport fuels for aviation and shipping
have also been included.

For biomass, only waste and residues from agriculture and forestry are
permitted, using the most conservative potential estimates from the JRC-EU-TIMES
model \cite{jrcbiomass2015}. This results in 352~TWh per year of biogas and
1261~TWh per year of solid biomass residues and waste for the whole of Europe.

For industry, we change some industrial processes to low-emission ones (e.g.
switching to hydrogen for direct reduction of iron ore \cite{voglAssessmentHydrogen2018}), allow
more recycling of steel and aluminium \cite{circular_economy}, switch fuel
sources for process heat, use synthetic fuels for ammonia and organic chemicals,
and allow carbon capture. It is assumed that no plastic or other non-energy
product is sequestered in landfill, but that all carbon in plastics eventually
makes its way back to the atmosphere, either through combustion or decay; this
approach is stricter than other models \cite{in-depth_2018}.

Transport and mobility comprises light and heavy road, rail, shipping and
aviation transport. For road and rail, electrification and fuel cell vehicles are
available. For shipping, liquid hydrogen is considered. For aviation, we
consider dense liquid hydrocarbons. Battery electric vehicles for passenger
transport can be enables with demand response as well as vehicle-to-grid
capabilities.

Energy sector coupling, storage and conversion is modelled to connect
electricity, heating (individual buildings, district heating and industry),
transport and gas (methane, hydrogen and carbon dioxide) in the different
sectors (buildings, transport and industry).

Electricity can be converted to heat via heat pumps or resistive heaters; to
hydrogen gas, or further to methane and liquid hydrocarbons; or to work in
various demand devices. Methane can be reformed to hydrogen, and most fuels can
be used for electricity generation in turbines or fuel cells.

Biomass can be used in electricity generation with and
without CCS, as well as to provide low- to medium-temperature process heat in
industry.

Energy/material storage can be optimised including conventional
pumped hydro storage, electrochemical storage like Lithium ion batteries,
storage of gases including methane, hydrogen and carbon dioxide, storage of
liquid fuels, as well as thermal energy storage in the form of hot water both in
individual buildings and in district heating networks.

Carbon capture is
needed in the model both to capture and sequester process emissions with a
fossil origin, such as those from calcination of fossil limestone in the cement
industry, as well as to provide carbon for the production of hydrocarbons for
dense transport fuels and as a chemical feedstock, for example for the plastics
industry.

\section*{Acknowledgements}

% F.N. and T.B. gratefully acknowledge funding from the Helmholtz
% Association under grant no. VH-NG-1352.

\section*{License}

This work is licensed under a \href{http://creativecommons.org/licenses/by/4.0/}{Creative Commons Attribution 4.0
International License (CC-BY-4.0)}.

\section*{Author Contributions}

% following https://casrai.org/credit/

\textbf{Author FN}:
Conceptualization --
Data curation --
Formal Analysis --
Investigation --
Methodology --
Software --
Supervision --
Validation --
Visualization --
Writing - original draft --
Writing - review \& editing
\textbf{Author EZ}:
Data curation --
Formal Analysis --
Investigation --
Software --
Validation --
Writing - review \& editing
\textbf{Author MV}:
Formal Analysis --
Investigation --
Methodology --
Software --
Writing - review \& editing
\textbf{Author TB}:
Conceptualization --
Data curation --
Formal Analysis --
Funding acquisition --
Investigation --
Methodology --
Project administration --
Resources --
Software --
Supervision --
Writing - original draft --
Writing - review \& editing


\section*{Data and Code Availability}

A dataset of the model results is available at \href{zenodoTBA}{doi:zenodoTBA}.
The code to reproduce the experiments is available at \href{https://github.com/fneum/spatial-sector}{github.com/fneum/spatial-sector}.
We also refer to the documentation of PyPSA (\href{https://pypsa.readthedocs.io}{pypsa.readthedocs.io}),
PyPSA-Eur (\href{https://pypsa-eur.readthedocs.io}{pypsa-eur.readthedocs.io}), and
PyPSA-Eur-Sec (\href{https://pypsa-eur-sec.readthedocs.io}{pypsa-eur-sec.readthedocs.io}).

% tidy with https://flamingtempura.github.io/bibtex-tidy/
\addcontentsline{toc}{section}{References}
\renewcommand{\ttdefault}{\sfdefault}
%\bibliography{library}
\bibliography{/home/fneum/zotero}

% supplementary information

\newpage

\makeatletter
\renewcommand \thesection{S\@arabic\c@section}
\renewcommand\thetable{S\@arabic\c@table}
\renewcommand \thefigure{S\@arabic\c@figure}
\makeatother

\renewcommand{\citenumfont}[1]{S#1}

\setcounter{equation}{0}
\setcounter{figure}{0}
\setcounter{table}{0}
\setcounter{section}{0}

\section{Model Overview}

Not all of the sectors are at the full nodal resolution, and some demand for
some sectors is distributed to nodes using heuristics that need to be corrected.
Some networks are copper-plated to reduce computational times.

\section{Electricity Sector}

Electricity supply and demand follows the electricity generation and
transmission model PyPSA-Eur, except that hydrogen storage is integrated into
the hydrogen supply, demand and network, and PyPSA-Eur-Sec includes CHPs.


\subsection{Electricity Demand}

distribution electricity demand for
industry uses the geographical data from
the Hotmaps Industrial Database.

subtracts existing electrified heating from
the existing electricity demand, so that power-to-heat can be optimised
separately.
Building heating demand: nodal, distributed in each country based on population.

The remaining electricity demand for households and services is distributed
inside each country proportional to GDP and population.

\subsection{Electricity Supply}

OCGT CCGT

hydro

run of river

\subsection{Electricity Storage}

battery

Distinguish costs for home battery storage and inverter from utility-scale battery costs.

pumped-hydro

\subsection{Electricity Transport}

clusters down the electricity transmission substations in each European country
based on the k-means algorithm

ENTSO-E

TYNDP

\section{Transport Sector}

\subsection{Land Transport}

Land transport is separated by energy carrier (fossil, hydrogen fuel cell
electric vehicle, and electric vehicle), but still needs to be separated into
heavy and light vehicles (the data is there, just not the code yet).

\subsection{Aviation}

kerosene

\subsection{Shipping}

hydrogen liquefaction costs for hydrogen demand in shipping

\section{Industry Sector}

Demand

Based on materials demand from JRC-IDEES and other sources such as the USGS for ammonia.

Industry is split into many sectors, including iron and steel, ammonia, other basic chemicals, cement, non-metalic minerals, alumuninium, other non-ferrous metals, pulp, paper and printing, food, beverages and tobacco, and other more minor sectors.

Inside each country the industrial demand is distributed using the Hotmaps Industrial Database.

Supply

Process switching (e.g. from blast furnaces to direct reduction and electric arc furnaces for steel) is defined exogenously.

Fuel switching for process heat is mostly also done exogenously.

Solid biomass is used for up to 500 Celsius, mostly in paper and pulp and food and beverages.

Higher temperatures are met with methane.

\subsection{Iron and Steel}

\subsection{Chemicals Industry}

basic chemicals: HVC (high-value chemicals), chlorine, methanol and ammonia

specify reuse, primary production, and mechanical and chemical recycling fraction of platics

Ammonia production data is taken from the USGS

\subsection{Non-metallic Mineral Products}

Cement

Ceramics

Glass

\subsection{Non-ferrous Metals}

Aluminium

\subsection{Other Industry Subsectors}

energy demands and CO2 emissions for the agriculture, forestry and fishing sector

\section{Heating Sector}

\subsection{Heating Demand}

Heat demand is split into:

urban central: large-scale district heating networks in urban areas with dense
heat demand

residential/services urban decentral: heating for individual buildings in urban
areas

residential/services rural: heating for individual buildings in rural areas,
agriculture heat uses

Building heating demand: nodal, distributed in each country based on population.


\subsection{Heating Supply}

Oil and gas boilers

Heat pumps

Either air-to-water or ground-to-water heat pumps are implemented.

They have coefficient of performance (COP) based on either the external air or the soil hourly temperature.

Ground-source heat pumps are only allowed in rural areas because of space constraints.

Only air-source heat pumps are allowed in urban areas. This is a conservative
assumption, since there are many possible sources of low-temperature heat that
could be tapped in cities (waste water, rivers, lakes, seas, etc.).

Resistive heaters

Large Combined Heat and Power (CHP) plants

https://doi.org/10.1016/j.energy.2018.10.044

PyPSA-Eur-Sec includes CHP plants fuelled by methane, hydrogen and solid biomass from waste and residues.

Hydrogen CHPs are fuel cells.

Methane and biomass CHPs are based on back pressure plants operating with a
fixed ratio of electricity to heat output. The methane CHP is modelled on the
Danish Energy Agency (DEA) “Gas turbine simple cycle (large)” while the solid
biomass CHP is based on the DEA’s “09b Wood Pellets Medium”.

The efficiencies of each are given on the back pressure line, where the back
pressure coefficient $c_b$ is the electricity output divided by the heat output.
The plants are not allowed to deviate from the back pressure line and are
implement as Link objects with a fixed ratio of heat to electricity output.

Micro-CHP for individual buildings

Waste heat from Fuel Cells, Methanation and Fischer-Tropsch plants

Solar thermal collectors

District heating!

\subsection{Heat Storage}

Thermal energy storage using hot water tanks
Small for decentral applications.
Big water pit storage for district heating.

\section{Wind}

\subsection{Wind Potentials}

\subsection{Wind Time Series}

\section{Solar}

\subsection{Solar Potentials}

utility PV

Installable potentials for rooftop PV are included with an assumption of 1 kWp
per person.

Solar thermal

\subsection{Solar Time series}

\citeS{zappa2019}

\section{Hydrogen}

\subsection{Hydrogen Demand}

Stationary fuel cell CHP.

Transport applications.

Industry (ammonia, precursor to hydrocarbons for chemicals and iron/steel).

\subsection{Hydrogen Supply}

Steam Methane Reforming (SMR), SMR+CCS, electrolysers.

\subsection{Hydrogen Transport}

retrofitting

new pipelines

\subsection{Hydrogen Storage}

cavern storage

steel tanks

\section{Methane}

\subsection{Methane Demand}

Can be used in boilers, in CHPs, in industry for high temperature heat, in OCGT.

Not used in transport because of engine slippage.

\subsection{Methane Supply}

Fossil, biogas, Sabatier (hydrogen to methane), HELMETH (directly power to
methane with efficient heat integration).

\subsection{Methane Transport}

Scigrid Gas dataset

single node for Europe, since future demand is so low and no bottlenecks are expected.

\section{Oil-based Products}

\subsection{Oil-based Product Demand}

Transport fuels, agriculture machinery and naphtha as a feedstock for the
chemicals industry.

\subsection{Oil-based Product Supply}

Fossil or Fischer-Tropsch.

\subsection{Oil-based Product Transport}

Liquid hydrocarbons: single node for Europe, since transport costs for liquids are low.


\section{Biomass}

\subsection{Biomass Potentials}

Only wastes and residues from the JRC ENSPRESO biomass dataset.

nodal where biomass potential is regionally disaggregated 

Use JRC ENSPRESO database to spatially disaggregate biomass potentials to
PyPSA-Eur regions based on overlaps with NUTS2 regions from ENSPRESO
(proportional to area)

\subsection{Biomass Demand}

Solid biomass provides process heat up to 500 Celsius in industry, as well as
feeding CHP plants in district heating networks.

solid biomass is used in the paper and pulp as well as food, beverages and
tobacco industries, where required temperatures are lower (see
DOI:10.1002/er.3436 and DOI:10.1007/s12053-017-9571-y).

\subsection{Biomass Transport}

solid biomass has to be consumed locally

biogas can be upgraded and then transported via methane network


\section{Carbon dioxide capture, usage and sequestration (CCU/S)}

Carbon dioxide can be captured from industry process emissions, emissions
related to industry process heat, combined heat and power plants, and directly
from the air (DAC).

Carbon dioxide can be used as an input for methanation and Fischer-Tropsch
fuels, or it can be sequestered underground.

CO2: single node for Europe, but a transport and storage cost is added for sequestered CO2. Optionally: nodal, with CO2 transport via pipelines.

\section{Mathematical Model Formulation}

\section{Techno-Economic Assumptions}


\addcontentsline{toc}{section}{Supplementary References}
\renewcommand{\ttdefault}{\sfdefault}
%\bibliographyS{library}
\bibliographyS{/home/fneum/zotero}

\begin{small}
	\tableofcontents
\end{small}

\end{document}