\documentclass[11pt,preprint]{elsarticle}
% \documentclass[12pt,1p]{elsarticle}
% \documentclass[11pt]{article}

\makeatletter
\def\ps@pprintTitle{%
 \let\@oddhead\@empty
 \let\@evenhead\@empty
 \def\@oddfoot{\centerline{\thepage}}%
 \let\@evenfoot\@oddfoot}
\makeatother

\journal{Joule}

% A typical research article will be
% 4000 words of text with
% 3–5 figures and
% 30 references.

\widowpenalty10000
\clubpenalty10000
\hyphenpenalty100

\abstracttitle{Summary}

\bibliographystyle{elsarticle-num}
\biboptions{numbers,sort&compress,super}

\usepackage{libertine}
\usepackage{libertinust1math}
\renewcommand{\ttdefault}{\sfdefault}

\usepackage{geometry}
\geometry{top=30mm, bottom=35mm}

\usepackage{amsmath}
\usepackage{bbold}
\usepackage{graphicx}
\usepackage{eurosym}
\usepackage{mathtools}
\usepackage{url}
\usepackage{booktabs}
\usepackage{epstopdf}
\usepackage{xfrac}
\usepackage{tabularx}
\usepackage[version=4]{mhchem}
\usepackage{bm}
\usepackage[colorlinks]{hyperref}
\usepackage[nameinlink,sort&compress,capitalise,noabbrev]{cleveref}
\usepackage[leftcaption,raggedright]{sidecap}
\usepackage{subcaption}
\usepackage{blindtext}
\usepackage[parfill]{parskip}


\usepackage[prependcaption,textsize=scriptsize]{todonotes}
% \usepackage[prependcaption,textsize=scriptsize,disable]{todonotes}

\usepackage{siunitx}
\sisetup{range-units=single, per-mode=symbol}
\DeclareSIUnit\year{a}
\DeclareSIUnit\tco{t_{\ce{CO2}}}
\DeclareSIUnit\sieuro{\mbox{\euro}}
\DeclareSIUnit\twh{TWh}
\DeclareSIUnit\mwh{MWh}
\DeclareSIUnit\kwh{kWh}

\newcommand{\co}{\ce{CO2}~}

\usepackage{longtable}
\usepackage{multirow}
\usepackage{threeparttable}
\usepackage{pdflscape}

\usepackage[export]{adjustbox}

\usepackage[resetlabels,labeled]{multibib}
\newcites{S}{Supplementary References}
\bibliographystyleS{elsarticle-num}

\graphicspath{
	{../results/graphics-20221227/},
	{../workflows/pypsa-eur-sec/results/}
}

\newcommand{\onwrun}{20221227-onw}
\newcommand{\gasrun}{20221227-gas}
\newcommand{\decrun}{20221227-decentral}
\newcommand{\lvrun}{20221227-lv}
\newcommand{\costrun}{20221227-costs}
\newcommand{\imprun}{20221227-import}
\newcommand{\shprun}{20221227-shipping}
\newcommand{\hyrun}{20221227-main}
\newcommand{\gasrunscen}{20221227-gas/elec_s_181_lv1.0__Co2L0-3H-T-H-B-I-A-solar+p3-linemaxext10_2050}
\newcommand{\runelec}{20221227-main/elec_s_181_lvopt__Co2L0-3H-T-H-B-I-A-solar+p3-linemaxext10-noH2network_2050}
\newcommand{\runhy}{20221227-main/elec_s_181_lv1.0__Co2L0-3H-T-H-B-I-A-solar+p3-linemaxext10_2050}

% \usepackage[
% 	type={CC},
% 	modifier={by},
% 	version={4.0},
% ]{doclicense}

\newcommand{\abs}[1]{\left|#1\right|}
\newcommand{\norm}[1]{\left\lVert#1\right\rVert}
\newcommand{\set}[1]{\left\{#1\right\}}
\DeclareMathOperator*{\argmin}{\arg\!\min}
\def\cT{\mathcal{T}}
\newcommand{\R}{\mathbb{R}}
\newcommand{\B}{\mathbb{B}}
\newcommand{\N}{\mathbb{N}}
%\def\co{CO${}_2$}
\def\el{${}_{\textrm{el}}$}
\def\th{${}_{\textrm{th}}$}
\def\hy{${}_{\textrm{H}_2}$}
\def\deg{${}^\circ$}

\usepackage{xcolor}
\usepackage{framed}
\definecolor{shadecolor}{rgb}{.95,.95,.95}

\usepackage{tocloft}
\cftsetindents{section}{1em}{2.75em}

\usepackage{orcidlink}

\newcommand{\lvbenefitabs}{-5.884402591919224}
\newcommand{\gridbenefitabs}{72}
\newcommand{\gridbenefitrel}{9.9}
\newcommand{\minacbenefitabs}{46}
\newcommand{\maxacbenefitabs}{61}
\newcommand{\minhybenefitabs}{12}
\newcommand{\maxhybenefitabs}{26}
\newcommand{\minacbenefitrel}{6.3}
\newcommand{\maxacbenefitrel}{8.1}
\newcommand{\minhybenefitrel}{1.6}
\newcommand{\maxhybenefitrel}{3.4}
\newcommand{\minsystemcost}{733}
\newcommand{\maxsystemcost}{805}
\newcommand{\acvshycost}{4.6}
\newcommand{\minaccost}{15.1}
\newcommand{\maxaccost}{37.9}
\newcommand{\minhycost}{3.2}
\newcommand{\maxhycost}{4.6}
\newcommand{\benefithyofac}{42.6}
\newcommand{\additivebenefitabs}{87}
\newcommand{\additivebenefitrel}{20.8}
\newcommand{\maxsystemcostnohydro}{762}
\newcommand{\minsystemcostnohydro}{690}
\newcommand{\minoffwind}{206}
\newcommand{\maxoffwind}{245}
\newcommand{\minonwind}{1691}
\newcommand{\maxonwind}{1776}
\newcommand{\minsolar}{2666}
\newcommand{\maxsolar}{3598}
\newcommand{\meanrooftopshare}{16}
\newcommand{\meanutilityshare}{84}
\newcommand{\minoffshoreshare}{10}
\newcommand{\maxoffshoreshare}{12}
\newcommand{\minelectrolysis}{937}
\newcommand{\maxelectrolysis}{1250}
\newcommand{\mincfFT}{59}
\newcommand{\maxcfFT}{68}
\newcommand{\mincfelectrolysis}{35}
\newcommand{\maxcfelectrolysis}{41}
\newcommand{\utilisationAC}{36}
\newcommand{\utilisationHy}{78}
\newcommand{\hydrogenstorageacyhyy}{26}
\newcommand{\hydrogenstorageacyhyn}{22}
\newcommand{\hydrogenstorageacnhyy}{43}
\newcommand{\hydrogenstorageacnhyn}{21}
\newcommand{\thermalstoragemin}{47}
\newcommand{\thermalstoragemax}{120}
\newcommand{\hydrogenproduction}{2376}
\newcommand{\ptlhydrogenusage}{1903}
\newcommand{\ptlwasteheat}{192}
\newcommand{\hydrogenindustrydemand}{195}
\newcommand{\hydrogentransportdemand}{275}
\newcommand{\hydrogenlosses}{139}
\newcommand{\hydrogenfuelcell}{287}
\newcommand{\hydrogenmethanation}{152}
\newcommand{\fossilgas}{366}
\newcommand{\biogas}{336}
\newcommand{\bluehydrogen}{78}
\newcommand{\mindac}{58}
\newcommand{\maxdac}{82}
\newcommand{\maxvres}{8972.792512116577}
\newcommand{\acoftotalbenefit}{84}
\newcommand{\hyoftotalbenefit}{36}
\newcommand{\maxretroshare}{69.1}
\newcommand{\minretroshare}{63.5}
\newcommand{\maxtwkmelectricity}{799.8}
\newcommand{\mintwkmelectricity}{357.7}
\newcommand{\mintwkmhydrogen}{0.0}
\newcommand{\maxtwkmhydrogen}{307.20000000000005}
\newcommand{\twkmhigher}{7.0}
\newcommand{\ewhkmelectricity}{1.04}
\newcommand{\ewhkmhydrogen}{2.16}
\newcommand{\ewhkmdiff}{0.9}


\begin{document}

\begin{frontmatter}

	\title{The Potential Role of a Hydrogen Network in Europe}

	\author[tubaddress]{Fabian Neumann\,\orcidlink{0000-0001-8551-1480}\corref{correspondingauthor}}
	\ead{f.neumann@tu-berlin.de}
	\author[tubaddress]{Elisabeth Zeyen\,\orcidlink{0000-0002-7262-3296}}
	\author[aarhus,aarhus2]{Marta Victoria\,\orcidlink{0000-0003-1665-1281}}
	\author[tubaddress]{Tom Brown\,\orcidlink{0000-0001-5898-1911}}
	% \cortext[correspondingauthor]{}
	\address[tubaddress]{Department of Digital Transformation in Energy Systems, Institute of Energy Technology, Technische Universität Berlin, Fakultät III, Einsteinufer 25 (TA 8), 10587 Berlin, Germany}
	\address[aarhus]{Department of Mechanical and Production Engineering, Aarhus University, Inge Lehmanns Gade 10, 8000 Aarhus, Denmark}
	\address[aarhus2]{Novo Nordisk Foundation CO$_2$ Research Center, Aarhus University, Aarhus, Denmark}

	\begin{abstract}
		Electricity transmission expansion has suffered many delays in Europe in recent
decades, despite its significance for integrating renewable electricity into the
energy system. A hydrogen network which reuses the existing fossil gas network
could not only help to supply demand for low-emission fuels, but could also to
balance variations in wind and solar energy across the continent and thus avoid
power grid expansion. We pursue this idea by varying the allowed expansion of
electricity and hydrogen grids in net-zero \co scenarios for a sector-coupled
and self-sufficient European energy system with high shares of renewables. We
cover the electricity, buildings, transport, agriculture, and industry sectors
across 181 regions and model every third hour of a year. With this high
spatio-temporal resolution, the model can capture bottlenecks in transmission
networks, the variability of demand and renewable supply, as well as regional
opportunities for the retrofitting of legacy gas infrastructure and development
of geological hydrogen storage. Our results show consistent system cost
reductions with a pan-continental hydrogen network that connects regions with
low-cost and abundant renewable potentials to demand centers, synthetic fuel
production and cavern storage sites. Developing a hydrogen network reduces
system costs by up to \maxhybenefitabs~bn\euro/a (\maxhybenefitrel\%), with
highest benefits when electricity grid reinforcements cannot be realised.
Between 64\% and 69\% of this network could be built from repurposed natural gas
pipelines. However, we find that hydrogen networks can only partially substitute
for power grid expansion. While the expansion of both networks together can
achieve the largest cost savings of \gridbenefitrel\%, the expansion of neither
is truly essential as long as higher costs can be accepted and regulatory
changes are made to manage grid bottlenecks.


	\end{abstract}

	\begin{keyword}
		hydrogen network, sector-coupling, retrofitting, energy systems, transmission expansion, Europe, climate-neutral, renewables
	\end{keyword}

	% \begin{graphicalabstract}
	% \end{graphicalabstract}

\end{frontmatter}

\section*{Graphical Abstract}

% https://www.cell.com/pb/assets/raw/shared/figureguidelines/GA_guide.pdf

\includegraphics[width=\textwidth]{graphics/graphical-abstract-v2-vector.pdf}

\newpage
\par\noindent\rule{\textwidth}{0.4pt}

\section*{Highlights}


\begin{itemize}
	\item examines cost benefit of a European hydrogen network in net-zero emission scenarios
	\item H$_2$ network reduces system costs by up to 3.4\%, highest without power grid expansion
	\item between 64\% and 69\% of hydrogen network uses retrofitted gas network pipelines
	\item power grid expansion saves more than hydrogen network, but strongest savings together
\end{itemize}

\par\noindent\rule{\textwidth}{0.4pt}

\section*{Context \& Scale}


%1000 characters in 1-2 paragraphs

Many different combinations of infrastructure would allow Europe to become
climate-neutral by 2050. But not all solutions meet the same level of
acceptance. For example, power transmission reinforcements have suffered many
delays in recent decades, despite their importance for integrating renewable
electricity. A hydrogen network which can reuse natural gas pipelines could
offer a substitute for moving cheap but remote renewable energy across the
continent to where demand is.

We study such trade-offs between building new electricity transmission lines and
developing a new network of hydrogen pipelines in the European energy system
with all sectors represented and net-zero \co emissions. We find that a hydrogen
network is consistently beneficial infrastructure and that a large part could
repurpose unused gas pipelines. Energy transport as electrons and molecules
offer complementary strengths, achieving highest cost savings together.
But neither is truly essential for a low-cost system.


\par\noindent\rule{\textwidth}{0.4pt}

\section*{eTOC Blurb}

We examine the interplay between a pan-continental hydrogen network and
electricity grid expansion in Europe to support renewable energy integration. By
repurposing gas pipelines, a hydrogen network can reduce energy system costs by
up to 3.4\% and help balance variations in wind and solar energy. Although the
cost benefit of electricity grid expansion is higher, combining both
infrastructures maximizes savings at 9.9\%. Together, they offer cost-effective
options for achieving a European energy system with net-zero \co emissions.

\newpage
\section*{Introduction}
\label{sec:intro}
\addcontentsline{toc}{section}{\nameref{sec:intro}}

There are many different combinations of technologies that would allow
the European Union to reach net-zero greenhouse gas emissions by
2050. However, not all technologies meet the same level of acceptance
among the public. The last few decades have seen public resistance to
new and existing nuclear power plants, projects with carbon capture
and sequestration (CCS), onshore wind power plants, and overhead
transmission lines. The lack of public acceptance can both delay the
deployment of a technology and stop its deployment altogether. This
may make it harder to reach greenhouse gas reduction targets in time,
or cause rising costs through the substitution with other
technologies.

In this contribution we examine the trade-off between costs and public
acceptance for onshore wind and overhead transmission lines in highly
renewable scenarios for Europe with net-zero carbon dioxide emissions
in the whole energy system (electricity, buildings, transport and
industry). For this purpose we use a capacity expansion model of the
European energy system, PyPSA-Eur-Sec, which, in contrast to previous
studies \cite{Henning20141003,mathiesen2014smart,IEESWV,Connolly20161634,en10101468,BLANCO2018617,BROWN2018720,in-depth_2018,victoria2020},
combines a fully sector-coupled approach with a high-resolution grid
model (181 nodes) so that it can capture the grid bottlenecks that
constrain the integration of renewable energy.  We successively
constrain the allowed onshore wind and new overhead grid projects down
to zero to examine the effects on total system cost and the
composition of technologies in the system.

We find that the overall system costs are not significantly affected
by restrictions on onshore wind and transmission. The high level of
synthetic fuel production and exchange between the nodes, which is
necessary for industry, transport and backup electricity and heating
applications, provides sufficient flexibility to manage these
restrictions. Moderate onshore wind and grid expansion provide cost
savings, but they are small (5-10~\%) compared to total system
costs. The restriction of onshore wind capacity results in a
substitution with offshore wind and solar PV; the restriction of grid
expansion leads to more local production from solar and more hydrogen
production. The limited rise in system costs relies on a new network
of underground hydrogen storage and pipelines in Europe, which help to balance
generation from renewables in time and space.

\section*{Common features across scenarios of European carbon-neutrality}
\label{sec:es}
\addcontentsline{toc}{section}{\nameref{sec:es}}

Overall, the total system costs are dominated by investments in generation from wind
and solar, and conversion from power to heat (primarily heat pumps)
and to hydrogen and liquid hydrocarbons (for transport fuels and as a
feedstock for the chemicals industry).

\cref{fig:tsc:w-el-w-h2} shows the spatial distribution of the investments
for the least-cost solution with full electricity and hydrogen network expansion.

While solar capacity is spread relatively evenly around the continent with a
slight skew torwards Southern Europe, both onshore and offshore wind are
concentrated around the North Sea and British Isles. New transmission capacity
is concentrated in regions so that they help the integration of wind in these
regions, and the transport of wind energy to inland locations.

Electrolyzer capacities for power-to-hydrogen see a massive scale-up between
1057~GW and 1297~GW depending on the permitted energy transport infrastructure.
Their locations correlate strongly with wind capacities ($R^2=0.85$).

A new network transports hydrogen from these sites of production to the rest of
Europe where hydrogen is consumed by industry (for ammonia, organic chemicals
and steel production), for heavy-duty transport and in fuel cells for power and
heat backup.

Of the huge hydrogen production (2437~TWh/a), most of it (1425~TWh/a) goes to
Fischer-Tropsch fuels for organic chemicals and transport fuels, of which
356~TWh/a waste heat enter the district heating networks. A total of 775~TWh/a
is used in shipping (two thirds) and land (one third) transport. 226~TWh/a are
used in the industry sector and for methanation. 79~TWh/a of hydrogen are lost
in conversion.

If the electricity grid expansion is restricted but hydrogen can be transported,
more hydrogen is produced to be re-electrified in critical phases and locations
of system operation (100 TWh\el).

Methane production is limited to biogas (346~TWh/a) and some fossil gas
(371~TWh/a), the latter of whose emissions are offset by bioenergy with carbon
capture. Direct air capture with sequestration and synthetic methane production
were only observed if both hydrogen and electricity network could not be
expanded. Methane is used for process heat in some industry applications and as
a heating backup for power-to-heat units.

From our model we infer a required carbon price between
\SI{435}{\sieuro\per\tco} with and \SI{513}{\sieuro\per\tco} without network
expansion to achieve \co neutrality in Europe.

compared to electricity, heating and transport are strongly peaked
- heating is strongly seasonal but also with synoptic variations
- transport has strong daily periodicity
There are difficult periods in winter with
- low wind and solar (high prices)
- high space heating demand
- low air temperatures, which are bad for air-sourced heat pump performance
- less smart solution: backup gas boilers
- smart solution: building retrofiting, TES in district heating, CHP

average capacity factors of electrolysis 36\% to 40\%, flexible operation
leveraging periods with high wind speeds across Europe (see
\cref{fig:output-ts-1}), buffered in hydrogen cavern storage for stable
production of synthetic hydrocarbons, whereas Fischer-Tropsch runs 87\% to 95\%,
only interrupted in winter periods with low wind speeds and high heat demands

dominant flow directions of energy by carrier?
- supplemental figures %\cref{}
- hydrogen: Ireland and England to Netherlands, Belgium and Western Germany; Scotland via North Sea to all of Germany; France to Southwest Germany; North East Spain via France to Switzerland
- electricity: more mixed patterns - balancing rather net transport over long distances

hydrogen storage:
- storage capacities? 62-66 TWh with hydrogen network, 32-35 TWh without hydrogen network; similar injection patterns; \cite{caglayanImpactDifferent2019} has 130~TWh of cavern storage
- all storage is cavern storage; no storage in steel tanks (neither hydrogen nor electricity network: 1.3 TWh steel tank (4\%))
- placed Northern Ireland, Denmark, England

What drives the hydrogen network?
- industry demand for hydrogen in areas with less attractive renewable potentials
- electricity grid bottlenecks / alternative energy transport
- waste heat of synthetic fuel production can only be leveraged in district heating networks of urban areas
- move produced hydrogen to cavern storage locations, rather than storage in steel tanks
- because gas and oil can be moved freely in the model, the spatial distribution of their demands is not a siting factor for synfuel production

\section*{Hydrogen network benefit is robust, strongest without power grid expansion}
\label{sec:h2}
\addcontentsline{toc}{section}{\nameref{sec:h2}}

\cref{fig:sensitivity-h2}

% electricity grid restriction

If the electricity grid can be expanded, total costs decrease
slightly, despite the increasing costs of the grid.

The
total cost benefit of power grid expansion is around
\euro47~billion per year.

The grid enables
renewable resources with better capacity factors to be integrated from
further away, resulting in lower capacity needs for solar and
wind. The grid also allows renewable variations to be smoothed in space,
resulting in lower hydrogen demand for balancing power and heat.

The restriction of grid
expansion leads to more local production from solar and more hydrogen
production.

consequently, as grid is expanded, costs reduce from solar, PtX and H2 network, more offshore wind

\cref{sec:si:lv} presents additional intermediate results between a doubling of
power grid capacity and no grid expansion.

% hydrogen network benefit

The presence of the hydrogen network can reduce system costs by up to 5.7\%.

The cost of hydrogen network 5.6-7.9 billion per year

Net benefit is much higher: 31-46 billion per year (4.1-5.7\%)

hydrogen network is robustly beneficial infrastructure
benefit is strongest when there is no power grid expansion

conversely, the benefit of electrcity grid reinforcement is strongest
when there is no hydrogen network option

a new network
of underground hydrogen storage and pipelines in Europe helps to balance
generation from renewables in time and space.

% overall / compared

We find that the overall system costs are not overly affected
by restrictions on electricity or hydrogen transmission.

Moderate levels of hydrogen and/or electricity grid expansion provide cost
savings, but they are small compared to total system costs.

systems without grid expansion are feasible

The high level of synthetic fuel production and exchange between the nodes, which is
necessary for industry, transport and backup electricity and heating
applications, provides sufficient flexibility to manage these
restrictions.

% comparison of H2 to power grid

the total net benefit of combined hydrogen and electricity grid expansion (beyond today) is 93 bn\euro/a (12.2\%)
- half of this benefit can be achieved by exclusive hydrogen network expansion
- whereas two thirds of the benefit can be reaped by only electricity grid expansion
- both are important for costs, but power grid expansion brings more cost benefit
- hydrogen grid is not a perfect substitute for electricity transmission
- rather transmission via eletrons or molecules have complementary strengths for long-distance transport of energy
- cost reductions are largely additive (total benefit 93, 46+62=108 for each individually, 108/93=1.16)
- hydrogen network can partially substitute transmission expansion (up to 46/62=75\% of electricity system benefit), strongest together

Depending on the level of power grid expansion, between 342 and 422 TWkm of
hydrogen pipelines are built. The higher value is obtained when the hydrogen
network partially compensates for the lack of electricity grid reinforcement.

\begin{SCfigure}
    \centering
    \includegraphics[width=0.8\textwidth]{sensitivity-h2-new.pdf}
    \caption{Benefits of electricity and hydrogen network infrastructure.}
    \label{fig:sensitivity-h2}
\end{SCfigure}


\section*{Repurposing gas pipelines lowers costs and shapes routes}
\label{sec:repurposed}
\addcontentsline{toc}{section}{\nameref{sec:repurposed}}


\cref{fig:h2-network}


\cref{fig:network-stats:twkm} shows statistics on the total electricity and
hydrogen transmission capacity built as well as how much energy is moved through
the respective networks. For the hydrogen network a distinction between
retrofitted and new pipelines is made. For the electricity network a distinction
is made between existing and added capacity or how much energy is moved via HVAC
or HVDC power lines.

A hydrogen network can take over and exceed the electricity grid
in terms of the amount of energy transported over long distances, balancing
renewables both in space (with the network) and time (with underground storage).

% h2 network topology

Restricting hydrogen expansion has only a small effect on cost-optimal levels of
grid expansion. While with a hydrogen network, the power grid capacity is a
little more than doubled, without it the cost-optimal power grid capacities are
10\% higher.

The pipeline connections between regions may have optimised capacities as high as 50 GW.

H2 pipeline is particularly attractive when end-use is hydrogen-based product
- hydrogen: 106-226~\euro/MW/km
- electricity: 400~\euro/MW/km
- annualised cost basis: electricity 1.6 times more expensive than H2, if retrofitted even 3.4 times
- not attractive for electricity end-use: round-trip efficiency electricity-H2-electricity 34\% (a third), more generation capacity needed
- however, electricity demand and direct electrification dominate the demand for hydrogen-based products; can explain benefit of power grid expansion

% repurposing capacities

Repurposing a gas pipeline to transport hydrogen is assumed to cost around half
that of building a new hydrogen pipeline (117 versus 250 \euro/MW/km). This
estimate includes the cost for compressor substitution. In consequence, the
hydrogen network topology does not always follow the shortest routes.

With power grid expansion, 58\% of the hydrogen network uses repurposed gas
pipelines. When the electricity grid cannot be reinforced, this share rises to
66\%.

Up to a third of the existing gas network (TWkm) are retrofitted to transport
hydrogen instead. This still leaves large  gas network capacities that are
neither used for hydrogen nor methane transport, particularly in Germany,
Poland, Italy and the North Sea as supplementary runs with full gas network
resolution demonstrate (\cref{fig:si:gas-leftover})

A little more than 40\% of retrofittable pipelines fully use their conversion
potential to hydrogen.

The largest new hydrogen pipelines are built on the British Isles, between
Denmark and Germany, inside Belgium and the Netherlands, and in the North-East
of Spain.

The most notable corridors for gas pipeline retrofitting, are located offshore
within the North Sea, in South-East England and crossing the English Channel, as
well as inside Germany, Austria and Northern Italy.

The large existing gas transmission capacities in Southern Italy and Eastern
Europe are not repurposed for hydrogen transport. However, this result would
likely change if attractive energy import options through these corridors were
considered.

% repurposing energy moved

\cref{fig:network-stats:ewhkm}
When both hydrogen and electricity grid expansion are allowed, both networks
transport approximately the same amount of energy. As the nominal capacity of
the hydrogen network is less than half that of the optimised electricity grid,
this means that the utilisation rate of the hydrogen network is higher (59\%
versus 35\%).

The hydrogen network plays a dominant role transporting energy around Europe
when grid expansion is restricted: around three times more energy is moved in
the hydrogen network (2.84~TWhkm/h) than in the electricity network
(0.95~EWhkm). At the same time, the total amount of energy moved as hydrogen or
electricity is reduced by only 22\%.

% import exports

\cref{fig:io}

overall pattern: energy surplus in coastal and most Southern regions supplying
to the inland regions of Europe with high demands but less attractive renewable potentials.

energy surplus (up to 200 TWh net surplus)
- wind-rich regions of Europe
- offshore wind in Denmark (in particular with eletricity grid reinforcement)
- onshore wind in Ireland (if grid expansion is restricted, production site for feedstock which can still be transported)
- but also other individual regions in Spain, Greece, France and Germany

energy deficit (up to 150 TWh net deficit)
- urban areas around London and Paris
- industrial cluster between Rotterdam and Ruhr valley

hydrogen supply is more regionally imbalanced than electricityl supply

if hydrogen transport is restricted

\begin{figure}
    \centering
    % \makebox[\textwidth][c]{
    \begin{subfigure}[t]{0.49\textwidth}
        \centering
        \caption{transmission capacity built}
        \includegraphics[width=\textwidth]{twkm}
        \label{fig:network-stats:twkm}
    \end{subfigure}
    \begin{subfigure}[t]{0.49\textwidth}
        \centering
        \caption{energy moved}
        \includegraphics[width=\textwidth]{ewhkm}
        \label{fig:network-stats:ewhkm}
    \end{subfigure}
    % }
    \caption{Transmission capacity built and energy moved for various scenarios.
        For the hydrogen network a distinction between retrofitted and new pipelines is made.
        For the electricity network a distinction is made between existing and added capacity
        or how much energy is moved via HVAC or HVDC power lines.}
    \label{fig:network-stats}
\end{figure}

\begin{figure}
    \centering
    \makebox[\textwidth][c]{
        \begin{subfigure}[t]{0.6\textwidth}
            \centering
            \caption{With grid reinforcement, with hydrogen network}
            \includegraphics[width=\textwidth, trim=0cm 0cm 7cm 0cm, clip]{\hyrun/maps/elec_s_181_lvopt__Co2L0-3H-T-H-B-I-A-solar+p3-linemaxext10-costs-all_2030.pdf}
            \label{fig:tsc:w-el-w-h2}
        \end{subfigure}
        \begin{subfigure}[t]{0.6\textwidth}
            \centering
            \caption{With grid reinforcement, without hydrogen network}
            \includegraphics[width=\textwidth, trim=0cm 0cm 7cm 0cm, clip]{\hyrun/maps/elec_s_181_lvopt__Co2L0-3H-T-H-B-I-A-solar+p3-linemaxext10-noH2network-costs-all_2030.pdf}
            \label{fig:tsc:w-el-wo-h2}
        \end{subfigure}
    }
    \makebox[\textwidth][c]{
        \begin{subfigure}[t]{0.6\textwidth}
            \centering
            \caption{Without grid reinforcement, with hydrogen network}
            \includegraphics[width=\textwidth, trim=0cm 0cm 7cm 0cm, clip]{\hyrun/maps/elec_s_181_lv1.0__Co2L0-3H-T-H-B-I-A-solar+p3-linemaxext10-noH2network-costs-all_2030.pdf}
            \label{fig:tsc:wo-el-w-h2}
        \end{subfigure}
        \begin{subfigure}[t]{0.6\textwidth}
            \centering
            \caption{Without grid reinforcement, without hydrogen network}
            \includegraphics[width=\textwidth, trim=0cm 0cm 7cm 0cm, clip]{\hyrun/maps/elec_s_181_lv1.0__Co2L0-3H-T-H-B-I-A-solar+p3-linemaxext10-noH2network-costs-all_2030.pdf}
            \label{fig:tsc:wo-el-wo-h2}
        \end{subfigure}
    } \caption{}
    \label{fig:tsc}
\end{figure}

\begin{figure}
    \centering
    \makebox[\textwidth][c]{
        \begin{subfigure}[t]{0.6\textwidth}
            \centering
            \caption{with grid reinforcement}
            \includegraphics[width=\textwidth]{\hyrun/maps/elec_s_181_lvopt__Co2L0-3H-T-H-B-I-A-solar+p3-linemaxext10-h2_network_2030.pdf}
            \label{fig:h2-network:w-el}
        \end{subfigure}
        \begin{subfigure}[t]{0.6\textwidth}
            \centering
            \caption{without grid reinforcement}
            \includegraphics[width=\textwidth]{\hyrun/maps/elec_s_181_lv1.0__Co2L0-3H-T-H-B-I-A-solar+p3-linemaxext10-h2_network_2030.pdf}
            \label{fig:h2-network:wo-el}
        \end{subfigure}
    }
    \caption{Optimised hydrogen network and production sites with and without electricity grid reinforcement.}
    \label{fig:h2-network}
\end{figure}



\begin{figure}
    \centering
    \makebox[\textwidth][c]{
        \begin{subfigure}[t]{0.6\textwidth}
            \centering
            \caption{With grid reinforcement, with hydrogen network}
            \includegraphics[width=\textwidth]{\hyrun/elec_s_181_lvopt__Co2L0-3H-T-H-B-I-A-solar+p3-linemaxext10_2030/import-export-total-200.pdf}
            \label{fig:io:w-el-w-h2}
        \end{subfigure}
        \begin{subfigure}[t]{0.6\textwidth}
            \centering
            \caption{With grid reinforcement, without hydrogen network}
            \includegraphics[width=\textwidth]{\hyrun/elec_s_181_lvopt__Co2L0-3H-T-H-B-I-A-solar+p3-linemaxext10-noH2network_2030/import-export-total-200.pdf}
            \label{fig:io:w-el-wo-h2}
        \end{subfigure}
    }
    \makebox[\textwidth][c]{
        \begin{subfigure}[t]{0.6\textwidth}
            \centering
            \caption{Without grid reinforcement, with hydrogen network}
            \includegraphics[width=\textwidth]{\hyrun/elec_s_181_lv1.0__Co2L0-3H-T-H-B-I-A-solar+p3-linemaxext10_2030/import-export-total-200.pdf}
            \label{fig:io:wo-el-w-h2}
        \end{subfigure}
        \begin{subfigure}[t]{0.6\textwidth}
            \centering
            \caption{Without grid reinforcement, without hydrogen network}
            \includegraphics[width=\textwidth]{\hyrun/elec_s_181_lv1.0__Co2L0-3H-T-H-B-I-A-solar+p3-linemaxext10-noH2network_2030/import-export-total-200.pdf}
            \label{fig:io:wo-el-wo-h2}
        \end{subfigure}
    }
    \caption{Regional total energy balances for scenarios with and without
        electricity or hydrogen network expansion. The Lorenz curves on the upper
        left of each map depict the regional inequity of electricity, hydrogen,
        methane and oil supply relative to demand.}
    \label{fig:io}
\end{figure}

\section*{Onshore wind expansion restrictions shift hydrogen infrastructure}
\label{sec:onwind}
\addcontentsline{toc}{section}{\nameref{sec:onwind}}

\begin{figure}
    \centering
    \makebox[\textwidth][c]{
        \begin{subfigure}[t]{0.48\textwidth}
            \centering
            \caption{system cost}
            \includegraphics[height=0.27\textheight]{\hyrun/maps/elec_s_181_lv1.0__Co2L0-3H-T-H-B-I-A-solar+p3-linemaxext10-onwind+p0-costs-all_2030.pdf}
            \label{fig:no-onw:tsc}
        \end{subfigure}
        \begin{subfigure}[t]{0.4\textwidth}
            \centering
            \caption{hydrogen network}
            \includegraphics[height=0.27\textheight]{\hyrun/maps/elec_s_181_lv1.0__Co2L0-3H-T-H-B-I-A-solar+p3-linemaxext10-onwind+p0-h2_network_2030.pdf}
            \label{fig:no-onw:h2}
        \end{subfigure}
        \begin{subfigure}[t]{0.43\textwidth}
            \centering
            \caption{energy balance}
            \includegraphics[height=0.27\textheight]{\hyrun/elec_s_181_lv1.0__Co2L0-3H-T-H-B-I-A-solar+p3-linemaxext10-onwind+p0_2030/import-export-total-200.pdf}
            \label{fig:no-onw:io}
        \end{subfigure}
    }
    \caption{Maps of regional energy balance, hydrogen network and production sites, and spatial and technological distribution of system costs for a scenario without onshore wind and without power grid expansion.}
    \label{fig:no-onw}
\end{figure}

% cost impact and overall system

By restricting the installable potentials of onshore down to zero, costs rise by
\euro~104 bn/a (12\%) as onshore wind is eliminated if the electricity grid is
fixed to today's capacities. \cref{sec:si:onw} presents intermediate results
between full and no onshore wind expansion. The model substitutes onshore wind,
particularly in the British Isles, for higher investment in offshore wind in the
North Sea and solar generators in Southern Europe (\cref{fig:no-onw:tsc}).
Because offshore capacities are concentrated near coastlines, and grid capacity
is restricted, total spending on hydrogen electrolyzers and networks also
increases to absorb the increased offshore generation. Without onshore wind, the
potentials for rooftop solar PV and offshore wind in Europe are largely
exhausted, such that in this self-sufficient scenario for Europe, the
effect of installable potentials becomes critical.

% changes in hydrogen infrastructure

Whereas with onshore wind, the  British Isles and North Sea dominate hydrogen
production, Southern Europe becomes a large exporter of solar-based hydrogen if
the development of onshore wind capacities is restricted
(\crefrange{fig:no-onw:h2}{fig:no-onw:io}). This shift in hydrogen
infrastructure also impacts the share of gas pipelines being retrofitted for
hydrogen transport. As the Iberian Peninsula becomes a preferred region for
hydrogen production but has a more sparse gas transmission network, the rate of
retrofitted pipeline capacity reduces from 66\% to 53\%. Many new hydrogen
pipelines are built to connect Spain with France, but also to connect increased
hydrogen production from Danish offshore wind to Germany. Gas pipeline
retrofitting is concentrated in Germany, Austria and Italy.



\section*{Discussion}
\label{sec:discussion}
\addcontentsline{toc}{section}{\nameref{sec:discussion}}

\subsection*{Comparison to Related Literature}

Compared to the net-zero scenarios from the European Commission
released in 2018 \cite{in-depth_2018}, we see much larger renewable
generation, reaching beyond 10000~TWh/a. This represents a
tripling of today's electricity generation (compared to at most a
145\% increase by 2050 in \cite{in-depth_2018}), with one third going
to regular electricity demand, one third going to newly-electrified
sectors in heating, transport and industry, and one third going to
hydrogen production (dominated by Fischer-Tropsch fuels). The major
difference is caused by lower electrification rates in other sectors
in \cite{in-depth_2018}, the higher biomass potentials
in \cite{in-depth_2018}, and the fact that \cite{in-depth_2018} relies
on imports of fossil oil for non-energy uses such as plastics (and
does not count them towards net emissions like we do). We are also
conservative on energy efficiency; building renovations and
circular economy concepts would reduce the demand in our model.

In \cite{brownSynergiesSector2018} an optimal grid expansion brought a benefit
of \euro~64~billion per year compared to the case with no transmission
between European countries, which is higher than the \euro~44~billion
per year benefit found here. There are at least four causes for this
difference: the model here has higher resolution (181 versus 30 nodes)
which allows better placement of wind at good sites; here we start
from today's grid, which has international transmission;
in \cite{brownSynergiesSector2018} there was no hydrogen pipeline network and no
underground hydrogen storage (just steel tanks); and finally we have
higher demand for hydrogen from industry and synthetic fuels, which
provides a large flexible load that helps to integrate wind and solar.

In \cite{schlachtbergerCostOptimal2018} only a small change
(8.8-12.2\%) in system costs was found in a model where onshore wind potentials
were restricted, with the biggest change when the electricity grid was
restricted. Onshore wind was largely replaced with offshore wind in
that model. Unlike that model, here we have a higher grid resolution
(181 versus 30 nodes) which allows us to better assess the grid
integration costs of offshore wind.

Caglayan and co-authors \cite{Caglayan2019} also consider European
decarbonisation scenarios with both electricity transmission and new
hydrogen pipelines, but at a lower spatial resolution (96 nodes). A
similar pattern of hydrogen pipeline expansion towards the British
Isles and North Sea is seen, but lower overall hydrogen capacities
because industry, shipping, aviation and non-electrified heating are
not included.

\subsection*{Broad ranges of options with similar costs}

The flatness of the total system costs as we vary grid expansion and
onshore wind potentials is a general feature of energy system models:
there are many directions in the feasible space where we can change
the system composition with only a small change in total system
costs. This flatness can be explored systematically using techniques similar to Modelling to
Generate Alternatives (MGA), and was investigated for an
electricity-only version of this model in \cite{Neumann2019}.

\subsection*{Reliance on hydrogen infrastructure}

One of the biggest changes seen in this model compared to today's
system is the huge built-out of hydrogen infastructure: huge new
electrolyzer capacities, underground storage in salt caverns as well
as a new hydrogen pipeline network. It is not clear that a new
hydrogen network will have any higher public acceptance than the power
grid. However, if existing natural gas pipelines can be reused for hydrogen,
or if at least the pipeline routes can be used, and are always discreetly buried underground, the disturbance to
the public is minimised.


\subsection*{Direct versus indirect costs}

Although our model includes a constraint on \co emissions, we have
not examined the indirect costs to environment, climate and health of
the energy system. The German Environmental Agency (UBA) calculated
the global environmental and health damages of greenhouse gas
emissions in Germany in 2016 to be \euro~164~billion \cite{UBA2019}
(these costs are dominated by climate-related damages amounting to
\SI{180}{\sieuro\per\tco}). This is comparable to the direct costs of the
German energy system, using our assumptions. Many of these indirect
costs would be avoided in the net-zero-emission systems presented
here.

\subsection*{Imports and Self-Sufficiency}

limitation no imports

in these scenarios, Europe is largely self-sufficient (fossil gas imports allowed)

very uneven infrastructure distribution

with imports
- strong point sources
- higher role of hydrogen network

future: increase self-sufficiency constraints of individual regions

\subsection*{Policy prerequisites}

A net-zero-emission energy system will require policy support.
Smaller bidding zones and dynamic pricing for flexible loads would
ease the management of grid congestion. From our model we see a need
for high, increasing and transparent price for \co pollution
(renewable support schemes alone are not sufficient to decarbonise
transport, heating and industry \cite{zhuImpactCO22018}). The model requires
a \co price of at least \SI{435}{\sieuro\per\tco} to achieve \co
neutrality. As discussed in \cite{brownSynergiesSector2018}, these high abatement
prices arise in building heating from the large price difference
between low-carbon heat and natural gas (\SI{22}{\sieuro\per\mwh} in the model);
if existing taxes and surcharges were included, the \co price would
be lower.



Add biomass supply and industry demand, aviation, shipping, agriculture, and non-energy
feedstocks for chemicals industry.

Examine different levels of: material efficiency, recycling, carbon
sequestration, biomass and import of low-carbon synthetic fuels.

Conclusions: efficiency is important, sequestration below \SI{200}{\mega\tco\per\year} is hard,
costs reduced with lots of biomass and sequestration, but this is challenging.

Without CCS/biomass/nuclear/efficiency/recycling/import need ~10,000 (update!) TWh/a of
wind+solar - very hard without acceptance problems!

\section*{Conclusion}
\label{sec:conclusion}
\addcontentsline{toc}{section}{\nameref{sec:conclusion}}

cross-sectoral approaches are important to reduce CO2 emissions cost-effectively and for flexibility

there are many trade-offs between unpopular infrastructure and system cost

through limiting power grid expansion, onshore wind potentials, hydrogen network infrastructure

e.g. limiting power grid expansion costs 40-50 billion per year more

If onshore wind expansion is restricted too, costs rise by further 120 billion per year

If all sectors included and Europe self-sufficient, effect of installable potentials is critical

but many near-optimal compromise energy systems with equally low cost but higher acceptance

hydrogen networks can partially substitute for power grid expansion, but system costs are
3-5\% higher

can also get away with neither power grid expansion nor hydrogen network

all results depend strongly on assumptions and modelling approach
- therefore openness and transparency are critical, guaranteed by open licenses for data and code

\section*{Experimental Procedures}
\label{sec:methods}
\addcontentsline{toc}{section}{\nameref{sec:methods}}

The model uses linear optimisation to minimise total annual operational and
investment costs subject to technical and physical constraints, assuming perfect
competition and perfect foresight over one year of 3-hourly operation. Apart
from existing electricity transmission and hydroelectric facilities, no other
existing assets are assumed (so-called `greenfield optimisation'), so that the
model represents an ideal steady state.  Cost assumptions are taken, where
possible, from predictions for the year 2030 by the Danish Energy Agency
\cite{dea2019}. The model is implemented in the free software framework Python
for Power System Analysis (PyPSA) \cite{brownPyPSAPython2018}.

In this section the main assumptions used in the model PyPSA-Eur-Sec are
presented. PyPSA-Eur-Sec builds upon the model from \cite{brown2018}, which
covered electricity, heating in buildings and ground transport in Europe with
one node per country. PyPSA-Eur-Sec adds biomass on the supply side, and
industry, aviation and shipping on the demand side. Unavoidable process
emissions, as well as the need for feedstocks for the chemicals industry and
dense hydrocarbon fuels for aviation, necessitate careful management of the
carbon cycle, including carbon capture from industry, biomass combustion and
directly from the air.

Figure gives an overview of the supply, transmission,
storage and demand sectors implemented in the European sector-coupled model
PyPSA-Eur-Sec. Generator capacities (for onshore wind, offshore wind, solar
photovoltaic (PV), biomass and natural gas), storage capacities (for batteries,
hydrogen, methane, liquid hydrocarbons, carbon dioxide and hot water tanks),
heating capacities (for heat pumps, resistive heaters, gas boilers, combined
heat and power (CHP) plants and solar thermal collector units), carbon capture
(from industry, CHP plants and directly from the air), energy converters
(electrolyzers, methanation, Fischer-Tropsch) and transmission capacities for
electricity and hydrogen are all subject to optimisation, as well as the
operational dispatch of each unit in each hour. Demand curves for the different
sectors, the ratio of district heating to decentralised heating, the number of
electric vehicles, methane storage and hydroelectricity capacities (for
reservoir and run-of-river generators and pumped hydro storage) are exogenous to
the model.

The European transmission network model is based on the open model PyPSA-Eur
presented in \cite{horschPyPSAEurOpen2018}. The state of the network in 2018 is plotted. The full European transmission network is clustered
down to 181 representative nodes based on the methodology used in
\cite{Hoersch2017}, thereby preserving the most important transmission corridors
that cause bottlenecks. The linearised optimal power flow uses a cycle-based
formulation from \cite{horschLinearOptimal2018} that significantly improves computational
performance; as transmission lines are expanded, impedances are updated
iteratively until convergence is achieved.

The sector-coupling model is based on the open model PyPSA-Eur-Sec from
\cite{brownSynergiesSector2018} which added to the electricity-only model of
\cite{schlachtbergerBenefitsCooperation2017} both land transport as well as space and water heating
in the residential and commercial sectors. In the model presented in this
contribution, biomass, industry demand (separately for sectors including iron
and steel, concrete, chemicals) and transport fuels for aviation and shipping
have also been included.

For biomass, only waste and residues from agriculture and forestry are
permitted, using the most conservative potential estimates from the JRC-EU-TIMES
model \cite{jrcbiomass2015}. This results in 352~TWh per year of biogas and
1261~TWh per year of solid biomass residues and waste for the whole of Europe.

For industry, we change some industrial processes to low-emission ones (e.g.
switching to hydrogen for direct reduction of iron ore \cite{voglAssessmentHydrogen2018}), allow
more recycling of steel and aluminium \cite{circular_economy}, switch fuel
sources for process heat, use synthetic fuels for ammonia and organic chemicals,
and allow carbon capture. It is assumed that no plastic or other non-energy
product is sequestered in landfill, but that all carbon in plastics eventually
makes its way back to the atmosphere, either through combustion or decay; this
approach is stricter than other models \cite{in-depth_2018}.

Transport and mobility comprises light and heavy road, rail, shipping and
aviation transport. For road and rail, electrification and fuel cell vehicles are
available. For shipping, liquid hydrogen is considered. For aviation, we
consider dense liquid hydrocarbons. Battery electric vehicles for passenger
transport can be enables with demand response as well as vehicle-to-grid
capabilities.

% Since the model is still being refined, the results in this contribution are
% only preliminary. Planned refinements include more precise placement of biomass
% resources and industry demand within each country, the option to convert the
% existing natural gas transmission network to carry 100\% hydrogen (currently the
% hydrogen network is built based on the optimal topology between the nodes
% regardless of the existing network) and more scenarios for circular economy
% concepts, different biomass potentials, different levels of carbon sequestration
% and non-zero import volumes for synthetic fuels.

\section*{Supplemental Information}

Supplemental information is included in \nameref{sec:si}.

\section*{Acknowledgements}

We are grateful for helpful comments by Johannes Hampp. We thank the
reviewers for their valuable feedback and suggestions.

\section*{Declaration of Interests}

The authors declare no competing interests.

% \section*{License}

% \href{http://creativecommons.org/licenses/by/4.0/}{Creative Commons Attribution 4.0
% International License (CC-BY-4.0)}

\section*{Author Contributions}

% following https://casrai.org/credit/

\textbf{F.N.}:
Conceptualization --
Data curation --
Formal Analysis --
Investigation --
Methodology --
Software --
Supervision --
Validation --
Visualization --
Writing - original draft --
Writing - review \& editing
\textbf{E.Z.}:
Data curation --
Formal Analysis --
Investigation --
Software --
Validation --
Writing - review \& editing
\textbf{M.V.}:
Formal Analysis --
Investigation --
Methodology --
Software --
Writing - review \& editing
\textbf{T.B.}:
Conceptualization --
Data curation --
Formal Analysis --
Funding acquisition --
Investigation --
Methodology --
Project administration --
Resources --
Software --
Supervision --
Writing - original draft --
Writing - review \& editing

% tidy with https://flamingtempura.github.io/bibtex-tidy/
\addcontentsline{toc}{section}{References}
\renewcommand{\ttdefault}{\sfdefault}
%\bibliography{library}
\bibliography{/home/fneum/zotero}

% supplementary information

\newpage

\makeatletter
\renewcommand \thesection{S\@arabic\c@section}
\renewcommand\thetable{S\@arabic\c@table}
\renewcommand \thefigure{S\@arabic\c@figure}
\makeatother

\renewcommand{\citenumfont}[1]{S#1}

\setcounter{equation}{0}
\setcounter{figure}{0}
\setcounter{table}{0}
\setcounter{section}{0}

\section{Model Overview}

Not all of the sectors are at the full nodal resolution, and some demand for
some sectors is distributed to nodes using heuristics that need to be corrected.
Some networks are copper-plated to reduce computational times.

\section{Electricity Sector}

Electricity supply and demand follows the electricity generation and
transmission model PyPSA-Eur, except that hydrogen storage is integrated into
the hydrogen supply, demand and network, and PyPSA-Eur-Sec includes CHPs.


\subsection{Electricity Demand}

distribution electricity demand for
industry uses the geographical data from
the Hotmaps Industrial Database.

subtracts existing electrified heating from
the existing electricity demand, so that power-to-heat can be optimised
separately.
Building heating demand: nodal, distributed in each country based on population.

The remaining electricity demand for households and services is distributed
inside each country proportional to GDP and population.

\subsection{Electricity Supply}

OCGT CCGT

hydro

run of river

\subsection{Electricity Storage}

battery

Distinguish costs for home battery storage and inverter from utility-scale battery costs.

pumped-hydro

\subsection{Electricity Transport}

clusters down the electricity transmission substations in each European country
based on the k-means algorithm

ENTSO-E

TYNDP

\section{Transport Sector}

\subsection{Land Transport}

Land transport is separated by energy carrier (fossil, hydrogen fuel cell
electric vehicle, and electric vehicle), but still needs to be separated into
heavy and light vehicles (the data is there, just not the code yet).

\subsection{Aviation}

kerosene

\subsection{Shipping}

hydrogen liquefaction costs for hydrogen demand in shipping

\section{Industry Sector}

Demand

Based on materials demand from JRC-IDEES and other sources such as the USGS for ammonia.

Industry is split into many sectors, including iron and steel, ammonia, other basic chemicals, cement, non-metalic minerals, alumuninium, other non-ferrous metals, pulp, paper and printing, food, beverages and tobacco, and other more minor sectors.

Inside each country the industrial demand is distributed using the Hotmaps Industrial Database.

Supply

Process switching (e.g. from blast furnaces to direct reduction and electric arc furnaces for steel) is defined exogenously.

Fuel switching for process heat is mostly also done exogenously.

Solid biomass is used for up to 500 Celsius, mostly in paper and pulp and food and beverages.

Higher temperatures are met with methane.

\subsection{Iron and Steel}

\subsection{Chemicals Industry}

basic chemicals: HVC (high-value chemicals), chlorine, methanol and ammonia

specify reuse, primary production, and mechanical and chemical recycling fraction of platics

Ammonia production data is taken from the USGS

\subsection{Non-metallic Mineral Products}

Cement

Ceramics

Glass

\subsection{Non-ferrous Metals}

Aluminium

\subsection{Other Industry Subsectors}

energy demands and CO2 emissions for the agriculture, forestry and fishing sector

\section{Heating Sector}

\subsection{Heating Demand}

Heat demand is split into:

urban central: large-scale district heating networks in urban areas with dense
heat demand

residential/services urban decentral: heating for individual buildings in urban
areas

residential/services rural: heating for individual buildings in rural areas,
agriculture heat uses

Building heating demand: nodal, distributed in each country based on population.


\subsection{Heating Supply}

Oil and gas boilers

Heat pumps

Either air-to-water or ground-to-water heat pumps are implemented.

They have coefficient of performance (COP) based on either the external air or the soil hourly temperature.

Ground-source heat pumps are only allowed in rural areas because of space constraints.

Only air-source heat pumps are allowed in urban areas. This is a conservative
assumption, since there are many possible sources of low-temperature heat that
could be tapped in cities (waste water, rivers, lakes, seas, etc.).

Resistive heaters

Large Combined Heat and Power (CHP) plants

https://doi.org/10.1016/j.energy.2018.10.044

PyPSA-Eur-Sec includes CHP plants fuelled by methane, hydrogen and solid biomass from waste and residues.

Hydrogen CHPs are fuel cells.

Methane and biomass CHPs are based on back pressure plants operating with a
fixed ratio of electricity to heat output. The methane CHP is modelled on the
Danish Energy Agency (DEA) “Gas turbine simple cycle (large)” while the solid
biomass CHP is based on the DEA’s “09b Wood Pellets Medium”.

The efficiencies of each are given on the back pressure line, where the back
pressure coefficient $c_b$ is the electricity output divided by the heat output.
The plants are not allowed to deviate from the back pressure line and are
implement as Link objects with a fixed ratio of heat to electricity output.

Micro-CHP for individual buildings

Waste heat from Fuel Cells, Methanation and Fischer-Tropsch plants

Solar thermal collectors

District heating!

\subsection{Heat Storage}

Thermal energy storage using hot water tanks
Small for decentral applications.
Big water pit storage for district heating.

\section{Wind}

\subsection{Wind Potentials}

\subsection{Wind Time Series}

\section{Solar}

\subsection{Solar Potentials}

utility PV

Installable potentials for rooftop PV are included with an assumption of 1 kWp
per person.

Solar thermal

\subsection{Solar Time series}

\citeS{zappa2019}

\section{Hydrogen}

\subsection{Hydrogen Demand}

Stationary fuel cell CHP.

Transport applications.

Industry (ammonia, precursor to hydrocarbons for chemicals and iron/steel).

\subsection{Hydrogen Supply}

Steam Methane Reforming (SMR), SMR+CCS, electrolysers.

\subsection{Hydrogen Transport}

retrofitting

new pipelines

\subsection{Hydrogen Storage}

cavern storage

steel tanks

\section{Methane}

\subsection{Methane Demand}

Can be used in boilers, in CHPs, in industry for high temperature heat, in OCGT.

Not used in transport because of engine slippage.

\subsection{Methane Supply}

Fossil, biogas, Sabatier (hydrogen to methane), HELMETH (directly power to
methane with efficient heat integration).

\subsection{Methane Transport}

Scigrid Gas dataset

single node for Europe, since future demand is so low and no bottlenecks are expected.

\section{Oil-based Products}

\subsection{Oil-based Product Demand}

Transport fuels, agriculture machinery and naphtha as a feedstock for the
chemicals industry.

\subsection{Oil-based Product Supply}

Fossil or Fischer-Tropsch.

\subsection{Oil-based Product Transport}

Liquid hydrocarbons: single node for Europe, since transport costs for liquids are low.


\section{Biomass}

\subsection{Biomass Potentials}

Only wastes and residues from the JRC ENSPRESO biomass dataset.

nodal where biomass potential is regionally disaggregated 

Use JRC ENSPRESO database to spatially disaggregate biomass potentials to
PyPSA-Eur regions based on overlaps with NUTS2 regions from ENSPRESO
(proportional to area)

\subsection{Biomass Demand}

Solid biomass provides process heat up to 500 Celsius in industry, as well as
feeding CHP plants in district heating networks.

solid biomass is used in the paper and pulp as well as food, beverages and
tobacco industries, where required temperatures are lower (see
DOI:10.1002/er.3436 and DOI:10.1007/s12053-017-9571-y).

\subsection{Biomass Transport}

solid biomass has to be consumed locally

biogas can be upgraded and then transported via methane network


\section{Carbon dioxide capture, usage and sequestration (CCU/S)}

Carbon dioxide can be captured from industry process emissions, emissions
related to industry process heat, combined heat and power plants, and directly
from the air (DAC).

Carbon dioxide can be used as an input for methanation and Fischer-Tropsch
fuels, or it can be sequestered underground.

CO2: single node for Europe, but a transport and storage cost is added for sequestered CO2. Optionally: nodal, with CO2 transport via pipelines.

\section{Mathematical Model Formulation}

\section{Techno-Economic Assumptions}


\addcontentsline{toc}{section}{Supplementary References}
\renewcommand{\ttdefault}{\sfdefault}
%\bibliographyS{library}
\bibliographyS{/home/fneum/zotero}

\newpage
\begin{small}
	\tableofcontents
\end{small}

\end{document}