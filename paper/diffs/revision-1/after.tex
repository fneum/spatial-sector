\documentclass{article}
\usepackage{libertine}
\usepackage{libertinust1math}
\usepackage[margin=2cm]{geometry}
\usepackage[parfill]{parskip}
\def\co{CO${}_2$}

\begin{document}

\section*{Abstract}

Electricity transmission expansion has suffered many delays in Europe in recent
decades, despite its importance for integrating renewable electricity into the
energy system. A hydrogen network which reuses the existing fossil gas network
would not only help supply demand for low-emission fuels, but could also help to
balance variations in wind and solar energy across the continent and thus avoid
power grid expansion. We pursue this idea by varying the allowed expansion of
electricity and hydrogen grids in net-zero \co scenarios for a sector-coupled
European energy system with high shares of renewables and self-sufficient
supply. We cover the electricity, buildings, transport, agriculture, and
industry sectors across 181 regions and model every third hour of a year. With
this high spatio-temporal resolution, the model can capture bottlenecks in
transmission networks, the variability of demand and renewable supply, as well
as regional opportunities for the reuse of legacy gas infastructure and
geological hydrogen storage. Our results show a consistent benefit of a
pan-continental hydrogen backbone that connects high-yield regions with demand
centers, synthetic fuel production and cavern storage sites. Developing a
hydrogen network reduces system costs by up to 6\%, with highest benefits when
electricity grid reinforcements cannot be realised. Between 58\% and 66\% of
this backbone could be built from repurposed natural gas pipelines. However, we
find that hydrogen networks can only partially substitute for power grid
expansion, and that both can achieve strongest cost savings of 12\% together. 

\section*{Introduction}

\dots pipelines already exist, the perceivable impact would be minimal.

However, few studies have evaluated the benefit of a hydrogen network in Europe
so far. The industry-oriented EHB reports do not include an assessment based on
the co-optimisation of energy system components. Other sector-coupling studies
have not included hydrogen networks at all, or when they do, model Europe only
at country-level resolution, have a country-specific focus with limited
geographical scope or detail outside the focus area, or neglect some energy
sectors or non-energy demands that involve hydrogen. However, high spatial
resolution at continental scope is critical to assess electricity grid
bottlenecks that constrain the integration of Europe's best renewable sites and
how they can be relieved by a hydrogen network. Great regional detail is also
essential to evaluate where there are potentials for lowering hydrogen network
development costs through gas pipeline retrofitting and where there are
geological hydrogen storage sites to support the development of a hydrogen
supply chain. Previous one-node-per-country studies could not have suitably
assessed this.

This paper provides the first high-resolution examination of the trade-offs
between electricity grid expansion and a new hydrogen network in scenarios for a
European energy system with net-zero carbon dioxide emissions and high shares of
renewable electricity production. By leveraging recent computational advances,
we elevate the model's spatial resolution to 181 regions to study what role
hydrogen infrastructure can play in a future sector-coupled system. For the
first time, such an investigation considers regionalised potentials for the
repurposing of legacy gas pipelines and the geological storage of hydrogen in
salt caverns. Our analysis covers four main scenarios to examine if a hydrogen
network can compensate for a potential lack of power grid expansion. These
scenarios differ based on whether or not electricity and hydrogen grids can be
expanded. Evaluation criteria include the total system cost, the composition and
spatial distribution of technologies and transmission infrastructure in the
system. As a supplementary sensitivity analysis, we also evaluate the impact of
restricted onshore wind potentials on these scenarios.

For our analysis, we use an open capacity expansion model\dots

\section*{Discussion}

\dots which provides a large flexible load that helps to integrate wind and
solar in time.

Victoria et al.~investigate the timing of
when technologies like carbon capture, electrolysis and a hydrogen network
become important for the European energy transition. The authors find a hydrogen
network consistently appearing after 2035 in all scenarios. However, owing to a
focus on transition pathways, the presented scenarios only resolve a single
region per country. Thereby, little can be said about network infrastructure
needs beyond cross-border transmission capacities, retrofitting opportunities
for legacy gas pipelines or the availability of geological hydrogen storage
potentials within countries. Compared to our findings, limited network expansion
options affect system costs much less. A doubling of today's transmission volume
reduces cost by 2\%, compared to 6.2\% in this study. Disabling hydrogen network
expansion increases system cost by 0.5\%, compared to 5.7\% in this study. This
substantial discrepancy arises because the one-node-per-country model version in
Victoria et al.~does not capture
country-internal transmission bottlenecks of either electricity or hydrogen
networks. As the system integration costs of remote resources like offshore wind
are thereby neglected within the countries, the benefit of network
infrastructure is degraded. By boosting the spatial resolution to subnational
scales in this study, more can be said about the competition between hydrogen
and electricity networks.

Caglayan et al. also consider European decarbonisation
scenarios\dots

\end{document}