\documentclass{article}
\usepackage{libertine}
\usepackage{libertinust1math}
\usepackage[margin=2cm]{geometry}
\usepackage[parfill]{parskip}
\def\co{CO${}_2$}

\begin{document}

\section*{Abstract}

Electricity transmission expansion has suffered many delays in Europe in recent
decades, despite its importance for integrating renewable electricity into the
energy system. A hydrogen network which reuses the existing fossil gas network
would not only help supply demand for low-emission fuels, but could also help to
balance variations in wind and solar energy across the continent and thus avoid
power grid expansion. We pursue this idea by varying the allowed expansion of
electricity and hydrogen grids in net-zero \co scenarios for a sector-coupled
European energy system with high shares of renewables and self-sufficient
supply. We cover the electricity, buildings, transport, agriculture, and
industry sectors across 181 regions and model every third hour of a year. With
this high spatio-temporal resolution, we can capture bottlenecks in transmission
and the variability of demand and renewable supply. Our results show a
consistent benefit of a pan-continental hydrogen backbone that connects
high-yield regions with demand centers, synthetic fuel production and geological
storage sites. Developing a hydrogen network reduces system costs by up to 6\%,
with highest benefits when electricity grid reinforcements cannot be realised.
Between 58\% and 66\% of this backbone could be built from repurposed natural gas
pipelines. However, we find that hydrogen networks can only partially substitute
for power grid expansion, and that both can achieve strongest cost savings of
12\% together.

\section*{Introduction}

\dots pipelines already exist, the perceivable impact would be minimal.

However, few studies have evaluated the benefit of a hydrogen network in Europe
so far. The EHB reports do not include an assessment based on the
co-optimisation of energy system components. Other sector-coupling studies have
not included hydrogen networks at all, or when they do, model Europe only at
country-level resolution, have a country-specific focus with limited
geographical scope or detail outside the focus area, or neglect some energy
sectors or non-energy demands that involve hydrogen. None of the studies have
explored the interplay between hydrogen network expansion and electricity grid
reinforcements. Neither have the potentials for lower development costs through
pipeline retrofitting been taken into account so far.

This paper provides the first high-resolution examination of the trade-offs
between electricity grid expansion and a new hydrogen network in scenarios for a
European energy system with net-zero carbon dioxide emissions and high shares of
renewable electricity production. We analyse four main scenarios to investigate
if a hydrogen network can compensate for a potential lack of power grid
expansion. The scenarios differ based on whether or not electricity and hydrogen
grids can be expanded, including potentials for gas pipeline retrofitting.
Evaluation criteria include the total system cost, the composition and spatial
distribution of technologies and transmission infrastructure in the system. As a
supplementary sensitivity analysis, we also evaluate the impact of restricted
onshore wind potentials on these scenarios.

For our analysis, we use an open capacity expansion model\dots

\section*{Discussion}

\dots which provides a large flexible load that helps to integrate wind and
solar in time.

Caglayan et al. also consider European decarbonisation
scenarios\dots

\end{document}