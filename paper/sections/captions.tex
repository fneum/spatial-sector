%% Figure 1

\begin{figure}[!ht]
    \caption{ \textbf{Energy, hydrogen and carbon dioxide balances across all
    scenarios.} Energy consumption includes final energy and non-energy demands
    by carrier as well as conversion losses in thermal storage and electrofuel
    synthesis processes (e.g.~power-to-hydrogen, power-to-liquid). The ambient
    heat retrieved by heat pumps is counted as energy supply. A breakdown of
    final energy and non-energy demands by sector is shown by sector in
    \cref{fig:demand-by-sector-carrier}, by time in \cref{fig:demand-time}, and
    by region in \cref{fig:demand-space}. For technologies with carbon capture
    (CC) option, the carbon dioxide balance shows residual emissions due to
    imperfect capture rates. For detailed Sankey diagrams of energy and carbon
    flows see \crefrange{fig:si:sankey}{fig:si:carbon-sankey-2}.}
    \label{fig:balance}
\end{figure}

%% Figure 2

\begin{figure}
    \caption{ \textbf{Cost reductions achieved by developing electricity and
    hydrogen network infrastructure.} (A) Comparison of four scenarios with and
    without expansion of a hydrogen network (left to right) and the electricity
    grid (top to bottom). Each bar depicts the total system cost of one scenario
    alongside its cost composition. Arrows between the bars indicate absolute
    and relative cost increases as network infrastructures are successively
    restricted. (B) System cost difference of grid expansion restrictions
    relative to the least-cost solution with full hydrogen and power grid
    expansion. For similar graphics in different settings, e.g.~more optimistic
    cost assumptions, imports of liquid hydrocarbons, liquid hydrogen as
    shipping fuel, intermediate levels of power grid expansion and onshore wind
    availability, see
    \crefrange{fig:lv-onw-restriction}{fig:sensitivity-shipping-diff}. }
    \label{fig:sensitivity-h2}
\end{figure}

%% Figure 3

\begin{figure}
    \caption{ \textbf{Regional distribution of system costs and electricity grid
    expansion for scenarios with and without electricity or hydrogen network
    expansion.} The pie charts in each of the network expansion scenarios
    (A)-(D) depict the annualised system cost alongside the shares of the
    various technologies for each region. The line widths depict the level of
    added grid capacity between two regions, which was capped at 10 GW. For the
    regional distribution of average electricity and hydrogen prices per
    scenario, see \cref{fig:si:lmp-ac} and \cref{fig:si:lmp-h2}. Corresponding
    regionally averaged price time series and price duration curves are shown in
    \crefrange{fig:si:lmp-ts-ac}{fig:si:lmp-dc}. Total installed capacities are
    presented in \cref{fig:si:capacities}.}
    \label{fig:tsc}
\end{figure}

%% Figure 4

\begin{figure}
    \caption{\textbf{Transmission capacity built and energy volume transported
        for various network expansion scenarios.} (A) shows transmission
        capacity built, whereas (B) shows energy volume transported. For the hydrogen
        network, a distinction between retrofitted and new pipelines is made.
        For the electricity network, a distinction is made between existing and
        added capacity or how much energy is moved via HVAC or HVDC power lines.
        Both measures weight capacity (TW) or energy (EWh) by the length (km) of
        the network connection.}
    \label{fig:network-stats}
\end{figure}

%% Figure 5


\begin{figure}
    \caption{ \textbf{Hydrogen network infrastructure and energy flows.}
    (A)-(B) Optimised hydrogen network, storage, reconversion and production
    sites with and without electricity grid reinforcement. The size of the
    circles depicts the electrolysis and fuel cell capacities in the respective
    region. The line widths depict the optimised hydrogen pipeline capacities.
    The darker shade depicts the share of capacity built from retrofitted gas
    pipelines. The coloring of the regions indicates installed hydrogen storage
    capacities.
    (C)-(D) Net flows of hydrogen in the network and the respective energy
    balance  with and without electricity grid reinforcement. Flows larger than
    2~TWh are shown with arrow sizes proportional to net flow volume.
    For a map outlining net power flows in the electricity transmission network,
    see \cref{fig:si:flow-ac}. }
    \label{fig:h2-network}
\end{figure}

%% Figure 6


\begin{figure}
    \caption{ \textbf{Total energy balances for scenarios with and without
    electricity or hydrogen network expansion for the 181 model regions.} For
    each scenario (A)-(D), the maps reveal regions with net energy surpluses and
    deficits. The Lorenz curves on the upper left of each map depict the
    regional imbalances of electricity, hydrogen, methane and liquid hydrocarbon
    supply relative to demand. Methane and liquid hydrocarbon supply can be of
    fossil, biogenic or synthetic origin. If the annual sums of supply and
    demand are equal in each region, the Lorenz curve resides on the identiy
    line. But the more imbalanced the regional supply is relative to demand, the
    further the curve dents into the bottom right corner of the graph.}
    \label{fig:io}
\end{figure}