\section{Mathematical Model Formulation}
\label{sec:si:math}

The objective is to minimise the total annual energy system costs of the energy system
that comprises both investment costs and operational expenditures of generation,
storage, transmission and conversion infrastructure. To express both as annual
costs, we use the annuity factor $(1-(1+\tau)^{-n}) / \tau$ that, like a
mortgage, converts the upfront investment of an asset to annual payments
considering its lifetime $n$ and cost of capital $\tau$. Thus, the objective
includes on one hand the annualised capital costs $c_*$ for investments at bus
$i$ in generator capacity $G_{i,r}\in\R^+$ of technology $r$, storage energy
capacity $E_{i,s}\in\R^+$ of technology $s$, electricity transmission line
capacities $P_{\ell}\in\R^+$, and energy conversion and transport capacities
$F_k\in\R^+$ (links), as well as the variable operating costs $o_*$ for
generator dispatch $g_{i,r,t}\in\R^+$ and link dispatch $f_{k,t}\in\R^+$ on the
other:
\begin{align}
  \label{eq:objective}
  \min_{G,E,P,F,g} \quad &\left[\sum_{i,r} c_{i,r}\cdot G_{i,r} + \sum_{i,s} c_{i,s}\cdot E_{i,s} + \sum_{\ell}c_{\ell}\cdot P_{\ell}+ \sum_{k}c_{k}\cdot F_k +\right. \\
  & \left.  \sum_{t} w_t \cdot \left( \sum_{i,r} o_{i,r} \cdot g_{i,r,t} + \sum_k o_k \cdot f_{k,t} \right) \right].
\end{align}
Thereby, the representative time snapshots $t$ are weighted by the time span
$w_t$ such that their total duration adds up to one year; \mbox{$\sum_{t\in \cT}
w_t=365\cdot 24\text{h}=8760\text{h}$}. A bus $i$ represents both a regional
scope and an energy carrier. Represented carriers include electricity, heat
(various subdivisions), hydrogen, methane, oil and carbon dioxide.

% constraints

In addition to the cost-minimising objective function, we further impose a set
of linear constraints that define limits on (i) the capacities of generation,
storage, conversion and transmission infrastructure from geographical and
technical potentials, (ii) the availability of variable renewable energy sources
for each location and point in time (iii) the limit for \co emissions or transmission expansion, (iv)
storage consistency equations, and (v) a multi-period linearised optimal power
flow (LOPF) formulation. Overall, this results in a large linear problem (LP).

The capacities of generation, storage, conversion and transmission
infrastructure are constrained from above by their installable potentials and
from below by any existing components:
\begin{align}
  \underline{G}_{i,r}  &  & \leq &  & G_{i,r}  &  & \leq &  & \overline{G}_{i,r}  & \qquad\forall i, r \label{eq:genlimit} \\
  \underline{E}_{i,s}  &  & \leq &  & E_{i,s}  &  & \leq &  & \overline{E}_{i,s}  & \qquad\forall i, s \\
  \underline{P}_{\ell} &  & \leq &  & P_{\ell} &  & \leq &  & \overline{P}_{\ell} & \qquad\forall \ell \\
  \underline{F}_{k} &  & \leq &  & F_{k} &  & \leq &  & \overline{F}_{k} & \qquad\forall k
\end{align}

Moreover, the dispatch of generators and links may not only be constrained by their rated capacity but also by the weather-dependent availability of
variable renewable energy or must-run conditions.
This can be expressed as a time- and location-dependent availability
factor $\overline{g}_{i,r,t}$/$\overline{f}_{k,t}$  and must-run factor $\underline{g}_{i,r,t}$/$\underline{f}_{k,t}$, given per unit of the nominal capacity:
\begin{align}
    \underline{g}_{i,r,t}  G_{i,r} &  & \leq &  & g_{i,r,t} &  & \leq &  & \overline{g}_{i,r,t} G_{i,r} & \qquad\forall i, r, t \\
    \underline{f}_{k,t}  F_{k} &  & \leq &  & f_{k,t} &  & \leq &  & \overline{f}_{k,t} F_{i,r} & \qquad\forall k, t
\end{align}
The parameter $\underline{f}_{k,t}$ can also be used to define whether a link is
bidirectional or unidirectional. For instance, for HVDC links
$\underline{f}_{k,t}=-1$ allows power flows in either direction. On the other
hand, a heat resistor has $\underline{f}_{k,t}=0$ since it can only convert
electricity to heat.

The energy levels $e_{i,s,t}$ of all stores are constrained by their energy capacity
\begin{align}
  0 &  & \leq &  & e_{i,s,t} &  & \leq &  & E_{i,s} & \qquad\forall i, s, t,
\end{align}
and have to be consistent with the dispatch variable $h_{i,s,t}\in\R$ in all
hours
\begin{align}
  e_{i,s,t} =\: & \eta_{i,s,0}^{w_t} \cdot e_{i,s,t-1} + w_t \cdot h_{i,s,t}, \label{eq:stoe}
\end{align}
where $\eta_{i,s,0}$ denotes the standing loss . Furthermore, the storage energy
levels are either assumed to be cyclic or given an initial state of charge,
\begin{align}
  e_{i,s,0} = e_{i,s,\abs{\mathcal{T}}} \qquad\forall i, s, or\\
  e_{i,s,0} = e_{i,s,\text{initial}} \qquad\forall i, s.
\end{align}

The modelling of hydroelectricity storage deviates from regular storage to
additionally account for natural inflow and spillage of water. We also assume
fixed power ratings $H_{i,s}$ for hydroelectricity storage. The dispatch of
hydroelectricity storage units is split into two positive variables; one each
for charging $h_{i,s,t}^+$ and discharging $h_{i,s,t}^-$, and limited by
$H_{i,s}$.
\begin{align}
  0 &  & \leq &  & h_{i,s,t}^+ &  & \leq &  & H_{i,s} & \qquad\forall i, s, t \label{eq:sto1} \\
  0 &  & \leq &  & h_{i,s,t}^- &  & \leq &  & H_{i,s} & \qquad\forall i, s, t \label{eq:sto2}
\end{align}
% This formulation does not prevent simultaneous charging and discharging, in
% order to maintain the computational benefit of a convex feasible space that can
% be counteracted by adding a small marginal cost to the storage dispatch
% variables.
The energy levels $e_{i,s,t}$ of all hydroelectric storage also have to match
the dispatch across all hours
\begin{align}
  e_{i,s,t} =\: & \eta_{i,s,0}^{w_t} \cdot e_{i,s,t-1} + w_t \cdot h_{i,s,t}^\text{inflow} - w_t \cdot h_{i,s,t}^\text{spillage} & \quad\forall i, s, t \nonumber \\
                & + \eta_{i,s,+} \cdot w_t \cdot h_{i,s,t}^+ - \eta_{i,s,-}^{-1} \cdot w_t \cdot h_{i,s,t}^-, \label{eq:stoe}
\end{align}
whereby hydropower storage units can additionally have a charging efficiency
$\eta_{i,s,+}$, a discharging efficiency $\eta_{i,s,-}$, natural inflow
$h_{i,s,t}^\text{inflow}$ and spillage $h_{i,s,t}^\text{spillage}$, besides the
standing loss $\eta_{i,s,0}$.

The nodal balance constraint for supply and demand (Kirchoff's current law for electricity
buses) requires local generators and storage units as well as incoming or
outgoing energy flows $f_{\ell,t}$ of incident transmission lines $\ell$ to
balance the perfectly inelastic electricity demand $d_{i,t}$ at each location
$i$ and snapshot $t$
\begin{align}
    \sum_r g_{i,r,t} + \sum_s \left(h_{i,s,t}^- - h_{i,s,t}^+ \right) + \sum_s h_{i,s,t} + \sum_\ell K_{i\ell} f_{\ell,t} + \sum_k L_{ikt} f_{k,t} = d_{i,t}  \quad \leftrightarrow \quad \lambda_{i,t} \quad \forall i,t,
\end{align}
where $K_{i\ell}$ is the incidence matrix of the electricity network with
non-zero values $-1$ if line $\ell$ starts at node $i$ and $1$ if it ends at
node $i$. $L_{ikt}$ is the lossy incidence matrix of the network with
non-zero values $-1$ if link $k$ starts at node $i$ and $\eta_{i,k,t}$ if one of
its terminal buses is node $i$. For a link with more than two outputs (e.g. CHP
converts gas to heat and electricity in a fixed ratio), the respective column of
the lossy incidence matrix has more than two non-zero entries (hypergraph). The
efficiency may be time-dependent and greater than one for certain technologies
(e.g. for heat pumps converting electricity and ambient heat to hot water).

The Lagrange multiplier (KKT multiplier) $\lambda_{i,t}$ associated with the
nodal balance constraint indicates the marginal price of the respective energy
carrier and location of bus $i$ at time $t$, e.g. the local marginal price (LMP)
of electricity at the electricity bus.

The power flows $p_{\ell,t}$ are limited by their nominal capacities $P_\ell$
\begin{align}
	|p_{\ell,t}| \leq \overline{p}_{\ell} P_{\ell} & \qquad\forall \ell, t,
	\label{eq:cap}
\end{align}
where $\overline{p}_\ell$ acts as an additional per-unit security margin on the line capacity
to allow a buffer for the failure of single circuits ($N-1$ criterion) and reactive power flows.

Kirchoff's voltage law (KVL) imposes further constraints on the flow of AC
transmission lines and there are several ways to formulate KVL with large
impacts on performance. Here, we use linearised load flow assumptions, where the
voltage angle difference around every closed cycle in the electricity
transmission network must add up to zero. Using a cycle basis $C_{\ell c}$ of
the network graph where the independent cycles $c$ are expressed as directed
linear combinations of lines $\ell$ \citeS{horschLinearOptimal2018}, we can write
KVL as
\begin{align}
    \sum_\ell C_{\ell c} \cdot x_\ell \cdot p_{\ell,t} = 0 \qquad\forall c,t
    \label{eq:kvl}
\end{align}
where $x_\ell$ is the series inductive reactance of line $\ell$.

We may further regard a constraint on the total annual CO$_2$ emissions $\Gamma_{\text{CO}_2}$
to achieve sustainability goals.
The emissions are determined from the time-weighted generator dispatch $ w_t \cdot g_{i,r,t}$ using the specific emissions $\rho_r$ of technology $r$
and the generator efficiencies $\eta_{i,r}$
\begin{align}
	\sum_{i,r,t}  \rho_r \cdot \eta_{i,r}^{-1} \cdot w_t \cdot g_{i,r,t} + \sum_{i,s} \rho_s \left(e_{i,s,t=0} - e_{i,s,t=\abs{\mathcal{T}}}\right) \leq \Gamma_{\text{CO}_2}  \quad \leftrightarrow \quad \mu_{\text{CO}_2}.
\end{align}
In this case, the Lagrange multiplier (KKT multiplier) $\mu_{\text{CO}_2}$ denotes the shadow price of emitting an additional tonne of \co, i.e. the \co price necessary to achieve the respective \co emission reduction target.

Additionally, another global constraint may be set on the volume of electricity transmission network expansion
\begin{align}
    \sum_\ell l_\ell \cdot P_\ell \leq \Gamma_{LV} \quad \leftrightarrow \quad \mu_{LV},
\end{align}
where the sum of transmission capacities $P_\ell$ multiplied by their lengths $l_\ell$ is bounded by a transmission volume cap
$\Gamma_{LV}$. In this case, the Lagrange multiplier (KKT multiplier) $\mu_{LV}$ denotes the shadow price of a marginal increase in transmission volume.

This formulation does not include pathway optimisation (i.e.~no
sequences of investments), but searches for a cost-optimal layout corresponding
to a given \co emission reduction level and assumes perfect foresight for
the reference year based on which capacities are optimised. This optimisation
problem is implemented in the open-source Python-based modelling framework
PyPSA \citeS{brownPyPSAPython2018}.
