overnight scenario, disregarding pathway

weather year 2013
- Detailed transmission grid representation in EU-wide model
- Detailed representation of demand sectors: Industry, buildings, transport
- so that the variability of demand and variable renewable supply can be represented, and so that existing grid bottlenecks are visible.
In this study, we used the historical year 2013 for weather-dependent inputs and applied a net-zero
emission constraint

The model uses linear optimisation to minimise total annual operational and
investment costs subject to technical and physical constraints, assuming perfect
competition and perfect foresight over one year of 3-hourly operation. Apart
from existing electricity transmission and hydroelectric facilities, no other
existing assets are assumed (so-called `greenfield optimisation'), so that the
model represents an ideal steady state.  Cost assumptions are taken, where
possible, from predictions for the year 2030 by the Danish Energy Agency
\cite{dea2019}. The model is implemented in the free software framework Python
for Power System Analysis (PyPSA) \cite{brownPyPSAPython2018}.

In this section the main assumptions used in the model PyPSA-Eur-Sec are
presented. PyPSA-Eur-Sec builds upon the model from \cite{brown2018}, which
covered electricity, heating in buildings and ground transport in Europe with
one node per country. PyPSA-Eur-Sec adds biomass on the supply side, and
industry, aviation and shipping on the demand side. Unavoidable process
emissions, as well as the need for feedstocks for the chemicals industry and
dense hydrocarbon fuels for aviation, necessitate careful management of the
carbon cycle, including carbon capture from industry, biomass combustion and
directly from the air.

Figure gives an overview of the supply, transmission,
storage and demand sectors implemented in the European sector-coupled model
PyPSA-Eur-Sec. Generator capacities (for onshore wind, offshore wind, solar
photovoltaic (PV), biomass and natural gas), storage capacities (for batteries,
hydrogen, methane, liquid hydrocarbons, carbon dioxide and hot water tanks),
heating capacities (for heat pumps, resistive heaters, gas boilers, combined
heat and power (CHP) plants and solar thermal collector units), carbon capture
(from industry, CHP plants and directly from the air), energy converters
(electrolyzers, methanation, Fischer-Tropsch) and transmission capacities for
electricity and hydrogen are all subject to optimisation, as well as the
operational dispatch of each unit in each hour. Demand curves for the different
sectors, the ratio of district heating to decentralised heating, the number of
electric vehicles, methane storage and hydroelectricity capacities (for
reservoir and run-of-river generators and pumped hydro storage) are exogenous to
the model.

The European transmission network model is based on the open model PyPSA-Eur
presented in \cite{horschPyPSAEurOpen2018}. The state of the network in 2018 is plotted. The full European transmission network is clustered
down to 181 representative nodes based on the methodology used in
\cite{Hoersch2017}, thereby preserving the most important transmission corridors
that cause bottlenecks. The linearised optimal power flow uses a cycle-based
formulation from \cite{horschLinearOptimal2018} that significantly improves computational
performance; as transmission lines are expanded, impedances are updated
iteratively until convergence is achieved.

The sector-coupling model is based on the open model PyPSA-Eur-Sec from
\cite{brownSynergiesSector2018} which added to the electricity-only model of
\cite{schlachtbergerBenefitsCooperation2017} both land transport as well as space and water heating
in the residential and commercial sectors. In the model presented in this
contribution, biomass, industry demand (separately for sectors including iron
and steel, concrete, chemicals) and transport fuels for aviation and shipping
have also been included.

For biomass, only waste and residues from agriculture and forestry are
permitted, using the most conservative potential estimates from the JRC-EU-TIMES
model \cite{jrcbiomass2015}. This results in 352~TWh per year of biogas and
1261~TWh per year of solid biomass residues and waste for the whole of Europe.

For industry, we change some industrial processes to low-emission ones (e.g.
switching to hydrogen for direct reduction of iron ore \cite{voglAssessmentHydrogen2018}), allow
more recycling of steel and aluminium \cite{circular_economy}, switch fuel
sources for process heat, use synthetic fuels for ammonia and organic chemicals,
and allow carbon capture. It is assumed that no plastic or other non-energy
product is sequestered in landfill, but that all carbon in plastics eventually
makes its way back to the atmosphere, either through combustion or decay; this
approach is stricter than other models \cite{in-depth_2018}.

Transport and mobility comprises light and heavy road, rail, shipping and
aviation transport. For road and rail, electrification and fuel cell vehicles are
available. For shipping, liquid hydrogen is considered. For aviation, we
consider dense liquid hydrocarbons. Battery electric vehicles for passenger
transport can be enables with demand response as well as vehicle-to-grid
capabilities.

Energy sector coupling, storage and conversion is modelled to connect
electricity, heating (individual buildings, district heating and industry),
transport and gas (methane, hydrogen and carbon dioxide) in the different
sectors (buildings, transport and industry).

Electricity can be converted to heat via heat pumps or resistive heaters; to
hydrogen gas, or further to methane and liquid hydrocarbons; or to work in
various demand devices. Methane can be reformed to hydrogen, and most fuels can
be used for electricity generation in turbines or fuel cells.

Biomass can be used in electricity generation with and
without CCS, as well as to provide low- to medium-temperature process heat in
industry.

Energy/material storage can be optimised including conventional
pumped hydro storage, electrochemical storage like Lithium ion batteries,
storage of gases including methane, hydrogen and carbon dioxide, storage of
liquid fuels, as well as thermal energy storage in the form of hot water both in
individual buildings and in district heating networks.

Carbon capture is
needed in the model both to capture and sequester process emissions with a
fossil origin, such as those from calcination of fossil limestone in the cement
industry, as well as to provide carbon for the production of hydrocarbons for
dense transport fuels and as a chemical feedstock, for example for the plastics
industry.