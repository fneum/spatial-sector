In this section the core characteristics and assumptions of the model
PyPSA-Eur-Sec are presented. More detailed descriptions of specific sectors,
energy carriers, renewable potentials, transmission infrastructure modelling,
and mathematical problem formulation are covered in the supplementary material
under \crefrange{sec:si:model-overview}{sec:si:math}.

The European sector-coupled energy system model PyPSA-Eur-Sec uses linear
\textbf{optimisation} to minimise total annual operational and investment costs subject
to technical and physical constraints, assuming perfect competition and perfect
foresight over one year of 3-hourly operation (see \cref{sec:si:math} for
mathematical formulation). In this study, we used the historical year 2013 for
weather-dependent inputs. Apart from existing electricity transmission and
hydroelectric facilities, no other existing assets are assumed (so-called
`greenfield optimisation' or `overnight scenario'), so that the model represents
an ideal steady state and disregards pathway dependencies. Cost assumptions are
taken, where possible, from predictions for the year 2030 by the Danish Energy
Agency \cite{dea2019} (see \cref{sec:si:costs}). The model is implemented in the
free software framework Python for Power System Analysis (PyPSA)
\cite{brownPyPSAPython2018}.

PyPSA-Eur-Sec builds upon the model from \cite{brownSynergiesSector2018}, which
covered electricity, heating in buildings and ground transport in Europe with
one node per country. PyPSA-Eur-Sec adds biomass on the supply side, industry,
agriculture, aviation and shipping on the demand side, and higher spatial
resolution to suitably assess infrastructure requirements (here: 181 regions).
Unavoidable process emissions, as well as the need for feedstocks for the
chemicals industry and dense hydrocarbon fuels for aviation, also made careful
management of the carbon cycle necessary, including carbon capture from
industry, biomass combustion and directly from the air.

\cref{fig:multisector} gives an \textbf{overview} of the supply, transmission,
storage and demand sectors implemented in the model. To model interactions in
the sector-coupled energy system, we explicity represent the energy carriers
electricity, heat, methane, hydrogen, carbon dioxide and liquid hydrocarbons
(oil, naphtha) across the different energy sectors. Generator capacities (for
onshore wind, offshore wind, solar photovoltaic (PV), biomass, hydro, oil and
natural gas), storage capacities (for batteries, hydrogen, methane, liquid
hydrocarbons, carbon dioxide and hot water tanks), heating capacities (for heat
pumps, resistive heaters, gas boilers, combined heat and power (CHP) plants and
solar thermal collector units), carbon capture (from industry process emissions,
CHP plants and directly from the air), energy converters (electrolyzers,
methanation, Fischer-Tropsch, steam methane reforming, fuel cells), energy
storage (stationary and electric vehicle batteries, hydrogen storage in caverns
and steel tanks, pit thermal energy storage, pumped-hydro and reservoirs, and
carbon-based fuels like methane and oil), and transmission capacities for
electricity, hydrogen and, for some scenarios, natural gas are all subject to
optimisation, as well as the operational dispatch of each unit in each
represented hour. Demand time series for the different sectors, the ratio of
district heating to decentralised heating, the number of electric vehicles,
methane storage and hydroelectricity capacities (for reservoir and run-of-river
generators and pumped hydro storage) are exogenous to the model.

The time series and potentials of variable renewable electricity \textbf{supply}
are computed from historical weather data (ERA5 and SARAH-2) and various land
eligibility constraints. Moreover, limited amounts of fossil oil and gas can
still be used as primary energy supply. The full \textbf{transmission} network
for European electricity transport \cite{horschPyPSAEurOpen2018} is clustered
down to 181 representative regions based on the methodology used in
\cite{Hoersch2017,frysztackiStrongEffect2021}, thereby preserving the most
important transmission corridors that cause bottlenecks. The linearised optimal
power flow uses a cycle-based formulation from \cite{horschLinearOptimal2018}
that significantly improves computational performance.

For \textbf{industry}, we change some industrial processes to low-emission
alternatives (e.g. switching to hydrogen for direct reduction of iron ore
\cite{voglAssessmentHydrogen2018}), allow more recycling of steel, plastics and
aluminium \cite{circular_economy}, switch fuel sources for process heat, use
synthetic fuels for ammonia and organic chemicals, and allow carbon capture. It
is assumed that no plastic or other non-energy product is sequestered in
landfill, but that all carbon in plastics eventually makes its way back to the
atmosphere, either through combustion or decay; this approach is stricter than
other models \cite{in-depth_2018}.

The \textbf{transport} sector comprises light and heavy road, rail, shipping and
aviation transport. For road and rail, electrification and fuel cell vehicles
are available. For shipping, liquid hydrogen is considered. For aviation, we
consider dense liquid hydrocarbons. Battery electric vehicles for passenger
transport can be enabled with demand response as well as vehicle-to-grid
capabilities.

The \textbf{buildings} sector includes decentral heat supply in individual
housing as well as centralised district heating for urban areas. Heating demand
can be met through air- and ground-sourced heat pumps, gas boilers, CHPs,
resitive heaters as well as waste heat from synthetic fuel production in
district heating networks. For district heating networks, seasonal heat storage
options are also available.

For \textbf{biomass}, only waste and residues from agriculture and forestry are
permitted, using the most conservative potential estimates from the JRC ENSPRESO
database \cite{jrcbiomass2015}. This results in 347~TWh per year of biogas and
1186~TWh per year of solid biomass residues and waste for the whole of Europe.
Biomass can be used in electricity generation with and without CCS, as well as
to provide low- to medium-temperature process heat in industry.

\textbf{Carbon capture} is needed in the model both to capture and sequester
process emissions with a fossil origin, such as those from calcination of fossil
limestone in the cement industry, as well as to use carbon for the production of
hydrocarbons for dense transport fuels and as a chemical feedstock, for example
for the plastics industry.