\subsection*{Resource Availability}

\subsubsection*{Lead Contact}

Requests for further information, resources and materials should be directed to
the lead contact, Fabian Neumann
(\href{mailto:f.neumann@tu-berlin.de}{f.neumann@tu-berlin.de}).

\subsubsection*{Materials availability}

A dataset of the model results is available on Zenodo at \href{https://doi.org/10.5281/zenodo.6821258}{doi:10.5281/zenodo.6821258}.
The code to reproduce the experiments is available on GitHub at \href{https://github.com/fneum/spatial-sector}{github.com/fneum/spatial-sector}.
We also refer to the documentation of PyPSA (\href{https://pypsa.readthedocs.io}{pypsa.readthedocs.io}),
PyPSA-Eur (\href{https://pypsa-eur.readthedocs.io}{pypsa-eur.readthedocs.io}), and
PyPSA-Eur-Sec (\href{https://pypsa-eur-sec.readthedocs.io}{pypsa-eur-sec.readthedocs.io}).
Technology data was taken from \href{https://github.com/pypsa/technology-data}{github.com/pypsa/technology-data} (v0.4.0).
An interactive scenario explorer can be found at \href{https://h2-network.streamlit.app}{h2-network.streamlit.app}.

\subsubsection*{Data and Code Availability}

A dataset of the model inputs and results has been deposited to Zenodo at
\href{https://doi.org/10.5281/zenodo.6821258}{doi:10.5281/zenodo.6821258}. The
code to reproduce the experiments is available on GitHub at
\href{https://github.com/fneum/spatial-sector}{github.com/fneum/spatial-sector}.

\subsection*{Modelling Setup}

In this section the core characteristics and assumptions of the model
PyPSA-Eur-Sec are presented. More detailed descriptions of specific sectors,
energy carriers, renewable potentials, transmission infrastructure modelling,
and mathematical problem formulation are covered in the supplementary material
under \crefrange{sec:si:model-overview}{sec:si:math}.

The European sector-coupled energy system model PyPSA-Eur-Sec uses linear
\textbf{optimisation} to minimise total annual operational and investment costs
subject to technical and physical constraints, assuming perfect competition and
perfect foresight over one uninterrupted year of 3-hourly operation (see
\cref{sec:si:math} for mathematical formulation). In this study, we used the
historical year 2013 for weather-dependent inputs. Apart from existing
electricity and gas transmission infrastructure and hydroelectric power plants,
no other existing assets are assumed (\textit{greenfield optimisation} or
\textit{overnight scenario}), so that the model assumes a long-term equilibrium
in a market with perfect competition and foresight, and disregards pathway
dependencies. The model is implemented in the free and open
software framework Python for Power System Analysis
(PyPSA).\cite{brownPyPSAPython2018}

PyPSA-Eur-Sec builds upon the model from Brown et
al.,~\cite{brownSynergiesSector2018} which covered electricity, heating in
buildings and ground transport in Europe with one node per country.
PyPSA-Eur-Sec adds biomass on the supply side, industry, agriculture, aviation
and shipping on the demand side, and higher spatial resolution to suitably
assess infrastructure requirements. In this study, the European continent is
divided into 181 regions. Unavoidable process emissions, feedstock demands in
the chemicals industry and the need for dense fuels for aviation and shipping,
also required the addition of a detailed representation of the carbon cycles,
including carbon capture from industry processes, biomass combustion and
directly from the air (DAC).

\cref{fig:multisector} gives an \textbf{overview} of the supply, transmission,
storage and demand sectors implemented in the model. To render interactions in
the sector-coupled energy system, we model the energy carriers electricity,
heat, methane, hydrogen, carbon dioxide and liquid hydrocarbons (oil, methanol,
naphtha) across the different energy sectors. Generator capacities (for onshore
wind, offshore wind, utility-scale and rooftop solar photovoltaic (PV), biomass,
hydroelectricity, oil and natural gas), heating capacities (for heat pumps,
resistive heaters, gas boilers, combined heat and power (CHP) plants and solar
thermal collector units), synthetic fuel production (electrolysers, methanation,
Fischer-Tropsch, steam methane reforming, fuel cells), storage capacities
(stationary and electric vehicle batteries, hydrogen storage in caverns and
steel tanks, pit thermal energy storage, pumped-hydro and reservoirs, and
carbon-based fuels like methane, methanol, and Fischer-Tropsch fuels), carbon
capture (from industry process emissions, steam methane reforming, CHP plants
and directly from the air), and transport capacities of electricity transmission
lines, new hydrogen and repurposed natural gas pipelines are all subject to
optimisation, as well as the operational dispatch of each unit in each
represented hour.

\textbf{Exogenous demand and supply assumptions} in the model include a fully
price inelastic and spatially-fixed demand for the different materials and
energy services in each sector, the extent of land transport electrification,
the use of methanol as shipping fuel and kerosene in aviation, process switching
in industry, the reuse and recycling rates of steel (70\%), aluminium (80\%) and
plastics (55\%) manufacturing, the ratio of district heating to decentralised
heating in densely populated regions, efficiency gains of 29\% due to building
retrofitting, hydroelectricity capacities (for reservoir and run-of-river
generators and pumped hydro storage).

For the \textbf{technology and cost assumptions}, we take estimates for the year
2030 for the main scenarios and run a sensitivity analysis with more progressive
cost projections for the year 2050 in \cref{sec:si:sensitivity-costs}. We take
technology projections for the year 2030 for the main scenarios to account for
expected technology cost reductions in the near-term while acknowledging that
the gradual transition to climate neutrality implies that much of the
infrastructure must be built well in advance of reaching net-zero emissions.
Many numbers come from the technology database published by the Danish Energy
Agency (DEA).\citeS{DEA} A complete referenced list of techno-economic
assumptions is compiled in \cref{tab:si:costs}. Among many other technologies,
for overnight costs we assume 636~\euro/kW$_e$ for rooftop PV, 487 \euro/kW$_e$
for utility-scale PV, 142~\euro/kWh and 160~\euro/kW for batteries,
1035~\euro/kW for onshore wind, 1524 \euro/kW for offshore wind, 450
\euro/kW$_e$ for electrolysers, 1100 \euro/kW$_e$ for fuel cell CHPs,
2~\euro/kWh for underground hydrogen storage, 0.54~\euro/kWh for central pit
thermal energy storage, 628-651~\euro/kW$_{out}$ for methanation,
methanolisation and Fischer-Tropsch processes, 572~\euro/kW$_{CH_4}$ for steam
methane reforming with carbon capture (i.e.~blue hydrogen), and 685~\euro/t for
direct air capture with uninterrupted operation.

The time series and potentials of variable renewable \textbf{energy supply}
(wind, solar, hydro, ambient heat) are computed from historical weather data
(ERA5 \cite{ecmwf} and SARAH-2 \cite{SARAH}). Potentials for wind and solar
generation take various land eligibility constraints into account, e.g.~suitable
land types and exclusion zones around populated and protected areas. As long as
emissions can be offset by negative emission technologies and sequestration
potentials are not exhausted, limited amounts of fossil oil and gas can still be
used as primary energy supply. While no assumption about the origin of fossil
energy is made, imports of renewables-based products into Europe are not
considered.

The full \textbf{transmission} network for European electricity transport is
taken from the electricity-only model version,
PyPSA-Eur,\cite{horschPyPSAEurOpen2018} and is clustered down to 181
representative regions based on the k-means network clustering methodology used
in Hörsch and Brown\cite{Hoersch2017} and Frysztacki et
al.~\cite{frysztackiStrongEffect2021}. This level of aggregation reflects, at
the upper end, the computational limit to solve a temporally resolved
sector-coupled energy system optimisation problem and, at the lower end, the
requirements to preserve the most important transmission corridors that cause
bottlenecks and limit the system integration of renewables. The impacts of
spatial aggregation are evaluated in \cref{sec:si:sensitivity-space}. Power
flows are modelled using a cycle-based load flow linearization from Hörsch et
al.~\cite{horschLinearOptimal2018} that significantly improves computational
performance. The power flow linearisation implies that no transmission losses
are considered. Hydrogen pipeline flows assume a simple transport model. This
means that while incoming and outbound flows must balance for each region and
pipes can transport hydrogen only within their capacity limits, no further
physical gaseous flow constraints are applied. The potential for gas pipeline
retrofitting is estimated based on consolidated network data from the
SciGRID\_gas project \cite{plutaSciGRIDGas2022a} such that for every unit of gas
pipeline decommissioned, 60\% of its capacity becomes available for hydrogen
transport.\cite{gasforclimateEuropeanHydrogen2020}

For \textbf{industry}, we assume that the demand for materials (such as steel,
cement, and high-value chemicals) remain constant, and disregard options for
industry relocation.\cite{toktarovaInteractionElectrified2022} The assumed
industry transformation is characterised by electrification, process switching
to low-emission alternatives (e.g. switching to hydrogen for direct reduction of
iron ore\cite{voglAssessmentHydrogen2018}), more recycling of steel, plastics
and aluminium\cite{circular_economy}, fuel switching for high- and
mid-temperature process heat to biomass and methane, use synthetic fuels for
ammonia and organic chemicals, and allow carbon capture. It is assumed that no
plastic or other non-energy product is sequestered in landfill, but that all
carbon in plastics eventually makes its way back to the atmosphere, either
through combustion or decay; this approach is stricter than other models.
\cite{in-depth_2018}

The \textbf{transport} sector comprises light and heavy road, rail, shipping and
aviation transport. For road and rail, electrification and fuel cell vehicles
for heavy-duty transport are available. For shipping, methanol is considered.
Aviation consumes kerosene whose origin (fossil or synthetic) is endogenously
determined. Half of the battery electric vehicle fleet for passenger transport
is assumed to engage in demand response schemes as well as vehicle-to-grid
operation.

The \textbf{buildings} sector includes decentral heat supply in individual
housing as well as centralised district heating for urban areas. Heating demand
can be met through air- and ground-sourced heat pumps, gas boilers,
gas/biomass/hydrogen CHPs, resitive heaters as well as waste heat from synthetic
fuel production in district heating networks. For district heating networks,
seasonal heat storage options are also available. Efficiency gains from building
retrofitting of 29\% are exogenous to the model based on Zeyen et
al.\cite{zeyenMitigatingHeat2021}

For \textbf{biomass}, only waste and residues from agriculture and forestry are
permitted, using the medium potential estimates from the JRC ENSPRESO database.
\cite{ruizENSPRESOOpen2019} This results in 336~TWh per year of biogas that can
be upgraded and 1038~TWh per year of solid biomass residues and waste for the
whole of Europe. Biomass can be used in combined electricity and heat generation
with and without carbon capture, as well as to provide low- to
medium-temperature process heat in industry.

\textbf{Carbon capture} is needed in the model both to capture and sequester
process emissions with a fossil origin, such as those from calcination of fossil
limestone in the cement industry, as well as to use carbon for the production of
hydrocarbons for dense transport fuels and as a chemical feedstock, for example
to produce plastic. \co can be captured from exhaust gases (industry process
emissions, steam methane reforming, CHP plants) or by direct air capture.
Captured \co can be used to produce synthetic hydrocarbons via Sabatier,
Fischer-Tropsch or methanolisation processes. Up to 200 Mt\co/a may be
sequestered underground, which is sufficient to capture process emissions but
limits the system's reliance on negative emission technologies. Landfill of
plastics is not considered as long-term sequestration option.