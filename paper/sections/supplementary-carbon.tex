\section{Carbon dioxide capture, usage and sequestration (CCU/S)}
\label{sec:si:carbon-management}

Carbon management becomes important in net-zero scenarios.
PyPSA-Eur-Sec includes carbon capture from air, electricity generators and
industrial facilities, carbon dioxide storage and transport, the usage of carbon
dioxide in synthetic hydrocarbons, as well as the ultimate sequestration of
carbon dioxide underground.

\subsection{Carbon Capture}

Carbon dioxide can be captured from industry process emissions, steam methane
reforming, methane or biomass used for process heat in the industry, combined
heat and power plants (CHP using biomass or methane), and directly from the air
using direct air capture (DAC). The capacities of each carbon capture technology
are co-optimised.

As shown in \cref{fig:process-emissions}, the model includes industrial process
emissions with fossil-origin totalling 127 Mt\co/a based on
the JRC-IDEES database \cite{IDEES}. Process emissions originate, for instance,
from limestone in cement production. These emissions need to be captured and
sequestered or offset to achieve net-zero emissions. Industry process emissions
are captured assuming a capture rate of 90\% and assuming costs of \co capturing
like in the cement industry \citeS{DEA}. The electricity and heat demand of process
emission carbon capture is currently ignored.

For steam methane reforming (SMR), CHP units, and biomass and methane demand in
industry the model can decide between two options (with and without carbon
capture) with different costs. Here, we also apply a capture rate of 90\%.

DAC includes the energy requirements of the adsorption phase with inputs
electricity and heat to assist adsorption process and regenerate adsorbent, as
well as the compression of \co prior to storage which consumes electricity and
rejects heat. We assume a net energy consumption of 1.8 MWh/t\co heat and 0.47
MWh/t\co electricity based on DEA data \citeS{DEA}. These values are a bit higher
compared to Breyer et al.~\citeS{breyerCarbonDioxide2020} who assume
requirements of 1.2 MWh/t\co heat at \SI{100}{\celsius} and 0.2 MWh\el/t\co
electricity.

\subsection{Carbon Usage}

Captured \co can be used to produce synthetic methane and liquid hydrocarbons
(e.g. naphtha). See \cref{sec:si:methane:supply} and \cref{sec:si:oil:supply}.
If carbon captured from biomass is used, the \co emissions of the synthetic
fuels are net-neutral.

\subsection{Carbon Transport and Sequestration}

Captured \co can also be stored underground up to an annual sequestration limit
of 200 Mt\co/a. Compared to other studies, this is a conservative assumption but
sufficient to capture and sequester process emissions. The sequestration of
captured \co from bioenergy results in net negative emissions. As stored carbon
dioxide is modelled as a single node for Europe, transport constraints are
neglected. For for \co transport and sequestration we assume a cost of
\SI{20}{\sieuro\per\tco} based on IEA data \citeS{ieaCarbonCapture}.

\citeS{fasihiTechnoeconomicAssessment2019, martin-robertsCarbonCapture2021}