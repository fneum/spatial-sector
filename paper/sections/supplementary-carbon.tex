\section{Carbon dioxide capture, usage and sequestration (CCU/S)}
\label{sec:si:carbon-management}

Carbon management becomes important in net-zero scenarios. PyPSA-Eur-Sec
includes carbon capture from electricity generators and industrial facilities,
carbon dioxide storage and transport, the usage of carbon dioxide in synthetic
hydrocarbons, as well as the ultimate sequestration of carbon dioxide
underground.


direct air capture DAC

synthetic methane and liquid hydrocarbons

transport and sequestration is \SI{20}{\sieuro\per\tco}

yearly sequestration limit to 200 Mt\co / a

carbon cycle

carbon is tracked through system

Carbon capture is needed in the model both to capture and sequester process
emissions with a fossil origin, such as those from calcination of fossil
limestone in the cement industry, as well as to provide carbon for the
production of hydrocarbons for dense transport fuels and as a chemical
feedstock, for example for the plastics industry.

Carbon dioxide can be captured from industry process emissions, emissions
related to industry process heat (methane and biomass), combined heat and power plants, and directly
from the air wit direct air capture (DAC).

DAC includes
- adsorption phase with inputs electricity and heat to assist adsorption process and regenerate adsorbent
- compression of \co prior to storage which consumes electricity and rejects heat

process emissions captured at 90\% capture rate at cost as in cement industry

SMR, CHP, biomass and methane demand in industry the model can decide w/wo carbon capture

these capacities are co-optimised

Carbon dioxide can be used as an input for methanation and Fischer-Tropsch
fuels/naphtha, or it can be sequestered underground.

Can also be sequestered:
- 20 \euro/t\co for transport and sequestration
- 2-14 USD/t\co for pipe transport IEA
- 10 USD/t\co for underground sequestration
- limit is 200 Mt (rather conservative but enough to capture process emissions)

unavoidable process emissions

\co: single node for Europe, but a transport and storage cost is added for sequestered \co. Optionally: nodal, with \co transport via pipelines.

Why net-zero target for \co? Since don't include LULUCF or non-\co (waste management and agriculture), which balance each other in EU analysis

CCU/S needed for synthetic fuels AND to deal with process emissions (from e.g.
cement)

need for feedstocks in chemicals industry and dense hydrocarbon fuels for aviation

we have capture on

- industrial process emissions, using same assumptions as the cement kiln
example described above from DEA  (90\% capture rate, 0.72 MWh/t\co heat and
0.1 MWh/t\co electricity, including compression and dehydration) - heat is
taken from urban heat buses, electricity from public grid - steam methane
reforming - biomass CHPs, using DEA assumptions for post-combustion capture on a
small CHP in 2030 (90\% capture rate, 0.72 MWh/t\co heat and 0.11 MWh/t\co
electricity, including compression and dehydration) - heat and electricity is
taken from CHP output - direct air capture (2 MWh/t\co heat, 0.47 MWh/t\co
electricity including compression and dehydration) - heat is taken from urban
heat buses (even though T is below \SI{100}{\celsius}), electricity from public grid; waste
heat from compression is used for amine washing

TODO: as of 210202: electricity and heat demand of process emission CC ignored
(only capital costs are used); also for fuel-based emissions, simple 10\% of
fuel is taken

127 Mt\co/a fossil-origin process emissions in industry (limestone for cement,
soda ash and graphite electrodes); need sequestration for this otherwise not
net-zero.

For FT-fuels demand need 0.27*1391 = 376 Mt\co/a. This comes from BECCU and
industry CCU with synfuels. Could also come from waste + CCU.


Comparison of heat/electricity requirements for capture:

\citeS{kuramochiComparativeAssessment2012} 5.2: Can integrate low-grade steam from power plants with
\co capture from cement

``In the case of post-combustion \co capture, a CHP plant will likely be built
together with the \co capture unit because this is the only way to generate
steam efficiently for \co capture solvent regeneration.''

Beware: need to take account of heat need for regenerating \co capture solvent.
In power plants, can use waste heat, but in industrial plants there is less
low-grade heat (need ~110 C). \citeS{kuramochiComparativeAssessment2012}

Need around 3.9 GJ/t\co heat for regeneration of aqueous MEA ~ 1 MWh/t\co
\citeS{zhangParametricStudy2016}. Solid biomass has 0.3 t\co/MWh\th => need 0.3 MWh heat for
each MWh\th of solid biomass => heat output is hugely reduced!!! Unless we have
oxyfuel biomass combustion...

Breyer \citeS{breyerCarbonDioxide2020}
has 1.2 MWh/t\co heat at \SI{100}{\celsius} and 0.2 MWh\el/t\co electricity for DAC.
Dittmeyer has energy requirements twice as high... DEA is closer to Dittmeyer

\citeS{fasihiTechnoeconomicAssessment2019, martin-robertsCarbonCapture2021}