\section{Heating Sector}
\label{sec:si:heat}

\subsection{Heat Demand}
\label{sec:si:heat:demand}

Building heating considering space and water heating in the residential and
services sectors is resolved for each region, both for individual buildings and
district heating systems, which include different supply options.

Annual heat demands per country are retrieved from JRC-IDEES\citeS{europeancommission.jointresearchcentre.JRCIDEESIntegrated2017} for
the year 2011 and split into space and water heating. The space heating demand
is reduced by retrofitting measures that improve the buildings' thermal
envelopes. This reduction is exogenously fixed at 29\%.\citeS{zeyenMitigatingHeat2021} For space
heating, the annual demands are converted to daily values based on the
population-weighted Heating Degree Day (HDD) using the \textit{atlite} tool,
\citeS{hofmannAtliteLightweight2021} where space heat demand is proportional to the difference between the
daily average ambient temperature (read from ERA5\citeS{ecmwf}) and a threshold
temperature above which space heat demand is zero. A threshold temperature of
\SI{15}{\celsius} is assumed. The daily space heat demand is distributed to the
hours of the day following heat demand profiles from BDEW.\citeS{bdewBDEWHeat2021} These differ
for weekdays and weekends/holidays and between residential and services demand.
Hot water demand is assumed to be constant throughout the year.

For every country, heat demand is split between low and high population density
areas. These country-level totals are then distributed to each region in
proportion to their rural and urban populations respectively. Urban areas with
dense heat demand can be supplied with large-scale district heating systems. We
assume that by mid-century, 60\% of urban heat demand is supplied by district heating
networks. Lump-sum losses of 15\% are assumed in district heating systems.
Cooling demand is supplied by electricity and included in the electricity
demand. Cooling demand is assumed to remain at current levels.

The regional distribution of the total heat demand is depicted in
\cref{fig:demand-space:heat}. As \cref{fig:demand-time} reveals, the total heat
demand is similar to the total electricity demand but features much more
pronounced seasonal variations. The total building heating demand adds up to
\SI{3084}{\twh\per\year} of which 78\% occurs in urban areas.

\subsection{Heat Supply}
\label{sec:si:heat:supply}

Different supply options are available depending on whether demand is met
centrally through district heating systems or decentrally through appliances in
individual buidlings. Supply options in individual buildings include gas and oil
boilers, air- and ground-sourced heat pumps, resistive heaters, and solar
thermal collectors. For large-scale district heating systems more options are
available: combined heat and power (CHP) plants consuming gas or biomass from
waste and residues with and without carbon capture (CC), large-scale air-sourced
heat pumps, gas and oil boilers, resistive heaters and fuel cell CHPs.
Additionally, waste heat from the Fischer-Tropsch and Sabatier processes for the
production of synthetic hydrocarbons can supply district heating systems.
Ground-source heat pumps are only allowed in rural areas because of space constraints.
Thus, only air-source heat pumps are allowed in urban areas. This is a conservative
assumption, since there are many possible sources of low-temperature heat that
could be tapped in cities (e.g.~waste water, ground water, or natural bodies of water).
Costs, lifetimes and efficiencies for these technologies are listed in \cref{sec:si:costs}.

% CHPs

CHPs are based on back pressure plants operating with a fixed ratio of
electricity to heat output. The efficiencies of each are given on the back
pressure line, where the back pressure coefficient $c_b$ is the electricity
output divided by the heat output. For biomass CHP, we assume $c_b=0.46$,
whereas for gas CHP, we assume $c_b=1$.

% heat pumps

The coefficient of performance (COP) of air- and ground-sourced heat pumps
depends on the ambient or soil temperature respectively. Hence, the COP is a
time-varying parameter. Generally, the COP will be lower during winter when
temperatures are low. Because the ambient temperature is more volatile than the
soil temperature, the COP of ground-sourced heat pumps is less variable.
Moreover, the COP depends on the difference between the source and sink
temperatures
\begin{equation}
    \Delta T = T_{sink} - T_{source}.
\end{equation}
For the sink water temperature $T_{sink}$ we assume \SI{55}{\celsius} For the
time- and location-dependent source temperatures $T_{source}$, we rely on the
ERA5 reanalysis weather data.\citeS{ecmwf} The temperature differences are
converted into COP time series using results from a regression analysis
performed in.\citeS{staffellReviewDomestic2012} For air-sourced heat pumps
(ASHP), we use the function
\begin{equation}
    COP(\Delta T) = 6.81 + 0.121 \Delta T + 0.000630 \Delta T^2;
\end{equation}
for ground-sourced heat pumps (GSHP), we use the function
\begin{equation}
    COP(\Delta T) = 8.77 + 0.150 \Delta T + 0.000734 \Delta T^2.
\end{equation}
The resulting time series are displayed in \cref{fig:cfs-ts}.
The spatial diversity of heat pump coefficients is shown in \cref{fig:cfs-maps}.

\subsection{Heat Storage}
\label{sec:si:heat:storage}

Thermal energy storage (TES) is available in large water pits associated with
district heating networks for seasonal storage and small water tanks for
decentral short-term storage. A thermal energy density
46.8~kWh$_{\text{th}}$/m$^3$ is assumed, corresponding to temperature difference
of \SI{40}{\kelvin}. The decay of thermal energy $1-\exp(-\sfrac{1}{24\tau})$ is
assumed to have a time constant of $\tau=180$ days for central TES and $\tau=3$
days for individual TES. The charging and discharging efficiencies are 90\% due
to pipe losses.