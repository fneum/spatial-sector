\section{Transport Sector}
\label{sec:si:transport}

Transport and mobility comprises light and heavy road, rail, shipping and
aviation transport. Annual energy demands for this sector are derived from the
JRC-IDEES database \citeS{IDEES}.

\subsection{Land Transport}
\label{sec:si:transport:land}

The diffusion of battery electric vehicles (BEV) and fuel cell electric vehicles
(FCEV) in land transport is exogenously defined. For 2050, we assume that 85\%
of land transport is electrified and 15\% uses hydrogen fuel cells. No more
internal combustion engines exist.

The energy savings gained by electrifying road transport, are computed through
country-specific factors that compare the current final energy consumption of
cars per distance travelled (average for Europe
\SI{0.7}{\kilo\watt\hour\per\kilo\metre}, \citeS{}) to the
\SI{0.18}{\kilo\watt\hour\per\kilo\metre} assumed for the battery-to-wheel
efficiency of electric vehicles.

Weekly profiles of distances travelled published by the Germand Federal Highway
Research Institute (BASt) \citeS{} are used to generate hourly time series for
each European country taking into account their local time. Furthermore, a
temperature dependence is included in the time series to account for
heating/cooling demand in transport. For temperatures below \SI{15}{\celsius}
and above \SI{20}{\celsius} temperature coefficients of
\SI{0.98}{\percent\per\celsius} and \SI{0.63}{\percent\per\celsius} are assumed
\citeS{brown2018}

For battery electric vehicles, we assume a storage capacity of
\SI{50}{\kilo\watt\hour}, a charging capacity of \SI{11}{\kilo\watt} and a 90\%
charging efficiency. We assume that half of the BEV fleet can shift their
charging time and participate in vehicle-to-grid (V2G) services to facilitate
system operation. The BEV state of charge is forced to be higher than 75\% at
7am every day to ensure that the batteries are sufficiently charged for the peak
usage in the morning. This also restricts BEV demand to be shifted within a day
and prevent EV batteries from becoming seasonal storage. The percentage of BEV
connected to the grid at any time is inversely proportional to the transport
demand profile, which translates into an average/minimum availability of
80\%/62\% of the time. These values are conservative compared to most of the
literature, where average parking times of the European vehicle fleet is
estimated at 92\% \citeS{}. The battery cost of BEV is not included in the model
since it is assumed that BEV owners buy them to primarily satisfy their mobility
needs.

\subsection{Aviation}
\label{sec:si:transport:aviation}

The aviation sector consumes kerosene that is synthetically produced or of
fossil origin (see \cref{sec:si:oil:supply}).

\subsection{Shipping}
\label{sec:si:transport:shipping}

The shipping sector consumes liquid hydrogen.
The liquefaction costs for hydrogen are taken into account.
Other fuel options, like methanol or ammonia, are currently not considered.