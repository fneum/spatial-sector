\section{Grid Reinforcement and Onshore Wind Potential Restrictions}
\label{sec:si:sensitivity-lv-onw}

\subsection{Cost of Electricity Grid Reinforcement Restrictions}
\label{sec:si:lv}

\begin{figure}
    \centering
    \makebox[\textwidth][c]{
    \begin{subfigure}[t]{1.2\textwidth}
        \centering
        \caption{Sensitivity of total system cost towards electricity transmission grid expansion limits}
        \includegraphics[width=\textwidth]{lv-sensitivity.pdf}
        \label{fig:lv-restriction}
    \end{subfigure}
    }
    \makebox[\textwidth][c]{
    \begin{subfigure}[t]{1.2\textwidth}
        \centering
        \caption{Sensitivity of total system cost towards onshore wind expansion limits}
        \includegraphics[width=\textwidth]{onw-sensitivity.pdf}
        \label{fig:onw-restriction}
    \end{subfigure}
    }
    \caption{Sensitivity of total system cost towards electricity transmission grid expansion limits and onshore wind restrictions.
    The sweep for grid expansion restrictions allows full onshore wind potentials.
    The extreme case (a, left) removes all existing power transmission lines as well.
    The sweep for onshore wind potential restrictions allows power grid reinforcements by up to 25\% of today's transmission capacities.
    The extreme case (b, left) combines no grid expansion with no onshore wind potentials.}

    \label{fig:lv-onw-restriction}
\end{figure}

In the following sensitivity runs, the model is allowed to build new electricity
transmission infrastructure wherever is cost-optimal, but the total volume of
new transmission capacity (sum of line length times capacity, TWkm) is
successively limited. The volume limit is given in fractions of today's grid
volume: a line volume limit of 100\% means no new capacity is allowed beyond
today's grid (since the model cannot remove existing lines); a limit of 125\%
means the total grid capacity can grow by 25\% (25\% is similar to the planned
extra capacity in the European network operators' Ten Year Network Development
Plan (TYNDP) \citeS{TYNDP2016}). For this investigation, a hydrogen network
could be built.

\cref{fig:lv-restriction} shows the composition of total yearly system costs
(including all investment and operational costs) as we vary the allowed grid
expansion, from no expansion (only today's grid) to a doubling of today's grid
capacities (the model optimises where new capacity is placed). As the grid is
expanded, total costs decrease only slightly, despite the increasing costs of
the grid. The total cost benefit of a doubling of grid capacity is around
\euro47~billion per year corresponding to an expansion of 644 TWkm. However, over
half of the benefit (\euro29~billion per year) is available already at a 25\%
expansion corresponding to an expansion of 403 TWkm.

\cref{fig:lv-restriction} also includes a scenario where today's electricity
transmission infrastructure is completely removed from the model. While doubling
the transmission grid yields a benefit of \euro47~billion per year, removing
what exists incurs a cost of \euro~86~billion per year. The lack of electricity
grid is mostly compensated by more solar PV generation, battery storage and
re-electrified hydrogen.

\subsection{Cost of Onshore Wind Potential Elimination}
\label{sec:si:onw}

\begin{figure}
    \centering
    \makebox[\textwidth][c]{
        \begin{subfigure}[t]{0.6\textwidth}
            \centering
            \caption{hydrogen network}
            \includegraphics[height=0.42\textheight]{\hyrun/maps/elec_s_181_lv1.0__Co2L0-3H-T-H-B-I-A-solar+p3-linemaxext10-onwind+p0-h2_network_2030.pdf}
            \label{fig:no-onw:h2}
        \end{subfigure}
        \begin{subfigure}[t]{0.6\textwidth}
            \centering
            \caption{energy balance}
            \includegraphics[height=0.42\textheight]{\hyrun/elec_s_181_lv1.0__Co2L0-3H-T-H-B-I-A-solar+p3-linemaxext10-onwind+p0_2030/import-export-total-200.pdf}
            \label{fig:no-onw:io}
        \end{subfigure}
        }
    \begin{subfigure}[t]{0.6\textwidth}
        \centering
        \caption{system cost}
        \includegraphics[height=0.42\textheight]{\hyrun/maps/elec_s_181_lv1.0__Co2L0-3H-T-H-B-I-A-solar+p3-linemaxext10-onwind+p0-costs-all_2030.pdf}
        \label{fig:no-onw:tsc}
    \end{subfigure}
    \caption{Maps of regional energy balance, hydrogen network and production sites, and spatial and technological distribution of system costs for a scenario without onshore wind and without power grid expansion.}
    \label{fig:no-onw}
\end{figure}

Like building new power transmission lines, the deployment of onshore wind may
not always be socially accepted, such that it may not be possible to leverage
its full potential. In the following additional sensitivity analysis, we explore
the hypothetical impact of restricting the installable potentials of onshore
wind down to zero (\cref{fig:no-onw}).

We find that as onshore wind is eliminated, costs rise by \euro~104 bn/a (12\%)
when the electricity grid is fixed to today's capacities, but a hydrogen network
can still be developed. In comparison to the least-cost solution with full
network expansion, this solution is 23\% more expensive. A solution in which
neither a hydrogen network could be developed would be 30\% more expensive.
\cref{sec:si:onw-compromise} presents further intermediate results between full
and no onshore wind expansion for scenarios with hydrogen network expansion and
TYNDP-equivalent power grid reinforcements. The model substitutes onshore wind,
particularly in the British Isles, for higher investment in offshore wind in the
North Sea and solar generators in Southern Europe (\cref{fig:no-onw:tsc}).
Because offshore capacities are concentrated near coastlines, and grid capacity
is restricted, total spending on hydrogen electrolysers and networks also
increases to absorb the increased offshore generation. Without onshore wind, the
potentials for rooftop solar PV and offshore wind in Europe are largely
exhausted, such that in this self-sufficient scenario for Europe, the effect of
installable potentials becomes critical.

Whereas with onshore wind, the British Isles and North Sea dominate hydrogen
production, Southern Europe becomes a large exporter of solar-based hydrogen if
the development of onshore wind capacities is restricted
(\cref{fig:no-onw:h2,fig:no-onw:io}). This shift in hydrogen infrastructure also
impacts the share of gas pipelines being retrofitted for hydrogen transport. As
the Iberian Peninsula becomes a preferred region for hydrogen production but has
a more sparse gas transmission network today, the rate of retrofitted pipeline
capacity reduces from 66\% to 53\%. Many new hydrogen pipelines are built to
connect Spain with France, but also to connect increased hydrogen production
from Danish offshore wind to Germany. Gas pipeline retrofitting is then
concentrated in Germany, Austria and Italy.

\begin{figure}
    \centering
    \includegraphics[width=\textwidth]{sensitivity-h2.pdf}
    \caption{Varying benefits of hydrogen network infrastructure and changes in system composition as power grid and onshore wind expansion options are altered. The benefit of a hydrogen backbone is robust across all scenarios varying between 3.5\% and 5.7\%.}
    \label{fig:h2-restriction-w-onw}
\end{figure}

The benefit of a hydrogen network is similar whether or not onshore wind
capacities are built in Europe, even though the hydrogen network topology is
then built around supply from offshore wind in the North Sea and solar PV from
Southern Europe rather than from onshore wind in North-West Europe. As
\cref{fig:h2-restriction-w-onw} illustrates, the net benefit is again strongest
when power grid expansion is restricted. If both onshore wind and power grid
expansion are excluded, costs for a system without a hydrogen network option
were by \euro~53~billion per year (5.6\%) higher. With cost-optimal electricity
grid reinforcement, the net benefit of a hydrogen network is still as high as
\euro~30~billion per year (3.5\%).

\subsection{Cost of Compromises on Onshore Wind Potential Restrictions}
\label{sec:si:onw-compromise}

In the following sensitivity runs, the maximum installable capacity of onshore
wind is successively restricted down to zero at each node. The upper limit is
derived from land use restriction and yields a maximum technical potential
corresponding to about \SI{481}{\giga\watt} for Germany. For this investigation,
a compromise electricity grid expansion by 25\% compared to today and no limits
on hydrogen network infrastructure are assumed.

As previously discussed in \cref{sec:si:onw}, costs rise by \euro~103~billion
per year (13\%) by restricting the installable potentials of onshore down to
zero. Just as in the case of restricted line volumes, \cref{fig:onw-restriction}
reveals a nonlinear rise in system costs: if we constrain the model to 25\% of
the onshore potential (around 120~GW for Germany), costs rise by only
\euro~64~billion per year (5\%). Thereby, 25\% of the onshore wind potential may
represent a social compromise between total system cost, and social concerns
about onshore wind development.

In comparison, Schlachtberger et al.~\citeS{schlachtbergerCostOptimal2018} found
a smaller change between 9\% and 12\% in system costs in an electricity-only
model when onshore wind potentials were restricted across various grid expansion
limitations. The the biggest change was observed when the power grid could not
be reinforced. Onshore wind was largely replaced with offshore wind in that
model. Unlike that model, here we have a higher grid resolution (181 versus 30
regions) which allows us to better assess the grid integration costs of offshore
wind. Our results show that moderate power grid expansion is particularly
important when onshore wind development is severely limited. For the extreme
case where no onshore wind capacities would be built, reducing power grid
expansion from 25\% to none incurs another rise in system cost of an additional
\euro~50~billion per year (6\%).


\section{Supplementary Results for Network Expansion Scenarios}
\label{sec:si:results-network-expansion}

In this section, supplementary results for the different network expansion
scenarios are presented. \cref{fig:si:flow-ac,fig:si:flow-h2} display net
electricity and hydrogen flows in their respective transmission networks.
Further figures show the variation of average nodal prices of electricity and
hydrogen in space (\cref{fig:si:lmp-ac,fig:si:lmp-h2}), in time
(\cref{fig:si:lmp-ts-ac,fig:si:lmp-ts-h2}), and as duration curves
(\cref{fig:si:lmp-dc}). Sankey diagrams in \cref{fig:si:sankey} illustrate
energy flows in the system. Across the scenarios, we infer that a carbon price
between \SI{435}{\sieuro\per\tco} with transmission infrastructure expansion and
\SI{513}{\sieuro\per\tco} without would be required to achieve both climate
neutrality and self-sufficiency in Europe.

\begin{figure}
    \begin{subfigure}{0.49\textwidth}
        \centering
        \caption{With power grid expansion, with hydrogen network}
        \includegraphics[width=\textwidth]{20211218-181-h2/elec_s_181_lvopt__Co2L0-3H-T-H-B-I-A-solar+p3-linemaxext10_2030/elec-flow-map-backbone.pdf}
    \end{subfigure}
    \begin{subfigure}{0.49\textwidth}
        \centering
        \caption{With power grid expansion, without hydrogen network}
        \includegraphics[width=\textwidth]{20211218-181-h2/elec_s_181_lvopt__Co2L0-3H-T-H-B-I-A-solar+p3-linemaxext10-noH2network_2030/elec-flow-map-backbone.pdf}
    \end{subfigure}
    \begin{subfigure}{0.49\textwidth}
        \centering
        \caption{Without power grid expansion, with hydrogen network}
        \includegraphics[width=\textwidth]{20211218-181-lv/elec_s_181_lv1.0__Co2L0-3H-T-H-B-I-A-solar+p3-linemaxext10_2030/elec-flow-map-backbone.pdf}
    \end{subfigure}
    \begin{subfigure}{0.49\textwidth}
        \centering
        \caption{Without power grid expansion, without hydrogen network}
        \includegraphics[width=\textwidth]{20211218-181-h2/elec_s_181_lv1.0__Co2L0-3H-T-H-B-I-A-solar+p3-linemaxext10-noH2network_2030/elec-flow-map-backbone.pdf}
    \end{subfigure}
    \caption{Net flow of electricity in the network. The maps shows net flows larger than 10 TWh with arrow sizes proportional to net flow volume. Only power grid expansion enables bulk energy transport in form of electricity. With the existing transmission network, net flows are limited and the transmission infrastructure is rather used for synoptic balancing as weather systems pass the continent.}
    \label{fig:si:flow-ac}
\end{figure}

\begin{figure}
    \begin{subfigure}{0.49\textwidth}
        \centering
        \caption{With power grid expansion, with hydrogen network}
        \includegraphics[width=\textwidth]{20211218-181-h2/elec_s_181_lvopt__Co2L0-3H-T-H-B-I-A-solar+p3-linemaxext10_2030/H2-flow-map-backbone.pdf}
    \end{subfigure}
    \begin{subfigure}{0.49\textwidth}
        \centering
        \caption{Without power grid expansion, with hydrogen network}
        \includegraphics[width=\textwidth]{20211218-181-lv/elec_s_181_lv1.0__Co2L0-3H-T-H-B-I-A-solar+p3-linemaxext10_2030/H2-flow-map-backbone.pdf}
    \end{subfigure}
    \caption{Net flow of hydrogen in the network. The maps shows net flows larger than 10 TWh with arrow sizes proportional to net flow volume. The flows indicate the integration of cheap production sites in Spain and the British Isles and high demands in Central Europe.}
    \label{fig:si:flow-h2}
\end{figure}

\begin{figure}
    \begin{subfigure}{0.49\textwidth}
        \centering
        \caption{With power grid expansion, with hydrogen network}
        \includegraphics[width=\textwidth]{20211218-181-h2/elec_s_181_lvopt__Co2L0-3H-T-H-B-I-A-solar+p3-linemaxext10_2030/nodal-prices-AC.pdf}
    \end{subfigure}
    \begin{subfigure}{0.49\textwidth}
        \centering
        \caption{With power grid expansion, without hydrogen network}
        \includegraphics[width=\textwidth]{20211218-181-h2/elec_s_181_lvopt__Co2L0-3H-T-H-B-I-A-solar+p3-linemaxext10-noH2network_2030/nodal-prices-AC.pdf}
    \end{subfigure}
    \begin{subfigure}{0.49\textwidth}
        \centering
        \caption{Without power grid expansion, with hydrogen network}
        \includegraphics[width=\textwidth]{20211218-181-lv/elec_s_181_lv1.0__Co2L0-3H-T-H-B-I-A-solar+p3-linemaxext10_2030/nodal-prices-AC.pdf}
    \end{subfigure}
    \begin{subfigure}{0.49\textwidth}
        \centering
        \caption{Without power grid expansion, without hydrogen network}
        \includegraphics[width=\textwidth]{20211218-181-h2/elec_s_181_lv1.0__Co2L0-3H-T-H-B-I-A-solar+p3-linemaxext10-noH2network_2030/nodal-prices-AC.pdf}
    \end{subfigure}
    \caption{Regional distribution of average nodal electricity prices. The reinforcement of the electricity grid mitigates regional price differences. Some price differences persist because of expansion constraints on individual lines.}
    \label{fig:si:lmp-ac}
\end{figure}

\begin{figure}
    \begin{subfigure}{0.49\textwidth}
        \centering
        \caption{With power grid expansion, with hydrogen network}
        \includegraphics[width=\textwidth]{20211218-181-h2/elec_s_181_lvopt__Co2L0-3H-T-H-B-I-A-solar+p3-linemaxext10_2030/nodal-prices-H2.pdf}
    \end{subfigure}
    \begin{subfigure}{0.49\textwidth}
        \centering
        \caption{With power grid expansion, without hydrogen network}
        \includegraphics[width=\textwidth]{20211218-181-h2/elec_s_181_lvopt__Co2L0-3H-T-H-B-I-A-solar+p3-linemaxext10-noH2network_2030/nodal-prices-H2.pdf}
    \end{subfigure}
    \begin{subfigure}{0.49\textwidth}
        \centering
        \caption{Without power grid expansion, with hydrogen network}
        \includegraphics[width=\textwidth]{20211218-181-lv/elec_s_181_lv1.0__Co2L0-3H-T-H-B-I-A-solar+p3-linemaxext10_2030/nodal-prices-H2.pdf}
    \end{subfigure}
    \begin{subfigure}{0.49\textwidth}
        \centering
        \caption{Without power grid expansion, without hydrogen network}
        \includegraphics[width=\textwidth]{20211218-181-h2/elec_s_181_lv1.0__Co2L0-3H-T-H-B-I-A-solar+p3-linemaxext10-noH2network_2030/nodal-prices-H2.pdf}
    \end{subfigure}
    \caption{Regional distribution of average nodal hydrogen prices. The
    development of a hydrogen network evens out regional price differences. With
    limited \ce{H2} network expansion prices are almost twice as high in
    Europe's industrial clusters than the most cost-effective hydrogen
    production sites.}
    \label{fig:si:lmp-h2}
\end{figure}

\begin{figure}
    \begin{subfigure}{0.49\textwidth}
        \centering
        \caption{With power grid expansion, with hydrogen network}
        \includegraphics[width=\textwidth]{20211218-181-h2/elec_s_181_lvopt__Co2L0-3H-T-H-B-I-A-solar+p3-linemaxext10_2030/price-ts-AC.pdf}
    \end{subfigure}
    \begin{subfigure}{0.49\textwidth}
        \centering
        \caption{With power grid expansion, without hydrogen network}
        \includegraphics[width=\textwidth]{20211218-181-h2/elec_s_181_lvopt__Co2L0-3H-T-H-B-I-A-solar+p3-linemaxext10-noH2network_2030/price-ts-AC.pdf}
    \end{subfigure}
    \begin{subfigure}{0.49\textwidth}
        \centering
        \caption{Without power grid expansion, with hydrogen network}
        \includegraphics[width=\textwidth]{20211218-181-h2/elec_s_181_lv1.0__Co2L0-3H-T-H-B-I-A-solar+p3-linemaxext10_2030/price-ts-AC.pdf}
    \end{subfigure}
    \begin{subfigure}{0.49\textwidth}
        \centering
        \caption{Without power grid expansion, without hydrogen network}
        \includegraphics[width=\textwidth]{20211218-181-h2/elec_s_181_lv1.0__Co2L0-3H-T-H-B-I-A-solar+p3-linemaxext10-noH2network_2030/price-ts-AC.pdf}
    \end{subfigure}
    \caption{Temporal distribution of average nodal electricity prices. The graphs show daily patterns with price troughs during the day, especially in summer, as well as seasonal patterns with higher prices in winter than in the summer. A few periods in January and February are particularly challenging to the system, resulting in very high electricity prices.}
    \label{fig:si:lmp-ts-ac}
\end{figure}

\begin{figure}
    \begin{subfigure}{0.49\textwidth}
        \centering
        \caption{With power grid expansion, with hydrogen network}
        \includegraphics[width=\textwidth]{20211218-181-h2/elec_s_181_lvopt__Co2L0-3H-T-H-B-I-A-solar+p3-linemaxext10_2030/price-ts-H2.pdf}
    \end{subfigure}
    \begin{subfigure}{0.49\textwidth}
        \centering
        \caption{With power grid expansion, without hydrogen network}
        \includegraphics[width=\textwidth]{20211218-181-h2/elec_s_181_lvopt__Co2L0-3H-T-H-B-I-A-solar+p3-linemaxext10-noH2network_2030/price-ts-H2.pdf}
    \end{subfigure}
    \begin{subfigure}{0.49\textwidth}
        \centering
        \caption{Without power grid expansion, with hydrogen network}
        \includegraphics[width=\textwidth]{20211218-181-h2/elec_s_181_lv1.0__Co2L0-3H-T-H-B-I-A-solar+p3-linemaxext10_2030/price-ts-H2.pdf}
    \end{subfigure}
    \begin{subfigure}{0.49\textwidth}
        \centering
        \caption{Without power grid expansion, without hydrogen network}
        \includegraphics[width=\textwidth]{20211218-181-h2/elec_s_181_lv1.0__Co2L0-3H-T-H-B-I-A-solar+p3-linemaxext10-noH2network_2030/price-ts-H2.pdf}
    \end{subfigure}
    \caption{Temporal distribution of average nodal hydrogen prices. Compared to
    electricity prices, the seasonal component dominates daily patterns. Price
    spikes occur with limited \ce{H2} network expansion in winter periods that
    are challenging to the system.}
    \label{fig:si:lmp-ts-h2}
\end{figure}

\begin{figure}
    \begin{subfigure}{0.49\textwidth}
        \centering
        \caption{With power grid expansion, with hydrogen network}
        \includegraphics[width=.8\textwidth]{20211218-181-h2/elec_s_181_lvopt__Co2L0-3H-T-H-B-I-A-solar+p3-linemaxext10_2030/price-duration-AC.pdf}
    \end{subfigure}
    \begin{subfigure}{0.49\textwidth}
        \centering
        \caption{With power grid expansion, without hydrogen network}
        \includegraphics[width=.8\textwidth]{20211218-181-h2/elec_s_181_lvopt__Co2L0-3H-T-H-B-I-A-solar+p3-linemaxext10-noH2network_2030/price-duration-AC.pdf}
    \end{subfigure}
    \begin{subfigure}{0.49\textwidth}
        \centering
        \includegraphics[width=.8\textwidth]{20211218-181-h2/elec_s_181_lvopt__Co2L0-3H-T-H-B-I-A-solar+p3-linemaxext10_2030/price-duration-H2.pdf}
    \end{subfigure}
    \begin{subfigure}{0.49\textwidth}
        \centering
        \includegraphics[width=.8\textwidth]{20211218-181-h2/elec_s_181_lvopt__Co2L0-3H-T-H-B-I-A-solar+p3-linemaxext10-noH2network_2030/price-duration-H2.pdf}
    \end{subfigure}
    \begin{subfigure}{0.49\textwidth}
        \centering
        \caption{Without power grid expansion, with hydrogen network}
        \includegraphics[width=.8\textwidth]{20211218-181-lv/elec_s_181_lv1.0__Co2L0-3H-T-H-B-I-A-solar+p3-linemaxext10_2030/price-duration-AC.pdf}
    \end{subfigure}
    \begin{subfigure}{0.49\textwidth}
        \centering
        \caption{Without power grid expansion, without hydrogen network}
        \includegraphics[width=.8\textwidth]{20211218-181-h2/elec_s_181_lv1.0__Co2L0-3H-T-H-B-I-A-solar+p3-linemaxext10-noH2network_2030/price-duration-AC.pdf}
    \end{subfigure}
    \begin{subfigure}{0.49\textwidth}
        \centering
        \includegraphics[width=.8\textwidth]{20211218-181-lv/elec_s_181_lv1.0__Co2L0-3H-T-H-B-I-A-solar+p3-linemaxext10_2030/price-duration-H2.pdf}
    \end{subfigure}
    \begin{subfigure}{0.49\textwidth}
        \centering
        \includegraphics[width=.8\textwidth]{20211218-181-h2/elec_s_181_lv1.0__Co2L0-3H-T-H-B-I-A-solar+p3-linemaxext10-noH2network_2030/price-duration-H2.pdf}
    \end{subfigure}
    \caption{Duration curve of nodal electricity and hydrogen prices.}
    \label{fig:si:lmp-dc}
\end{figure}

\begin{figure}
    \makebox[\textwidth][c]{
    \begin{subfigure}{0.6\textwidth}
        \centering
        \caption{With power grid expansion, with hydrogen network}
        \includegraphics[trim=1.5cm 2cm 1.5cm 2.5cm, clip, width=\textwidth]{20211218-181-h2/elec_s_181_lvopt__Co2L0-3H-T-H-B-I-A-solar+p3-linemaxext10_2030/sankey.pdf}
    \end{subfigure}
    \begin{subfigure}{0.6\textwidth}
        \centering
        \caption{With power grid expansion, without hydrogen network}
        \includegraphics[trim=1.5cm 2cm 1.5cm 2.5cm, clip, width=\textwidth]{20211218-181-h2/elec_s_181_lvopt__Co2L0-3H-T-H-B-I-A-solar+p3-linemaxext10-noH2network_2030/sankey.pdf}
    \end{subfigure}
    }
    \makebox[\textwidth][c]{
    \begin{subfigure}{0.6\textwidth}
        \centering
        \caption{Without power grid expansion, with hydrogen network}
        \includegraphics[trim=1.5cm 2cm 1.5cm 2.5cm, clip, width=\textwidth]{20211218-181-h2/elec_s_181_lv1.0__Co2L0-3H-T-H-B-I-A-solar+p3-linemaxext10_2030/sankey.pdf}
    \end{subfigure}
    \begin{subfigure}{0.6\textwidth}
        \centering
        \caption{Without power grid expansion, without hydrogen network}
        \includegraphics[trim=1.5cm 2cm 1.5cm 2.5cm, clip, width=\textwidth]{20211218-181-h2/elec_s_181_lv1.0__Co2L0-3H-T-H-B-I-A-solar+p3-linemaxext10-noH2network_2030/sankey.pdf}
    \end{subfigure}
    }
    \caption{Sankey diagrams of energy flows in the European system.}
    \label{fig:si:sankey}
\end{figure}



\section{Detailed Results of Least-Cost Solution with Full Grid Expansion}
\label{sec:si:detailed}

In the following section we present more detailed results from the scenario
where both the hydrogen and electricity grid could be expanded. Among the
scenarios we investigated, this represents the least-cost solution.
\crefrange{fig:output-ts-1}{fig:output-ts-4} show temporally resolved energy
balances for different carriers: electricity, hydrogen, heat, methane, oil-based
products, and carbon dioxide. These are daily sampled time series for a year and
3-hourly sampled time series for the month February, and indicate how different
technologies are operated both seasonally and daily. \cref{fig:si:lcoe} displays
the regional distribution of levelised cost of electricity for wind and solar
generation. \cref{fig:si:utilisation-rate-map} indicates how synthetic fuel
production facilities are operated regionally. How selected energy system
components are operated throughout the year is shown in
\cref{fig:si:utilisation-rate-ts}. The utilisation of electricity and hydrogen
network assets are presented in \cref{fig:si:grid-utilisation}, alongside
information about where energy is curtailed and what congestion rents are
incurred.

\begin{figure}
    \centering
    % \makebox[\textwidth][c]{
    \begin{subfigure}[t]{\textwidth}
        \centering
        \caption{electricity}
        \includegraphics[width=\textwidth]{20211218-181-lv/elec_s_181_lv1.0__Co2L0-3H-T-H-B-I-A-solar+p3-linemaxext10_2030/ts-balance-total electricity-D-2013.pdf}
    \end{subfigure}
    \begin{subfigure}[t]{\textwidth}
        \centering
        \caption{space and water heating}
        \includegraphics[width=\textwidth]{20211218-181-lv/elec_s_181_lv1.0__Co2L0-3H-T-H-B-I-A-solar+p3-linemaxext10_2030/ts-balance-total heat-D-2013.pdf}
    \end{subfigure}
    \begin{subfigure}[t]{\textwidth}
        \centering
        \caption{hydrogen}
        \includegraphics[width=\textwidth]{20211218-181-lv/elec_s_181_lv1.0__Co2L0-3H-T-H-B-I-A-solar+p3-linemaxext10_2030/ts-balance-H2-D-2013.pdf}
    \end{subfigure}
    %}
    \caption{Daily sampled time series for (a) electricity, (b) heat, and (c) hydrogen supply (above zero) and consumption (below zero) composition. Supply and consumption balance for each bar by definition.}
    \label{fig:output-ts-1}
\end{figure}

\begin{figure}
    \centering
    % \makebox[\textwidth][c]{
    \begin{subfigure}[t]{\textwidth}
        \centering
        \caption{methane}
        \includegraphics[width=\textwidth]{20211218-181-lv/elec_s_181_lv1.0__Co2L0-3H-T-H-B-I-A-solar+p3-linemaxext10_2030/ts-balance-gas-D-2013.pdf}
    \end{subfigure}
    \begin{subfigure}[t]{\textwidth}
        \centering
        \caption{oil-based products}
        \includegraphics[width=\textwidth]{20211218-181-lv/elec_s_181_lv1.0__Co2L0-3H-T-H-B-I-A-solar+p3-linemaxext10_2030/ts-balance-oil-D-2013.pdf}
    \end{subfigure}
    \begin{subfigure}[t]{\textwidth}
        \centering
        \caption{stored \co}
        \includegraphics[width=\textwidth]{20211218-181-lv/elec_s_181_lv1.0__Co2L0-3H-T-H-B-I-A-solar+p3-linemaxext10_2030/ts-balance-co2 stored-D-2013.pdf}
    \end{subfigure}
    %}
    \caption{Daily sampled time series for (a) methane, (b) oil-based products, and (c) carbon dioxide supply (above zero) and consumption (below zero) composition. Supply and consumption balance for each bar.}
    \label{fig:output-ts-2}
\end{figure}


\begin{figure}
    \centering
    % \makebox[\textwidth][c]{
    \begin{subfigure}[t]{\textwidth}
        \centering
        \caption{electricity}
        \includegraphics[width=\textwidth]{20211218-181-lv/elec_s_181_lv1.0__Co2L0-3H-T-H-B-I-A-solar+p3-linemaxext10_2030/ts-balance-total electricity--2013-02.pdf}
    \end{subfigure}
    \begin{subfigure}[t]{\textwidth}
        \centering
        \caption{space and water heating}
        \includegraphics[width=\textwidth]{20211218-181-lv/elec_s_181_lv1.0__Co2L0-3H-T-H-B-I-A-solar+p3-linemaxext10_2030/ts-balance-total heat--2013-02.pdf}
    \end{subfigure}
    \begin{subfigure}[t]{\textwidth}
        \centering
        \caption{hydrogen}
        \includegraphics[width=\textwidth]{20211218-181-lv/elec_s_181_lv1.0__Co2L0-3H-T-H-B-I-A-solar+p3-linemaxext10_2030/ts-balance-H2--2013-02.pdf}
    \end{subfigure}
    %}
    \caption{Hourly sampled time series of February for (a) electricity, (b) heat, and (c) hydrogen supply (above zero) and consumption (below zero) composition. Supply and consumption balance for each bar.}
    \label{fig:output-ts-3}
\end{figure}

\begin{figure}
    \centering
    % \makebox[\textwidth][c]{
    \begin{subfigure}[t]{\textwidth}
        \centering
        \caption{methane}
        \includegraphics[width=\textwidth]{20211218-181-lv/elec_s_181_lv1.0__Co2L0-3H-T-H-B-I-A-solar+p3-linemaxext10_2030/ts-balance-gas--2013-02.pdf}
    \end{subfigure}
    \begin{subfigure}[t]{\textwidth}
        \centering
        \caption{oil-based products}
        \includegraphics[width=\textwidth]{20211218-181-lv/elec_s_181_lv1.0__Co2L0-3H-T-H-B-I-A-solar+p3-linemaxext10_2030/ts-balance-oil--2013-02.pdf}
    \end{subfigure}
    \begin{subfigure}[t]{\textwidth}
        \centering
        \caption{stored \co}
        \includegraphics[width=\textwidth]{20211218-181-lv/elec_s_181_lv1.0__Co2L0-3H-T-H-B-I-A-solar+p3-linemaxext10_2030/ts-balance-co2 stored--2013-02.pdf}
    \end{subfigure}
    %}
    \caption{Hourly sampled time series of February for (a) methane, (b) oil-based products, and (c) carbon dioxide supply (above zero) and consumption (below zero) composition. Supply and consumption balance for each bar.}
    \label{fig:output-ts-4}
\end{figure}

\begin{figure}
    \begin{subfigure}{0.49\textwidth}
        \centering
        \caption{utility-scale solar}
        \includegraphics[width=\textwidth]{20211218-181-h2/elec_s_181_lvopt__Co2L0-3H-T-H-B-I-A-solar+p3-linemaxext10_2030/lcoe-solar.pdf}
    \end{subfigure}
    \begin{subfigure}{0.49\textwidth}
        \centering
        \caption{onshore wind}
        \includegraphics[width=\textwidth]{20211218-181-h2/elec_s_181_lvopt__Co2L0-3H-T-H-B-I-A-solar+p3-linemaxext10_2030/lcoe-onwind.pdf}
    \end{subfigure}
    \begin{subfigure}{0.49\textwidth}
        \centering
        \caption{offshore wind (AC-connected)}
        \includegraphics[width=\textwidth]{20211218-181-h2/elec_s_181_lvopt__Co2L0-3H-T-H-B-I-A-solar+p3-linemaxext10_2030/lcoe-offwind-dc.pdf}
    \end{subfigure}
    \begin{subfigure}{0.49\textwidth}
        \centering
        \caption{offshore wind (DC-connected)}
        \includegraphics[width=\textwidth]{20211218-181-h2/elec_s_181_lvopt__Co2L0-3H-T-H-B-I-A-solar+p3-linemaxext10_2030/lcoe-offwind-dc.pdf}
    \end{subfigure}
    \caption{Levelised cost of electricity of wind and solar generation. Only shows locations where more than 1 GW of the respective technology were built.}
    \label{fig:si:lcoe}
\end{figure}

\begin{figure}
    \centering
    \begin{subfigure}{0.66\textwidth}
        \centering
        \caption{electrolysis}
        \includegraphics[width=\textwidth]{20211218-181-h2/elec_s_181_lvopt__Co2L0-3H-T-H-B-I-A-solar+p3-linemaxext10_2030/utilisation-factors-H2 Electrolysis.pdf}
    \end{subfigure}
    \begin{subfigure}{0.66\textwidth}
        \centering
        \caption{Fischer-Tropsch conversion}
        \includegraphics[width=\textwidth]{20211218-181-h2/elec_s_181_lvopt__Co2L0-3H-T-H-B-I-A-solar+p3-linemaxext10_2030/utilisation-factors-Fischer-Tropsch.pdf}
    \end{subfigure}
    \caption{Utilisation rate of synthetic fuel production capacities. Only shows locations where more than 1 GW of the respective technology were built.}
    \label{fig:si:utilisation-rate-map}
\end{figure}

\begin{figure}
    \begin{subfigure}{0.49\textwidth}
        \centering
        \includegraphics[width=\textwidth]{20211218-181-h2/elec_s_181_lvopt__Co2L0-3H-T-H-B-I-A-solar+p3-linemaxext10_2030/cf-ts-H2 Electrolysis.pdf}
    \end{subfigure}
    \begin{subfigure}{0.49\textwidth}
        \centering
        \includegraphics[width=\textwidth]{20211218-181-h2/elec_s_181_lvopt__Co2L0-3H-T-H-B-I-A-solar+p3-linemaxext10_2030/cf-ts-Fischer-Tropsch.pdf}
    \end{subfigure}
    \begin{subfigure}{0.49\textwidth}
        \centering
        \includegraphics[width=\textwidth]{20211218-181-h2/elec_s_181_lvopt__Co2L0-3H-T-H-B-I-A-solar+p3-linemaxext10_2030/cf-ts-electricity distribution grid.pdf}
    \end{subfigure}
    \begin{subfigure}{0.49\textwidth}
        \centering
        \includegraphics[width=\textwidth]{20211218-181-h2/elec_s_181_lvopt__Co2L0-3H-T-H-B-I-A-solar+p3-linemaxext10_2030/cf-ts-OCGT.pdf}
    \end{subfigure}
    \begin{subfigure}{0.49\textwidth}
        \centering
        \includegraphics[width=\textwidth]{20211218-181-h2/elec_s_181_lvopt__Co2L0-3H-T-H-B-I-A-solar+p3-linemaxext10_2030/cf-ts-air-sourced heat pump.pdf}
    \end{subfigure}
    \begin{subfigure}{0.49\textwidth}
        \centering
        \includegraphics[width=\textwidth]{20211218-181-h2/elec_s_181_lvopt__Co2L0-3H-T-H-B-I-A-solar+p3-linemaxext10_2030/cf-ts-ground-sourced heat pump.pdf}
    \end{subfigure}
    \begin{subfigure}{0.49\textwidth}
        \centering
        \includegraphics[width=\textwidth]{20211218-181-h2/elec_s_181_lvopt__Co2L0-3H-T-H-B-I-A-solar+p3-linemaxext10_2030/cf-ts-gas boiler.pdf}
    \end{subfigure}
    \begin{subfigure}{0.49\textwidth}
        \centering
        \includegraphics[width=\textwidth]{20211218-181-h2/elec_s_181_lvopt__Co2L0-3H-T-H-B-I-A-solar+p3-linemaxext10_2030/cf-ts-CHP.pdf}
    \end{subfigure}
    \begin{subfigure}{0.49\textwidth}
        \centering
        \includegraphics[width=\textwidth]{20211218-181-h2/elec_s_181_lvopt__Co2L0-3H-T-H-B-I-A-solar+p3-linemaxext10_2030/cf-ts-battery.pdf}
    \end{subfigure}
    \begin{subfigure}{0.49\textwidth}
        \centering
        \includegraphics[width=\textwidth]{20211218-181-h2/elec_s_181_lvopt__Co2L0-3H-T-H-B-I-A-solar+p3-linemaxext10_2030/cf-ts-hydrogen storage.pdf}
    \end{subfigure}
    \caption{Operations and storage filling levels of selected energy system components. The figure outlines
    the flexible operation of electrolysers (both seasonally and daily)
    the near-constant operation of synthetic fuel production
    the backup role of gas power plants (OCGT),
    the seasonal operation of heat pumps, gas boilers, CHP, and hydrogen storage,
    the daily pattern of battery storage filling levels, and
    periods of peak loading of the power distribution grid.
    }
    \label{fig:si:utilisation-rate-ts}
\end{figure}


\begin{figure}
    \centering
    \begin{subfigure}{0.49\textwidth}
        \centering
        \caption{electricity network loading}
        \includegraphics[width=\textwidth]{20211218-181-h2/elec_s_181_lvopt__Co2L0-3H-T-H-B-I-A-solar+p3-linemaxext10_2030/map_avg_lineloading.pdf}
    \end{subfigure}
    \begin{subfigure}{0.49\textwidth}
        \centering
        \caption{hydrogen network loading}
        \includegraphics[width=\textwidth]{20211218-181-h2/elec_s_181_lvopt__Co2L0-3H-T-H-B-I-A-solar+p3-linemaxext10_2030/map_avg_pipelineloading.pdf}
    \end{subfigure}
    \begin{subfigure}{0.65\textwidth}
        \vspace{1cm}
        \centering
        \caption{electricity grid congestion and curtailment}
        \includegraphics[width=\textwidth]{20211218-181-h2/elec_s_181_lvopt__Co2L0-3H-T-H-B-I-A-solar+p3-linemaxext10_2030/map_congestion.pdf}
    \end{subfigure}
    \caption{Utilisation rate of electricity and hydrogen network, curtailment and congestion. Subplot (a) shows average electricity network loading relative to $N-1$ compliant line rating (70\% of nominal rating) and the corresponding duration curve of line loadings. Subplot (b) shows the average hydrogen pipeline loading relative to the nominal pipeline capacity and also the corresponding duration curve of pipeline loadings. Subplot (c) shows the regional and technological distribution of curtailment in the system as well as realised congestion rents in the electricity network.}
    \label{fig:si:grid-utilisation}
\end{figure}
