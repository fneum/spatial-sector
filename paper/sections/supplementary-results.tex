\section{Grid Reinforcement and Onshore Wind Potential Restrictions}
\label{sec:si:sensitivity-lv-onw}

\subsection{Cost of Electricity Grid Reinforcement Restrictions}
\label{sec:si:lv}

\begin{figure}
    \centering
    \begin{subfigure}[t]{\textwidth}
        \centering
        \caption{Sensitivity of electricity transmission grid expansion limits}
        \includegraphics[width=\textwidth]{lv-sensitivity.pdf}
        \label{fig:lv-restriction}
    \end{subfigure}
    \begin{subfigure}[t]{\textwidth}
        \centering
        \caption{Sensitivity of onshore wind expansion limits}
        \includegraphics[width=\textwidth]{onw-sensitivity.pdf}
        \label{fig:onw-restriction}
    \end{subfigure}
    \caption{Sensitivity of electricity transmission grid expansion limits and onshore wind restrictions.}
    \label{fig:lv-onw-restriction}
\end{figure}

In the following sensitivity runs, the model is allowed to build new electricity
transmission infrastructure wherever is cost-optimal, but the total volume of
new transmission capacity (sum of line length times capacity, TWkm) is
successively limited. The volume limit is given in fractions of today's grid
volume: a line volume limit of 100\% means no new capacity is allowed beyond
today's grid (since the model cannot remove existing lines); a limit of 125\%
means the total grid capacity can grow by 25\% (25\% is similar to the planned
extra capacity in the European network operators' Ten Year Network Development
Plan (TYNDP) \cite{TYNDP2016}). For this investigation, a hydrogen network
could be built.

\cref{fig:lv-restriction} shows the composition of total yearly system costs
(including all investment and operational costs) as we vary the allowed grid
expansion, from no expansion (only today's grid) to a doubling of today's grid
capacities (the model optimises where new capacity is placed). As the grid is
expanded, total costs decrease only slightly, despite the increasing costs of
the grid. The total cost benefit of a doubling of grid capacity is around
\euro47~billion per year corresponding to an expansion of 644 TWkm. However, over
half of the benefit (\euro29~billion per year) is available already at a 25\%
expansion corresponding to an expansion of 403 TWkm.

\cref{fig:lv-restriction} also includes a scenario where today's electricity
transmission infrastructure is completely removed from the model. While doubling
the transmission grid yields a benefit of \euro47~billion per year, removing
what exists incurs a cost of \euro~86~billion per year. The lack of electricity
grid is mostly compensated by more solar PV generation, battery storage and
re-electrified hydrogen.

\subsection{Cost of Onshore Wind Potential Restrictions}
\label{sec:si:onw}

In the following sensitivity runs, the maximum installable capacity of onshore
wind is successively restricted down to zero at each node. The upper limit is
derived from land use restriction and yields a maximum technical potential
corresponding to about \SI{481}{\giga\watt} for Germany. For this investigation,
a compromise electricity grid expansion by 25\% compared to today and no limits
on hydrogen network infrastructure are assumed.

By restricting the installable potentials of onshore down to zero, costs rise by
\euro~103~billion per year (13\%). Just as in the case of restricted line volumes,
\cref{fig:onw-restriction} reveals a nonlinear rise in system costs: if we
constrain the model to 25\% of the onshore potential (around 120~GW for Germany),
costs rise by only \euro~64~billion per year (5\%). Thereby, 25\% of the onshore
wind potential may represent a social compromise between total system cost, and
social concerns about onshore wind development.

In comparison, Schlachtberger et al.~\cite{schlachtbergerCostOptimal2018} found
a smaller change between 9\% and 12\% in system costs in an electricity-only
model when onshore wind potentials were restricted across various grid expansion
limitations. The the biggest change was observed when the power grid could not
be reinforced. Onshore wind was largely replaced with offshore wind in that
model. Unlike that model, here we have a higher grid resolution (181 versus 30
regions) which allows us to better assess the grid integration costs of offshore
wind. Our results show that moderate power grid expansion is particularly
important when onshore wind development is severely limited. For the extreme
case where no onshore wind capacities would be built, reducing power grid
expansion from 25\% to none incurs another rise in system cost of an additional
\euro~50~billion per year (6\%).

\begin{figure}
    \centering
    \includegraphics[width=0.95\textwidth]{sensitivity-h2.pdf}
    \caption{Sensitivity of hydrogen network infrastructure.}
    \label{fig:h2-restriction-w-onw}
\end{figure}

The benefit of a hydrogen network is similar whether or not onshore wind
capacities are built in Europe, even though the hydrogen network topology is
then built around supply from offshore wind in the North Sea and solar PV from
Southern Europe rather than from onshore wind in North-West Europe. As
\cref{fig:h2-restriction-w-onw} illustrates, the net benefit is again strongest
when power grid expansion is restricted. If both onshore wind and power grid
expansion are excluded, costs for a system without a hydrogen network option
were by \euro~53~billion per year (5.6\%) higher. With cost-optimal electricity
grid reinforcement, the net benefit of a hydrogen network is still as high as
\euro~30~billion per year (3.5\%).

\section{Detailed Results of Least-Cost Solution with Gas Network}
\label{sec:si:detailed}

In the following section we describe detailed results from a scenario where the
carrier gas was nodally resolved and which included the methane network
alongside the hydrogen and electricity grid. In previously shown results, gas
was not nodally resolved and gas pipelines were only considered to determine the
potential for retrofitted hydrogen pipelines.

competition of H2 and gas flow? can gas network be completely removed?

no loss modelling in gas networks

\begin{figure}
    \centering
    % \makebox[\textwidth][c]{
    \begin{subfigure}[t]{\textwidth}
        \centering
        \caption{electricity}
        \includegraphics[width=\textwidth]{20211218-181-lv/elec_s_181_lv1.0__Co2L0-3H-T-H-B-I-A-solar+p3-linemaxext10_2030/ts-balance-total electricity-D-2013.pdf}
    \end{subfigure}
    \begin{subfigure}[t]{\textwidth}
        \centering
        \caption{heat}
        \includegraphics[width=\textwidth]{20211218-181-lv/elec_s_181_lv1.0__Co2L0-3H-T-H-B-I-A-solar+p3-linemaxext10_2030/ts-balance-total heat-D-2013.pdf}
    \end{subfigure}
    \begin{subfigure}[t]{\textwidth}
        \centering
        \caption{hydrogen}
        \includegraphics[width=\textwidth]{20211218-181-lv/elec_s_181_lv1.0__Co2L0-3H-T-H-B-I-A-solar+p3-linemaxext10_2030/ts-balance-H2-D-2013.pdf}
    \end{subfigure}
    %}
    \caption{Daily time series}
    \label{fig:output-ts-1}
\end{figure}

\begin{figure}
    \centering
    % \makebox[\textwidth][c]{
    \begin{subfigure}[t]{\textwidth}
        \centering
        \caption{methane}
        \includegraphics[width=\textwidth]{20211218-181-lv/elec_s_181_lv1.0__Co2L0-3H-T-H-B-I-A-solar+p3-linemaxext10_2030/ts-balance-gas-D-2013.pdf}
    \end{subfigure}
    \begin{subfigure}[t]{\textwidth}
        \centering
        \caption{oil}
        \includegraphics[width=\textwidth]{20211218-181-lv/elec_s_181_lv1.0__Co2L0-3H-T-H-B-I-A-solar+p3-linemaxext10_2030/ts-balance-oil-D-2013.pdf}
    \end{subfigure}
    \begin{subfigure}[t]{\textwidth}
        \centering
        \caption{stored \co}
        \includegraphics[width=\textwidth]{20211218-181-lv/elec_s_181_lv1.0__Co2L0-3H-T-H-B-I-A-solar+p3-linemaxext10_2030/ts-balance-co2 stored-D-2013.pdf}
    \end{subfigure}
    %}
    \caption{Daily time series}
    \label{fig:output-ts-2}
\end{figure}


\begin{figure}
    \centering
    % \makebox[\textwidth][c]{
    \begin{subfigure}[t]{\textwidth}
        \centering
        \caption{electricity}
        \includegraphics[width=\textwidth]{20211218-181-lv/elec_s_181_lv1.0__Co2L0-3H-T-H-B-I-A-solar+p3-linemaxext10_2030/ts-balance-total electricity--2013-02.pdf}
    \end{subfigure}
    \begin{subfigure}[t]{\textwidth}
        \centering
        \caption{heat}
        \includegraphics[width=\textwidth]{20211218-181-lv/elec_s_181_lv1.0__Co2L0-3H-T-H-B-I-A-solar+p3-linemaxext10_2030/ts-balance-total heat--2013-02.pdf}
    \end{subfigure}
    \begin{subfigure}[t]{\textwidth}
        \centering
        \caption{hydrogen}
        \includegraphics[width=\textwidth]{20211218-181-lv/elec_s_181_lv1.0__Co2L0-3H-T-H-B-I-A-solar+p3-linemaxext10_2030/ts-balance-H2--2013-02.pdf}
    \end{subfigure}
    %}
    \caption{hourly time series}
    \label{fig:output-ts-3}
\end{figure}

\begin{figure}
    \centering
    % \makebox[\textwidth][c]{
    \begin{subfigure}[t]{\textwidth}
        \centering
        \caption{methane}
        \includegraphics[width=\textwidth]{20211218-181-lv/elec_s_181_lv1.0__Co2L0-3H-T-H-B-I-A-solar+p3-linemaxext10_2030/ts-balance-gas--2013-02.pdf}
    \end{subfigure}
    \begin{subfigure}[t]{\textwidth}
        \centering
        \caption{oil}
        \includegraphics[width=\textwidth]{20211218-181-lv/elec_s_181_lv1.0__Co2L0-3H-T-H-B-I-A-solar+p3-linemaxext10_2030/ts-balance-oil--2013-02.pdf}
    \end{subfigure}
    \begin{subfigure}[t]{\textwidth}
        \centering
        \caption{stored \co}
        \includegraphics[width=\textwidth]{20211218-181-lv/elec_s_181_lv1.0__Co2L0-3H-T-H-B-I-A-solar+p3-linemaxext10_2030/ts-balance-co2 stored--2013-02.pdf}
    \end{subfigure}
    %}
    \caption{hourly time series}
    \label{fig:output-ts-4}
\end{figure}