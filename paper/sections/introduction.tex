There are many different combinations of technologies that would allow
the European Union to reach net-zero greenhouse gas emissions by
2050. However, not all technologies meet the same level of acceptance
among the public. The last few decades have seen public resistance to
new and existing nuclear power plants, projects with carbon capture
and sequestration (CCS), onshore wind power plants, and overhead
transmission lines. The lack of public acceptance can both delay the
deployment of a technology and stop its deployment altogether. This
may make it harder to reach greenhouse gas reduction targets in time,
or cause rising costs through the substitution with other
technologies.

In this contribution we examine the trade-off between costs and public
acceptance for onshore wind and overhead transmission lines in highly
renewable scenarios for Europe with net-zero carbon dioxide emissions
in the whole energy system (electricity, buildings, transport and
industry). For this purpose we use a capacity expansion model of the
European energy system, PyPSA-Eur-Sec, which, in contrast to previous
studies \cite{Henning20141003,mathiesen2014smart,IEESWV,Connolly20161634,en10101468,BLANCO2018617,BROWN2018720,in-depth_2018,victoria2020},
combines a fully sector-coupled approach with a high-resolution grid
model (181 nodes) so that it can capture the grid bottlenecks that
constrain the integration of renewable energy.  We successively
constrain the allowed onshore wind and new overhead grid projects down
to zero to examine the effects on total system cost and the
composition of technologies in the system.

We find that the overall system costs are not significantly affected
by restrictions on onshore wind and transmission. The high level of
synthetic fuel production and exchange between the nodes, which is
necessary for industry, transport and backup electricity and heating
applications, provides sufficient flexibility to manage these
restrictions. Moderate onshore wind and grid expansion provide cost
savings, but they are small (5-10~\%) compared to total system
costs. The restriction of onshore wind capacity results in a
substitution with offshore wind and solar PV; the restriction of grid
expansion leads to more local production from solar and more hydrogen
production. The limited rise in system costs relies on a new network
of underground hydrogen storage and pipelines in Europe, which help to balance
generation from renewables in time and space.