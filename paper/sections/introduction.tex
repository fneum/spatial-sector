\topic{public acceptance problems / importance of power grid}

There are many different combinations of infrastructure that would allow the
Europe to reach net-zero greenhouse gas emissions by 2050. However, not all
technologies meet the same level of acceptance among the public. The last few
decades have seen public resistance to new and existing nuclear power plants,
projects with carbon capture and sequestration (CCS), onshore wind power plants,
and overhead transmission lines. The lack of public acceptance can both delay
the deployment of a technology and stop its deployment altogether. This may make
it harder to reach greenhouse gas reduction targets in time, or cause rising
costs through the substitution with other technologies. In particular,
electricity transmission network expansion has suffered many delays in Europe in
recent decades, despite its importance for integrating large amounts of
renewable electricity into the energy system such that all energy sectors can be
decarbonized.

\topic{hydrogen: an important energy carrier}

In such a climate-neutral energy system, hydrogen will likely become a pivotal
energy carrier. Hydrogen is needed in the industry sector to produce ammonia for
fertilisers and direct reduced iron for steelmaking. It is also an important
feedstock to produce synthetic methane and liquid hydrocarbons for use as
aviation fuel and as precursor to high value chemicals. Further uses of hydrogen
extend to heavy-duty land transport, shipping as liquified hydrogen, and backup
heat and power supply through re-electrification in stationary fuel cells.

\topic{idea of European hydrogen backbone as replacement?}

The limited social acceptance for electricity grid reinforcement and the
advancing role of hydrogen, raise the question whether a new network of hydrogen
pipelines could offer a replacement for balancing variable renewable electricity
generation and moving energy across the continent. Such a vision for a
\textit{European Hydrogen Backbone (EHB)} has recently been expressed by
Europe's gas industry in a series of reports
\cite{gasforclimateEuropeanHydrogen2020,gasforclimateEuropeanHydrogen2021,gasforclimateExtendingEuropean2021,gasforclimateEuropeanHydrogen2022},
as it would offer an alternative route to connecting remote regions with
abundant and cost-effective hydrogen supply potentials to densely-populated and
industry-heavy regions with high demand but limited potentials.

\topic{large retrofitting potential of gas network / higher acceptance?}

Since Europe has a large existing gas transmission network that is set to become
increasingly expendable as the system transitions towards climate-neutrality,
the option to repurpose parts of the network to transport hydrogen instead may
add additional appeal to this idea. This is because retrofitting gas pipelines
would significantly lower the development costs compared to building new
hydrogen pipelines. Moreover, repurposed as well as new pipelines may
also meet higher levels of acceptance among the local populations than
transmission lines. Unlike transmission towers, pipelines are less visible
because they usually run below or close to the ground. Particularly where gas
pipelines already exist, the perceivable impact would be minimal.

\topic{shortcomings in literature/EHB}

However, few studies have evaluated the benefit of a hydrogen network in Europe
so far.
- no co-planning of infrastructures (EHB)
- no included (Brown, Pickering)
- country-level resolution (Bossmann)
- country-specific focus (Gils, Goeke)
- no non-energy demands for hydrogen (Gils, Caglayan)

\topic{in this paper}

In this paper, we examine the economic benefit of a hydrogen backbone in
scenarios for a European energy system with net-zero carbon dioxide emissions
and high shares of renewable electricity production. To investigate if a
hydrogen network can compensate for a potential lack of power grid expansion, we
analyse four main scenarios. These vary whether or not electricty and/or
hydrogen grids can be expanded. As supplementary sensitivity analysis, we also
evaluate the impact of restricted onshore wind potentials. Evaluation criteria
are the total system cost as well as the composition and spatial distribution of
technologies and transmission infrastructure in the system.

\topic{model description and novelty}

For our analysis, we use an open-source capacity expansion model of the European
energy system, PyPSA-Eur-Sec, which, in contrast to previous studies
\cite{henningComprehensiveModel2014,mathiesenSmartEnergy2015,connollySmartEnergy2016,lofflerDesigningModel2017,blancoPotentialHydrogen2018,brownSynergiesSector2018,in-depth_2018,victoria2020},
combines a fully sector-coupled approach with a high spatio-temporal resolution
so that it can capture the transmission bottlenecks that constrain the
cost-effective integration of variable renewable energy.
% co-optimisation
The model co-optimises the investment and operation of generation, storage,
conversion and transmission infrastructures in a single linear optimisation
problem, while covering 181 regions and a 3-hourly time resolution for a full
year.
% demands
It incorporates all the demands of the electricity, industry, buildings, and
transport sectors, including shipping and aviation as well as non-energy
feedstock demands in the chemicals industry.
% generation
Primary energy supply comes from onshore and offshore wind, rooftop and
utility-scale solar PV, residual biomass, hydro and run-of-river plants as well
as limited amounts of fossil oil and gas.
% conversion
The energy flows between the system's energy carriers are modelled by a range of
technologies and storage options, including air- and ground-sourced heat pumps,
gas boilers, CHPs, resistive heaters, long- and short-term thermal energy
storage, electric vehicles (incl.~vehicle-to-grid), batteries, electrolysers,
Fischer-Tropsch and methanation plants, stationary fuel cells, and hydrogen
storage in caverns or steel tanks.
% transmission
Moreover, the model incorporates data on existing electricity and gas transmission
network infrastructure in Europe to estimate grid expansion needs and retrofitting
potentials. Power flows are modelled with linearised load flow assumptions.
% carbon management
Finally, the model also features a detailed management of carbon flows between
capture, usage and sequestration.

% challenges to the model
% compared to electricity, heating and transport are strongly peaked
% - heating is strongly seasonal but also with synoptic variations
% - transport has strong daily periodicity
% There are difficult periods in winter with
% - low wind and solar (high prices)
% - high space heating demand
% - low air temperatures, which are bad for air-sourced heat pump performance
% - less smart solution: backup gas boilers
% - smart solution: building retrofiting, TES in district heating, CHP

\topic{scenario assumptions}

All investigations are carried out with a constraint that carbon dioxide
emissions balance out to zero over the year. The model is allowed to sequester
up to 200 Mt\co per year, which allows it to sequester industry process
emissions, such as calcination in cement manufacture, that have a fossil origin.
% Higher sequestration rates would be favoured in the model. For comparison, the
% JRC-EU-TIMES model sees optimal sequestration levels up to 1.4 Gt\co per year
% for a 95\% reduction of \co emissions in the EU compared to 1990 levels
% \cite{blancoPotentialHydrogen2018}.
Moreover, in our scenarios we do not consider energy imports into Europe while
applying technology cost assumptions for the year 2030, taken predominantly from
the Danish Energy Agency \cite{}.

% general:
% \cite{
%     mckennaScenicnessAssessment2021,
%     krummModellingSocial2022,
%     weinandImpactPublic2021,
%     weinandExploringTrilemma,
%     trondleTradeOffsGeographic2020,
%     sasseDistributionalTradeoffs2019,
%     sasseRegionalImpacts2020,
%     ludererImpactDeclining2021,
%     victoria2020,
%     victoriaSpeedTechnological2021,
%     lombardiPolicyDecision2020,
%     tsiropoulosNetzeroEmissions2020,
%     europeancommission.directorategeneralforenergy.METISStudy2021,
%     tafarteQuantifyingTrade,
%     lehmannManagingSpatial}

% PV cost
% \cite{
%     jaxa-rozenSourcesUncertainty2021,
%     victoriaSolarPhotovoltaics2021,
%     xiaoPlummetingCosts2021}