% public acceptance problems

There are many different combinations of technologies that would allow
the European Union to reach net-zero greenhouse gas emissions by
2050. However, not all technologies meet the same level of acceptance
among the public. The last few decades have seen public resistance to
new and existing nuclear power plants, projects with carbon capture
and sequestration (CCS), onshore wind power plants, and overhead
transmission lines. The lack of public acceptance can both delay the
deployment of a technology and stop its deployment altogether. This
may make it harder to reach greenhouse gas reduction targets in time,
or cause rising costs through the substitution with other
technologies.

2050 scenarios for EU: power demand doubles, mostly met by VRE

hydrogen is an important energy carrier
- store electricity
- used in industry
- feedstock for synthetic fuels

This collides with low acceptance for power grid expansion and onshore wind

Offshore wind can certainly help, but still needed to transport the electricity to loads inland

trade-offs with unpopular infrastructure

Electricity transmission network expansion has suffered many delays in
Europe in recent decades, despite its importance for integrating
renewable electricity into the energy system.

But could a hydrogen
network that reuses the existing fossil gas network offer a replacement
for balancing variations in wind and solar energy across the continent?

Can we substitute for the electricity grid by producing eleoctrolytic hydrogen
and transport it through new or repurposed hydrogen pipeline network?

% Idea of European hydrogen backbone

- European hydrogen backbone report
- hydrogen network largely based on repurposed natural gas infrastructure
- substition options of infrastructure (electricity and hydrogen)?
- is h2 grid necessary / useful

% in this paper / model description

We examine interplay of electricity and hydrogen transmission grids in the high
resolution (181 nodes) all-energy-sector open-source model of the European
energy system, PyPSA-Eur-Sec.

In this contribution we examine the trade-off between costs and public
acceptance for onshore wind and overhead transmission lines in highly
renewable scenarios for Europe with net-zero carbon dioxide emissions
in the whole energy system (electricity, buildings, transport and
industry).

For this purpose we use a capacity expansion model of the European energy
system, PyPSA-Eur-Sec, which, in contrast to previous studies
\cite{henningComprehensiveModel2014,mathiesenSmartEnergy2015,IEESWV,connollySmartEnergy2016,lofflerDesigningModel2017,blancoPotentialHydrogen2018,brownSynergiesSector2018,in-depth_2018,victoria2020},
combines a fully sector-coupled approach with a high spatio-temporal resolution
so that it can capture the grid bottlenecks that constrain the integration of
variable renewable energy.

We constrain the expansion of electricity and/or hydrogen grid to examine the
effects on total system cost and the composition of technologies in the system.

The model is detailed enough to capture existing
bottlenecks and long time series, and has been highly optimised to run
sector-coupled capacity expansion optimisations.

Scenarios are run with cost
assumptions for the year 2030 assuming zero net greenhouse gas emissions, while
varying the allowed expansion of the electricity and hydrogen grids
respectively.

A hydrogen network can take over and exceed the electricity grid
in terms of the amount of energy transported over long distances, balancing
renewables both in space (with the network) and time (with underground storage).

Need to co-optimise balancing solutions with generation. Optimising separately won't work.

PyPSA-Eur-Sec represents all energy flows and bottlenecks in energy networks

couples all energy sectors (power, heat, transport, industry)

% results

The presence of the hydrogen network can reduce system costs by up to 10\%.

We find that the overall system costs are not overly affected
by restrictions on electricity or hydrogen transmission.

The high level of synthetic fuel production and exchange between the nodes, which is
necessary for industry, transport and backup electricity and heating
applications, provides sufficient flexibility to manage these
restrictions.

Moderate ... provide cost
savings, but they are small (5-10~\%) compared to total system
costs.

The restriction of grid
expansion leads to more local production from solar and more hydrogen
production.

The limited rise in system costs relies on a new network
of underground hydrogen storage and pipelines in Europe, which help to balance
generation from renewables in time and space.

H2 and power: are not complements but their cost reductions are largely additive!

All investigations are carried out with a constraint that carbon dioxide
emissions balance out to zero over the year. The model is allowed to sequester
up to 200 Mt\co per year, which allows it to sequester industry process
emissions, such as calcination in cement manufacture, that have a fossil origin.
Higher sequestration rates would be favoured in the model. For comparison, the
JRC-EU-TIMES model sees optimal sequestration levels up to 1.4 Gt\co per year
for a 95\% reduction of \co emissions in the EU compared to 1990 levels
\cite{blancoPotentialHydrogen2018}.

In these scenarios we do not allow the import of electricity or synthetic fuels
(hydrogen, methane and liquid hydrocarbons) from outside Europe. Imports from
regions with good renewable resources should be considered in future work
\cite{fasihiTechnoeconomicAssessment2019,heuserTechnoeconomicAnalysis2019}.

% challenges in high-RES energy system

compared to electricity, heating and transport are strongly peaked
- heating is strongly seasonal but also with synoptic variations
- transport has strong daily periodicity
There are difficult periods in winter with
- low wind and solar (high prices)
- high space heating demand
- low air temperatures, which are bad for air-sourced heat pump performance
- less smart solution: backup gas boilers
- smart solution: building retrofiting, TES in district heating, CHP


general:
\cite{
    mckennaScenicnessAssessment2021,
    krummModellingSocial2022,
    weinandImpactPublic2021,
    weinandExploringTrilemma,
    trondleTradeOffsGeographic2020,
    sasseDistributionalTradeoffs2019,
    sasseRegionalImpacts2020,
    ludererImpactDeclining2021,
    EuropeanHydrogen,
    victoria2020,
    victoriaSpeedTechnological2021,
    lombardiPolicyDecision2020,
    tsiropoulosNetzeroEmissions2020,
    europeancommission.directorategeneralforenergy.METISStudy2021,
    deutschNoRegretHydrogen,
    tafarteQuantifyingTrade,
    lehmannManagingSpatial}

PV cost
\cite{
    jaxa-rozenSourcesUncertainty2021,
    victoriaSolarPhotovoltaics2021,
    xiaoPlummetingCosts2021}