\topic{public acceptance problems / importance of power grid}

There are many different combinations of infrastructure that would allow Europe
to reach net-zero greenhouse gas emissions by 2050. However, not all
technologies meet the same level of acceptance among the public. The last few
decades have seen public resistance to new and existing nuclear power plants,
projects with carbon capture and sequestration (CCS), onshore wind power plants,
and overhead transmission lines. The lack of public acceptance can both delay
the deployment of a technology and even stop its deployment altogether. This may
make it harder to reach greenhouse gas reduction targets in time or cause rising
costs through the substitution with other technologies. In particular,
electricity transmission network expansion has suffered many delays in Europe in
recent decades, despite its importance for integrating large amounts of
renewable electricity such that all energy sectors can be decarbonised.

\topic{hydrogen: an important energy carrier}

Hydrogen will likely become a pivotal energy carrier in such a climate-neutral
energy system. Hydrogen is needed in the industry to produce ammonia for
fertilisers and direct reduced iron for steelmaking. It is also a critical
feedstock to produce synthetic methane and liquid hydrocarbons for use as
aviation fuel and as a precursor to high-value chemicals. Further hydrogen use
extends to heavy-duty land transport, shipping as liquified hydrogen, and backup
heat and power supply through re-electrification in stationary fuel cells.

\topic{idea of European hydrogen backbone as replacement?}

The limited social acceptance for electricity grid reinforcement and the
advancing role of hydrogen raises the question of whether a new network of
hydrogen pipelines could offer a replacement for balancing variable renewable
electricity generation and moving energy across the continent. Such a vision for
a \textit{European Hydrogen Backbone (EHB)} has recently been expressed by
Europe's gas industry in a series of reports
\cite{gasforclimateEuropeanHydrogen2020,gasforclimateEuropeanHydrogen2021,gasforclimateExtendingEuropean2021,gasforclimateEuropeanHydrogen2022}.
It would offer an alternative route to connecting remote regions with abundant
and cost-effective wind and solar potentials to densely-populated and
industry-heavy regions with high demand but limited supply options.

\topic{large retrofitting potential of gas network / higher acceptance?}

Since Europe has a sizeable existing gas transmission network that is set to
become increasingly expendable as the system transitions towards climate
neutrality, the option to repurpose parts of the network, to transport hydrogen
instead, may reinforce this idea. This is because retrofitting gas pipelines
would significantly lower the development costs compared to building new
hydrogen pipelines. Moreover, repurposed and new pipelines may also meet higher
levels of acceptance among the local populations than transmission lines. Unlike
transmission towers, pipelines are less visible because they usually run below
or near the ground. Particularly where gas pipelines already exist, the
perceivable impact would be minimal.

\topic{shortcomings in literature/EHB}

However, few studies have evaluated the benefit of a hydrogen network in Europe
so far. The EHB reports do not include an assessment based on the
co-optimisation of energy system components
\cite{gasforclimateEuropeanHydrogen2020,gasforclimateEuropeanHydrogen2021,gasforclimateExtendingEuropean2021,gasforclimateEuropeanHydrogen2022}.
Other sector-coupling studies have not included hydrogen networks at all
\cite{brownSynergiesSector2018}, or when they do, model Europe only at
country-level resolution
\cite{europeancommission.directorategeneralforenergy.METISStudy2021,victoriaSpeedTechnological2021},
have a country-specific focus with limited geographical scope
\cite{gilsInteractionHydrogen2021}, or neglect some energy sectors or non-energy
demands that involve hydrogen
\cite{gilsInteractionHydrogen2021,Caglayan2019,caglayanRobustDesign2021}. None
of the studies have explored the interplay between hydrogen network expansion
and electricity grid reinforcements. Neither have the potentials for lower
development costs through pipeline retrofitting been taken into account so far.


\topic{in this paper}

This paper examines the economic benefit of a hydrogen backbone in scenarios for
a European energy system with net-zero carbon dioxide emissions and high shares
of renewable electricity production. We analyse four main scenarios to
investigate if a hydrogen network can compensate for a potential lack of power
grid expansion. These vary on whether or not electricity and hydrogen grids can
be expanded, including potentials for gas pipeline retrofitting. Evaluation
criteria are the total system cost, the composition and spatial distribution of
technologies and transmission infrastructure in the system.

% As a supplementary sensitivity analysis, we also evaluate the
% impact of restricted onshore wind potentials.

\topic{mini model description and novelty}

For our analysis, we use an open-source capacity expansion model of the European
energy system, PyPSA-Eur-Sec, which, in contrast to previous studies
\cite{henningComprehensiveModel2014,mathiesenSmartEnergy2015,connollySmartEnergy2016,lofflerDesigningModel2017,blancoPotentialHydrogen2018,brownSynergiesSector2018,in-depth_2018,victoria2020},
combines a fully sector-coupled approach with a high spatio-temporal resolution
so that it can capture the transmission bottlenecks that constrain the
cost-effective integration of variable renewable energy. The model co-optimises
the investment and operation of generation, storage, conversion and transmission
infrastructures in a single linear optimisation problem, covering 181 regions
and a 3-hourly time resolution for a full year. It incorporates spatially
distributed demands of the electricity, industry, buildings, agriculture and
transport sectors, including shipping and aviation as well as non-energy
feedstock demands in the chemicals industry. Primary energy supply comes from
wind, solar, biomass, hydro, and limited amounts of fossil oil and gas. The
energy flows between the system's energy carriers are modelled by various
technologies, including heat pumps, CHPs, thermal storage, electric vehicles,
batteries, power-to-X processes, fuel cells, and geological potentials of
underground hydrogen storage. Moreover, data on electricity and gas transmission
infrastructure is included to determine grid expansion needs and retrofitting
potentials. Finally, the model also features detailed management of carbon flows
between capture, usage, sequestration and the atmosphere. More details on the
model are presented in \nameref{sec:methods} and \nameref{sec:si}.

% challenges to the model compared to electricity, heating and transport are
% strongly peaked
% - heating is strongly seasonal but also with synoptic variations
% - transport has strong daily periodicity There are difficult periods in winter
%   with
% - low wind and solar (high prices)
% - high space heating demand
% - low air temperatures, which are bad for air-sourced heat pump performance
% - less smart solution: backup gas boilers
% - smart solution: building retrofitting, TES in district heating, CHP

\topic{scenario assumptions}

All investigations are conducted with a constraint that carbon dioxide emissions
into the atmosphere balance out to zero over the year. The model can sequester
up to 200 Mt\co per year, allowing it to sequester industry process emissions,
such as calcination in cement manufacture, that have a fossil origin but limits
the use of negative emission technologies compared to other works
\cite{blancoPotentialHydrogen2018}. In our scenarios, we do not
consider energy imports into Europe and apply technology assumptions for
the year 2030 \cite{DEA}.

% Higher sequestration rates would be
% favoured in the model. For comparison, the JRC-EU-TIMES model sees optimal
% sequestration levels up to 1.4 Gt\co per year for a 95\% reduction of \co
% emissions in the EU compared to 1990 levels .

% general:
% \cite{
%     mckennaScenicnessAssessment2021,
%     krummModellingSocial2022,
%     weinandImpactPublic2021,
%     weinandExploringTrilemma,
%     trondleTradeOffsGeographic2020,
%     sasseDistributionalTradeoffs2019,
%     sasseRegionalImpacts2020,
%     ludererImpactDeclining2021,
%     victoria2020,
%     victoriaSpeedTechnological2021,
%     lombardiPolicyDecision2020,
%     tsiropoulosNetzeroEmissions2020,
%     europeancommission.directorategeneralforenergy.METISStudy2021,
%     tafarteQuantifyingTrade,
%     lehmannManagingSpatial}

% PV cost
% \cite{
%     jaxa-rozenSourcesUncertainty2021,
%     victoriaSolarPhotovoltaics2021,
%     xiaoPlummetingCosts2021}