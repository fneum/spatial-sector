% public acceptance problems

There are many different combinations of infrastructure that would allow
the European Union to reach net-zero greenhouse gas emissions by
2050. However, not all technologies meet the same level of acceptance
among the public. The last few decades have seen public resistance to
new and existing nuclear power plants, projects with carbon capture
and sequestration (CCS), onshore wind power plants, and overhead
transmission lines. The lack of public acceptance can both delay the
deployment of a technology and stop its deployment altogether. This
may make it harder to reach greenhouse gas reduction targets in time,
or cause rising costs through the substitution with other
technologies.

% importance of electricity transmission and lack of acceptance

In particular, electricity transmission network expansion has suffered many
delays in Europe in recent decades, despite its importance for integrating
renewable electricity into the energy system.

2050 scenarios for EU: power demand doubles, mostly met by VRE

This collides with low acceptance for power grid expansion and onshore wind

Offshore wind can certainly help, but still needed to transport the electricity to loads inland

% hydrogen important energy carrier, hydrogen network as replacement?

At the same time, hydrogen is likely to become an important energy carrier:
- used in industry
- feedstock for synthetic fuels
- store electricity

But could a hydrogen
network that reuses the existing fossil gas network offer a replacement
for balancing variations in wind and solar energy across the continent?

Can we substitute for the electricity grid by producing eleoctrolytic hydrogen
and transport it through new or repurposed hydrogen pipeline network?

% idea of European hydrogen backbone, retrofitting potential, higher acceptance?

European hydrogen backbone report
- connect regions with abundant and cost-effective hydrogen supply
- regional differences lend impetus to a hydrogen backbone
- \cite{gasforclimateEuropeanHydrogen2020,gasforclimateEuropeanHydrogen2021,gasforclimateExtendingEuropean2021,gasforclimateEuropeanHydrogen2022}
- hydrogen network largely based on repurposed natural gas infrastructure

Retrofitting potential
- Europe has large existing gas transmission infrastructure
- current role: distribute natural gas imports from Russia, North Africa, and LNG terminals for use in electricity production, heat supply, and industry.
- gradually decreasing role for methane as energy system transitions towards carbon-neutrality
- makes it possible and attractive to repurpose pipelines to transport hydrogen instead
- achieved by replacing certain components like compressors
- the costs for that are estimated around half of new hydrogen pipeline

Higher acceptance for (retrofitted) hydrogen infastructure?
- less visible: underground or close to ground compared to soaring transmission towers
- reuse rather than new: retrofit infrastructure where it already exists

% in this paper

We examine interplay of electricity and hydrogen transmission grids in the high
resolution (181 nodes) all-energy-sector open-source model of the European
energy system, PyPSA-Eur-Sec.

whether is h2 grid necessary / useful in highly
renewable scenarios for Europe with net-zero carbon dioxide emissions
in the whole energy system (electricity, buildings, transport and
industry).

We constrain the expansion of electricity and/or hydrogen grid to examine the
effects on total system cost and the composition of technologies in the system.

while
varying the allowed expansion of the electricity and hydrogen grids
respectively.

We also run sensitivity of restricted onshore wind potentials

% model description and novelty

For this purpose we use a capacity expansion model of the European energy
system, PyPSA-Eur-Sec, which, in contrast to previous studies
\cite{henningComprehensiveModel2014,mathiesenSmartEnergy2015,IEESWV,connollySmartEnergy2016,lofflerDesigningModel2017,blancoPotentialHydrogen2018,brownSynergiesSector2018,in-depth_2018,victoria2020},
combines a fully sector-coupled approach with a high spatio-temporal resolution
so that it can capture the grid bottlenecks that constrain the integration of
variable renewable energy.

The model is detailed enough to capture existing
bottlenecks and long time series, and has been highly optimised to run
sector-coupled capacity expansion scenarios that represent all energy flows.

Scenarios are run with cost
assumptions for the year 2030 assuming zero net greenhouse gas emissions,

Need to co-optimise balancing solutions with generation. Optimising separately won't work.


Features:
- couples all energy sectors (power, heat, transport, industry)
- gas network data to estimate retrofitting potentials
- potential for hydrogen storage in salt caverns
- renewable potentials for wind, solar and biomass
- energy as well as non-energy demands
- detailed carbon management of sources, capture, usage and sequestration (last for transition)

% challenges to the model

compared to electricity, heating and transport are strongly peaked
- heating is strongly seasonal but also with synoptic variations
- transport has strong daily periodicity
There are difficult periods in winter with
- low wind and solar (high prices)
- high space heating demand
- low air temperatures, which are bad for air-sourced heat pump performance
- less smart solution: backup gas boilers
- smart solution: building retrofiting, TES in district heating, CHP

% scenario assumptions

All investigations are carried out with a constraint that carbon dioxide
emissions balance out to zero over the year. The model is allowed to sequester
up to 200 Mt\co per year, which allows it to sequester industry process
emissions, such as calcination in cement manufacture, that have a fossil origin.
Higher sequestration rates would be favoured in the model. For comparison, the
JRC-EU-TIMES model sees optimal sequestration levels up to 1.4 Gt\co per year
for a 95\% reduction of \co emissions in the EU compared to 1990 levels
\cite{blancoPotentialHydrogen2018}.

In these scenarios we do not allow the import of electricity or synthetic fuels
(hydrogen, methane and liquid hydrocarbons) from outside Europe. Imports from
regions with good renewable resources should be considered in future work
\cite{fasihiTechnoeconomicAssessment2019,heuserTechnoeconomicAnalysis2019}.

general:
\cite{
    mckennaScenicnessAssessment2021,
    krummModellingSocial2022,
    weinandImpactPublic2021,
    weinandExploringTrilemma,
    trondleTradeOffsGeographic2020,
    sasseDistributionalTradeoffs2019,
    sasseRegionalImpacts2020,
    ludererImpactDeclining2021,
    EuropeanHydrogen,
    victoria2020,
    victoriaSpeedTechnological2021,
    lombardiPolicyDecision2020,
    tsiropoulosNetzeroEmissions2020,
    europeancommission.directorategeneralforenergy.METISStudy2021,
    deutschNoRegretHydrogen,
    tafarteQuantifyingTrade,
    lehmannManagingSpatial}

PV cost
\cite{
    jaxa-rozenSourcesUncertainty2021,
    victoriaSolarPhotovoltaics2021,
    xiaoPlummetingCosts2021}