There are many different combinations of technologies that would allow
the European Union to reach net-zero greenhouse gas emissions by
2050. However, not all technologies meet the same level of acceptance
among the public. The last few decades have seen public resistance to
new and existing nuclear power plants, projects with carbon capture
and sequestration (CCS), onshore wind power plants, and overhead
transmission lines. The lack of public acceptance can both delay the
deployment of a technology and stop its deployment altogether. This
may make it harder to reach greenhouse gas reduction targets in time,
or cause rising costs through the substitution with other
technologies.

2050 scenarios for EU: power demand doubles, mostly met by VRE

This collides with low acceptance for power grid expansion and onshore wind

Offshore wind can certainly help, but still needed to transport the electricity to loads inland

- trade-offs with unpopular infrastructure
- triangle: transmission, h2 grid, onshore wind
- h2 grid vs. transmission grid
	- contrast H2 network w/wo grid expansion
	- contrast no grid expansion w/wo H2 network
- framing: is h2 grid necessary / useful
- framing: question of social potentials not cost
- refer to European hydrogen backbone report
- focus on substition options of infrastructure (electricity and hydrogen) with onshore wind as important sensitivity

- examine benefits of hydrogen network in high-VRE net-zero scenarios for Europe
- use PyPSA-Eur-Sec 181-region model with all energy sectors represented
- hydrogen network reduces total systems costs by 2.4-4.8\%
- X\% uses retrofitted gas network pipelines
- benefits highest when grid expansion restricted and onshore potentials limited
- hydrogen network reduces cost of system configurations with higher public acceptance

interplay of electricity and hydrogen transmission grids

Electricity transmission network expansion has suffered many delays in
Europe in recent decades, despite its importance for integrating
renewable electricity into the energy system. But could a hydrogen
network that reuses the existing fossil gas network offer a replacement
for balancing variations in wind and solar energy across the continent?

general:
\cite{
    mckennaScenicnessAssessment2021,
    krummModellingSocial2022,
    weinandImpactPublic2021,
    weinandExploringTrilemma,
    trondleTradeOffsGeographic2020,
    sasseDistributionalTradeoffs2019,
    sasseRegionalImpacts2020,
    ludererImpactDeclining2021,
    EuropeanHydrogen,
    victoria2020,
    victoriaSpeedTechnological2021,
    lombardiPolicyDecision2020,
    tsiropoulosNetzeroEmissions2020,
    europeancommission.directorategeneralforenergy.METISStudy2021,
    deutschNoRegretHydrogen,
    tafarteQuantifyingTrade,
    lehmannManagingSpatial}

PV cost
\cite{jaxa-rozenSourcesUncertainty2021,victoriaSolarPhotovoltaics2021,xiaoPlummetingCosts2021}


We examine these questions in the high resolution (181 nodes)
all-energy-sector open-source model of the European energy system,
PyPSA-Eur-Sec. The model is detailed enough to capture existing
bottlenecks and long time series, and has been highly optimised to run
sector-coupled capacity-expansion simulations. Scenarios are run for the
year 2050 assuming zero net greenhouse gas emissions, while varying the
allowed expansion of the electricity and hydrogen grids respectively. A
hydrogen network can take over and exceed the electricity grid in terms
of the amount of energy transported over long distances, balancing
renewables both in space (with the network) and time (with underground
storage). The presence of the hydrogen network can reduce system costs
by up to 10%. It is planned to also examine the impact of assumptions
about domestic versus imported hydrogen-based chemicals like ammonia and
methanol on the different grid topologies.

In this contribution we examine the trade-off between costs and public
acceptance for onshore wind and overhead transmission lines in highly
renewable scenarios for Europe with net-zero carbon dioxide emissions
in the whole energy system (electricity, buildings, transport and
industry). For this purpose we use a capacity expansion model of the
European energy system, PyPSA-Eur-Sec, which, in contrast to previous
studies \cite{henningComprehensiveModel2014,mathiesenSmartEnergy2015,IEESWV,connollySmartEnergy2016,lofflerDesigningModel2017,blancoPotentialHydrogen2018,brownSynergiesSector2018,in-depth_2018,victoria2020},
combines a fully sector-coupled approach with a high-resolution grid
model (181 regions) so that it can capture the grid bottlenecks that
constrain the integration of renewable energy.  We successively
constrain the allowed onshore wind and new overhead grid projects down
to zero to examine the effects on total system cost and the
composition of technologies in the system.

We find that the overall system costs are not overly affected
by restrictions on onshore wind and transmission. The high level of
synthetic fuel production and exchange between the nodes, which is
necessary for industry, transport and backup electricity and heating
applications, provides sufficient flexibility to manage these
restrictions. Moderate onshore wind and grid expansion provide cost
savings, but they are small (5-10~\%) compared to total system
costs. The restriction of onshore wind capacity results in a
substitution with offshore wind and solar PV; the restriction of grid
expansion leads to more local production from solar and more hydrogen
production. The limited rise in system costs relies on a new network
of underground hydrogen storage and pipelines in Europe, which help to balance
generation from renewables in time and space.

Need to co-optimise balancing solutions with generation. Optimising separately won't work.

PyPSA-Eur-Sec represents all energy flows and bottlenecks in energy networks

couple all energy sectors (power, heat, transport, industry)

To explore the effects of acceptance, we carry out three principal investigations
on the model:

\begin{enumerate}
\item \nameref{sec:lv}: The model is allowed to build new electricity
transmission infrastructure wherever is cost-optimal, but the total volume of new
transmission capacity (sum of line length times capacity, TWkm) is limited.

\item \nameref{sec:onw}: Each node has a maximum installable capacity of onshore
wind, based on land use restrictions. This gives a maximum social-technological
potential, corresponding to about \SI{480}{\giga\watt} for Germany. This potential is
successively restricted down to zero at each node. For this investigation, a
compromise electricity grid expansion by 25\% compared to today is assumed.

\item \nameref{sec:h2}:
\end{enumerate}

All investigations are carried out with a constraint that carbon dioxide
emissions balance out to zero over the year (see for
the possible paths carbon can take in the model). The model is allowed to
sequester up to 200 Mt\co per year, which allows it to sequester industry
process emissions, such as calcination in cement manufacture, that have a fossil
origin. Higher sequestration rates would be favoured in the model (the
JRC-EU-TIMES model sees optimal sequestration levels up to 1.4 Gt\co per year
for a 95\% reduction of \co emissions in the EU compared to 1990 levels
\cite{blancoPotentialHydrogen2018}).

In these scenarios we do not allow the import of electricity or synthetic fuels
(hydrogen, methane and liquid hydrocarbons) from outside Europe. Imports from
regions with good renewable resources should be considered in future work
\cite{fasihiTechnoeconomicAssessment2019,heuserTechnoeconomicAnalysis2019}.

Can a hydrogen network replace electricity transmission network expansion in a climate-neutral scenario for Europe?

Can we substitute for the electricity grid by producing eleoctrolytic hydrogen
and transport it through new or repurposed hydrogen pipeline network?

There are difficult periods in winter with
- low wind and solar (high prices)
- high space heating demand
- low air temperatures, which are bad for air-sourced heat pump performance
- less smart solution: backup gas boilers
- smart solution: building retrofiting, TES in district heating, CHP

H2 and power: are not complements but their cost reductions are largely additive!



\newgeometry{top=0.5cm, bottom=1.5cm}
\begin{figure}
    \centering
    \begin{subfigure}[t]{0.49\textwidth}
        \centering
        % \caption{clustered electricity network}
        \includegraphics[width=\textwidth]{electricity-network-today-map.pdf}
    \end{subfigure}
    \begin{subfigure}[t]{0.49\textwidth}
        \centering
        % \caption{gas network}
        \includegraphics[width=\textwidth]{gas-network-today-map.pdf}
    \end{subfigure}
    \begin{subfigure}[t]{0.49\textwidth}
        \centering
        % \caption{electricity demand}
        \includegraphics[width=\textwidth]{demand-map-electricity.pdf}
    \end{subfigure}
    \begin{subfigure}[t]{0.49\textwidth}
        \centering
        % \caption{hydrogen demand}
        \includegraphics[width=\textwidth]{demand-map-H2.pdf}
    \end{subfigure}
    \begin{subfigure}[t]{0.49\textwidth}
        \centering
        % \caption{methane demand}
        \includegraphics[width=\textwidth]{demand-map-gas.pdf}
    \end{subfigure}
    \begin{subfigure}[t]{0.49\textwidth}
        \centering
        % \caption{heat demand}
        \includegraphics[width=\textwidth]{demand-map-heat.pdf}
    \end{subfigure}
    \begin{subfigure}[t]{\textwidth}
        \centering
        \includegraphics[height=0.18\textheight]{total-annual-demand.pdf}
        \includegraphics[height=0.18\textheight]{ts-demand.pdf}
    \end{subfigure}
    \caption{Final energy demand diversity.}
    \label{fig:demand-space}
\end{figure}
\restoregeometry