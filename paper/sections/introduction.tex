There are many different combinations of infrastructure that would allow Europe
to reach net-zero greenhouse gas emissions by 2050. However, not all
technologies meet the same level of acceptance among the public. The last few
decades have seen public resistance to new and existing nuclear power plants,
projects with carbon capture and sequestration (CCS), onshore wind power plants,
and overhead transmission lines. The lack of public acceptance can both delay
the deployment of a technology and even stop its deployment altogether. This may
make it harder to reach greenhouse gas reduction targets in time or cause rising
costs through the substitution with other technologies. In particular,
electricity transmission network expansion has suffered many delays in Europe in
recent decades, despite its importance for integrating large amounts of
renewable electricity such that all energy sectors can be decarbonised.

Hydrogen will likely become a pivotal energy carrier in such a climate-neutral
energy system. Hydrogen is needed in the industry to produce ammonia for
fertilisers and can be used for direct reduced iron for steelmaking. It is also
a critical feedstock to produce synthetic methane and liquid hydrocarbons for
use as aviation fuel and as a precursor to high-value chemicals. Hydrogen could
also be used for heavy-duty land transport, shipping, and backup heat
and power supply.

The limited social acceptance for electricity grid reinforcement and the
advancing role of hydrogen raises the question of whether a new network of
hydrogen pipelines could offer a replacement for balancing variable renewable
electricity generation and moving energy across the continent. Such a vision for
a \textit{European Hydrogen Backbone (EHB)} has recently been expressed by
Europe's gas industry in a series of reports
\cite{gasforclimateEuropeanHydrogen2020,gasforclimateEuropeanHydrogen2021,gasforclimateExtendingEuropean2021,gasforclimateEuropeanHydrogen2022}.
It would offer an alternative route to connecting remote regions with abundant
and cost-effective wind and solar potentials to densely-populated and
industry-heavy regions with high demand but limited supply options.

Since Europe has a sizeable existing gas transmission network that is set to
become increasingly redundant as the system transitions towards climate
neutrality, the option to repurpose parts of the network to transport hydrogen
instead, may make hydrogen networks even more attractive. This is because
retrofitting gas pipelines would significantly lower the development costs
compared to building new hydrogen pipelines. Moreover, repurposed and new
pipelines may also meet higher levels of acceptance among the local populations
than transmission lines. Unlike transmission towers, pipelines are less visible
because they usually run below or near the ground. Particularly where gas
pipelines already exist, the perceivable impact would be minimal.

However, few studies have evaluated the benefit of a hydrogen network in Europe
so far. The EHB reports do not include an assessment based on the
co-optimisation of energy system components
\cite{gasforclimateEuropeanHydrogen2020,gasforclimateEuropeanHydrogen2021,gasforclimateExtendingEuropean2021,gasforclimateEuropeanHydrogen2022}.
Other sector-coupling studies have not included hydrogen networks at all
\cite{brownSynergiesSector2018,pickeringDiversityOptions}, or when they do,
model Europe only at country-level resolution
\cite{europeancommission.directorategeneralforenergy.METISStudy2021,victoriaSpeedTechnological2021},
have a country-specific focus with limited geographical scope
\cite{gilsInteractionHydrogen2021}, or neglect some energy sectors or non-energy
demands that involve hydrogen
\cite{gilsInteractionHydrogen2021,Caglayan2019,caglayanRobustDesign2021}. None
of the studies have explored the interplay between hydrogen network expansion
and electricity grid reinforcements. Neither have the potentials for lower
development costs through pipeline retrofitting been taken into account so far
in a highly resolved model.

This paper provides the first high-resolution examination of the trade-offs
between electricity grid expansion and a new hydrogen network in scenarios for a
European energy system with net-zero carbon dioxide emissions and high shares of
renewable electricity production. We analyse four main scenarios to investigate
if a hydrogen network can compensate for a potential lack of power grid
expansion. The scenarios differ based on whether or not electricity and hydrogen
grids can be expanded, including potentials for gas pipeline retrofitting.
Evaluation criteria include the total system cost, the composition and spatial
distribution of technologies and transmission infrastructure in the system. As a
supplementary sensitivity analysis, we also evaluate the impact of restricted
onshore wind potentials on these scenarios.

For our analysis, we use an open capacity expansion model of the European energy
system, PyPSA-Eur-Sec, which, in contrast to many previous studies
\cite{henningComprehensiveModel2014,mathiesenSmartEnergy2015,connollySmartEnergy2016,lofflerDesigningModel2017,blancoPotentialHydrogen2018,brownSynergiesSector2018,in-depth_2018,victoria2020},
combines a fully sector-coupled approach with a high spatio-temporal resolution
and multi-carrier transmission infrastructure representation so that it can
capture the various transport bottlenecks that constrain the cost-effective
integration of variable renewable energy. The model co-optimises the investment
and operation of generation, storage, conversion and transmission
infrastructures in a single linear optimisation problem, covering 181 regions
and a 3-hourly time resolution for a full year. It incorporates spatially
distributed demands of the electricity, industry, buildings, agriculture and
transport sectors, including shipping and aviation as well as non-energy
feedstock demands in the chemicals industry. Primary energy supply comes from
wind, solar, biomass, hydro, and limited amounts of fossil oil and gas. The
energy flows between the system's energy carriers are modelled by various
technologies, including heat pumps, combined heat and power (CHP) plants,
thermal storage, electric vehicles, batteries, power-to-X processes, fuel cells,
and geological potentials of underground hydrogen storage. Data on electricity
and gas transmission infrastructure is also included to determine grid expansion
needs and retrofitting potentials. The model also features detailed management
of carbon flows between capture, usage, sequestration and the atmosphere to
track carbon through the system. More details on the model are presented in the
\nameref{sec:methods} and \nameref{sec:si}. The model is open-source and based
on open data such that results can be reproduced and assumptions may be modified
(\href{https://github.com/pypsa/pypsa-eur-sec}{github.com/pypsa/pypsa-eur-sec}).

All investigations are conducted with a constraint that carbon dioxide emissions
into the atmosphere balance out to zero over the year. The model can sequester
up to 200 Mt\co per year, allowing it to sequester industry process emissions that have a fossil origin,
such as calcination in cement manufacture, but restricting
the use of negative emission technologies compared to other works
\cite{blancoPotentialHydrogen2018}. In our scenarios, we also do not consider
clean energy imports to Europe, thus assuming that Europe is self-sufficient in
electricity and green fuels and feedstocks. Technology assumptions are taken
widely from the Danish Energy Agency for the year 2030 \cite{DEA}.

% Higher sequestration rates would be
% favoured in the model. For comparison, the JRC-EU-TIMES model sees optimal
% sequestration levels up to 1.4 Gt\co per year for a 95\% reduction of \co
% emissions in the EU compared to 1990 levels .

% general:
% \cite{
%     mckennaScenicnessAssessment2021,
%     krummModellingSocial2022,
%     weinandImpactPublic2021,
%     weinandExploringTrilemma,
%     trondleTradeOffsGeographic2020,
%     sasseDistributionalTradeoffs2019,
%     sasseRegionalImpacts2020,
%     ludererImpactDeclining2021,
%     victoria2020,
%     victoriaSpeedTechnological2021,
%     lombardiPolicyDecision2020,
%     tsiropoulosNetzeroEmissions2020,
%     tafarteQuantifyingTrade,
%     lehmannManagingSpatial}

% PV cost
% \cite{
%     jaxa-rozenSourcesUncertainty2021,
%     victoriaSolarPhotovoltaics2021,
%     xiaoPlummetingCosts2021}

%