There are many different combinations of technologies that would allow
the European Union to reach net-zero greenhouse gas emissions by
2050. However, not all technologies meet the same level of acceptance
among the public. The last few decades have seen public resistance to
new and existing nuclear power plants, projects with carbon capture
and sequestration (CCS), onshore wind power plants, and overhead
transmission lines. The lack of public acceptance can both delay the
deployment of a technology and stop its deployment altogether. This
may make it harder to reach greenhouse gas reduction targets in time,
or cause rising costs through the substitution with other
technologies.

general:
\cite{
    mckennaScenicnessAssessment2021,
    krummModellingSocial2022,
    weinandImpactPublic2021,
    weinandExploringTrilemma,
    trondleTradeOffsGeographic2020,
    sasseDistributionalTradeoffs2019,
    sasseRegionalImpacts2020,
    ludererImpactDeclining2021,
    EuropeanHydrogen,
    victoria2020,
    victoriaSpeedTechnological2021,
    lombardiPolicyDecision2020,
    tsiropoulosNetzeroEmissions2020,
    europeancommission.directorategeneralforenergy.METISStudy2021,
    deutschNoRegretHydrogen,
    tafarteQuantifyingTrade,
    lehmannManagingSpatial,
}

\cite{}

PV cost
\cite{jaxa-rozenSourcesUncertainty2021,victoriaSolarPhotovoltaics2021,xiaoPlummetingCosts2021}

In this contribution we examine the trade-off between costs and public
acceptance for onshore wind and overhead transmission lines in highly
renewable scenarios for Europe with net-zero carbon dioxide emissions
in the whole energy system (electricity, buildings, transport and
industry). For this purpose we use a capacity expansion model of the
European energy system, PyPSA-Eur-Sec, which, in contrast to previous
studies \cite{henningComprehensiveModel2014,mathiesenSmartEnergy2015,IEESWV,connollySmartEnergy2016,lofflerDesigningModel2017,blancoPotentialHydrogen2018,brownSynergiesSector2018,in-depth_2018,victoria2020},
combines a fully sector-coupled approach with a high-resolution grid
model (181 nodes) so that it can capture the grid bottlenecks that
constrain the integration of renewable energy.  We successively
constrain the allowed onshore wind and new overhead grid projects down
to zero to examine the effects on total system cost and the
composition of technologies in the system.

We find that the overall system costs are not significantly affected
by restrictions on onshore wind and transmission. The high level of
synthetic fuel production and exchange between the nodes, which is
necessary for industry, transport and backup electricity and heating
applications, provides sufficient flexibility to manage these
restrictions. Moderate onshore wind and grid expansion provide cost
savings, but they are small (5-10~\%) compared to total system
costs. The restriction of onshore wind capacity results in a
substitution with offshore wind and solar PV; the restriction of grid
expansion leads to more local production from solar and more hydrogen
production. The limited rise in system costs relies on a new network
of underground hydrogen storage and pipelines in Europe, which help to balance
generation from renewables in time and space.

To explore the effects of acceptance we carry out two principal investigations
on the model:
\begin{enumerate}
\item {\bf Grid reinforcement restriction:} The model is allowed to build new
transmission capacity (high voltage alternating current (HVAC) and direct
current (HVDC)) wherever is optimal in the network, but the total volume of new
capacity (sum of capacity of each new line in MW multiplied by its length in km)
is limited. The volume limit is given in fractions of today's grid volume: a
line volume limit of 1.0 means no new capacity is allowed beyond today's grid
(since the model cannot remove existing lines); a limit of 1.25 means the total
grid capacity can grow by 25\% (25\% is similar to the planned extra capacity in
the European network operators' Ten Year Network Development Plan (TYNDP)
\cite{TYNDP2016}).
\item {\bf Onshore wind potential restriction:} Each node has a maximum
installable capacity of onshore wind, based on land use restrictions (e.g. no
wind turbines in cities) and minimum distances to buildings. This gives a
maximum social-technological potential, corresponding to about 400~GW for
Germany. This potential is successively restricted down to zero at each node, to
examine the effect on the results. For this investigation, no grid expansion
beyond today is assumed.
\end{enumerate}

\nameref{sec:discussion}

All investigations are carried out with a constraint that carbon dioxide
emissions balance out to zero over the year (see for
the possible paths carbon can take in the model). The model is allowed to
sequester up to 200 Mt\co per year, which allows it to sequester industry
process emissions, such as calcination in cement manufacture, that have a fossil
origin. Higher sequestration rates would be favoured in the model (the
JRC-EU-TIMES model sees optimal sequestration levels up to 1.4 Gt\co per year
for a 95\% reduction of \co emissions in the EU compared to 1990 levels
\cite{blancoPotentialHydrogen2018}).

In these scenarios we do not allow the import of electricity or synthetic fuels
(hydrogen, methane and liquid hydrocarbons) from outside Europe. Imports from
regions with good renewable resources should be considered in future work
\cite{fasihiTechnoeconomicAssessment2019,heuserTechnoeconomicAnalysis2019}.

Can a hydrogen network replace electricity transmission network expansion in a climate-neutral scenario for Europe?

Can we substitute for the electricity grid by producing eleoctrolytic hydrogen
and transport it through new or repurposed hydrogen pipeline network?

2050 scenarios for EU: power demand doubles, mostly met by VRE

This collides with low acceptance for power grid expansion and onshore wind

Offshore wind can certainly help, but still needed to transport the electricity to loads inland

Need spatial resolution to see grid bottlenecks and infrastructure trade-offs
one node per country or continent won't work

Need to co-optimise balancing solutions with generation. Optimising separately won't work.

PyPSA-Eur-Sec represents all energy flows and bottlenecks in energy networks

couple all energy sectors (power, heat, transport, industry)

reduce net CO2 emissions to zero

smaller bidding zones and widespread dynamic pricing

examine effect of limiting power grid expansion, onshore wind potentials, hydrogen grid

There are difficult periods in winter with
- low wind and solar (high prices)
- high space heating demand
- low air temperatures, which are bad for air-sourced heat pump performance
- less smart solution: backup gas boilers
- smart solution: building retrofiting, TES in district heating, CHP