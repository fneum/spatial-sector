There are many different combinations of infrastructure that would allow Europe
to reach net-zero greenhouse gas emissions by
mid-century.\cite{pickeringDiversityOptions2022} However, not all technologies
meet the same level of acceptance among the public. The last few decades have
seen public resistance to new and existing nuclear power plants, projects with
carbon capture and sequestration (CCS), onshore wind power plants, and overhead
transmission
lines.\cite{cohenRefocussingResearch2014a,bertschPublicAcceptance2016,boudetPublicPerceptions2019a}
The lack of public acceptance can both delay the deployment of a technology and
even stop its deployment altogether.\cite{batelSocialAcceptance2013a} This may
make it harder to reach greenhouse gas reduction targets in time or cause rising
costs through the substitution with other technologies. In particular,
electricity transmission network expansion has suffered many delays in Europe in
recent decades, despite its importance for integrating large amounts of
renewable electricity and electrifying the transport, buildings and industry
sectors.\cite{schlachtbergerBenefitsCooperation2017,trondleTradeOffsGeographic2020}

Hydrogen has the potential to become a pivotal energy carrier in such a climate-neutral
energy system.\cite{hanleyRoleHydrogen2018,staffellRoleHydrogen2019a} It is
needed in industry to produce ammonia for fertilisers and can be used for
direct reduced iron for
steelmaking.\cite{wangGreeningAmmonia2018,voglPhasingOut2021} It is also a
critical feedstock to produce synthetic methane and liquid hydrocarbons for use
as aviation and shipping fuels, and as a precursor to high-value chemicals in
industrial production.\cite{lechtenbohmerDecarbonisingEnergy2016} Hydrogen could
also be used for heavy-duty land transport and backup heat and power
supply.\cite{kluschkeMarketDiffusion2019,doddsHydrogenFuel2015}

The limited social acceptance for electricity grid reinforcement and the
advancing role of hydrogen raises the question of whether a new hydrogen network
could offer a replacement for balancing variable renewable electricity
generation and moving energy across the
continent.\cite{caglayanRobustDesign2021} Such a vision for a \textit{European
Hydrogen Backbone (EHB)} has recently been expressed by Europe's gas industry in
a series of reports.
\cite{gasforclimateEuropeanHydrogen2020,gasforclimateEuropeanHydrogen2021,gasforclimateExtendingEuropean2021,gasforclimateEuropeanHydrogen2022}
It would offer an alternative to connect remote regions with abundant
and cost-effective wind and solar potentials to densely-populated and
industry-heavy regions with high demand but limited supply options.

Since Europe's sizeable natural gas transmission network is set to become
increasingly redundant as the system transitions towards climate neutrality, the
option to repurpose parts of the network to transport hydrogen instead may
enhance the appeal of hydrogen networks further. This is because retrofitting
gas pipelines would greatly reduce the development costs of hydrogen
pipelines.\cite{cerniauskasOptionsNatural2020,tsikliosHydrogenTransport2022}
Moreover, repurposed and new pipelines may also meet higher levels of acceptance
among the local populations than transmission
lines.\cite{schonauerHydrogenFuture2022} Unlike transmission towers, pipelines
are less visible because they usually run below or near the ground. Particularly
where gas pipelines already exist, the perceivable impact would be minimal.

However, few studies have looked into how much building a hydrogen network in
Europe could reduce system costs. The industry-oriented EHB reports do not
include an assessment based on the co-optimisation of energy system components.
\cite{gasforclimateEuropeanHydrogen2020,gasforclimateEuropeanHydrogen2021,gasforclimateExtendingEuropean2021,gasforclimateEuropeanHydrogen2022}
Other sector-coupling studies have not included hydrogen networks at all,
\cite{brownSynergiesSector2018,pickeringDiversityOptions2022,childFlexibleElectricity2019,kendziorskiCentralizedDecentral2022a}
or when they do, model Europe only at country-level resolution,
\cite{europeancommission.directorategeneralforenergy.METISStudy2021,victoriaSpeedTechnological2022}
have a country-specific focus with limited geographical scope or detail outside
the focus area,\cite{gilsInteractionHydrogen2021} investigate the mid-term
role rather than the long-term role of a hydrogen
network,\cite{europeancommission.directorategeneralforenergy.METISStudy2021} or
neglect some energy sectors or non-energy demands that involve hydrogen.
\cite{gilsInteractionHydrogen2021,Caglayan2019,caglayanRobustDesign2021}
However, high resolution at continental scope is needed to understand how a
hydrogen network can relieve power grid bottlenecks, where the costs of hydrogen
network development can be reduced by retrofitting gas pipelines, and where
geological sites for hydrogen storage are located. Previous one-node-per-country
studies could not have suitably assessed this.

This paper provides the first high-resolution examination of the trade-offs
between electricity grid expansion and a new hydrogen network in scenarios for a
European energy system with net-zero carbon dioxide emissions, no energy imports
and high shares of renewable electricity production. By leveraging recent
computational advances, we resolve 181 regions to study what role hydrogen
infrastructure can play in a future sector-coupled system. This enables us to
take account of network bottlenecks inside countries, see more precise locations
of demand and supply in the network and capture the variability of renewable
resources. For the first time, such an investigation also considers regional
potentials for the repurposing of legacy gas pipelines and the geological
storage of hydrogen in salt caverns.

Our analysis covers four main scenarios to examine if a hydrogen network
composed of new and retrofitted pipelines can compensate for a potential lack of
power grid expansion. These scenarios differ based on whether or not electricity
and hydrogen grids can be expanded. As supplemental sensitivity analyses, we
also evaluate the impact of restricted onshore wind potentials
(\cref{sec:si:onw}), more progressive technology assumptions
(\cref{sec:si:sensitivity-costs}), the impact of importing most hydrogen
derivatives from outside of Europe on network benefits (\cref{sec:si:sensitivity-imports}), and the use
of alternative shipping fuels (\cref{sec:si:sensitivity-shipping}).

For our analysis, we use an open capacity expansion model of the European energy
system, PyPSA-Eur-Sec, which, in contrast to many previous studies,
\cite{henningComprehensiveModel2014,mathiesenSmartEnergy2015,connollySmartEnergy2016,Loffler_2019,blancoPotentialHydrogen2018,brownSynergiesSector2018,in-depth_2018,victoria2020,ludererImpactDeclining2021,gea-bermudezRoleSector2021}
combines a fully sector-coupled approach with a high spatio-temporal resolution
and multi-carrier transmission infrastructure representation so that it can
capture the various transport bottlenecks that constrain the cost-effective
integration of variable renewable energy.\cite{frysztackiStrongEffect2021} The
model co-optimises the investment and operation of generation, storage,
conversion and transmission infrastructures for the least-cost outcome in a
single linear optimisation problem, covering 181 regions and a 3-hourly time
resolution for a full year. A sensitivity analysis varying the model's
spatio-temporal resolution is included in
\cref{sec:si:sensitivity-time,sec:si:sensitivity-space}. The regional scope
comprises the European Union without Cyprus and Malta as well as the United
Kingdom, Norway, Switzerland, Albania, Bosnia and Herzegovina, Montenegro, North
Macedonia, Serbia and Kosovo. It incorporates spatially distributed demands of
the electricity, industry, buildings, agriculture and transport sectors,
including dense fuels in shipping and aviation as well as non-energy feedstock
demands in the chemicals industry. Primary energy supply comes from wind, solar,
biomass, hydro, and limited amounts of fossil oil and gas. The energy flows
between the system's energy carriers are modelled by various technologies,
including heat pumps, combined heat and power (CHP) plants, thermal storage,
electric vehicles, batteries, power-to-X processes, hydrogen fuel cells, and
geological potentials of underground hydrogen storage. Data on existing
electricity and gas transmission infrastructure is also included to determine
grid expansion needs and retrofitting potentials. The model also features
detailed management of carbon flows between capture, usage, sequestration,
removal and emissions to the atmosphere to track carbon through the system. More
details on the model are presented in the \nameref{sec:methods} and
\nameref{sec:si}. The model is open-source and based on open data
(\href{https://github.com/pypsa/pypsa-eur-sec}{github.com/pypsa/pypsa-eur-sec}).

All investigations are conducted with a constraint that carbon dioxide emissions
into the atmosphere balance out to zero over the year, disregarding other
greenhouse gas emissions. The model can sequester up to 200 Mt\co per year,
allowing it to sequester industry process emissions that have a fossil origin,
such as the calcination in cement manufacturing, but restricting the use of
negative emission technologies compared to other works.
\cite{blancoPotentialHydrogen2018} In our scenarios, we also do not consider
clean energy imports to Europe, thus assuming that Europe is self-sufficient in
electricity and green fuels and feedstocks. We relax this constraint in
\cref{sec:si:sensitivity-imports}. Technology assumptions are taken widely from
the Danish Energy Agency for the year 2030.\cite{DEA} A sensitivity with
technology assumptions for the year 2050 is presented in \cref{sec:si:sensitivity-costs}.
