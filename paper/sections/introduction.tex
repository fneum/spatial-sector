% public acceptance problems

There are many different combinations of infrastructure that would allow the
Europe to reach net-zero greenhouse gas emissions by 2050. However, not all
technologies meet the same level of acceptance among the public. The last few
decades have seen public resistance to new and existing nuclear power plants,
projects with carbon capture and sequestration (CCS), onshore wind power plants,
and overhead transmission lines. The lack of public acceptance can both delay
the deployment of a technology and stop its deployment altogether. This may make
it harder to reach greenhouse gas reduction targets in time, or cause rising
costs through the substitution with other technologies.

% importance of electricity transmission and lack of acceptance

In particular, electricity transmission network expansion has suffered many
delays in Europe in recent decades, despite its importance for integrating
renewable electricity into the energy system.

2050 scenarios for EU: power demand doubles, mostly met by VRE \cite{}

This outlook collides with low acceptance for power grid expansion and onshore
wind

Offshore wind and solar electricity may offer a partial remedy to limited
expansion options, but still a means to transport the energy to loads inland

% hydrogen important energy carrier, hydrogen network as replacement?

At the same time, hydrogen is likely to become a pivotal energy carrier in a
carbon-neutral energy system. Hydrogen is needed in the industry sector to
produce ammonia for fertilisers and direct reduced iron for steelmaking.
Hydrogen is also an important feedstock to produce synthetic methane and liquid
hydrocarbons, which have multiple uses in the industry and other sectors. For
example, Fischer-Tropsch fuels are used in aviation and also as a precursor to
high value chemicals and plastics. In the transport sector, hydrogen is used in
heavy-duty vehicles and as liquified hydrogen in shipping. Furthermore,
stationary fuel cell combined heat and power (CHP) plants may re-electrify
hydrogen as backup heat and power supply.

% idea of European hydrogen backbone, retrofitting potential, higher acceptance?

The limited social acceptance for grid reinforcement and the advancing role of
hydrogen in future sustainable energy systems, raise the question whether a new
hydrogen network could offer a replacement for balancing variations in wind and
solar energy across the continent. The vision for a \textit{European Hydrogen
Backbone (EHB)} has been expressed by Europe's gas industry in a series of
recent reports
\cite{gasforclimateEuropeanHydrogen2020,gasforclimateEuropeanHydrogen2021,gasforclimateExtendingEuropean2021,gasforclimateEuropeanHydrogen2022}.
The prospect of connecting remote regions with abundant and cost-effective
hydrogen supply potentials to densely-populated, industry-heavy regions with
high hydrogen demand but limited expansion potentials gives impetus to
developing a new network of hydrogen pipelines.

The potential appeal of a hydrogen backbone is further amplified if parts of
Europe's vast existing natural gas network could be repurposed to transport
hydrogen instead. The gas network's current role to distribute imports from
Russia, North Africa and various LNG terminals for use in industry, electricity
and heat supply is set to fade as the energy system transitions towards climate
neutrality. Hence, a rising momentum for hydrogen transport happens to coincide
with a reduced need for gas transmission. Retrofitting could be achieved by
moderately complicated measures like compressor replacements and plastic tube
inlays, leading to costs around half that of new pipelines \cite{}.

Pipelines may also meet higher levels of acceptance among the local populations
than transmission lines. Unlike transmission towers, pipelines are less visible
because they usually run below or close to the ground. Particularly where gas
pipelines already exist, the perceivable impact would be minimal. Combined with
lower retrofitting costs and low acceptance for grid reinforcement, this creates
ideal prerequisites to evaluate the potential cost-benefit of a hydrogen network
in detail.

% what is lacking in literature and EHB to evaluate the benefit of H2 network?

However, EHB has not done any detailed co-planning of infrastructures
-

% in this paper

In this paper, we examine whether a hydrogen backbone is necessary or useful in
scenarios for a European energy system with high shares of variable renewable
electricity production and net-zero carbon dioxide emissions. The sectoral
coverage encompasses electricity, buildings, transport and industry sectors.

interplay of electricity and hydrogen transmission grids

We constrain the expansion of electricity and/or hydrogen grid to examine the
effects on total system cost and the composition of technologies in the system.

while
varying the allowed expansion of the electricity and hydrogen grids
respectively.

We also evaluate the impact of restricted onshore wind potentials on these
results as additional sensitivity.

% model description and novelty

For this purpose we use an open-source capacity expansion model of the European
energy system, PyPSA-Eur-Sec, which, in contrast to previous studies
\cite{henningComprehensiveModel2014,mathiesenSmartEnergy2015,IEESWV,connollySmartEnergy2016,lofflerDesigningModel2017,blancoPotentialHydrogen2018,brownSynergiesSector2018,in-depth_2018,victoria2020},
combines a fully sector-coupled approach with a high spatio-temporal resolution
so that it can capture the transmission bottlenecks that constrain the
cost-effective integration of variable renewable energy.

With its 181 regions and 3-hourly time series for a full year, the model is
detailed enough to capture existing bottlenecks and long time series, and has
been highly optimised to solve sector-coupled capacity expansion problems that
represent all energy flows and infrastructures.

Scenarios are run with cost assumptions for the year 2030, taken mostly from the
Danish Energy Agency \cite{}.

Features:
- couples all energy sectors (power, heat, transport, industry)
- gas network data to estimate retrofitting potentials
- linearised power flow equations
- potential for hydrogen storage in salt caverns
- renewable potentials for wind, solar and biomass
- energy as well as non-energy demands
- detailed carbon management of sources, capture, usage and sequestration (last for transition!)

% challenges to the model

% compared to electricity, heating and transport are strongly peaked
% - heating is strongly seasonal but also with synoptic variations
% - transport has strong daily periodicity
% There are difficult periods in winter with
% - low wind and solar (high prices)
% - high space heating demand
% - low air temperatures, which are bad for air-sourced heat pump performance
% - less smart solution: backup gas boilers
% - smart solution: building retrofiting, TES in district heating, CHP

% scenario assumptions

All investigations are carried out with a constraint that carbon dioxide
emissions balance out to zero over the year. The model is allowed to sequester
up to 200 Mt\co per year, which allows it to sequester industry process
emissions, such as calcination in cement manufacture, that have a fossil origin.
Higher sequestration rates would be favoured in the model. For comparison, the
JRC-EU-TIMES model sees optimal sequestration levels up to 1.4 Gt\co per year
for a 95\% reduction of \co emissions in the EU compared to 1990 levels
\cite{blancoPotentialHydrogen2018}. Moreover, in our scenarios we do not allow
the import of electricity or synthetic fuels (hydrogen, methane and liquid
hydrocarbons) from outside Europe and assume material demands as observed in the
year 2016 \cite{IDEES}.

general:
\cite{
    mckennaScenicnessAssessment2021,
    krummModellingSocial2022,
    weinandImpactPublic2021,
    weinandExploringTrilemma,
    trondleTradeOffsGeographic2020,
    sasseDistributionalTradeoffs2019,
    sasseRegionalImpacts2020,
    ludererImpactDeclining2021,
    EuropeanHydrogen,
    victoria2020,
    victoriaSpeedTechnological2021,
    lombardiPolicyDecision2020,
    tsiropoulosNetzeroEmissions2020,
    europeancommission.directorategeneralforenergy.METISStudy2021,
    deutschNoRegretHydrogen,
    tafarteQuantifyingTrade,
    lehmannManagingSpatial}

PV cost
\cite{
    jaxa-rozenSourcesUncertainty2021,
    victoriaSolarPhotovoltaics2021,
    xiaoPlummetingCosts2021}

Gils, Goeke, Lombardi, Bossmann