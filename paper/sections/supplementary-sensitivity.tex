\section{Sensitivity Analysis}
\label{sec:si:sensitivity}

\subsection{Electricity Grid Reinforcement Restrictions}
\label{sec:si:lv}

\begin{figure}
    \centering
    \makebox[\textwidth][c]{
    \begin{subfigure}[t]{1.2\textwidth}
        \centering
        \caption{Sensitivity of total system cost towards electricity transmission grid expansion limits}
        \includegraphics[width=\textwidth]{lv-sensitivity.pdf}
        \label{fig:lv-restriction}
    \end{subfigure}
    }
    \makebox[\textwidth][c]{
    \begin{subfigure}[t]{1.2\textwidth}
        \centering
        \caption{Sensitivity of total system cost towards onshore wind expansion limits}
        \includegraphics[width=\textwidth]{onw-sensitivity.pdf}
        \label{fig:onw-restriction}
    \end{subfigure}
    }
    \caption{Sensitivity of total system cost towards electricity transmission grid expansion limits and onshore wind restrictions.
    The sweep for grid expansion restrictions allows full onshore wind potentials.
    The extreme case (a, left) removes all existing power transmission lines as well.
    The sweep for onshore wind potential restrictions allows power grid reinforcements by up to 25\% of today's transmission capacities.
    The extreme case (b, left) combines no grid expansion with no onshore wind potentials.}

    \label{fig:lv-onw-restriction}
\end{figure}

In the following sensitivity runs, the model is allowed to build new electricity
transmission infrastructure wherever is cost-optimal, but the total volume of
new transmission capacity (sum of line length times capacity, TWkm) is
successively limited. The volume limit is given in fractions of today's grid
volume: a line volume limit of 100\% means no new capacity is allowed beyond
today's grid (since the model cannot remove existing lines); a limit of 125\%
means the total grid capacity can grow by 25\% (25\% is similar to the planned
extra capacity in the European network operators' Ten Year Network Development
Plan (TYNDP)\citeS{tyndp2018}). For this investigation, a hydrogen network
could be built.

\cref{fig:lv-restriction} shows the composition of total annual energy system costs
(including all investment and operational costs) as we vary the allowed power
grid expansion, from no expansion (only today's grid) to a doubling of today's
grid capacities (the model optimises where new capacity is placed). As the grid
is expanded, total costs decrease only slightly, despite the increasing costs of
the grid. The total cost benefit of a doubling of grid capacity is around
46~bn\euro/a (6\%) corresponding to an expansion of 715 TWkm. However, over half
of the benefit (27~bn\euro/a, 3.5\%) is available already at a 25\% expansion
corresponding to an expansion of 447 TWkm.

\cref{fig:lv-restriction} also includes a scenario where today's electricity
transmission infrastructure is completely removed from the model, similar to an
electricity system study on geographic trade-offs by Tröndle et
al.\citeS{trondleTradeOffsGeographic2020} While doubling the transmission grid
yields a benefit of 46~bn\euro/a, removing what exists incurs a cost of
108~bn\euro/a. The lack of electricity grid is mostly compensated by more solar
PV generation, battery storage and re-electrified hydrogen.

\subsection{Onshore Wind Potential Elimination}
\label{sec:si:onw}

\begin{figure}
    \centering
    \makebox[\textwidth][c]{
        \begin{subfigure}[t]{0.6\textwidth}
            \centering
            \caption{hydrogen network}
            \includegraphics[height=0.42\textheight]{\hyrun/maps/elec_s_181_lv1.0__Co2L0-3H-T-H-B-I-A-solar+p3-linemaxext10-onwind+p0-h2_network_2050.pdf}
            \label{fig:no-onw:h2}
        \end{subfigure}
        \begin{subfigure}[t]{0.6\textwidth}
            \centering
            \caption{energy balance}
            \includegraphics[height=0.42\textheight]{\hyrun/elec_s_181_lv1.0__Co2L0-3H-T-H-B-I-A-solar+p3-linemaxext10-onwind+p0_2050/import-export-total-200.pdf}
            \label{fig:no-onw:io}
        \end{subfigure}
        }
    \begin{subfigure}[t]{0.6\textwidth}
        \centering
        \caption{system cost}
        \includegraphics[height=0.42\textheight]{\hyrun/maps/elec_s_181_lv1.0__Co2L0-3H-T-H-B-I-A-solar+p3-linemaxext10-onwind+p0-costs-all_2050.pdf}
        \label{fig:no-onw:tsc}
    \end{subfigure}
    \caption{Maps of regional energy balance, hydrogen network and production sites, and spatial and technological distribution of total energy system costs for a scenario without onshore wind and without power grid expansion.}
    \label{fig:no-onw}
\end{figure}

Like building new power transmission lines, the deployment of onshore wind may
not always be socially accepted, such that it may not be possible to leverage
its full potential.\citeS{mckennaScenicnessAssessment2021,weinandImpactPublic2021,weinandExploringTrilemma2021} In the following additional sensitivity analysis, we explore
the hypothetical impact of restricting the installable potentials of onshore
wind down to zero (\cref{fig:no-onw}).

We find that as onshore wind is eliminated, costs rise by \euro~92 bn/a (12\%)
when the electricity grid is fixed to today's capacities, but a hydrogen network
can still be developed. In comparison to the least-cost solution with full
network expansion, this solution is 19\% more expensive. A solution in which
neither a hydrogen network could be developed would be 23\% more expensive.
\cref{sec:si:onw-compromise} presents further intermediate results between full
and no onshore wind expansion for scenarios with hydrogen network expansion and
TYNDP-equivalent power grid reinforcements. The model substitutes onshore wind,
particularly in the British Isles, for higher investment in offshore wind in the
continental shores of the North Sea and solar generators plus batteries in
Southern and Central Europe (\cref{fig:no-onw:tsc}).  Without onshore wind, the
potentials for rooftop solar PV and fixed-pole offshore wind in Europe are
largely exhausted, such that in this self-sufficient scenario for Europe, the
effect of installable potentials becomes critical.

% Because offshore capacities are
% concentrated near coastlines, and grid capacity is restricted, total spending on
% hydrogen electrolysers and networks also increases to absorb the increased
% offshore generation.

Whereas with onshore wind, we observe both wind-backed electrolysis in
Northwestern Europe and solar-backed hydrogen production in Southern Europe, the
latter becomes the dominant producer of hydrogen if the development of onshore
wind capacities is restricted (\cref{fig:no-onw:h2,fig:no-onw:io}). This shift
in hydrogen infrastructure also impacts the share of gas pipelines being
retrofitted for hydrogen transport. As the Iberian Peninsula becomes a preferred
region for hydrogen production but has a more sparse gas transmission network
today, the rate of retrofitted pipeline capacity reduces from 65\% to 58\%. Many
new hydrogen pipelines are built to connect Spain with France, but also to
connect Denmark to Germany and Greece to Italy. Gas pipeline retrofitting is
then concentrated in Germany, Austria and Italy.

\begin{figure}
    \centering
    \includegraphics[width=.9\textwidth]{sensitivity-h2.pdf}
    \caption{Varying cost benefits of hydrogen network infrastructure and changes in system composition as power grid and onshore wind expansion options are altered. The cost benefit of a hydrogen network varies between 1.6\% and 3.7\% across all scenarios shown.}
    \label{fig:h2-restriction-w-onw}
\end{figure}

The cost benefit of a hydrogen network is similar whether or not onshore wind
capacities are built in Europe, even though the hydrogen network topology is
then built around supply from solar PV from Southern Europe and offshore wind in
the North Sea rather than from onshore wind in Northwestern Europe. As
\cref{fig:h2-restriction-w-onw} illustrates, the net benefit is again strongest
when power grid expansion is restricted. If both onshore wind and power grid
expansion are excluded, costs for a system without a hydrogen network option
were by 32~bn\euro/a (3.7\%) higher. With cost-optimal electricity grid
reinforcement, the net benefit of a hydrogen network is lower with
13~bn\euro/a (1.7\%).

\subsection{Compromises on Onshore Wind Potential Restrictions}
\label{sec:si:onw-compromise}

In the following sensitivity runs, the maximum installable capacity of onshore
wind is successively restricted down to zero at each node. The upper limit is
derived from land use restriction and yields a maximum technical potential
corresponding to about \SI{481}{\giga\watt} for Germany. For this investigation,
a compromise electricity grid expansion by 25\% compared to today and no limits
on hydrogen network infrastructure are assumed.

In this case, system costs rise by 77~bn\euro/a (10\%)
by restricting the installable potentials of onshore down to zero. Just as in
the case of restricted line volumes, \cref{fig:onw-restriction} reveals a
nonlinear rise in system costs: if we constrain the model to 25\% of the onshore
potential (around 120~GW for Germany), costs rise by only 46~bn\euro/a (6\%).
Thereby, 25\% of the onshore wind potential may represent a social compromise
between total system cost, and social concerns about onshore wind development.

In comparison, Schlachtberger et al.~\citeS{schlachtbergerCostOptimal2018} found
a similar change between 9\% and 12\% in system costs in an electricity-only
model when onshore wind potentials were restricted across various grid expansion
limitations. The the biggest change was observed when the power grid could not
be reinforced. Onshore wind was largely replaced with offshore wind in that
model. Unlike that model, here we have a higher grid resolution (181 versus 30
regions) which allows us to better assess the grid integration costs of offshore
wind. Our results show that moderate power grid expansion is particularly
important when onshore wind development is severely limited. For the extreme
case where no onshore wind capacities would be built, reducing power grid
expansion from 25\% to none incurs another rise in system cost of an additional
43~bn\euro/a (6\%).

\subsection{Using Technology and Cost Projections for 2050}
\label{sec:si:sensitivity-costs}

In this sensitivity analysis, we investigate the impact of using more
progressive technology cost projections.\citeS{wayEmpiricallyGrounded2022}
Rather than using assumptions for the year 2030, we use cost assumptions for
2050 as outlined in \cref{tab:si:costs}. These assumptions include cost
reductions of solar photovoltaics and power-to-liquid processes by 25\% beyond
2030, as well as a reduction by 33\% for direct air capture and 45\% to 60\% for
battery storage and electrolysers.

Using more progressive assumptions diminishes the cost benefit of power grid
reinforcements from 6-8\% to 4-5\% (\cref{fig:sensitivity-costs}). However, the
cost benefit of the hydrogen network is robust against variations in cost and
technology projections, changing merely from 1.6-3.4\% to 2.0-3.1\%. With cost
projections for 2050, we see a total cost reduction between 15\% and 18\% and a
shift towards more distributed and decentral solutions
(\cref{fig:sensitivity-costs-diff}). This includes significantly more solar and
battery deployment, more electrolysers with more flexible operation supported by
additional hydrogen storage for buffering, and less wind generation and
distribution grid capacities. Owing to plummeting costs of solar photovoltaics,
\cref{fig:hydrogen-inframaps:costs} reveals a shift towards more hydrogen
production in sunny Southern Europe. This leads to more hydrogen storage and a
stronger hydrogen network buildout in this region. The consequence is a more
balanced production of solar-based hydrogen in Southern Europe and wind-based
hydrogen in the North Sea region.

\begin{SCfigure}
    \centering
    \includegraphics[width=0.75\textwidth]{sensitivity-h2-new-costs.pdf}
    \caption{Cost benefits of electricity and hydrogen network infrastructure with cost projections for 2050.}
    \label{fig:sensitivity-costs}
\end{SCfigure}

\begin{figure}
    \centering
    \begin{subfigure}[t]{\textwidth}
        \centering
        \caption{differences in system cost compared to 2030 cost projections}
        \includegraphics[width=\textwidth]{diff-cost-costs}
        \label{fig:sensitivity-costs-cost}
    \end{subfigure}
    \begin{subfigure}[t]{\textwidth}
        \centering
        \caption{differences in generation and conversion capacities compared to 2030 cost projections}
        \includegraphics[width=\textwidth]{diff-capacity-costs}
        \label{fig:sensitivity-costs-cap}
    \end{subfigure}
    \begin{subfigure}[t]{\textwidth}
        \centering
        \caption{differences in storage capacities compared to 2030 cost projections}
        \includegraphics[width=\textwidth]{diff-storage-costs}
        \label{fig:sensitivity-costs-sto}
    \end{subfigure}
    \caption{Differences in total system cost and optimised capacities for more progressive 2050 cost projections compared to more conservative 2030 cost projections.}
    \label{fig:sensitivity-costs-diff}
\end{figure}


\subsection{Importing all Liquid Hydrocarbons}
\label{sec:si:sensitivity-imports}

In this sensitivity analysis, we explore the cost benefit of a hydrogen network
if all liquid hydrocarbons were imported from outside of Europe. We chose this
case as it constitutes an import scenario that should be detrimental to the
benefits of a hydrogen network. We assume uniform import costs of 115
\euro/MWh\citeS{staissOptionenFuer2022,schornMethanolRenewable2021} for
methanol and Fischer-Tropsch fuels, for a total import volume of 1573 TWh
(roughly one-third methanol and two-thirds Fischer-Tropsch fuels). By replacing the
domestic production of electrofuels with imports, 1903 TWh (80\%) of domestic
hydrogen demand fall away, which is much more compared to the 333 TWh for
domestic and foreign hydrogen supply each mentioned in the REPowerEU plans for
2030.\citeS{europeancommissionRepowerEUPlan}

As \cref{fig:sensitivity-imports} shows, the relative cost benefits of network
expansion do not change much (from 1.6-3.4\% to 1.9-2.8\%), while the overall
benefit of network expansion is slightly reduced from 10\% to 9\%. Moreover,
there is practically no change in total system costs
(\cref{fig:sensitivity-import-diff}). The costs of 189 bn\euro/a for electrofuel
imports (23.6-25.7\% of system costs) displace almost equal costs for the
domestic supply chain for fuel synthesis comprising wind and solar electricity
generation, direct air capture, hydrogen storage and power-to-X processes
(electrolysis, methanolisation, Fischer-Topsch). Since the domestic electrofuels
can mostly use captured carbon dioxide from point-sources, whereas imported
fuels rely on direct air capture as a carbon source, the costs for direct air
capture cancel out the savings from utilising better renewable resources abroad.

Regarding the spatial deployment of hydrogen infrastructure, as shown in
\cref{fig:hydrogen-inframaps:imports}, we see fewer hydrogen pipelines built
overall, reduced to a total network volume of 103-180 TWkm compared to 204-307
TWkm in scenarios without imports. However, the reduced network achieves higher
retrofitting shares of 78\%. Hydrogen production hubs in Southern Europe
disappear such that the remaining hubs are located in the broader North Sea region.

It is necessary to underline that this sensitivity analysis explores the impact
of importing the majority of hydrogen derivatives and does not analyse the
impact of direct hydrogen imports on network development. For instance, if
hydrogen were imported via pipelines from the MENA region to supply hydrogen to
domestic synthetic fuel production sites, much of the buildout of the European
hydrogen network would likely be diverted to Italy and Spain.

\begin{SCfigure}
    \centering
    \includegraphics[width=0.75\textwidth]{sensitivity-h2-new-import.pdf}
    \caption{Cost benefits of electricity and hydrogen network infrastructure if all liquid hydrocarbons are imported.}
    \label{fig:sensitivity-imports}
\end{SCfigure}

\begin{figure}
    \centering
    \begin{subfigure}[t]{\textwidth}
        \centering
        \caption{differences in system cost compared to scenarios without imports}
        \includegraphics[width=\textwidth]{diff-cost-import}
        \label{fig:sensitivity-import-cost}
    \end{subfigure}
    \begin{subfigure}[t]{\textwidth}
        \centering
        \caption{differences in generation and conversion capacities compared to scenarios without imports}
        \includegraphics[width=\textwidth]{diff-capacity-import}
        \label{fig:sensitivity-import-cap}
    \end{subfigure}
    \begin{subfigure}[t]{\textwidth}
        \centering
        \caption{differences in storage capacities compared to scenarios without imports}
        \includegraphics[width=\textwidth]{diff-storage-import}
        \label{fig:sensitivity-import-sto}
    \end{subfigure}
    \caption{Differences in total system cost and optimised capacities for scenarios with all liquid hydrocarbons imported compared to scenarios without imports.}
    \label{fig:sensitivity-import-diff}
\end{figure}


\subsection{Liquid Hydrogen in Shipping}
\label{sec:si:sensitivity-shipping}

In this sensitivity analysis, we examined the impact of changing the primary
fuel used in shipping from methanol to liquid hydrogen. The hydrogen demand for
international shipping was geographically distributed based on trade volumes of
international ports,\citeS{worldbankWorldBank} while the demand for domestic
shipping was distributed by population. The costs for hydrogen liquefaction were
also included in our analysis (see \cref{tab:si:costs}).

The results of this analysis are shown in \cref{fig:sensitivity-shipping} and
indicate that while the cost benefit of electricity grid reinforcements is
similar (6.6-9.0\%), the cost benefit of a hydrogen network almost doubles (from
1.6-3.4\% to 3.3-5.6\%). The overall cost benefit of network expansion rose from
9.9\% to 12.6\%. This difference can be attributed to the added need to
transport the hydrogen for the shipping sector from the most cost-effective
hydrogen production sites to the ports, whereas previously methanol offered
low-cost transport allowing for methanolisation directly where hydrogen was
produced.

As shown in \cref{fig:sensitivity-shipping-diff}, energy system costs are
between 2.5\% and 4.9\% cheaper when exchanging methanol with liquid hydrogen in
ships. Methanolisation plants are substituted by hydrogen liquefaction plants,
and because less carbon needs to be handled in the system, direct air capture is
no longer required. The higher energy efficiency of liquid hydrogen in shipping
also lowers the requirements for wind and solar buildout. However, the cost
differences should be viewed in the context that the costs of ships fueled by
liquid hydrogen are likely to be considerably higher than those fueled by
methanol.\citeS{johnstonShippingSunshine2022} The spatial patterns of hydrogen
infrastructure buildout remain largely unchanged
(\cref{fig:hydrogen-inframaps:lh2}).

\begin{SCfigure}
    \centering
    \includegraphics[width=0.75\textwidth]{sensitivity-h2-new-shipping.pdf}
    \caption{Cost benefits of electricity and hydrogen network infrastructure with use of liquid hydrogen in shipping instead of methanol.}
    \label{fig:sensitivity-shipping}
\end{SCfigure}

\begin{figure}
    \centering
    \includegraphics[width=\textwidth]{diff-cost-shipping}
    \caption{Differences in total system cost for usage of liquid hydrogen in shipping compared to methanol.}
    \label{fig:sensitivity-shipping-diff}
\end{figure}

\begin{figure}
    \centering
    \makebox[\textwidth][c]{
        \begin{subfigure}[t]{0.6\textwidth}
            \centering
            \caption{hydrogen infrastructure with 2030 costs, methanol in shipping, no imports}
            \includegraphics[width=\textwidth]{\hyrun/maps/elec_s_181_lv1.0__Co2L0-3H-T-H-B-I-A-solar+p3-linemaxext10-h2_network_2050.pdf}
            \label{fig:hydrogen-inframaps:base}
        \end{subfigure}
        \begin{subfigure}[t]{0.6\textwidth}
            \centering
            \caption{hydrogen infrastructure with 2050 cost assumptions}
            \includegraphics[width=\textwidth]{\costrun/maps/elec_s_181_lv1.0__Co2L0-3H-T-H-B-I-A-solar+p3-linemaxext10-h2_network_2050.pdf}
            \label{fig:hydrogen-inframaps:costs}
        \end{subfigure}
    }
    \makebox[\textwidth][c]{
        \begin{subfigure}[t]{0.6\textwidth}
            \centering
            \caption{hydrogen infrastructure with liquid hydrogen in shipping}
            \includegraphics[width=\textwidth]{\shprun/maps/elec_s_181_lv1.0__Co2L0-3H-T-H-B-I-A-solar+p3-linemaxext10-h2_network_2050.pdf}
            \label{fig:hydrogen-inframaps:lh2}
        \end{subfigure}
        \begin{subfigure}[t]{0.6\textwidth}
            \centering
            \caption{hydrogen infrastructure with all liquid hydrocarbons imported}
            \includegraphics[width=\textwidth]{\imprun/maps/elec_s_181_lv1.0__Co2L0-3H-T-H-B-I-A-solar+p3-linemaxext10-h2_network_2050.pdf}
            \label{fig:hydrogen-inframaps:imports}
        \end{subfigure}
        }
    \caption{Hydrogen infrastructure buildout with different cost assumptions (\cref{fig:hydrogen-inframaps:costs}), shipping fuel (\cref{fig:hydrogen-inframaps:lh2}) or import levels (\cref{fig:hydrogen-inframaps:imports}) in scenarios without electricity grid reinforcements.}
    \label{fig:hydrogen-inframaps}
\end{figure}

\subsection{Temporal Resolution}
\label{sec:si:sensitivity-time}

In \cref{fig:sensitivity-time}, we varied the temporal resolution for the
scenario with both power and hydrogen network expansion with a reduced spatial
resolution of 90 regions. In this way, we were computationally able to sweep the
time resolution from a 6-hourly model up to an hourly model. Total energy system
costs and optimised capacities are shown relative to the hourly model.

Overall, with a system cost difference of -0.35\% the error induced by
resampling the model from an hourly to a 3-hourly resolution is small and
justifies a model size reduction by factor 3. The temporal aggregation causes a
minor underestimation of short-term battery storage and offshore wind as well as
a minor overestimation of solar photovoltaics and hydrogen storage. This trend
intensifies with coarser temporal resolution, such that with a 6-hourly
resolution the system cost deviation exceeds 2.5\% since balancing needs for
solar electricity are discounted, which makes this technology more attractive.

\begin{figure}
    \centering
    \begin{subfigure}[t]{\textwidth}
        \centering
        \caption{differences in total energy system cost compared to hourly resolution}
        \includegraphics[width=.9\textwidth]{diff-time-opt-cost}
        \label{fig:sensitivity-time-cost}
    \end{subfigure}
    \begin{subfigure}[t]{\textwidth}
        \centering
        \caption{differences in generation and conversion capacities compared to hourly resolution}
        \includegraphics[width=.9\textwidth]{diff-time-opt-capacity}
        \label{fig:sensitivity-time-cap}
    \end{subfigure}
    \caption{ Total system costs and optimised capacities for varying temporal
    resolutions relative to hourly resolution. The comparison refers to scenario
    with both power grid reinforcements and hydrogen network expansion. }
    \label{fig:sensitivity-time}
\end{figure}

\subsection{Spatial Resolution}
\label{sec:si:sensitivity-space}

In \cref{fig:sensitivity-space}, we also varied the spatial resolution of the
model from a one-node-per-country version (37 regions) to 181 regions for the
scenario without power or hydrogen network expansion. Total energy system
costs and optimised capacities are shown relative to the 181-region model.

Compared to the model with 181 regions, a one-node-per-country resolution
underestimates system cost by 4.3\%, favouring remote offshore wind over more
localised production with solar photovoltaics and batteries. The differences can
be explained by a combination of two opposing
effects.\citeS{frysztackiStrongEffect2021a} The aggregation of the transmission
networks lifts bottlenecks within clustered regions, lowering system costs. On
the other hand, the aggregation of wind and solar capacity factors blur the most
productive sites, increasing costs. In terms of system costs, the error induced
by reducing the spatial resolution from 181 regions to 128 regions (-0.47\%) is
comparable to the error caused by choosing 3-hourly over hourly time resolution
(-0.35\%).

\begin{figure}
    \centering
    \begin{subfigure}[t]{\textwidth}
        \centering
        \caption{differences in system cost compared to 181-region model}
        \includegraphics[width=.9\textwidth]{diff-space-1p0-no_H2_grid-cost}
        \label{fig:sensitivity-space-cost}
    \end{subfigure}
    \begin{subfigure}[t]{\textwidth}
        \centering
        \caption{differences in generation and conversion capacities compared to 181-region model}
        \includegraphics[width=.9\textwidth]{diff-space-1p0-no_H2_grid-capacity}
        \label{fig:sensitivity-space-cap}
    \end{subfigure}
    \caption{ Total system costs and optimised capacities for varying spatial
    resolutions relative to 181-regions model. The case with 37 regions corresponds to
    a single node per country and synchronous zone. The comparison refers to the scenario
    with neither power grid reinforcements nor hydrogen network expansion. }
    \label{fig:sensitivity-space}
\end{figure}
