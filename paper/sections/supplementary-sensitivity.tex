\section{Sensitivity Analysis}
\label{sec:si:sensitivity}

\subsection{Electricity Grid Reinforcement Restrictions}
\label{sec:si:lv}

\begin{figure}
    \centering
    \makebox[\textwidth][c]{
    \begin{subfigure}[t]{1.2\textwidth}
        \centering
        \caption{Sensitivity of total system cost towards electricity transmission grid expansion limits}
        \includegraphics[width=\textwidth]{lv-sensitivity.pdf}
        \label{fig:lv-restriction}
    \end{subfigure}
    }
    \makebox[\textwidth][c]{
    \begin{subfigure}[t]{1.2\textwidth}
        \centering
        \caption{Sensitivity of total system cost towards onshore wind expansion limits}
        \includegraphics[width=\textwidth]{onw-sensitivity.pdf}
        \label{fig:onw-restriction}
    \end{subfigure}
    }
    \caption{Sensitivity of total system cost towards electricity transmission grid expansion limits and onshore wind restrictions.
    The sweep for grid expansion restrictions allows full onshore wind potentials.
    The extreme case (a, left) removes all existing power transmission lines as well.
    The sweep for onshore wind potential restrictions allows power grid reinforcements by up to 25\% of today's transmission capacities.
    The extreme case (b, left) combines no grid expansion with no onshore wind potentials.}

    \label{fig:lv-onw-restriction}
\end{figure}

In the following sensitivity runs, the model is allowed to build new electricity
transmission infrastructure wherever is cost-optimal, but the total volume of
new transmission capacity (sum of line length times capacity, TWkm) is
successively limited. The volume limit is given in fractions of today's grid
volume: a line volume limit of 100\% means no new capacity is allowed beyond
today's grid (since the model cannot remove existing lines); a limit of 125\%
means the total grid capacity can grow by 25\% (25\% is similar to the planned
extra capacity in the European network operators' Ten Year Network Development
Plan (TYNDP)\citeS{tyndp2018}). For this investigation, a hydrogen network
could be built.

\cref{fig:lv-restriction} shows the composition of total yearly system costs
(including all investment and operational costs) as we vary the allowed power
grid expansion, from no expansion (only today's grid) to a doubling of today's
grid capacities (the model optimises where new capacity is placed). As the grid
is expanded, total costs decrease only slightly, despite the increasing costs of
the grid. The total cost benefit of a doubling of grid capacity is around
46~bn\euro/a (6\%) corresponding to an expansion of 715 TWkm. However, over half
of the benefit (27~bn\euro/a, 3.5\%) is available already at a 25\% expansion
corresponding to an expansion of 447 TWkm.

\cref{fig:lv-restriction} also includes a scenario where today's electricity
transmission infrastructure is completely removed from the model, similar to an
electricity system study on geographic trade-offs by Tröndle et
al.\citeS{trondleTradeOffsGeographic2020} While doubling the transmission grid
yields a benefit of 46~bn\euro/a, removing what exists incurs a cost of
108~bn\euro/a. The lack of electricity grid is mostly compensated by more solar
PV generation, battery storage and re-electrified hydrogen.

\subsection{Onshore Wind Potential Elimination}
\label{sec:si:onw}

\begin{figure}
    \centering
    \makebox[\textwidth][c]{
        \begin{subfigure}[t]{0.6\textwidth}
            \centering
            \caption{hydrogen network}
            \includegraphics[height=0.42\textheight]{\hyrun/maps/elec_s_181_lv1.0__Co2L0-3H-T-H-B-I-A-solar+p3-linemaxext10-onwind+p0-h2_network_2050.pdf}
            \label{fig:no-onw:h2}
        \end{subfigure}
        \begin{subfigure}[t]{0.6\textwidth}
            \centering
            \caption{energy balance}
            \includegraphics[height=0.42\textheight]{\hyrun/elec_s_181_lv1.0__Co2L0-3H-T-H-B-I-A-solar+p3-linemaxext10-onwind+p0_2050/import-export-total-200.pdf}
            \label{fig:no-onw:io}
        \end{subfigure}
        }
    \begin{subfigure}[t]{0.6\textwidth}
        \centering
        \caption{system cost}
        \includegraphics[height=0.42\textheight]{\hyrun/maps/elec_s_181_lv1.0__Co2L0-3H-T-H-B-I-A-solar+p3-linemaxext10-onwind+p0-costs-all_2050.pdf}
        \label{fig:no-onw:tsc}
    \end{subfigure}
    \caption{Maps of regional energy balance, hydrogen network and production sites, and spatial and technological distribution of system costs for a scenario without onshore wind and without power grid expansion.}
    \label{fig:no-onw}
\end{figure}

Like building new power transmission lines, the deployment of onshore wind may
not always be socially accepted, such that it may not be possible to leverage
its full potential.\citeS{mckennaScenicnessAssessment2021,weinandImpactPublic2021,weinandExploringTrilemma2021} In the following additional sensitivity analysis, we explore
the hypothetical impact of restricting the installable potentials of onshore
wind down to zero (\cref{fig:no-onw}).

We find that as onshore wind is eliminated, costs rise by \euro~92 bn/a (12\%)
when the electricity grid is fixed to today's capacities, but a hydrogen network
can still be developed. In comparison to the least-cost solution with full
network expansion, this solution is 19\% more expensive. A solution in which
neither a hydrogen network could be developed would be 23\% more expensive.
\cref{sec:si:onw-compromise} presents further intermediate results between full
and no onshore wind expansion for scenarios with hydrogen network expansion and
TYNDP-equivalent power grid reinforcements. The model substitutes onshore wind,
particularly in the British Isles, for higher investment in offshore wind in the
continental shores of the North Sea and solar generators plus batteries in
Southern and Central Europe (\cref{fig:no-onw:tsc}).  Without onshore wind, the
potentials for rooftop solar PV and fixed-pole offshore wind in Europe are
largely exhausted, such that in this self-sufficient scenario for Europe, the
effect of installable potentials becomes critical.

% Because offshore capacities are
% concentrated near coastlines, and grid capacity is restricted, total spending on
% hydrogen electrolysers and networks also increases to absorb the increased
% offshore generation.

Whereas with onshore wind, we observe both wind-backed electrolysis in
Northwestern Europe and solar-backed hydrogen production in Southern Europe, the
latter becomes the dominant producer of hydrogen if the development of onshore
wind capacities is restricted (\cref{fig:no-onw:h2,fig:no-onw:io}). This shift
in hydrogen infrastructure also impacts the share of gas pipelines being
retrofitted for hydrogen transport. As the Iberian Peninsula becomes a preferred
region for hydrogen production but has a more sparse gas transmission network
today, the rate of retrofitted pipeline capacity reduces from 65\% to 58\%. Many
new hydrogen pipelines are built to connect Spain with France, but also to
connect Denmark to Germany and Greece to Italy. Gas pipeline retrofitting is
then concentrated in Germany, Austria and Italy.

\begin{figure}
    \centering
    \includegraphics[width=.9\textwidth]{sensitivity-h2.pdf}
    \caption{Varying cost benefits of hydrogen network infrastructure and changes in system composition as power grid and onshore wind expansion options are altered. The cost benefit of a hydrogen network varies between 1.6\% and 3.7\% across all scenarios shown.}
    \label{fig:h2-restriction-w-onw}
\end{figure}

The cost benefit of a hydrogen network is similar whether or not onshore wind
capacities are built in Europe, even though the hydrogen network topology is
then built around supply from solar PV from Southern Europe and offshore wind in
the North Sea rather than from onshore wind in Northwestern Europe. As
\cref{fig:h2-restriction-w-onw} illustrates, the net benefit is again strongest
when power grid expansion is restricted. If both onshore wind and power grid
expansion are excluded, costs for a system without a hydrogen network option
were by 32~bn\euro/a (3.7\%) higher. With cost-optimal electricity grid
reinforcement, the net benefit of a hydrogen network is lower with
13~bn\euro/a (1.7\%).

\subsection{Compromises on Onshore Wind Potential Restrictions}
\label{sec:si:onw-compromise}

In the following sensitivity runs, the maximum installable capacity of onshore
wind is successively restricted down to zero at each node. The upper limit is
derived from land use restriction and yields a maximum technical potential
corresponding to about \SI{481}{\giga\watt} for Germany. For this investigation,
a compromise electricity grid expansion by 25\% compared to today and no limits
on hydrogen network infrastructure are assumed.

In this case, system costs rise by 77~bn\euro/a (10\%)
by restricting the installable potentials of onshore down to zero. Just as in
the case of restricted line volumes, \cref{fig:onw-restriction} reveals a
nonlinear rise in system costs: if we constrain the model to 25\% of the onshore
potential (around 120~GW for Germany), costs rise by only 46~bn\euro/a (6\%).
Thereby, 25\% of the onshore wind potential may represent a social compromise
between total system cost, and social concerns about onshore wind development.

In comparison, Schlachtberger et al.~\citeS{schlachtbergerCostOptimal2018} found
a similar change between 9\% and 12\% in system costs in an electricity-only
model when onshore wind potentials were restricted across various grid expansion
limitations. The the biggest change was observed when the power grid could not
be reinforced. Onshore wind was largely replaced with offshore wind in that
model. Unlike that model, here we have a higher grid resolution (181 versus 30
regions) which allows us to better assess the grid integration costs of offshore
wind. Our results show that moderate power grid expansion is particularly
important when onshore wind development is severely limited. For the extreme
case where no onshore wind capacities would be built, reducing power grid
expansion from 25\% to none incurs another rise in system cost of an additional
43~bn\euro/a (6\%).


\subsection{Using Technology and Cost Projections for 2050}
\label{sec:si:sensitivity-costs}

Diminishes benefit of electricity grid expansion.

\begin{figure}
    \centering
    \includegraphics[width=0.7\textwidth]{sensitivity-h2-new-costs.pdf}
    \caption{Benefits of electricity and hydrogen network infrastructure with cost projections for 2050.}
    \label{fig:sensitivity-costs}
\end{figure}

\begin{figure}
    \centering
    \begin{subfigure}[t]{\textwidth}
        \centering
        \caption{differences in system cost}
        \includegraphics[width=\textwidth]{diff-cost-costs}
        \label{fig:sensitivity-costs-cost}
    \end{subfigure}
    \begin{subfigure}[t]{\textwidth}
        \centering
        \caption{differences in capacities}
        \includegraphics[width=\textwidth]{diff-capacity-costs}
        \label{fig:sensitivity-costs-cap}
    \end{subfigure}
    \begin{subfigure}[t]{\textwidth}
        \centering
        \caption{differences in storage capacities}
        \includegraphics[width=\textwidth]{diff-storage-costs}
        \label{fig:sensitivity-costs-sto}
    \end{subfigure}
    \caption{Differences to main scenarios}
    \label{fig:sensitivity-costs-diff}
\end{figure}

% \includegraphics[width=.9\textwidth]{diff-internal-cost-costs}

\subsection{Importing all Liquid Hydrocarbons}
\label{sec:si:sensitivity-imports}

\begin{figure}
    \centering
    \includegraphics[width=0.7\textwidth]{sensitivity-h2-new-import.pdf}
    \caption{Benefits of electricity and hydrogen network infrastructure with imports of all liquid hydrocarbons.}
    \label{fig:sensitivity-imports}
\end{figure}

Majority of hydrogen derivatives


\begin{figure}
    \centering
    \begin{subfigure}[t]{\textwidth}
        \centering
        \caption{differences in system cost}
        \includegraphics[width=\textwidth]{diff-cost-import}
        \label{fig:sensitivity-import-cost}
    \end{subfigure}
    \begin{subfigure}[t]{\textwidth}
        \centering
        \caption{differences in capacities}
        \includegraphics[width=\textwidth]{diff-capacity-import}
        \label{fig:sensitivity-import-cap}
    \end{subfigure}
    \begin{subfigure}[t]{\textwidth}
        \centering
        \caption{differences in storage capacities}
        \includegraphics[width=\textwidth]{diff-storage-import}
        \label{fig:sensitivity-import-sto}
    \end{subfigure}
    \caption{Differences to main scenarios}
    \label{fig:sensitivity-import-diff}
\end{figure}

% \includegraphics[width=.9\textwidth]{diff-internal-cost-import}

\subsection{Liquid Hydrogen in Shipping}
\label{sec:si:sensitivity-shipping}

\begin{figure}
    \centering
    \includegraphics[width=0.7\textwidth]{sensitivity-h2-new-shipping.pdf}
    \caption{Benefits of electricity and hydrogen network infrastructure with use of liquid hydrogen in shipping instead of methanol.}
    \label{fig:sensitivity-shipping}
\end{figure}


\begin{figure}
    \centering
    \begin{subfigure}[t]{\textwidth}
        \centering
        \caption{differences in system cost}
        \includegraphics[width=\textwidth]{diff-cost-shipping}
        \label{fig:sensitivity-shipping-cost}
    \end{subfigure}
    \begin{subfigure}[t]{\textwidth}
        \centering
        \caption{differences in capacities}
        \includegraphics[width=\textwidth]{diff-capacity-shipping}
        \label{fig:sensitivity-shipping-cap}
    \end{subfigure}
    \begin{subfigure}[t]{\textwidth}
        \centering
        \caption{differences in storage capacities}
        \includegraphics[width=\textwidth]{diff-storage-shipping}
        \label{fig:sensitivity-shipping-sto}
    \end{subfigure}
    \caption{Differences to main scenarios}
    \label{fig:sensitivity-shipping-diff}
\end{figure}

% \includegraphics[width=.9\textwidth]{diff-internal-cost-shipping}

How it increases benefit of hydrogen network.

\subsection{Temporal Resolution}
\label{sec:si:sensitivity-time}

Text

\begin{figure}
    \centering
    \begin{subfigure}[t]{\textwidth}
        \centering
        \caption{differences in system cost}
        \includegraphics[width=.9\textwidth]{diff-time-opt-cost}
        \label{fig:sensitivity-time-cost}
    \end{subfigure}
    \begin{subfigure}[t]{\textwidth}
        \centering
        \caption{differences in capacities}
        \includegraphics[width=.9\textwidth]{diff-time-opt-capacity}
        \label{fig:sensitivity-time-cap}
    \end{subfigure}
    \caption{Differences to hourly resolution: cost and capacities.}
    \label{fig:sensitivity-time}
\end{figure}

\subsection{Spatial Resolution}
\label{sec:si:sensitivity-space}

Text

\begin{figure}
    \centering
    \begin{subfigure}[t]{\textwidth}
        \centering
        \caption{differences in system cost}
        \includegraphics[width=.9\textwidth]{diff-space-1p0-no_H2_grid-cost}
        \label{fig:sensitivity-space-cost}
    \end{subfigure}
    \begin{subfigure}[t]{\textwidth}
        \centering
        \caption{differences in capacities}
        \includegraphics[width=.9\textwidth]{diff-space-1p0-no_H2_grid-capacity}
        \label{fig:sensitivity-space-cap}
    \end{subfigure}
    \caption{Differences to spatial resolution of 181 regions: cost and capacities.}
    \label{fig:sensitivity-space}
\end{figure}
