\section*{Common features across scenarios of European carbon-neutrality}
\label{sec:es}
\addcontentsline{toc}{section}{\nameref{sec:es}}

Overall, the total system costs are dominated by investments in generation from wind
and solar, and conversion from power to heat (primarily heat pumps)
and to hydrogen and liquid hydrocarbons (for transport fuels and as a
feedstock for the chemicals industry).

\cref{fig:tsc:w-el-w-h2} shows the spatial distribution of the investments
for the least-cost solution with full electricity and hydrogen network expansion.

While solar capacity is spread relatively evenly around the continent with a
slight skew torwards Southern Europe, both onshore and offshore wind are
concentrated around the North Sea and British Isles. New transmission capacity
is concentrated in regions so that they help the integration of wind in these
regions, and the transport of wind energy to inland locations.

Electrolyzer capacities for power-to-hydrogen see a massive scale-up between
1057~GW and 1297~GW depending on the permitted energy transport infrastructure.
Their locations correlate strongly with wind capacities ($R^2=0.85$).

A new network transports hydrogen from these sites of production to the rest of
Europe where hydrogen is consumed by industry (for ammonia, organic chemicals
and steel production), for heavy-duty transport and in fuel cells for power and
heat backup.

Of the huge hydrogen production (2437~TWh/a), most of it (1425~TWh/a) goes to
Fischer-Tropsch fuels for organic chemicals and transport fuels, of which
356~TWh/a waste heat enter the district heating networks. A total of 775~TWh/a
is used in shipping (two thirds) and land (one third) transport. 226~TWh/a are
used in the industry sector and for methanation. 79~TWh/a of hydrogen are lost
in conversion.

If the electricity grid expansion is restricted but hydrogen can be transported,
more hydrogen is produced to be re-electrified in critical phases and locations
of system operation (100 TWh\el).

Methane production is limited to biogas (346~TWh/a) and some fossil gas
(371~TWh/a), the latter of whose emissions are offset by bioenergy with carbon
capture. Direct air capture with sequestration and synthetic methane production
were only observed if both hydrogen and electricity network could not be
expanded. Methane is used for process heat in some industry applications and as
a heating backup for power-to-heat units.

From our model we infer a required carbon price between
\SI{435}{\sieuro\per\tco} with and \SI{513}{\sieuro\per\tco} without network
expansion to achieve \co neutrality in Europe.

compared to electricity, heating and transport are strongly peaked
- heating is strongly seasonal but also with synoptic variations
- transport has strong daily periodicity
There are difficult periods in winter with
- low wind and solar (high prices)
- high space heating demand
- low air temperatures, which are bad for air-sourced heat pump performance
- less smart solution: backup gas boilers
- smart solution: building retrofiting, TES in district heating, CHP

average capacity factors of electrolysis 36\% to 40\%, flexible operation
leveraging periods with high wind speeds across Europe (see
\cref{fig:output-ts-1}), buffered in hydrogen cavern storage for stable
production of synthetic hydrocarbons, whereas Fischer-Tropsch runs 87\% to 95\%,
only interrupted in winter periods with low wind speeds and high heat demands

dominant flow directions of energy by carrier?
- supplemental figures %\cref{}
- hydrogen: Ireland and England to Netherlands, Belgium and Western Germany; Scotland via North Sea to all of Germany; France to Southwest Germany; North East Spain via France to Switzerland
- electricity: more mixed patterns - balancing rather net transport over long distances

hydrogen storage:
- storage capacities? 62-66 TWh with hydrogen network, 32-35 TWh without hydrogen network; similar injection patterns; \cite{caglayanImpactDifferent2019} has 130~TWh of cavern storage
- all storage is cavern storage; no storage in steel tanks (neither hydrogen nor electricity network: 1.3 TWh steel tank (4\%))
- placed Northern Ireland, Denmark, England

What drives the hydrogen network?
- industry demand for hydrogen in areas with less attractive renewable potentials
- electricity grid bottlenecks / alternative energy transport
- waste heat of synthetic fuel production can only be leveraged in district heating networks of urban areas
- move produced hydrogen to cavern storage locations, rather than storage in steel tanks
- because gas and oil can be moved freely in the model, the spatial distribution of their demands is not a siting factor for synfuel production

\section*{Hydrogen network benefit is robust, strongest without power grid expansion}
\label{sec:h2}
\addcontentsline{toc}{section}{\nameref{sec:h2}}

\cref{fig:sensitivity-h2}

% electricity grid restriction

If the electricity grid can be expanded, total costs decrease
slightly, despite the increasing costs of the grid.

The
total cost benefit of power grid expansion is around
\euro47~billion per year.

The grid enables
renewable resources with better capacity factors to be integrated from
further away, resulting in lower capacity needs for solar and
wind. The grid also allows renewable variations to be smoothed in space,
resulting in lower hydrogen demand for balancing power and heat.

The restriction of grid
expansion leads to more local production from solar and more hydrogen
production.

consequently, as grid is expanded, costs reduce from solar, PtX and H2 network, more offshore wind

\cref{sec:si:lv} presents additional intermediate results between a doubling of
power grid capacity and no grid expansion.

% hydrogen network benefit

The presence of the hydrogen network can reduce system costs by up to 5.7\%.

The cost of hydrogen network 5.6-7.9 billion per year

Net benefit is much higher: 31-46 billion per year (4.1-5.7\%)

hydrogen network is robustly beneficial infrastructure
benefit is strongest when there is no power grid expansion

conversely, the benefit of electrcity grid reinforcement is strongest
when there is no hydrogen network option

a new network
of underground hydrogen storage and pipelines in Europe helps to balance
generation from renewables in time and space.

% overall / compared

We find that the overall system costs are not overly affected
by restrictions on electricity or hydrogen transmission.

Moderate levels of hydrogen and/or electricity grid expansion provide cost
savings, but they are small compared to total system costs.

systems without grid expansion are feasible

The high level of synthetic fuel production and exchange between the nodes, which is
necessary for industry, transport and backup electricity and heating
applications, provides sufficient flexibility to manage these
restrictions.

% comparison of H2 to power grid

the total net benefit of combined hydrogen and electricity grid expansion (beyond today) is 93 bn\euro/a (12.2\%)
- half of this benefit can be achieved by exclusive hydrogen network expansion
- whereas two thirds of the benefit can be reaped by only electricity grid expansion
- both are important for costs, but power grid expansion brings more cost benefit
- hydrogen grid is not a perfect substitute for electricity transmission
- rather transmission via eletrons or molecules have complementary strengths for long-distance transport of energy
- cost reductions are largely additive (total benefit 93, 46+62=108 for each individually, 108/93=1.16)
- hydrogen network can partially substitute transmission expansion (up to 46/62=75\% of electricity system benefit), strongest together

Depending on the level of power grid expansion, between 342 and 422 TWkm of
hydrogen pipelines are built. The higher value is obtained when the hydrogen
network partially compensates for the lack of electricity grid reinforcement.

\begin{SCfigure}
    \centering
    \includegraphics[width=0.8\textwidth]{sensitivity-h2-new.pdf}
    \caption{Benefits of electricity and hydrogen network infrastructure.}
    \label{fig:sensitivity-h2}
\end{SCfigure}


\section*{Repurposing gas pipelines lowers costs and shapes routes}
\label{sec:repurposed}
\addcontentsline{toc}{section}{\nameref{sec:repurposed}}


\cref{fig:h2-network}


\cref{fig:network-stats:twkm} shows statistics on the total electricity and
hydrogen transmission capacity built as well as how much energy is moved through
the respective networks. For the hydrogen network a distinction between
retrofitted and new pipelines is made. For the electricity network a distinction
is made between existing and added capacity or how much energy is moved via HVAC
or HVDC power lines.

A hydrogen network can take over and exceed the electricity grid
in terms of the amount of energy transported over long distances, balancing
renewables both in space (with the network) and time (with underground storage).

% h2 network topology

Restricting hydrogen expansion has only a small effect on cost-optimal levels of
grid expansion. While with a hydrogen network, the power grid capacity is a
little more than doubled, without it the cost-optimal power grid capacities are
10\% higher.

The pipeline connections between regions may have optimised capacities as high as 50 GW.

H2 pipeline is particularly attractive when end-use is hydrogen-based product
- hydrogen: 106-226~\euro/MW/km
- electricity: 400~\euro/MW/km
- annualised cost basis: electricity 1.6 times more expensive than H2, if retrofitted even 3.4 times
- not attractive for electricity end-use: round-trip efficiency electricity-H2-electricity 34\% (a third), more generation capacity needed
- however, electricity demand and direct electrification dominate the demand for hydrogen-based products; can explain benefit of power grid expansion

% repurposing capacities

Repurposing a gas pipeline to transport hydrogen is assumed to cost around half
that of building a new hydrogen pipeline (117 versus 250 \euro/MW/km). This
estimate includes the cost for compressor substitution. In consequence, the
hydrogen network topology does not always follow the shortest routes.

With power grid expansion, 58\% of the hydrogen network uses repurposed gas
pipelines. When the electricity grid cannot be reinforced, this share rises to
66\%.

Up to a third of the existing gas network (TWkm) are retrofitted to transport
hydrogen instead. This still leaves large  gas network capacities that are
neither used for hydrogen nor methane transport, particularly in Germany,
Poland, Italy and the North Sea as supplementary runs with full gas network
resolution demonstrate (\cref{fig:si:gas-leftover})

A little more than 40\% of retrofittable pipelines fully use their conversion
potential to hydrogen.

The largest new hydrogen pipelines are built on the British Isles, between
Denmark and Germany, inside Belgium and the Netherlands, and in the North-East
of Spain.

The most notable corridors for gas pipeline retrofitting, are located offshore
within the North Sea, in South-East England and crossing the English Channel, as
well as inside Germany, Austria and Northern Italy.

The large existing gas transmission capacities in Southern Italy and Eastern
Europe are not repurposed for hydrogen transport. However, this result would
likely change if attractive energy import options through these corridors were
considered.

% repurposing energy moved

\cref{fig:network-stats:ewhkm}
When both hydrogen and electricity grid expansion are allowed, both networks
transport approximately the same amount of energy. As the nominal capacity of
the hydrogen network is less than half that of the optimised electricity grid,
this means that the utilisation rate of the hydrogen network is higher (59\%
versus 35\%).

The hydrogen network plays a dominant role transporting energy around Europe
when grid expansion is restricted: around three times more energy is moved in
the hydrogen network (2.84~TWhkm/h) than in the electricity network
(0.95~EWhkm). At the same time, the total amount of energy moved as hydrogen or
electricity is reduced by only 22\%.

% import exports

\cref{fig:io}

overall pattern: energy surplus in coastal and most Southern regions supplying
to the inland regions of Europe with high demands but less attractive renewable potentials.

energy surplus (up to 200 TWh net surplus)
- wind-rich regions of Europe
- offshore wind in Denmark (in particular with eletricity grid reinforcement)
- onshore wind in Ireland (if grid expansion is restricted, production site for feedstock which can still be transported)
- but also other individual regions in Spain, Greece, France and Germany

energy deficit (up to 150 TWh net deficit)
- urban areas around London and Paris
- industrial cluster between Rotterdam and Ruhr valley

hydrogen supply is more regionally imbalanced than electricityl supply

if hydrogen transport is restricted

\begin{figure}
    \centering
    % \makebox[\textwidth][c]{
    \begin{subfigure}[t]{0.49\textwidth}
        \centering
        \caption{transmission capacity built}
        \includegraphics[width=\textwidth]{twkm}
        \label{fig:network-stats:twkm}
    \end{subfigure}
    \begin{subfigure}[t]{0.49\textwidth}
        \centering
        \caption{energy moved}
        \includegraphics[width=\textwidth]{ewhkm}
        \label{fig:network-stats:ewhkm}
    \end{subfigure}
    % }
    \caption{Transmission capacity built and energy moved for various scenarios.
        For the hydrogen network a distinction between retrofitted and new pipelines is made.
        For the electricity network a distinction is made between existing and added capacity
        or how much energy is moved via HVAC or HVDC power lines.}
    \label{fig:network-stats}
\end{figure}

\begin{figure}
    \centering
    \makebox[\textwidth][c]{
        \begin{subfigure}[t]{0.6\textwidth}
            \centering
            \caption{With grid reinforcement, with hydrogen network}
            \includegraphics[width=\textwidth, trim=0cm 0cm 7cm 0cm, clip]{\hyrun/maps/elec_s_181_lvopt__Co2L0-3H-T-H-B-I-A-solar+p3-linemaxext10-costs-all_2030.pdf}
            \label{fig:tsc:w-el-w-h2}
        \end{subfigure}
        \begin{subfigure}[t]{0.6\textwidth}
            \centering
            \caption{With grid reinforcement, without hydrogen network}
            \includegraphics[width=\textwidth, trim=0cm 0cm 7cm 0cm, clip]{\hyrun/maps/elec_s_181_lvopt__Co2L0-3H-T-H-B-I-A-solar+p3-linemaxext10-noH2network-costs-all_2030.pdf}
            \label{fig:tsc:w-el-wo-h2}
        \end{subfigure}
    }
    \makebox[\textwidth][c]{
        \begin{subfigure}[t]{0.6\textwidth}
            \centering
            \caption{Without grid reinforcement, with hydrogen network}
            \includegraphics[width=\textwidth, trim=0cm 0cm 7cm 0cm, clip]{\hyrun/maps/elec_s_181_lv1.0__Co2L0-3H-T-H-B-I-A-solar+p3-linemaxext10-noH2network-costs-all_2030.pdf}
            \label{fig:tsc:wo-el-w-h2}
        \end{subfigure}
        \begin{subfigure}[t]{0.6\textwidth}
            \centering
            \caption{Without grid reinforcement, without hydrogen network}
            \includegraphics[width=\textwidth, trim=0cm 0cm 7cm 0cm, clip]{\hyrun/maps/elec_s_181_lv1.0__Co2L0-3H-T-H-B-I-A-solar+p3-linemaxext10-noH2network-costs-all_2030.pdf}
            \label{fig:tsc:wo-el-wo-h2}
        \end{subfigure}
    } \caption{}
    \label{fig:tsc}
\end{figure}

\begin{figure}
    \centering
    \makebox[\textwidth][c]{
        \begin{subfigure}[t]{0.6\textwidth}
            \centering
            \caption{with grid reinforcement}
            \includegraphics[width=\textwidth]{\hyrun/maps/elec_s_181_lvopt__Co2L0-3H-T-H-B-I-A-solar+p3-linemaxext10-h2_network_2030.pdf}
            \label{fig:h2-network:w-el}
        \end{subfigure}
        \begin{subfigure}[t]{0.6\textwidth}
            \centering
            \caption{without grid reinforcement}
            \includegraphics[width=\textwidth]{\hyrun/maps/elec_s_181_lv1.0__Co2L0-3H-T-H-B-I-A-solar+p3-linemaxext10-h2_network_2030.pdf}
            \label{fig:h2-network:wo-el}
        \end{subfigure}
    }
    \caption{Optimised hydrogen network and production sites with and without electricity grid reinforcement.}
    \label{fig:h2-network}
\end{figure}



\begin{figure}
    \centering
    \makebox[\textwidth][c]{
        \begin{subfigure}[t]{0.6\textwidth}
            \centering
            \caption{With grid reinforcement, with hydrogen network}
            \includegraphics[width=\textwidth]{\hyrun/elec_s_181_lvopt__Co2L0-3H-T-H-B-I-A-solar+p3-linemaxext10_2030/import-export-total-200.pdf}
            \label{fig:io:w-el-w-h2}
        \end{subfigure}
        \begin{subfigure}[t]{0.6\textwidth}
            \centering
            \caption{With grid reinforcement, without hydrogen network}
            \includegraphics[width=\textwidth]{\hyrun/elec_s_181_lvopt__Co2L0-3H-T-H-B-I-A-solar+p3-linemaxext10-noH2network_2030/import-export-total-200.pdf}
            \label{fig:io:w-el-wo-h2}
        \end{subfigure}
    }
    \makebox[\textwidth][c]{
        \begin{subfigure}[t]{0.6\textwidth}
            \centering
            \caption{Without grid reinforcement, with hydrogen network}
            \includegraphics[width=\textwidth]{\hyrun/elec_s_181_lv1.0__Co2L0-3H-T-H-B-I-A-solar+p3-linemaxext10_2030/import-export-total-200.pdf}
            \label{fig:io:wo-el-w-h2}
        \end{subfigure}
        \begin{subfigure}[t]{0.6\textwidth}
            \centering
            \caption{Without grid reinforcement, without hydrogen network}
            \includegraphics[width=\textwidth]{\hyrun/elec_s_181_lv1.0__Co2L0-3H-T-H-B-I-A-solar+p3-linemaxext10-noH2network_2030/import-export-total-200.pdf}
            \label{fig:io:wo-el-wo-h2}
        \end{subfigure}
    }
    \caption{Regional total energy balances for scenarios with and without
        electricity or hydrogen network expansion. The Lorenz curves on the upper
        left of each map depict the regional inequity of electricity, hydrogen,
        methane and oil supply relative to demand.}
    \label{fig:io}
\end{figure}

\section*{Onshore wind expansion restrictions shift hydrogen infrastructure}
\label{sec:onwind}
\addcontentsline{toc}{section}{\nameref{sec:onwind}}

\begin{figure}
    \centering
    \makebox[\textwidth][c]{
        \begin{subfigure}[t]{0.48\textwidth}
            \centering
            \caption{system cost}
            \includegraphics[height=0.27\textheight]{\hyrun/maps/elec_s_181_lv1.0__Co2L0-3H-T-H-B-I-A-solar+p3-linemaxext10-onwind+p0-costs-all_2030.pdf}
            \label{fig:no-onw:tsc}
        \end{subfigure}
        \begin{subfigure}[t]{0.4\textwidth}
            \centering
            \caption{hydrogen network}
            \includegraphics[height=0.27\textheight]{\hyrun/maps/elec_s_181_lv1.0__Co2L0-3H-T-H-B-I-A-solar+p3-linemaxext10-onwind+p0-h2_network_2030.pdf}
            \label{fig:no-onw:h2}
        \end{subfigure}
        \begin{subfigure}[t]{0.43\textwidth}
            \centering
            \caption{energy balance}
            \includegraphics[height=0.27\textheight]{\hyrun/elec_s_181_lv1.0__Co2L0-3H-T-H-B-I-A-solar+p3-linemaxext10-onwind+p0_2030/import-export-total-200.pdf}
            \label{fig:no-onw:io}
        \end{subfigure}
    }
    \caption{Maps of regional energy balance, hydrogen network and production sites, and spatial and technological distribution of system costs for a scenario without onshore wind and without power grid expansion.}
    \label{fig:no-onw}
\end{figure}

% cost impact and overall system

By restricting the installable potentials of onshore down to zero, costs rise by
\euro~104 bn/a (12\%) as onshore wind is eliminated if the electricity grid is
fixed to today's capacities. \cref{sec:si:onw} presents intermediate results
between full and no onshore wind expansion. The model substitutes onshore wind,
particularly in the British Isles, for higher investment in offshore wind in the
North Sea and solar generators in Southern Europe (\cref{fig:no-onw:tsc}).
Because offshore capacities are concentrated near coastlines, and grid capacity
is restricted, total spending on hydrogen electrolyzers and networks also
increases to absorb the increased offshore generation. Without onshore wind, the
potentials for rooftop solar PV and offshore wind in Europe are largely
exhausted, such that in this self-sufficient scenario for Europe, the
effect of installable potentials becomes critical.

% changes in hydrogen infrastructure

Whereas with onshore wind, the  British Isles and North Sea dominate hydrogen
production, Southern Europe becomes a large exporter of solar-based hydrogen if
the development of onshore wind capacities is restricted
(\crefrange{fig:no-onw:h2}{fig:no-onw:io}). This shift in hydrogen
infrastructure also impacts the share of gas pipelines being retrofitted for
hydrogen transport. As the Iberian Peninsula becomes a preferred region for
hydrogen production but has a more sparse gas transmission network, the rate of
retrofitted pipeline capacity reduces from 66\% to 53\%. Many new hydrogen
pipelines are built to connect Spain with France, but also to connect increased
hydrogen production from Danish offshore wind to Germany. Gas pipeline
retrofitting is concentrated in Germany, Austria and Italy.
