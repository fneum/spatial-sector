\section*{Cost of Electricity Grid Reinforcement Restrictions}
\label{sec:lv}
\addcontentsline{toc}{section}{\nameref{sec:lv}}



The volume limit is given in fractions of today's grid volume: a
line volume limit of 1.0 means no new capacity is allowed beyond today's grid
(since the model cannot remove existing lines); a limit of 1.25 means the total
grid capacity can grow by 25\% (25\% is similar to the planned extra capacity in
the European network operators' Ten Year Network Development Plan (TYNDP)\cite{TYNDP2016})

Figure shows the composition of total yearly system costs
(including all investment and operational costs) as we vary the allowed
grid expansion, from no expansion (only today's grid) to
a doubling of today's grid capacities (the model optimises where new
capacity is placed).

The total costs are dominated by investments in generation from wind
and solar, and conversion from power to heat (primarily heat pumps)
and to hydrogen and liquid hydrocarbons (for transport fuels and as a
feedstock for the chemicals industry). Vehicle costs are not included
here. The costs of today's system are \euro700~billion per year,
computed with the same assumptions, which is only slightly less than
these net-zero emission systems.

As the grid is expanded, total costs decrease
slightly, despite the increasing costs of the grid. The grid enables
renewable resources with better capacity factors to be integrated from
further away, resulting in lower capacity needs for solar and
wind. The grid also allows renewable variations to be smoothed in space,
resulting in lower hydrogen demand for balancing power and heat.  The
total cost benefit of a doubling of grid capacity is around
\euro44~billion per year, but over half of the benefit (\euro26~billion per year) is available already at a 25\% expansion (similar to the level seen in the TYNDP \cite{TYNDP2016}).

Figure shows the spatial distribution of the investments
for the case of a 25\% expansion. While solar capacity is spread
relatively evenly around the continent, both onshore and offshore wind
are concentrated around the North Sea and British Isles. New
transmission capacity is concentrated in HVDC lines that help the
integration of wind in these regions, and the transport of wind energy
to inland locations.

Electrolyzer capacities for power-to-hydrogen see a massive scale-up,
from several dozen MW today to 1048~GW in this scenario (note that
several 100~MW electrolyzer facilities are already in planning for the
2020s). Their locations correlate strongly with wind capacities,
particularly offshore wind. A new network
transports hydrogen from these sites of production to the rest of
Europe where hydrogen is consumed by industry (for ammonia, organic
chemicals and steel production), for heavy-duty transport and in fuel
cells for power and heat backup. Of the huge hydrogen production
(2949~TWh/a), most of it (1722~TWh/a) goes to Fischer-Tropsch fuels
for organic chemicals and transport fuels, 615~TWh/a to fuel cell CHPs
in district heating networks, and the rest to industry and transport.
The hydrogen network plays a dominant role transporting energy around
Europe when grid expansion is restricted: more energy is moved further
in the hydrogen network (209~TWhkm/h) than either the HVAC
(99~TWhkm/h) or HVDC (3.3~TWhkm/h) networks.

Methane production is limited to biogas (352~TWh/a) and some fossil
gas (388~TWh/a), the latter of whose emissions are offset by bioenergy
with carbon capture and direct air capture with sequestration.
Methane is used for process heat in some industry applications and as
a heating backup for power-to-heat units. The model has the option of
efficient power-to-methane conversion (methanation integrated with a
solid oxide electrolyzer) but does not choose to build it because of
its high capital costs compared to the direct hydrogen route.

50\% more power grid volume (expansion of 162 TWkm) and 260 TWkm of hydrogen grid

25\% more power grid volume (expansion of 81 TWkm) and 282 TWkm of hydrogen grid

no more power grid volume and 308 TWkm of hydrogen grid

direct system costs bit higher than today's system

systems without grid expansion are feasible, but more costly

as grid is expanded, costs reduce from solar, PtX and H2 network, more offshore wind

total cost benefit of extra grid: 47 billion euro per year

over half of the benefit is available at 25\% grid expansion (like TYNDP)


\newgeometry{top=0.5cm, bottom=1.5cm}
\begin{figure}
    \centering
    % \makebox[\textwidth][c]{
    \begin{subfigure}[t]{\textwidth}
        \centering
        \caption{Sensitivity of electricity transmission grid expansion limits}
        \includegraphics[width=\textwidth]{lv-sensitivity.pdf}
    \end{subfigure}
    \begin{subfigure}[t]{0.49\textwidth}
        \centering
        \caption{Cost without grid reinforcement}
        \includegraphics[width=\textwidth, trim=0cm 0cm 7cm 0cm, clip]{\lvrun/maps/elec_s_181_lv1.0__Co2L0-3H-T-H-B-I-A-solar+p3-linemaxext10-costs-all_2030.pdf}
    \end{subfigure}
    \begin{subfigure}[t]{0.49\textwidth}
        \centering
        \caption{Cost with 50\% higher grid volume}
        \includegraphics[width=\textwidth, trim=0cm 0cm 7cm 0cm, clip]{\lvrun/maps/elec_s_181_lv1.5__Co2L0-3H-T-H-B-I-A-solar+p3-linemaxext10-costs-all_2030.pdf}
    \end{subfigure}
    \begin{subfigure}[t]{0.49\textwidth}
        \centering
        \caption{Hydrogen network without grid reinforcment}
        \includegraphics[width=\textwidth]{\lvrun/maps/elec_s_181_lv1.0__Co2L0-3H-T-H-B-I-A-solar+p3-linemaxext10-h2_network_2030.pdf}
    \end{subfigure}
    \begin{subfigure}[t]{0.49\textwidth}
        \centering
        \caption{Hydrogen network with 50\% higher grid volume}
        \includegraphics[width=\textwidth]{\lvrun/maps/elec_s_181_lv1.5__Co2L0-3H-T-H-B-I-A-solar+p3-linemaxext10-h2_network_2030.pdf}
    \end{subfigure}
    % }
    \caption{Sensitivity of electricity transmission grid expansion limits.}
    \label{fig:lv-restriction}
\end{figure}
\restoregeometry


\section*{Cost of Onshore Wind Potential Restrictions}
\label{sec:onw}
\addcontentsline{toc}{section}{\nameref{sec:onw}}


If we take the case with no grid expansion and restrict the installable
potentials of onshore down to zero, costs rise by an additional \euro~42~billion
per year. The model substitutes onshore wind for
higher investment in offshore and solar generators. Because offshore capacities
are concentrated near coastlines, and grid capacity is restricted, total
spending on hydrogen electrolyzers and networks also increases to absorb the
offshore generation.

Just as in the case of restricted line volumes, the rise in costs is non-linear:
if we restrict to 25\% of the onshore potential (around 100~GW for Germany),
costs rise by only \euro~14~billion per year. 25\% of the onshore wind potential
represents a possible social compromise between total system cost, and public
acceptance concerns about onshore wind development.

Without onshore wind, solar rooftop and offshore potentials are maxxed out
- If all sectors included and Europe self-sufficient, effect of installable potentials is critical

Without onshore wind, hydrogen network looks much different:
- with: British Isles and North Sea dominate hydrogen production
- without: Southern Europe becomes much larger exporter of hydrogen

Technical potentials for onshore wind respect land usage, but do not represent
sociallly-acceptable potentials
- technical potential of 480 GW in Germany is unlikely to be built
- costs rise by 122 billion per year as we eliminate onshore wind
- rise is only 45 billion per year if we allow a quarter of technical potential (120 GW in Germany)


\newgeometry{top=0.5cm, bottom=1.5cm}
\begin{figure}
    \centering
    % \makebox[\textwidth][c]{
    \begin{subfigure}[t]{\textwidth}
        \centering
        \caption{Sensitivity of onshore wind expansion limits}
        \includegraphics[width=\textwidth]{onw-sensitivity.pdf}
    \end{subfigure}
    \begin{subfigure}[t]{0.49\textwidth}
        \centering
        \caption{Cost without onshore wind}
        \includegraphics[width=\textwidth, trim=0cm 0cm 7cm 0cm, clip]{\onwrun/maps/elec_s_181_lv1.25__Co2L0-3H-T-H-B-I-A-solar+p3-linemaxext10-onwind+p0-costs-all_2030.pdf}
    \end{subfigure}
    \begin{subfigure}[t]{0.49\textwidth}
        \centering
        \caption{Cost with full onshore wind potential}
        \includegraphics[width=\textwidth, trim=0cm 0cm 7cm 0cm, clip]{\onwrun/maps/elec_s_181_lv1.25__Co2L0-3H-T-H-B-I-A-solar+p3-linemaxext10-onwind+p0.5-costs-all_2030.pdf}
    \end{subfigure}
    \begin{subfigure}[t]{0.49\textwidth}
        \centering
        \caption{Hydrogen network without onshore wind}
        \includegraphics[width=\textwidth]{\onwrun/maps/elec_s_181_lv1.25__Co2L0-3H-T-H-B-I-A-solar+p3-linemaxext10-onwind+p0-h2_network_2030.pdf}
    \end{subfigure}
    \begin{subfigure}[t]{0.49\textwidth}
        \centering
        \caption{Hydrogen network with full onshore wind potential}
        \includegraphics[width=\textwidth]{\onwrun/maps/elec_s_181_lv1.25__Co2L0-3H-T-H-B-I-A-solar+p3-linemaxext10-onwind+p0.5-h2_network_2030.pdf}
    \end{subfigure}
    % }
    \caption{Sensitivity of onshore wind expansion limits.}
    \label{fig:onw-restriction}
\end{figure}
\restoregeometry

\section*{Benefit of a Hydrogen Network with Repurposed Gas Pipelines}
\label{sec:h2}
\addcontentsline{toc}{section}{\nameref{sec:h2}}

The cost of hydrogen network 6-8 billion per year

Net benefit is much higher: 30-53 billion per year (2.7-4.8\%)

hydrogen network is robustly beneficial infrastructure, with strongest benefits when there is no power grid expansion

Compare power grid to H2 network:
- both are important for costs
- but power grid expansion brings more cost benefit
- hydrogen network can partially substitute transmission expansion, but at higher system cost

restricted storage potential makes H2 network more valuable

which pipes are always retrofitted?

\begin{figure}
    \centering
    % \makebox[\textwidth][c]{
    \begin{subfigure}[t]{\textwidth}
        \centering
        \includegraphics[width=\textwidth]{sensitivity-h2.pdf}
    \end{subfigure}
    % }
    \caption{Sensitivity of hydrogen network infrastructure.}
    \label{fig:h2-restriction}
\end{figure}