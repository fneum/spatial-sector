\section*{Common features across scenarios of European carbon-neutrality}
\label{sec:es}
\addcontentsline{toc}{section}{\nameref{sec:es}}

Overall, the total system costs are dominated by investments in generation from wind
and solar, and conversion from power to heat (primarily heat pumps)
and to hydrogen and liquid hydrocarbons (for transport fuels and as a
feedstock for the chemicals industry).

Figure shows the spatial distribution of the investments
for the least-cost solution with full electricity and hydrogen network expansion. While solar capacity is spread
relatively evenly around the continent, both onshore and offshore wind
are concentrated around the North Sea and British Isles. New
transmission capacity is concentrated in HVDC lines that help the
integration of wind in these regions, and the transport of wind energy
to inland locations.

Electrolyzer capacities for power-to-hydrogen see a massive scale-up to XX~GW in
this scenario. Their locations correlate strongly with wind capacities,
particularly offshore wind. A new network transports hydrogen from these sites
of production to the rest of Europe where hydrogen is consumed by industry (for
ammonia, organic chemicals and steel production), for heavy-duty transport and
in fuel cells for power and heat backup. Of the huge hydrogen production
(2437~TWh/a), most of it (1425~TWh/a) goes to Fischer-Tropsch fuels for organic
chemicals and transport fuels, of which 356~TWh/a waste heat enter the district
heating networks. A total of 775~TWh/a is used in shipping (two thirds) and land
(one third) transport. 226~TWh/a are used in the industry sector and for
methanation. 79~TWh/a of hydrogen are lost in conversion.

If the electricity grid expansion is restricted, more hydrogen is produced to be
re-electrified in critical phases and locations of system operation.

Methane production is limited to biogas (346~TWh/a) and some fossil gas
(371~TWh/a), the latter of whose emissions are offset by bioenergy with carbon
capture. Direct air capture with sequestration was only observed if both
hydrogen and electricity network could not be expanded. Methane is used for
process heat in some industry applications and as a heating backup for
power-to-heat units.

A net-zero-emission energy system will require policy support. From our model we
see a need for high, increasing and transparent price for \co pollution of
at least \SI{435}{\sieuro\per\tco} to achieve \co neutrality. As discussed in
\cite{brownSynergiesSector2018}, these high abatement prices arise in building
heating from the large price difference between low-carbon heat and natural gas
(\SI{22}{\sieuro\per\mwh} in the model); if existing taxes and surcharges were
included, the \co price would be lower.

Uses of hydrogen:
- amount of re-electrified hydrogen
- amount of use in industry
- amount of use for heating

hydrogen supply:
- how much hydrogen produced?
- electrolysis capacities
- spatial and temporal distribution of electrolysis capacity factors

hydrogen network:
- how much energy is transported via hydrogen vs electricity?
- how much TWkm built?
- what share of hydrogen is locally consumed?
- how much H2 network is submarine/offshore?
- what is the average loading of hydrogen vs electricity?
- what are the dominant flow directions of energy?

hydrogen storage:
- storage capacities?
- any steel tanks?
- state of charge development over time

What drives the hydrogen network?
- industry demand
- electricity grid bottlenecks / energy transport
- cavern storage locations (restricted storage potential makes H2 network more valuable)

\section*{Hydrogen network benefit is robust, strongest without power grid expansion}
\label{sec:h2}
\addcontentsline{toc}{section}{\nameref{sec:h2}}

If the electricity grid can be expanded, total costs decrease
slightly, despite the increasing costs of the grid. The grid enables
renewable resources with better capacity factors to be integrated from
further away, resulting in lower capacity needs for solar and
wind. The grid also allows renewable variations to be smoothed in space,
resulting in lower hydrogen demand for balancing power and heat. The
total cost benefit of power grid expansion is around
\euro47~billion per year.

\cref{sec:si:lv} presents additional intermediate results between a doubling of
power grid capacity and no grid expansion.

as grid is expanded, costs reduce from solar, PtX and H2 network, more offshore wind


\cref{fig:sensitivity-h2}

The cost of hydrogen network 6-8 billion per year

Net benefit is much higher: 30-53 billion per year (2.7-4.8\%)

hydrogen network is robustly beneficial infrastructure

benefit is strongest when there is no power grid expansion

Compare power grid to H2 network:
- both are important for costs
- but power grid expansion brings more cost benefit
- hydrogen grid is not a perfect substitute
- hydrogen network can partially substitute transmission expansion, but at higher system cost


no more power grid volume and 308 TWkm of hydrogen grid

systems without grid expansion are feasible, but more costly

compare annualised investment costs and volume to EHB

Broad ranges of options with similar costs. The flatness of the total system
costs as we vary grid expansion and onshore wind potentials is a general feature
of energy system models: there are many directions in the feasible space where
we can change the system composition with only a small change in total system
costs. This flatness can be explored systematically using techniques similar to
Modelling to Generate Alternatives (MGA), and was investigated for an
electricity-only version of this model in \cite{Neumann2019}.

\begin{SCfigure}
    \centering
    \includegraphics[width=0.8\textwidth]{sensitivity-h2-new.pdf}
    \caption{Benefits of electricity and hydrogen network infrastructure.}
    \label{fig:sensitivity-h2}
\end{SCfigure}


\section*{Repurposing gas pipelines lowers costs and shapes routes}
\label{sec:repurposed}
\addcontentsline{toc}{section}{\nameref{sec:repurposed}}


\cref{fig:h2-network}
The pipeline connections may have optimised capacities as high as 50 GW.


Repurposing a gas pipeline to transport hydrogen is assumed to cost around half
that of building a new hydrogen pipeline (117 versus 250 \euro/MW/km). This
estimate includes the cost for compressor substitution. In consequence, the
hydrogen network topology does not always follow the shortest routes.

\cref{fig:network-stats:twkm} shows statistics on the total electricity and
hydrogen transmission capacity built as well as how much energy is moved through
the respective networks. For the hydrogen network a distinction between
retrofitted and new pipelines is made. For the electricity network a distinction
is made between existing and added capacity or how much energy is moved via HVAC
or HVDC power lines.

With power grid expansion, 58\% of the hydrogen network uses repurposed gas
pipelines. When the electricity grid cannot be reinforced, this share rises to
66\%. Up to a third of the existing gas network are retrofitted to transport
hydrogen instead.

In relation to today's electricity transmission grid capacities, restricting
hydrogen expansion has only a small effect on cost-optimal levels of grid
expansion. While with a hydrogen network, the power grid capacity is a little
more than doubled, without it the cost-optimal power grid capacities are 10\%
higher.

\cref{fig:network-stats:ewhkm}
When both hydrogen and electricity grid expansion are allowed,
both networks transport approximately the same amount of energy.
As the nominal capacity of the hydrogen network is less than half that of the
optimised electricity grid, this means that the utilisation rate of
the hydrogen network is higher \todo{How much higher?, Reference figure!}

For the case where power grid expansion is disallowed, the hydrogen network
takes over bulk energy transmission. The hydrogen network now transmits around three
times more energy than electricity grid, while the total amount of energy moved
as hydrogen or electricity is reduced by only 22\%.

The hydrogen network plays a
dominant role transporting energy around Europe when grid expansion is
restricted: more energy is moved further in the hydrogen network (2.84~TWhkm/h)
than the electricity network (0.95~EWhkm).

A little more than 40\% of retrofittable pipelines fully use their conversion
potential to hydrogen transport.

gas pipelines are not needed anymore (SI)

which pipes are consistently retrofitted across scenarios?

- where H2 pipeline where no gas network?
- where H2 pipeline exceeds gas pipeline?
- where are gas pipelines left unused?

How many H2 pipelines are used in both directions?

\begin{figure}
    \centering
    % \makebox[\textwidth][c]{
    \begin{subfigure}[t]{0.49\textwidth}
        \centering
        \caption{transmission capacity built}
        \includegraphics[width=\textwidth]{twkm}
        \label{fig:network-stats:twkm}
    \end{subfigure}
    \begin{subfigure}[t]{0.49\textwidth}
        \centering
        \caption{energy moved}
        \includegraphics[width=\textwidth]{ewhkm}
        \label{fig:network-stats:ewhkm}
    \end{subfigure}
    % }
    \caption{Transmission capacity built and energy moved for various scenarios.
    For the hydrogen network a distinction between retrofitted and new pipelines is made.
    For the electricity network a distinction is made between existing and added capacity
    or how much energy is moved via HVAC or HVDC power lines.}
    \label{fig:network-stats}
\end{figure}

\begin{figure}
    \centering
    \makebox[\textwidth][c]{
    \begin{subfigure}[t]{0.6\textwidth}
        \centering
        \caption{With grid reinforcement, with hydrogen network}
        \includegraphics[width=\textwidth, trim=0cm 0cm 7cm 0cm, clip]{\hyrun/maps/elec_s_181_lvopt__Co2L0-3H-T-H-B-I-A-solar+p3-linemaxext10-costs-all_2030.pdf}
        \label{fig:tsc:w-el-w-h2}
    \end{subfigure}
    \begin{subfigure}[t]{0.6\textwidth}
        \centering
        \caption{With grid reinforcement, without hydrogen network}
        \includegraphics[width=\textwidth, trim=0cm 0cm 7cm 0cm, clip]{\hyrun/maps/elec_s_181_lvopt__Co2L0-3H-T-H-B-I-A-solar+p3-linemaxext10-noH2network-costs-all_2030.pdf}
        \label{fig:tsc:w-el-wo-h2}
    \end{subfigure}
    }
    \makebox[\textwidth][c]{
    \begin{subfigure}[t]{0.6\textwidth}
        \centering
        \caption{Without grid reinforcement, with hydrogen network}
        \includegraphics[width=\textwidth, trim=0cm 0cm 7cm 0cm, clip]{\hyrun/maps/elec_s_181_lv1.0__Co2L0-3H-T-H-B-I-A-solar+p3-linemaxext10-noH2network-costs-all_2030.pdf}
        \label{fig:tsc:wo-el-w-h2}
    \end{subfigure}
    \begin{subfigure}[t]{0.6\textwidth}
        \centering
        \caption{Without grid reinforcement, without hydrogen network}
        \includegraphics[width=\textwidth, trim=0cm 0cm 7cm 0cm, clip]{\hyrun/maps/elec_s_181_lv1.0__Co2L0-3H-T-H-B-I-A-solar+p3-linemaxext10-noH2network-costs-all_2030.pdf}
        \label{fig:tsc:wo-el-wo-h2}
    \end{subfigure}
    } \caption{Regional total energy balances for scenarios with and without
    electricity or hydrogen network expansion. The Lorenz curves on the upper
    left of each map depict the regional inequity of electricity, hydrogen,
    methane and oil supply relative to demand. }
    \label{fig:tsc}
\end{figure}

\begin{figure}
    \centering
    \makebox[\textwidth][c]{
    \begin{subfigure}[t]{0.6\textwidth}
        \centering
        \caption{with grid reinforcement}
        \includegraphics[width=\textwidth]{\hyrun/maps/elec_s_181_lvopt__Co2L0-3H-T-H-B-I-A-solar+p3-linemaxext10-h2_network_2030.pdf}
        \label{fig:h2-network:w-el}
    \end{subfigure}
    \begin{subfigure}[t]{0.6\textwidth}
        \centering
        \caption{without grid reinforcement}
        \includegraphics[width=\textwidth]{\hyrun/maps/elec_s_181_lv1.0__Co2L0-3H-T-H-B-I-A-solar+p3-linemaxext10-h2_network_2030.pdf}
        \label{fig:h2-network:wo-el}
    \end{subfigure}
    }
    \caption{Optimised hydrogen network and production sites with and without electricity grid reinforcement.}
    \label{fig:h2-network}
\end{figure}



\begin{figure}
    \centering
    \makebox[\textwidth][c]{
    \begin{subfigure}[t]{0.6\textwidth}
        \centering
        \caption{With grid reinforcement, with hydrogen network}
        \includegraphics[width=\textwidth]{\hyrun/elec_s_181_lvopt__Co2L0-3H-T-H-B-I-A-solar+p3-linemaxext10_2030/import-export-total-200.pdf}
        \label{fig:io:w-el-w-h2}
    \end{subfigure}
    \begin{subfigure}[t]{0.6\textwidth}
        \centering
        \caption{With grid reinforcement, without hydrogen network}
        \includegraphics[width=\textwidth]{\hyrun/elec_s_181_lvopt__Co2L0-3H-T-H-B-I-A-solar+p3-linemaxext10-noH2network_2030/import-export-total-200.pdf}
        \label{fig:io:w-el-wo-h2}
    \end{subfigure}
    }
    \makebox[\textwidth][c]{
    \begin{subfigure}[t]{0.6\textwidth}
        \centering
        \caption{Without grid reinforcement, with hydrogen network}
        \includegraphics[width=\textwidth]{\hyrun/elec_s_181_lv1.0__Co2L0-3H-T-H-B-I-A-solar+p3-linemaxext10_2030/import-export-total-200.pdf}
        \label{fig:io:wo-el-w-h2}
    \end{subfigure}
    \begin{subfigure}[t]{0.6\textwidth}
        \centering
        \caption{Without grid reinforcement, without hydrogen network}
        \includegraphics[width=\textwidth]{\hyrun/elec_s_181_lv1.0__Co2L0-3H-T-H-B-I-A-solar+p3-linemaxext10-noH2network_2030/import-export-total-200.pdf}
        \label{fig:io:wo-el-wo-h2}
    \end{subfigure}
    }
    \caption{Energy balance.}
    \label{fig:io}
\end{figure}

\section*{Onshore wind expansion restrictions shift hydrogen infrastructure}
\label{sec:onwind}
\addcontentsline{toc}{section}{\nameref{sec:onwind}}

\begin{figure}
    \centering
    \makebox[\textwidth][c]{
        \begin{subfigure}[t]{0.48\textwidth}
            \centering
            \caption{system cost}
            \includegraphics[height=0.27\textheight]{\hyrun/maps/elec_s_181_lv1.0__Co2L0-3H-T-H-B-I-A-solar+p3-linemaxext10-onwind+p0-costs-all_2030.pdf}
            \label{fig:no-onw:tsc}
        \end{subfigure}
        \begin{subfigure}[t]{0.4\textwidth}
            \centering
            \caption{hydrogen network}
            \includegraphics[height=0.27\textheight]{\hyrun/maps/elec_s_181_lv1.0__Co2L0-3H-T-H-B-I-A-solar+p3-linemaxext10-onwind+p0-h2_network_2030.pdf}
            \label{fig:no-onw:h2}
        \end{subfigure}
    \begin{subfigure}[t]{0.43\textwidth}
        \centering
        \caption{energy balance}
        \includegraphics[height=0.27\textheight]{\hyrun/elec_s_181_lv1.0__Co2L0-3H-T-H-B-I-A-solar+p3-linemaxext10-onwind+p0_2030/import-export-total-200.pdf}
        \label{fig:no-onw:io}
    \end{subfigure}
    }
    \caption{Maps of regional energy balance, hydrogen network and production sites, and spatial and technological distribution of system costs for a scenario without onshore wind and without power grid expansion.}
    \label{fig:no-onw}
\end{figure}

% cost impact and overall system

By restricting the installable potentials of onshore down to zero, costs rise by
\euro~104 bn/a (12\%) as onshore wind is eliminated if the electricity grid is
fixed to today's capacities. \cref{sec:si:onw} presents intermediate results
between full and no onshore wind expansion. The model substitutes onshore wind,
particularly in the British Isles, for higher investment in offshore wind in the
North Sea and solar generators in Southern Europe (\cref{fig:no-onw:tsc}).
Because offshore capacities are concentrated near coastlines, and grid capacity
is restricted, total spending on hydrogen electrolyzers and networks also
increases to absorb the offshore generation. Without onshore wind, the
potentials for rooftop solar PV and offshore wind in Europe are largely
exhausted, such that in this self-sufficient scenario for Europe, the
installable solar and offshore wind potentials become critical.

% changes in hydrogen infrastructure

Whereas with onshore wind, the  British Isles and North Sea dominate hydrogen
production, Southern Europe becomes a large exporter of solar-based hydrogen if
the development of onshore wind capacities is restricted
(\crefrange{fig:no-onw:h2}{fig:no-onw:io}). This shift in hydrogen
infrastructure also impacts the share of gas pipelines being retrofitted for
hydrogen transport. As the Iberian Peninsula becomes a preferred region for
hydrogen production but has a more sparse gas transmission network, the rate of
retrofitted pipeline capacity reduces from 66\% to 53\%. Many new hydrogen
pipelines are built to connect Spain with France, but also to connect increased
hydrogen production from Danish offshore wind to Germany. Gas pipeline
retrofitting is concentrated on Germany, Austria and Italy.
