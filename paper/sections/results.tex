\section*{Energy system}
\label{sec:es}
\addcontentsline{toc}{section}{\nameref{sec:es}}

Overall, the total system costs are dominated by investments in generation from wind
and solar, and conversion from power to heat (primarily heat pumps)
and to hydrogen and liquid hydrocarbons (for transport fuels and as a
feedstock for the chemicals industry).

Figure shows the spatial distribution of the investments
for the case of a XX\% expansion. While solar capacity is spread
relatively evenly around the continent, both onshore and offshore wind
are concentrated around the North Sea and British Isles. New
transmission capacity is concentrated in HVDC lines that help the
integration of wind in these regions, and the transport of wind energy
to inland locations.

Electrolyzer capacities for power-to-hydrogen see a massive scale-up to XX~GW in
this scenario. Their locations correlate strongly with wind capacities,
particularly offshore wind. A new network transports hydrogen from these sites
of production to the rest of Europe where hydrogen is consumed by industry (for
ammonia, organic chemicals and steel production), for heavy-duty transport and
in fuel cells for power and heat backup. Of the huge hydrogen production
(XX~TWh/a), most of it (XX~TWh/a) goes to Fischer-Tropsch fuels for organic
chemicals and transport fuels, XX~TWh/a to fuel cell CHPs in district heating
networks, and the rest to industry and transport.

The hydrogen network plays a
dominant role transporting energy around Europe when grid expansion is
restricted: more energy is moved further in the hydrogen network (XX~TWhkm/h)
than either the HVAC (XX~TWhkm/h) or HVDC (XX~TWhkm/h) networks.

Methane production is limited to biogas (XX~TWh/a) and some fossil
gas (XX~TWh/a), the latter of whose emissions are offset by bioenergy
with carbon capture and direct air capture with sequestration.
Methane is used for process heat in some industry applications and as
a heating backup for power-to-heat units.

A net-zero-emission energy system will require policy support. From our model we
see a need for high, increasing and transparent price for \co pollution of
at least \SI{435}{\sieuro\per\tco} to achieve \co neutrality. As discussed in
\cite{brownSynergiesSector2018}, these high abatement prices arise in building
heating from the large price difference between low-carbon heat and natural gas
(\SI{22}{\sieuro\per\mwh} in the model); if existing taxes and surcharges were
included, the \co price would be lower.

Uses of hydrogen:
- amount of re-electrified hydrogen
- amount of use in industry
- amount of use for heating

hydrogen supply:
- how much hydrogen produced?
- electrolysis capacities
- spatial and temporal distribution of electrolysis capacity factors

hydrogen network:
- how much energy is transported via hydrogen vs electricity?
- how much TWkm built?
- what share of hydrogen is locally consumed?
- how much H2 network is submarine/offshore?
- what is the average loading of hydrogen vs electricity?
- what are the dominant flow directions of energy?

hydrogen storage:
- storage capacities?
- any steel tanks?
- state of charge development over time

What drives the hydrogen network?
- industry demand
- electricity grid bottlenecks / energy transport
- cavern storage locations

\section*{Hydrogen network benefit is robust, strongest without power grid expansion}
\label{sec:h2}
\addcontentsline{toc}{section}{\nameref{sec:h2}}

If the electricity grid can be expanded, total costs decrease
slightly, despite the increasing costs of the grid. The grid enables
renewable resources with better capacity factors to be integrated from
further away, resulting in lower capacity needs for solar and
wind. The grid also allows renewable variations to be smoothed in space,
resulting in lower hydrogen demand for balancing power and heat.  The
total cost benefit of a doubling of grid capacity is around
\euro44~billion per year, but over half of the benefit (\euro26~billion per year).

as grid is expanded, costs reduce from solar, PtX and H2 network, more offshore wind


\cref{fig:sensitivity-h2}

The cost of hydrogen network 6-8 billion per year

Net benefit is much higher: 30-53 billion per year (2.7-4.8\%)

hydrogen network is robustly beneficial infrastructure

benefit is strongest when there is no power grid expansion

Compare power grid to H2 network:
- both are important for costs
- but power grid expansion brings more cost benefit
- hydrogen grid is not a perfect substitute
- hydrogen network can partially substitute transmission expansion, but at higher system cost

restricted storage potential makes H2 network more valuable

no more power grid volume and 308 TWkm of hydrogen grid

systems without grid expansion are feasible, but more costly

compare annualised investment costs and volume to EHB

Broad ranges of options with similar costs. The flatness of the total system
costs as we vary grid expansion and onshore wind potentials is a general feature
of energy system models: there are many directions in the feasible space where
we can change the system composition with only a small change in total system
costs. This flatness can be explored systematically using techniques similar to
Modelling to Generate Alternatives (MGA), and was investigated for an
electricity-only version of this model in \cite{Neumann2019}.

\begin{SCfigure}
    \centering
    \includegraphics[width=0.8\textwidth]{sensitivity-h2-new.pdf}
    \caption{Sensitivity of hydrogen network infrastructure.}
    \label{fig:sensitivity-h2}
\end{SCfigure}


\section*{Repurposing gas pipelines lowers costs and shapes routes}
\label{sec:repurposed}
\addcontentsline{toc}{section}{\nameref{sec:repurposed}}

\cref{fig:twkm}

How much does it cost to retrofit gas pipeline compared to new H2 pipeline?
- around half

Options to repurpose gas pipelines at lower cost than new pipelines means that
for topology not always the shortest route will be preferred.

How much of H2 network is retrofitted?
- around two-thirds of the hydrogen grid can repurpose existing gas network

how much energy is transmitted via electricity / hydrogen
- if grid expansion is forbidden, hydrogen grid transmits 3x more energy than electricity grid

How much of the methane network is retrofitted to hydrogen?

gas pipelines are not needed anymore (SI)

which pipes are consistently retrofitted across scenarios?

- where H2 pipeline where no gas network?
- where H2 pipeline exceeds gas pipeline?
- where are gas pipelines left unused?

How many H2 pipelines are used in both directions?

\begin{figure}
    \centering
    % \makebox[\textwidth][c]{
    \begin{subfigure}[t]{0.49\textwidth}
        \centering
        \caption{transmission capacity built}
        \includegraphics[width=\textwidth]{twkm}
    \end{subfigure}
    \begin{subfigure}[t]{0.49\textwidth}
        \centering
        \caption{energy moved}
        \includegraphics[width=\textwidth]{ewhkm}
    \end{subfigure}
    % }
    \caption{Transmission capacity built and energy moved.}
    \label{fig:twkm}
\end{figure}

\begin{figure}
    \centering
    \makebox[\textwidth][c]{
    \begin{subfigure}[t]{0.6\textwidth}
        \centering
        \caption{With grid reinforcement, with hydrogen network}
        \includegraphics[width=\textwidth, trim=0cm 0cm 7cm 0cm, clip]{\hyrun/maps/elec_s_181_lvopt__Co2L0-3H-T-H-B-I-A-solar+p3-linemaxext10-costs-all_2030.pdf}
    \end{subfigure}
    \begin{subfigure}[t]{0.6\textwidth}
        \centering
        \caption{With grid reinforcement, without hydrogen network}
        \includegraphics[width=\textwidth, trim=0cm 0cm 7cm 0cm, clip]{\hyrun/maps/elec_s_181_lvopt__Co2L0-3H-T-H-B-I-A-solar+p3-linemaxext10-noH2network-costs-all_2030.pdf}
    \end{subfigure}
    }
    \makebox[\textwidth][c]{
    \begin{subfigure}[t]{0.6\textwidth}
        \centering
        \caption{Without grid reinforcement, with hydrogen network}
        \includegraphics[width=\textwidth, trim=0cm 0cm 7cm 0cm, clip]{\hyrun/maps/elec_s_181_lv1.0__Co2L0-3H-T-H-B-I-A-solar+p3-linemaxext10-noH2network-costs-all_2030.pdf}
    \end{subfigure}
    \begin{subfigure}[t]{0.6\textwidth}
        \centering
        \caption{Without grid reinforcement, without hydrogen network}
        \includegraphics[width=\textwidth, trim=0cm 0cm 7cm 0cm, clip]{\hyrun/maps/elec_s_181_lv1.0__Co2L0-3H-T-H-B-I-A-solar+p3-linemaxext10-noH2network-costs-all_2030.pdf}
    \end{subfigure}
    }
    \caption{Energy balance.}
    \label{fig:tsc}
\end{figure}

\begin{figure}
    \centering
    \makebox[\textwidth][c]{
    \begin{subfigure}[t]{0.6\textwidth}
        \centering
        \caption{with grid reinforcement}
        \includegraphics[width=\textwidth]{\hyrun/maps/elec_s_181_lvopt__Co2L0-3H-T-H-B-I-A-solar+p3-linemaxext10-h2_network_2030.pdf}
    \end{subfigure}
    \begin{subfigure}[t]{0.6\textwidth}
        \centering
        \caption{without grid reinforcement}
        \includegraphics[width=\textwidth]{\hyrun/maps/elec_s_181_lv1.0__Co2L0-3H-T-H-B-I-A-solar+p3-linemaxext10-h2_network_2030.pdf}
    \end{subfigure}
    }
    \caption{Energy balance.}
    \label{fig:h2-network}
\end{figure}



\begin{figure}
    \centering
    \makebox[\textwidth][c]{
    \begin{subfigure}[t]{0.6\textwidth}
        \centering
        \caption{With grid reinforcement, with hydrogen network}
        \includegraphics[width=\textwidth]{\hyrun/elec_s_181_lvopt__Co2L0-3H-T-H-B-I-A-solar+p3-linemaxext10_2030/import-export-total-200.pdf}
    \end{subfigure}
    \begin{subfigure}[t]{0.6\textwidth}
        \centering
        \caption{With grid reinforcement, without hydrogen network}
        \includegraphics[width=\textwidth]{\hyrun/elec_s_181_lvopt__Co2L0-3H-T-H-B-I-A-solar+p3-linemaxext10-noH2network_2030/import-export-total-200.pdf}
    \end{subfigure}
    }
    \makebox[\textwidth][c]{
    \begin{subfigure}[t]{0.6\textwidth}
        \centering
        \caption{Without grid reinforcement, with hydrogen network}
        \includegraphics[width=\textwidth]{\hyrun/elec_s_181_lv1.0__Co2L0-3H-T-H-B-I-A-solar+p3-linemaxext10_2030/import-export-total-200.pdf}
    \end{subfigure}
    \begin{subfigure}[t]{0.6\textwidth}
        \centering
        \caption{Without grid reinforcement, without hydrogen network}
        \includegraphics[width=\textwidth]{\hyrun/elec_s_181_lv1.0__Co2L0-3H-T-H-B-I-A-solar+p3-linemaxext10-noH2network_2030/import-export-total-200.pdf}
    \end{subfigure}
    }
    \caption{Energy balance.}
    \label{fig:io}
\end{figure}

\section*{Hydrogen infrastructure shifts when onshore wind expansion is restricted}
\label{sec:onwind}
\addcontentsline{toc}{section}{\nameref{sec:onwind}}

\begin{figure}
    \centering
    \makebox[\textwidth][c]{
    \begin{subfigure}[t]{0.43\textwidth}
        \centering
        \caption{energy balance}
        \includegraphics[height=0.27\textheight]{\hyrun/elec_s_181_lv1.0__Co2L0-3H-T-H-B-I-A-solar+p3-linemaxext10-onwind+p0_2030/import-export-total-200.pdf}
    \end{subfigure}
    \begin{subfigure}[t]{0.4\textwidth}
        \centering
        \caption{hydrogen network}
        \includegraphics[height=0.27\textheight]{\hyrun/maps/elec_s_181_lv1.0__Co2L0-3H-T-H-B-I-A-solar+p3-linemaxext10-onwind+p0-h2_network_2030.pdf}
    \end{subfigure}
    \begin{subfigure}[t]{0.48\textwidth}
        \centering
        \caption{system cost}
        \includegraphics[height=0.27\textheight]{\hyrun/maps/elec_s_181_lv1.0__Co2L0-3H-T-H-B-I-A-solar+p3-linemaxext10-onwind+p0-costs-all_2030.pdf}
    \end{subfigure}
    }
    \caption{Energy balance.}
    \label{fig:no-onw}
\end{figure}

\cref{fig:no-onw}

By restricting the installable
potentials of onshore down to zero, costs rise by an additional \euro~42~billion
per year. The model substitutes onshore wind for
higher investment in offshore and solar generators. Because offshore capacities
are concentrated near coastlines, and grid capacity is restricted, total
spending on hydrogen electrolyzers and networks also increases to absorb the
offshore generation.

Without onshore wind, solar rooftop and offshore potentials are maxxed out
- If all sectors included and Europe self-sufficient, effect of installable potentials is critical
- less retrofitting (X\%)
- costs rise by \euro~104 bn/a (12\%) as onshore wind is eliminated (25\% grid expansion)

Without onshore wind, hydrogen network looks much different:
- with: British Isles and North Sea dominate hydrogen production
- without: Southern Europe becomes much larger exporter of hydrogen