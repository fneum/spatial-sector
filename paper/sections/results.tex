
\begin{figure}
    \centering
    \makebox[\textwidth][c]{
    \includegraphics[width=1.3\textwidth]{balance-h}
    } \caption{ Energy, hydrogen and carbon dioxide balances across all
    scenarios. Energy consumption includes final energy and non-energy demands
    by carrier as well as conversion losses in thermal storage and electrofuel
    synthesis processes (e.g.~power-to-hydrogen, power-to-liquid). The ambient
    heat retrieved by heat pumps is counted as energy supply. A breakdown of
    final energy and non-energy demands by sector is shown by sector in
    \cref{fig:demand-by-sector-carrier}, by time in \cref{fig:demand-time}, and
    by region in \cref{fig:demand-space}. For technologies with carbon capture
    (CC) option, the carbon dioxide balance shows residual emissions due to
    imperfect capture rates. }
    \label{fig:balance}
\end{figure}

\section*{Energy, hydrogen and carbon balances show key technologies needed to satisfy European energy needs with net-zero emissions}
\label{sec:balances}
\addcontentsline{toc}{section}{\nameref{sec:balances}}

First of all, with the energy balance in \cref{fig:balance}, we underline the
central role of wind and solar electricity supply in all scenarios.
Hydroelectricity, biomass and the recovery of ambient heat through heat pumps
further support the energy supply, whereas fossil oil and gas only play a small
role, since carbon dioxide removal options to offset their unabated emissions
are limited by the assumed sequestration potentials. Electricity demand for
industrial processes, electrified transport and the residential sector,
alongside heat for hot water provision, space heating and industrial processes,
dominate the energy consumption. Conversion losses of power-to-X processes are
also shown in the energy balance and are most pronounced for electrolysis.
Overall, differences between the scenarios are small. With restricted network
expansion options, the energy supply shifts towards solar photovoltaics and the
total increases slightly. This rise compensates for the higher heat losses in
thermal energy storage and increased handling of added synthetic gas in these
scenarios.

\cref{fig:balance} also presents the balance of hydrogen consumption and supply.
The supply-side is dominated by the production of large amounts of green
electrolytic hydrogen between 2376~TWh/a and 2665~TWh/a depending on the
scenario. We only observe a limited production of blue hydrogen from steam
methane reforming with carbon capture in scenarios without hydrogen network
expansion (\SI{\bluehydrogen}{\twh\per\year}). A glance at the demand-side
reveals that, for the most part, hydrogen is only an intermediate product
between electricity and derivative products. There are only a few direct uses of
hydrogen, for instance, in the industry sector for producing ammonia and steel
with hydrogen-based direct reduction of iron, as well as for heavy-duty land
transport. Most hydrogen is used to produce derivatives like Fischer-Tropsch
fuels, methane, ammonia and methanol, which are used for dense aviation and
shipping fuels, fertilisers and as a feedstock for producing high-value
chemicals.

The production of liquid hydrocarbons consumes \ptlhydrogenusage~TWh/a of
hydrogen, of which \ptlwasteheat~TWh/a is useable in the form of waste heat for
district heating networks. Around \hydrogenlosses~TWh/a of hydrogen is lost
during synthetic fuel production. A total of \hydrogentransportdemand~TWh/a is
used in land transport, while the industry sector consumes
\hydrogenindustrydemand~TWh/a for ammonia and steel production, excluding the
consumption of hydrogen for other industry feedstocks (e.g.~for high-value
chemicals).  If the electricity grid expansion is restricted, but hydrogen can
be transported, some more hydrogen is produced to be re-electrified in fuel
cells during critical phases of system operation
(\hydrogenfuelcell~TWh$_{\ce{H2}}$). These fuel cells would mostly be built
inland in Central Europe (see later section \nameref{sec:es}), where the lack of
a strong grid connection requires local dispatchable heat and power supply as a
backup for periods of low renewables feed-in and cold weather. However, in terms
of energy consumed the reconversion of hydrogen to electricity only assumes a
secondary role. In all scenarios with network expansion, no synthetic methane
for process heat in some industrial applications and as a heating backup for
power-to-heat units is produced. This is because the model prefers to use the
full potential for biogas (\biogas~TWh/a) and limited amounts of fossil gas
(\fossilgas~TWh/a), which are offset by sequestering biogenic carbon dioxide,
over synthetic production.

Only when neither hydrogen nor power network expansion were allowed, do we see
methanation (\ce{H2}-to-\ce{CH4}, \SI{\hydrogenmethanation}{\twh} hydrogen). In
this case, despite the associated conversion losses, synthetic methane is used
as a transport medium for hydrogen to utilise the existing gas network to bypass
the restricted transport options for hydrogen and electricity. Apart from
imperfect capture rates of 90\% that requires supplementing some CO$_2$, the
combination of carbon-capturing steam methane reforming creates a carbon cycle
provided that the \co is returned to the methanation sites with an appropriate
\co transport infrastructure.

The atmospheric \co balance in \cref{fig:balance} shows that liquid hydrocarbons
in shipping, aviation and the incineration or eventual decay of plastics
constitute the major uncaptured carbon dioxide emissions in the system. Some
additional \co is emitted through using unabated methane (natural gas, biogas or
synthetic) in gas boilers and CHP plants in the heating sector during the
challenging cold winter periods with low renewable energy supply and high space
heating demand. Industrial process emissions are largely captured such that,
owing to imperfect capture rates, only residual emissions are released into the
atmosphere. Most carbon dioxide removal is achieved through biomass
technologies. For instance, biogenic \co is captured in biomass CHP plants or
industrial low-temperature heat applications. Direct air capture was used in all
scenarios, but takes a much smaller part supplementing the \co available from
biogenic or fossil sources once they are exhausted. Of the \co handled by the
system for the synthesis of electrofuels and long-term sequestration, the
largest share is of biogenic origin (62\%) followed by captured fossil \co
emissions from fuel combustion and process emissions (25\%). Direct air capture
has the smallest share with \SIrange{\mindac}{\maxdac}{\mega\tco\per\year}
(13\%). The broad availability of captured \co from industrial processes and
biofuel combustion is advantageous for the system, as it lowers the cost of fuel
synthesis by avoiding costly and energy-intensive direct air capture.

For a comprehensive overview of energy and carbon flows between carriers in each
scenario see \cref{fig:si:sankey,fig:si:carbon-sankey} which can be
interactively explored at
\href{https://h2-network.streamlit.app}{h2-network.streamlit.app}.


\begin{figure}
    \centering
    \begin{subfigure}[t]{\textwidth}
        \centering
        \caption{cost reductions induced by hydrogen and power grid expansion}
        \includegraphics[width=0.85\textwidth]{sensitivity-h2-new.pdf}
        \label{fig:sensitivity-h2-a}
    \end{subfigure}
    \begin{subfigure}[t]{\textwidth}
        \centering
        \caption{system cost difference to full hydrogen and power grid expansion scenario}
        \includegraphics[width=.85\textwidth]{diff-internal-cost.pdf}
        \label{fig:sensitivity-h2-b}
    \end{subfigure}
    \caption{Cost reductions achieved by developing electricity and hydrogen
    network infrastructure. \cref{fig:sensitivity-h2-a} compares four scenarios
    with and without expansion of a hydrogen network (left to right) and the
    electricity grid (top to bottom). Each bar depicts the total system cost of
    one scenario alongside its cost composition. Arrows between the bars
    indicate absolute and relative cost increases as network infrastructures are
    successively restricted. \cref{fig:sensitivity-h2-b} shows in monetary terms
    how the model reacts to grid expansion restrictions relative to the
    least-cost solution with full hydrogen and power grid expansion.}
    \label{fig:sensitivity-h2}
\end{figure}

\section*{Cost benefit of hydrogen network is consistent, and strongest without power grid expansion}
\label{sec:h2}
\addcontentsline{toc}{section}{\nameref{sec:h2}}

In \cref{fig:sensitivity-h2}, we first compare the total energy system costs and
their composition between the four main scenarios, which vary in whether or not
the power grid can be expanded beyond today's levels and whether a hydrogen
network based on new and retrofitted pipelines can be built. Across all
scenarios, the total costs are dominated by investments in wind and solar
capacities, power-to-heat applications (primarily heat pumps), electrolysers,
and electrofuel synthesis plants (for transport fuels and as a feedstock for the
chemicals industry). Total energy system costs vary between \minsystemcost~and
\maxsystemcost~bn\euro/a, depending on available network expansion options.

Overall, we find that energy system costs are not overly affected by
restrictions on the development of electricity or hydrogen transmission
infrastructure, and systems without grid expansion appear as equally feasible
alternatives. Nonetheless, realisable cost savings range in the order of tens of
billions of euros per year. The combined net benefit of hydrogen and electricity
grid expansion beyond today's levels is \gridbenefitabs~bn\euro/a; a system no
further network expansion would be around \gridbenefitrel\% more expensive. This
limited cost increase can be attributed to the high level of synthetic fuel
production for industry, transport, and backup electricity and heating
applications. The option for a flexible operation of conversion plants,
inexpensive energy storage and low-cost energy transport as hydrocarbons between
regions offer sufficient leeway to manage electricity and hydrogen transport
restrictions effectively (see \nameref{sec:es}). However, regulatory changes
would be needed in order to manage the network bottlenecks (see
\nameref{sec:policy}).

The total net benefit of power grid expansion is between
\minacbenefitabs-\maxacbenefitabs~bn\euro/a
(\minacbenefitrel-\maxacbenefitrel\%) compared to costs for transmission
reinforcements between \minaccost-\maxaccost~bn\euro/a. System costs decrease
despite the increasing investments in electricity transmission infrastructure.
Power grid reinforcements enable renewable resources with higher capacity
factors to be integrated from further away, resulting in lower capacity needs
for solar and wind. The electricity grid also allows renewable variations to be
smoothed in space and facilitates the integration of offshore wind, resulting in
lower hydrogen demand for balancing power and heat and less hydrogen
infrastructure (comprising electrolysis, cavern storage, re-conversion,
pipelines). Restrictions on power grid expansion conversely raise costs by
forcing more local production from solar photovoltaics and increased hydrogen
production. As a hydrogen network could compensate for the lack of grid capacity
to transport energy over long distances, the benefit of electricity grid
reinforcements is strongest if no hydrogen network can be developed.
\cref{sec:si:lv} presents in more detail the progression of system cost changes
in intermediate steps between a doubling of power grid capacity and no grid
expansion.

The presence of a new hydrogen network can reduce system costs by up to
\maxhybenefitrel\%. The net benefit between
\minhybenefitabs-\maxhybenefitabs~bn\euro/a
(\minhybenefitrel-\maxhybenefitrel\%) largely exceeds the cost of the hydrogen
network, which costs between \minhycost-\maxhycost~bn\euro/a. The hydrogen
network offers an alternative for bulk energy transport from the windiest and
sunniest regions in Europe's periphery to low-cost geological storage sites and
the industrial clusters in Central Europe with high energy demand but less
attractive and more constrained renewable potentials (see
\nameref{sec:energy-moved}). We find that its system cost benefit is strongest
when the electricity grid is not expanded. However, even with high levels of
power grid expansion, the hydrogen network is still beneficial infrastructure.

Although power grid reinforcements provide higher cost reductions, hydrogen and
electricity networks are stronger together. Around \hyoftotalbenefit\% of the
combined cost benefit of transmission infrastructure can be achieved solely with
a new hydrogen network. In contrast, \acoftotalbenefit\% of the combined cost
benefit can be reached by just reinforcing the electricity transmission system.
Compared to the combined net benefit of \gridbenefitabs~bn\euro/a, the
individual benefits sum up to a value that is only \additivebenefitrel\% higher
(\maxacbenefitabs{} + \maxhybenefitabs{} = \additivebenefitabs{} bn\euro/a).
Thus, offered cost reductions are mainly additive.

This also means that a hydrogen network cannot substitute perfectly for power
grid reinforcements. It can only partially compensate for the lack of grid
expansion, yielding \benefithyofac\% of the cost reductions achieved by
electricity grid expansion. This is because electricity has more versatile uses
in the newly electrified transport, buildings and industry sectors. Hydrogen can
only be used directly in a few specialised sectors, and if it has to be produced
only to be re-electrified later there are expensive efficiency losses.  A system
built exclusively around hydrogen network expansion is just \acvshycost\% more
expensive than an alternative system that only allows electricity grid
expansion. Overall, our results show that energy transport as electrons and
molecules offer complementary strengths. From a system-level perspective,
network expansion leads to small cost reductions.

\section*{Common design features in four net-zero carbon dioxide emission scenarios for Europe}
\label{sec:es}
\addcontentsline{toc}{section}{\nameref{sec:es}}

Across all scenarios, we see \SIrange{\minoffwind}{\maxoffwind}{\giga\watt}
offshore wind, \SIrange{\minonwind}{\maxonwind}{\giga\watt} onshore wind, and
\SIrange{\minsolar}{\maxsolar}{\giga\watt} solar photovoltaics
(\cref{fig:si:capacities}). The wide range of solar capacities is due to an
increased localisation of electricity generation through solar photovoltaics
when the expansion of transmission infrastructure is limited. As network
expansion options are constrained, we see demand for local daily storage with
batteries almost quadrupling (from 73 to 272 GW with a typical energy-to-power
ratio of 6 hours) and doubling for weekly and seasonal storage with hydrogen and
thermal storage (from 73 to 141~TWh, see \cref{fig:si:capacities}). For all
scenarios, the capacities of photovoltaics split on average into
\meanrooftopshare\% rooftop PV and \meanutilityshare\% utility-scale PV. The
offshore share of wind generation capacities varies between \minoffshoreshare\%
and \maxoffshoreshare\% and is highest when networks can be fully expanded.

The spatial distribution of investments per scenario is shown in \cref{fig:tsc}.
While solar capacities are found throughout Europe, especially in the South,
onshore and offshore wind capacities are mostly found in the North Sea region
and the British Isles. When allowed, new electricity transmission capacity is
built where they help the integration of remote wind production and the
transport to inland demand centres. Consequently, most grid expansion is seen in
and between Northwestern and Central Europe. Battery storage pairs with solar
generation in Southern Europe, particularly when power grid reinforcement is
limited. Besides their wider use overall, battery deployment also progresses
northbound in this case.

Furthermore, electrolyser capacities for power-to-hydrogen conversion see a
massive scale-up ranging from
\SIrange{\minelectrolysis}{\maxelectrolysis}{\giga\watt} depending on the
scenario. The capacities are lowest when the electricity grid can be expanded.
In this case, their locations correlate strongly with wind and solar capacities
(Pearson correlation coefficient $R^2=0.64$ for each, \cref{fig:tsc}). If no
hydrogen or electricity transmission expansion is allowed, the electrolysis
correlates more strongly with wind ($R^2=0.74$) than solar ($R^2=0.46$). The
build-out of hydrogen production facilities is accompanied by a network of
pipelines and hydrogen underground storage in Europe to help balance generation
from renewables in time and space.

In space, a new pipeline network transports hydrogen from preferred production
sites to the rest of Europe, where hydrogen is consumed by industry (for
ammonia, high-value chemicals and steel production), aviation and shipping, as
well as fuel cell CHPs for combined power and heat backup. Varying in magnitude
per scenario, we see major net flows of hydrogen from Great Britain to the
Benelux countries, Germany and Norway, from Northern Germany to the South, and
from the East of Spain to Southern France. The favoured network topology
strongly depends on the potentials for cheap renewable electricity. If onshore
wind potentials were restricted, e.g. due to limited social acceptance in
Northern Europe, the network infrastructure would be tailored to deliver larger
amounts of solar-based hydrogen from Southern Europe to Central Europe. We
discuss this supplementary sensitivity analysis in
\cref{sec:si:onw,sec:si:onw-compromise}.

The development of a hydrogen network is driven by the fact that (i)
spatially-fixed hydrogen demand for steelmaking and ammonia industry as well as
heavy-duty land transport is located in areas with less attractive renewable
potentials (\cref{fig:demand-space:hydrogen}), (ii) the best wind and solar
potentials are located in the periphery of Europe (\cref{fig:energy-density}),
(iii) bottlenecks in the electricity transmission network give impetus to
alternative energy transport options and re-electrification capacities as backup
supply in weakly connected areas, and (iv) moving hydrogen from production sites
to where the geological conditions allow for cheap underground storage is
significantly more cost-effective than local storage in steel tanks
(\cref{fig:clustered-caverns}). Another subsidiary location factor for hydrogen
network infrastructure is linked to the siting of electrofuel production.
Because we assume that waste heat from these processes can be recovered for
district heating networks, urban areas with attractive renewable potentials
nearby are preferred sites for fuel synthesis to which the hydrogen needs to be
transferred. Since we assume no constraints for the transport of liquid
hydrocarbons, the spatial distribution of hydrogen consumption for fuel
synthesis is not a siting factor that is considered. Just like the positioning
of hydrogen fuel cells, the location of hydrogen consumption for electrofuel
production is endogenously optimised. Because we further assume sufficient
infrastructure for the transport of captured carbon dioxide, the location of
carbon sources and sinks neither influences the siting of fuel synthesis plants.

The flexible operation of electrolysers further supports the system integration
of variable renewables in time. Hydrogen production leverages periods with
exceptionally high wind speeds across Europe by running the electrolysis with
average utilisation rates between \mincfelectrolysis\% and \maxcfelectrolysis\%
(see \cref{fig:output-ts-1,fig:output-ts-3}). The produced hydrogen is buffered
in salt caverns which then allows for higher full load hours of fuel synthesis
processes. For Fischer-Tropsch and methanolisation plants, we see combined
average utilisation rates between \mincfFT\% and \maxcfFT\% which aligns with
the higher upfront investment costs of these processes. Their operation is very
steady in the summer months and mostly interrupted in winter periods with low
wind speeds and low ambient temperatures to give way to backup heat and power
supply options (see \cref{fig:output-ts-1,fig:output-ts-3}). By exploiting
periods of peak generation and curbing production in periods of scarcity, large
amounts of variable renewable power generation that serves the system's abundant
synthetic fuel demands can be incorporated into the system cost-effectively.
This ultimately leads to little curtailment of renewables between 2\% and 3\%
(\cref{fig:si:curtailment}) even without grid reinforcements, and low levels of
firm capacity. In relation to a peak electricity consumption of 2626~GW\el, we
observe OCGT and CHP plant capacities between 106 and 218~GW\el, most of which
are gas CHP plants. The lowest values were attained when additional power
transmission could be built.

Hydrogen storage is required to benefit from temporal balancing through flexible
electrolyser operation. We find cost-optimal storage capacities between
\SIrange{\hydrogenstorageacyhyy}{\hydrogenstorageacnhyy}{\twh} with a hydrogen
network and \SIrange{\hydrogenstorageacnhyn}{\hydrogenstorageacyhyn}{\twh}
without a hydrogen network while featuring similar filling level patterns
throughout the year (\cref{fig:si:soc}). Almost all hydrogen is stored in salt
caverns, exploiting vast geological potentials across Europe mostly in Northern
Ireland, England and Denmark. We observe no storage in steel tanks unless both
hydrogen and electricity networks cannot be expanded. In this case, we see up to
1~TWh of steel tank capacity, which represents 5\% of the total hydrogen storage
capacity. If the options for network development are restricted, more hydrogen
storage is built to balance renewables in time rather than in space.

\begin{figure}
    \centering
    \vspace{-2cm}
    \makebox[\textwidth][c]{
        \begin{subfigure}[t]{0.5\textwidth}
            \centering
            \caption{with power grid reinforcement, with hydrogen network}
            \includegraphics[width=\textwidth, trim=0cm .3cm 7cm 0cm, clip]{\hyrun/maps/elec_s_181_lvopt__Co2L0-3H-T-H-B-I-A-solar+p3-linemaxext10-costs-all_2050.pdf}
            \label{fig:tsc:w-el-w-h2}
        \end{subfigure}
        \begin{subfigure}[t]{0.5\textwidth}
            \centering
            \caption{with power grid reinforcement, without hydrogen network}
            \includegraphics[width=\textwidth, trim=0cm .3cm 7cm 0cm, clip]{\hyrun/maps/elec_s_181_lvopt__Co2L0-3H-T-H-B-I-A-solar+p3-linemaxext10-noH2network-costs-all_2050.pdf}
            \label{fig:tsc:w-el-wo-h2}
        \end{subfigure}
    } \makebox[\textwidth][c]{
        \begin{subfigure}[t]{0.5\textwidth}
            \centering
            \caption{without power grid reinforcement, with hydrogen network}
            \includegraphics[width=\textwidth, trim=0cm .3cm 7cm 0cm, clip]{\hyrun/maps/elec_s_181_lv1.0__Co2L0-3H-T-H-B-I-A-solar+p3-linemaxext10-costs-all_2050.pdf}
            \label{fig:tsc:wo-el-w-h2}
        \end{subfigure}
        \begin{subfigure}[t]{0.5\textwidth}
            \centering
            \caption{without power grid reinforcement, without hydrogen network}
            \includegraphics[width=\textwidth, trim=0cm .3cm 7cm 0cm, clip]{\hyrun/maps/elec_s_181_lv1.0__Co2L0-3H-T-H-B-I-A-solar+p3-linemaxext10-noH2network-costs-all_2050.pdf}
            \label{fig:tsc:wo-el-wo-h2}
        \end{subfigure}
    } \vspace{-.5cm} \makebox[1\textwidth][c]{
        \centering
    \includegraphics[width=1.1\textwidth]{color_legend}
    } \caption{ Regional distribution of system costs and electricity grid
    expansion for scenarios with and without electricity or hydrogen network
    expansion. The pie charts depict the annualised system cost alongside the
    shares of the various technologies for each region. The line widths depict
    the level of added grid capacity between two regions, which was capped at 10
    GW.}
    \label{fig:tsc}
\end{figure}


\section*{Hydrogen network takes over role of bulk energy transport}
\label{sec:energy-moved}
\addcontentsline{toc}{section}{\nameref{sec:energy-moved}}

\begin{figure}
    \centering
        \begin{subfigure}[t]{0.49\textwidth}
            \centering
            \caption{transmission capacity built}
            \includegraphics[width=\textwidth]{twkm-main}
            \label{fig:network-stats:twkm}
        \end{subfigure}
        \begin{subfigure}[t]{0.49\textwidth}
            \centering
            \caption{energy volume transported}
            \includegraphics[width=\textwidth]{ewhkm-main}
            \label{fig:network-stats:ewhkm}
        \end{subfigure}
    \caption{Transmission capacity built and energy volume transported for
        various network expansion scenarios. For the hydrogen network, a
        distinction between retrofitted and new pipelines is made. For the
        electricity network, a distinction is made between existing and added
        capacity or how much energy is moved via HVAC or HVDC power lines. Both
        measures weight capacity (TW) or energy (EWh) by the length (km) of the
        network connection.}
    \label{fig:network-stats}
\end{figure}

Depending on the level of power grid expansion, between 204 and 307~TWkm of
hydrogen pipelines are built (\cref{fig:network-stats:twkm}). The higher value
is obtained when the hydrogen network partially offsets the lack of electricity
grid reinforcement. On the other hand, restricting hydrogen expansion only has a
small effect on cost-optimal levels of power grid expansion. The length-weighted
power grid capacity is more than doubled in the least-cost scenario; without a
hydrogen network, the cost-optimal power grid capacity is 7\% higher.

When both hydrogen and electricity grid expansion is allowed, the hydrogen
network transports approximately half the amount of energy transmitted via the
electricity network (\cref{fig:network-stats:ewhkm}). This is striking because
the hydrogen network capacity is little more than a quarter that of the power
grid (\cref{fig:network-stats:twkm}). In consequence, the utilisation rate of
\utilisationHy\% of the hydrogen network is much higher than the
\utilisationAC\% of the electricity grid (\cref{fig:si:grid-utilisation}). One
plausible explanation for this observation is that the buffering of produced
hydrogen in cavern storage allows more coordinated bulk energy tranport in
hydrogen networks, whereas the power grid directly balances the variability of
renewable electricity supply and is subject to linearised power flow physics
(Kirchhoff's circuit laws).

When electricity grid expansion is restricted, the hydrogen network plays a
dominant role in transporting energy around Europe. In this case, around twice
as much energy is moved in the hydrogen network (\ewhkmhydrogen~EWhkm) than in
the electricity network (\ewhkmelectricity~EWhkm). Between only power grid
expansion and only hydrogen network expansion, the difference in the total
volume of energy transported is only \ewhkmdiff\%.

\section*{New hydrogen network can leverage repurposed natural gas pipelines}
\label{sec:repurposed}
\addcontentsline{toc}{section}{\nameref{sec:repurposed}}

\begin{figure}
    \centering
    \makebox[\textwidth][c]{
        \begin{subfigure}[t]{0.65\textwidth}
            \centering
            \caption{hydrogen infrastructure with power grid reinforcement}
            \includegraphics[width=\textwidth]{\hyrun/maps/elec_s_181_lvopt__Co2L0-3H-T-H-B-I-A-solar+p3-linemaxext10-h2_network_2050.pdf}
            \label{fig:h2-network:w-el}
        \end{subfigure}
        \begin{subfigure}[t]{0.65\textwidth}
            \centering
            \caption{hydrogen infrastructure without power grid reinforcement}
            \includegraphics[width=\textwidth]{\hyrun/maps/elec_s_181_lv1.0__Co2L0-3H-T-H-B-I-A-solar+p3-linemaxext10-h2_network_2050.pdf}
            \label{fig:h2-network:wo-el}
        \end{subfigure}
    } \makebox[\textwidth][c]{
    \begin{subfigure}[t]{0.65\textwidth}
        \centering
        \caption{hydrogen flows with power grid reinforcement}
        \includegraphics[width=\textwidth]{\hyrun/elec_s_181_lvopt__Co2L0-3H-T-H-B-I-A-solar+p3-linemaxext10_2050/H2-flow-map-backbone.pdf}
    \end{subfigure}
    \begin{subfigure}[t]{0.65\textwidth}
        \centering
        \caption{hydrogen flows without power grid reinforcement}
        \includegraphics[width=\textwidth]{\hyrun/elec_s_181_lv1.0__Co2L0-3H-T-H-B-I-A-solar+p3-linemaxext10_2050/H2-flow-map-backbone.pdf}
    \end{subfigure}
    } \caption{Optimised hydrogen network, storage, reconversion and production
    sites with and without electricity grid reinforcement. The size of the
    circles depicts the electrolysis and fuel cell capacities in the respective
    region. The line widths depict the optimised hydrogen pipeline capacities.
    The darker shade depicts the share of capacity built from retrofitted gas
    pipelines. The coloring of the regions indicates installed hydrogen storage
    capacities. The second row shows net flow of hydrogen in the network and the
    respective energy balance. Flows larger than 2~TWh are shown with arrow
    sizes proportional to net flow volume.}
    \label{fig:h2-network}
\end{figure}

With our assumptions, developing electricity transmission lines is approximately
60\% more expensive than building new hydrogen pipelines. We assume costs for a
new hydrogen pipeline of 250~\euro/MW/km, whereas, for a new high-voltage
transmission line, we assume 400~\euro/MW/km (see~\cref{sec:si:costs}). Despite
higher costs, we observe that electricity grid reinforcements are preferred over
hydrogen pipelines. Part of the reason is that electricity has more versatile
end uses in transport, buildings and industry in our scenarios with high levels
of direct electrification. Hydrogen can only be used directly in a few
specialised sectors, and if hydrogen has to be produced only to be
re-electrified later, the efficiency losses mean additional generation capacity
would be needed to compensate. This makes energy transport in form of hydrogen
less competitive. However, hydrogen pipelines are particularly attractive where
the end-use is hydrogen-based.

The appeal of a hydrogen network is further spurred when existing natural gas
pipelines are available for retrofitting. Repurposing costs just around half
that of building a new hydrogen pipeline (117 versus 250 \euro/MW/km;
see~\cref{sec:si:costs}). For the capacity retrofit we include costs for
required compressor substitutions and assume that for every unit of gas pipeline
decommissioned, 60\% of its capacity becomes available for hydrogen transport.
The threefold lower volumetric energy density of hydrogen compared to natural
gas is offset by the possibility to attain higher volume flows with hydrogen. In
consequence, even detours of the hydrogen network topology may be cost-effective
if, through rerouting, more repurposing potentials can be tapped.

As \cref{fig:h2-network} illustrates, the optimised hydrogen network topology is
built around supporting flows into the industrial and population centres of
Central Europe. We see strong pipeline connections in Northwestern Europe to
integrate wind-based hydrogen hubs as well as connections for the transport of
solar hydrogen hubs from Spain, Italy and Greece. Individual pipeline
connections between regions have optimised capacities up to 30 GW. Of the total
hydrogen network volume, between 64\% and 69\% consists of repurposed gas
pipelines. The share is highest when the electricity grid is not permitted to be
reinforced. Up to a quarter of the existing natural gas network is retrofitted
to transport hydrogen instead, leaving large capacities that are used neither
for hydrogen nor methane transport. In our scenarios, 29-42\% of retrofittable
gas pipelines fully exhaust their conversion potential to hydrogen. The most
notable corridors for gas pipeline retrofitting are located offshore across the
North Sea and the English Channel and in Great Britain, Germany, Austria,
Switzerland, Northern France and Italy. The most prominent new hydrogen
pipelines are built in the British Isles particularly to connect Ireland,
Northern France, the Netherlands, and in Spain and Portugal. The sizeable
existing natural gas transmission capacities in Southern Italy and Eastern
Europe are largely not repurposed for hydrogen transport in this self-sufficient
scenario for Europe.

However, this picture would change if clean energy import options were
considered. Since most hydrogen is used to produce synthetic hydrocarbons and
ammonia, much of the hydrogen demand would fall away if these derivatives were
imported. In a sensitivity analysis in \cref{sec:si:sensitivity-imports}, we
show that the relative cost benefits of hydrogen network expansion are not
strongly affected by importing all liquid hydrocarbons, even though this action
would reduce the cost-optimal extent of hydrogen infrastructure by more than
50\%. Moreover, direct hydrogen imports into Europe by pipeline or ship could
alter cost-effective network topologies as new import locations need to be
connected rather than domestic production sites. For instance, the networks role
might change from distributing energy from North Sea hydrogen hubs to
integrating inbound pipelines from North Africa with increased network
capacities in Southern Europe.\cite{wetzelGreenEnergy2022}

\section*{Regional imbalance of supply and demand is reinforced by transmission}
\label{sec:imbalance}
\addcontentsline{toc}{section}{\nameref{sec:imbalance}}

\begin{figure}
    \centering
    \makebox[\textwidth][c]{
        \begin{subfigure}[t]{0.65\textwidth}
            \centering
            \caption{with power grid reinforcement, with hydrogen network}
            \includegraphics[width=\textwidth]{\hyrun/elec_s_181_lvopt__Co2L0-3H-T-H-B-I-A-solar+p3-linemaxext10_2050/import-export-total-200.pdf}
            \label{fig:io:w-el-w-h2}
        \end{subfigure}
        \begin{subfigure}[t]{0.65\textwidth}
            \centering
            \caption{with power grid reinforcement, without hydrogen network}
            \includegraphics[width=\textwidth]{\hyrun/elec_s_181_lvopt__Co2L0-3H-T-H-B-I-A-solar+p3-linemaxext10-noH2network_2050/import-export-total-200.pdf}
            \label{fig:io:w-el-wo-h2}
        \end{subfigure}
    } \makebox[\textwidth][c]{
        \begin{subfigure}[t]{0.65\textwidth}
            \centering
            \caption{without power grid reinforcement, with hydrogen network}
            \includegraphics[width=\textwidth]{\hyrun/elec_s_181_lv1.0__Co2L0-3H-T-H-B-I-A-solar+p3-linemaxext10_2050/import-export-total-200.pdf}
            \label{fig:io:wo-el-w-h2}
        \end{subfigure}
        \begin{subfigure}[t]{0.65\textwidth}
            \centering
            \caption{without power grid reinforcement, without hydrogen network}
            \includegraphics[width=\textwidth]{\hyrun/elec_s_181_lv1.0__Co2L0-3H-T-H-B-I-A-solar+p3-linemaxext10-noH2network_2050/import-export-total-200.pdf}
            \label{fig:io:wo-el-wo-h2}
        \end{subfigure}
    } \caption{Total energy balances for scenarios with and without electricity
    or hydrogen network expansion for the 181 model regions, revealing regions
    with net energy surpluses and deficits. The Lorenz curves on the upper left
    of each map depict the regional imbalances of electricity, hydrogen, methane
    and liquid hydrocarbon supply relative to demand. Methane and liquid
    hydrocarbon supply can be of fossil, biogenic or synthetic origin. If the
    annual sums of supply and demand are equal in each region, the Lorenz curve
    resides on the identiy line. But the more imbalanced the regional supply is
    relative to demand, the further the curve dents into the bottom right corner
    of the graph.}
    \label{fig:io}
\end{figure}

In line with previously shown capacity expansion plans, energy surplus is found
largely in the windy coastal and sunny Southern regions that supply the inland
regions of Europe, which have high demands but less attractive renewable
potentials (\cref{fig:io}). The net energy surplus of individual regions amounts
to up to 260~TWh. Examples are Danish offshore wind power exports and large
wind-based production sites for synthetic fuels in Ireland. For Denmark, this
surplus is more than twice as high as its final energy demand, resulting in the
situation that three quarters of Denmark's energy production is exported. Net
deficits of single regions can have similarly high values, close to 200~TWh.
Examples are, in particular, the industrial cluster between Rotterdam and the
Ruhr valley as well as other European metropolises.

Energy transport infrastructure fuels the uneven regional distribution of supply
relative to demand. This is illustrated by the Lorenz curves in \cref{fig:io}
for different energy carriers. The Lorenz curves plot the carrier's cumulative
share of supply versus the cumulative share of demand, sorted by the ratio of
supply and demand in ascending order. If the annual sums of supply and demand
are equal in each region, the Lorenz curve resides on the identity line.
However, the more unequal the regional supply is relative to demand, the further
the curves dent into the graph's bottom right corner. For the least-cost
scenario, \cref{fig:io:w-el-w-h2} highlights that supply and demand of hydrogen
is slightly more regionally imbalanced than electricity. Reduced power grid
expansion causes more evenly distributed electricity supply
(\cref{fig:io:wo-el-w-h2,fig:io:wo-el-wo-h2}), and when hydrogen transport is
restricted (\cref{fig:io:w-el-wo-h2,fig:io:wo-el-wo-h2}), the production of
liquid hydrocarbons is increased in regions with attractive renewable potentials
because they can be transported at low cost.

