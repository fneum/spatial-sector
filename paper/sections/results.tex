\section*{Hydrogen network benefit is consistent, strongest without power grid expansion}
\label{sec:h2}
\addcontentsline{toc}{section}{\nameref{sec:h2}}


\begin{figure}
    \centering
    \begin{subfigure}[t]{\textwidth}
        \centering
        \caption{cost reductions induced by hydrogen and power grid expansion}
        \includegraphics[width=0.85\textwidth]{sensitivity-h2-new.pdf}
        \label{fig:sensitivity-h2-a}
    \end{subfigure}
    \begin{subfigure}[t]{\textwidth}
        \centering
        \caption{system cost difference to full grid expansion scenario}
        \includegraphics[width=.85\textwidth]{diff-internal-cost.pdf}
        \label{fig:sensitivity-h2-b}
    \end{subfigure}
    \caption{Benefits of electricity and hydrogen network infrastructure.
    \cref{fig:sensitivity-h2-a} compares four scenarios with and without
    expansion of a hydrogen network (left to right) and the electricity grid
    (top to bottom). Each bar depicts the total system cost of one scenario
    alongside its cost composition. Arrows between the bars indicate absolute
    and relative cost increases as network infrastructures are successively
    restricted. \cref{fig:sensitivity-h2-b} shows how the model reacts to grid
    expansion restrictions relative to the least-cost solution.}
    \label{fig:sensitivity-h2}
\end{figure}

In \cref{fig:sensitivity-h2}, we first compare the total system costs and their
composition between the four main scenarios, which vary in whether or not the
power grid can be expanded beyond today's levels and if a new hydrogen network
based on new and retrofitted pipelines can be built. Across all four scenarios,
the total system costs are dominated by investments in generation from wind and
solar and conversion from power to heat (primarily heat pumps) and to hydrogen
and liquid hydrocarbons (for transport fuels and as a feedstock for the
chemicals industry). System costs vary between \minsystemcost and
\maxsystemcost~bn\euro/a, depending on available network expansion options.

Overall, we find that system costs are not overly affected by restrictions on
the development of electricity or hydrogen transmission infrastructure. The
realisable cost savings are small compared to total system costs, and systems
without grid expansion present themselves as equally feasible alternatives. The
combined net benefit of hydrogen and electricity grid expansion is
\gridbenefitabs~bn\euro/a; a system without either would be around
\gridbenefitrel\% more expensive. This limited cost increase can be attributed
to the high level of synthetic fuel production for industry, transport, and
backup electricity and heating applications. The option for a flexible operation
of conversion plants, cheap energy storage and low-cost energy transport as
hydrocarbons between regions offer sufficient leeway to manage electricity and
hydrogen transport restrictions effectively (see \nameref{sec:es}).

The total net benefit of power grid expansion is between
\minacbenefitabs-\maxacbenefitabs~bn\euro/a
(\minacbenefitrel-\maxacbenefitrel\%) compared to costs for transmission line
reinforcements between \minaccost-\maxaccost~bn\euro/a. System costs decrease
despite the increasing investments in electricity transmission infrastructure.
The benefit is strongest if no hydrogen network can compensate for the lack of
electricity grid capacity to transport energy over long distances.
\cref{sec:si:lv} presents additional intermediate results about the system cost
sensitivity between a doubling of power grid capacity and no grid expansion.
Electricity grid reinforcement enables renewable resources with higher capacity
factors to be integrated from further away, resulting in lower capacity needs
for solar and wind. The grid also allows renewable variations to be smoothed in
space and facilitates the integration of offshore wind, resulting in lower
hydrogen demand for balancing power and heat and less hydrogen infrastructure
(electrolysis, cavern storage, re-conversion, pipelines). A restriction on the
level of power grid expansion leads to more local production from solar
photovoltaics and increased hydrogen production.

The presence of a new hydrogen network can reduce system costs by up to
\maxhybenefitrel\%. The net benefit between
\minhybenefitabs-\maxhybenefitabs~bn\euro/a
(\minhybenefitrel-\maxhybenefitrel\%) largely exceeds the cost of the hydrogen
network, which costs between \minhycost-\maxhycost~bn\euro/a. Its system cost
benefit is strongest when the electricity grid is not expanded. However, even
with high levels of power grid expansion, the hydrogen network is still
beneficial infrastructure.

Although grid reinforcements provide higher cost reductions, hydrogen and
electricity transmission infrastructure are strongest together. Around
\hyoftotalbenefit\% of the combined cost benefit of transmission infrastructure
can be achieved by only building a new hydrogen network. In contrast,
\acoftotalbenefit\% of the combined cost benefit can be reached by exclusively
reinforcing the electricity transmission system. Compared to the combined net
benefit of \gridbenefitabs bn\euro/a, the individual benefits sum up to a value
that is only \additivebenefitrel\% higher (\maxacbenefitabs + \maxhybenefitabs =
\additivebenefitabs bn\euro/a). Thus, offered cost reductions are mainly
additive.

This also means that a hydrogen network cannot substitute perfectly for power
grid reinforcements. It can only partially compensate for the lack of grid
expansion, yielding only \benefithyofac\% of the system cost reductions achieved
by electricity grid expansion. Instead, energy transport as electrons and
molecules seem to offer complementary strengths. From a system-level
perspective, reductions of total costs achieved by network expansion are small.
A system built exclusively around hydrogen network expansion is just \acvshycost\%
more expensive than an alternative system that only allows electricity grid
expansion.

\section*{Common features across four scenarios of European climate neutrality}
\label{sec:es}
\addcontentsline{toc}{section}{\nameref{sec:es}}

Across all scenarios, we see \SIrange{\minoffwind}{\maxoffwind}{\giga\watt}
offshore wind, \SIrange{\minonwind}{\maxonwind}{\giga\watt} onshore wind, and
\SIrange{\minsolar}{\maxsolar}{\giga\watt} solar photovoltaics. The wide range
of solar capacities are due to an increased localisation of electricity
generation when the expansion of transmission infrastructure is limited. Across
all scenarios, the capacities of photovoltaics split on average into
\meanrooftopshare\% rooftop PV and \meanutilityshare\% utility-scale PV. The
offshore share of wind generation capacities varies between \minoffshoreshare\%
and \maxoffshoreshare\% and is highest when transmission networks can be fully
expanded.

The spatial distribution of investments per scenario is shown in \cref{fig:tsc}.
While solar capacities are spread relatively evenly around the continent with a
stronger presence in Southern Europe, both onshore and offshore wind are
concentrated around the North Sea and the British Isles. When allowed, new
electricity transmission capacity is built where they help the integration of
wind and the transport to inland demand centres (see \nameref{sec:h2}).
Consequently, most grid expansion is seen in and between Northwestern and
Central Europe.

Furthermore, electrolyser capacities for power-to-hydrogen see a massive
scale-up to between \SIrange{\minelectrolysis}{\maxelectrolysis}{\giga\watt}
depending on the permitted energy transport infrastructure. The capacities are
lowest when the electricity grid can be expanded. In this case, their locations
correlate strongly with wind and solar capacities (Pearson correlation
coefficient $R^2=0.64$ for each, \cref{fig:tsc}). If no hydrogen or electricity
transmission expansion is allowed, the electrolysis correlates more strongly
with wind ($R^2=0.74$ ) than solar ($R^2=0.46$). The build-out of hydrogen
production facilities is accompanied by a network of pipelines and hydrogen
underground storage in Europe to help balance generation from renewables in time
and space.

In space, a new pipeline network transports hydrogen from preferred production
sites to the rest of Europe, where hydrogen is consumed by industry (for
ammonia, high-value chemicals and steel production), heavy-duty transport, and
fuel cell CHPs for combined power and heat backup. Varying in magnitude per
scenario, we see major net flows of hydrogen from Great Britain to the Benelux
Union, Germany and Norway, from Northern Germany to the South, and from the East
of Spain to Southern France. The favoured network topology strongly
depends on the potentials for cheap renewable electricity. If onshore wind
potentials were restricted, e.g. due to limited social acceptance in Northern
Europe, the network infrastructure would be tailored to deliver larger amounts
of solar-based hydrogen from Southern Europe to Central Europe. We discuss this
supplementary sensitivity analysis in \cref{sec:si:onw,sec:si:onw-compromise}.

% Compared to net flows in the electricity network,
% which also balances renewable generation back and forth as weather systems pass
% the continent, the hydrogen network more distinctly targets energy transport
% over long distances (see also \nameref{sec:imbalance} and
% \cref{fig:si:flow-ac,fig:h2-network}).

The development of a hydrogen network is driven by the fact that (i) industry
demand for hydrogen is located in areas with less attractive renewable
potentials, (ii) the best wind and solar potentials are located in the periphery
of Europe, (iii) bottlenecks in the electricity transmission network exist and
give impetus to alternative energy transport options, and (iv) moving produced
hydrogen to locations where the geological conditions allow for cheap
underground storage is more cost-effective than local storage where there are no
salt deposits and only steel tanks are available to the model. Another location
factor for hydrogen network infrastructure is linked to the siting of synthetic
liquid hydrocarbon production. Because we assume that waste heat from these
processes can be recovered in district heating networks, urban areas with
attractive renewable potentials nearby appear to be preferred sites for
synthetic fuels production to which additional hydrogen would need to be
transferred (\cref{fig:si:waste-heat-synfuels}). Because of our assumption that
oil and gas can be moved freely in the model, the spatial distribution of their
demands is not a siting factor that is taken into account. Neither is the
location of carbon dioxide sources.

The flexible operation of electrolysers further supports the system integration
of variable renewables in time. Hydrogen production leverages periods with
exceptionally high wind speeds across Europe by running the electrolysis with
average utilisation rates between \mincfelectrolysis\% and \maxcfelectrolysis\%
(see \cref{fig:output-ts-1,fig:output-ts-3}). The produced hydrogen is buffered
in salt caverns which then allows again for a stable subsequent production of
synthetic hydrocarbons. For Fischer-Tropsch plants, for instance, we see higher
average utilisation rates between \mincfFT\% and \maxcfFT\% which is caused by the high upfront
investment costs we assume. Their operation is by tendency interrupted in winter
periods with low wind speeds and low ambient temperatures to give way to backup
heat and power supply options (see \cref{fig:output-ts-1,fig:output-ts-3}). By
exploiting periods of peak generation and curbing production in periods of
scarcity, large amounts of variable renewable power generation that serves the
systems' abundant synthetic fuel demands can be incorporated into the system
cost-effectively. This ultimately leads to low levels of firm capacity. In
relation to a peak electricity consumption of 2626~GW\el, we observe OCGT and
CHP plant capacities between 106 and 218~GW\el, most of which are gas CHP plants.
The lowest values were attained when additional power transmission could be
built.
% TODO adjust text for new utilisation rates of FT TODO recalculate firm

Hydrogen storage is required to benefit from temporal balancing through flexible
electrolyser operation. We find cost-optimal storage capacities between
\SIrange{\hydrogenstorageacyhyy}{\hydrogenstorageacnhyy}{\twh} with a hydrogen
network and \SIrange{\hydrogenstorageacyhyn}{\hydrogenstorageacnhyn}{\twh}
without a hydrogen network while featuring similar filling level patterns
throughout the year. Almost all hydrogen is stored in salt caverns, exploiting
vast geological potentials across Europe mostly in Northern Ireland, England and
Denmark. We observe no storage in steel tanks unless neither a hydrogen nor the
electricity network can be expanded. In this case, we see up to 1~TWh of steel
tank capacity, which represents only 5\% of the total hydrogen
storage capacity. If the options for network development are restricted, more
hydrogen storage is built to balance renewables in time rather than space.

Together with the supporting infrastructure, the production of huge amounts of
hydrogen (\hydrogenproduction~TWh/a) offers versatile use cases. Most of the
hydrogen is used to produce methanol and Fischer-Tropsch fuels for organic
chemicals and transport fuels in aviation and shipping
(\ptlhydrogenusage~TWh/a), of which \ptlwasteheat~TWh/a is useable as a
by-product in the form of waste heat for district heating networks. A total of
\hydrogentransportdemand~TWh/a is used inland transport, while the industry
sector consumes \hydrogenindustrydemand~TWh/a, excluding use of hydrogen for
industry feedstock (e.g.~high-value chemicals). Around \hydrogenlosses~TWh/a of
hydrogen is lost during synthetic fuel production. If the electricity grid
expansion is restricted, but hydrogen can be transported, even more hydrogen is
produced to be re-electrified in fuel cells during critical phases of system
operation (\hydrogenfuelcell TWh$_{\ce{H2}}$). According to \cref{fig:tsc},
these fuel cells would mostly be built inland in Germany. In all scenarios with
network expansion, the Sabatier process to produce synthetic methane for process
heat in some industrial applications and as a heating backup for power-to-heat
units is not used. This is because the model prefers to use the full potential
for biogas (\biogas~TWh/a) and limited amounts of fossil gas (\fossilgas~TWh/a),
which are offset by sequestering biogenic carbon dioxide, over synthetic
production.

Only in scenarios where neither hydrogen nor power network expansion were
allowed, we observe notable synthetic methane production (\ce{H2}-to-\ce{CH4},
\SI{\hydrogenmethanation}{\twh}) and steam methane reforming with carbon capture
(\ce{CH4}-to-\ce{H2}, \SI{\bluehydrogen}{\twh}). In these cases, the route via
the European gas transmission network, is used to partially counteract
transmission bottlenecks. Direct air capture
(\SIrange{\mindac}{\maxdac}{\mega\tco\per\year}) was made use of in all
scenarios, supplementing carbon available from biogenic or fossil sources, while
still being able to stay within set sequestration limits. For a comprehensive
overview of energy and carbon flows in each scenario see
\cref{fig:si:sankey,fig:si:carbon-sankey}.

\begin{figure}
    \centering
    \vspace{-3cm}
    \makebox[\textwidth][c]{
        \begin{subfigure}[t]{0.55\textwidth}
            \centering
            \caption{with power grid reinforcement, with hydrogen network}
            \includegraphics[width=\textwidth, trim=0cm .3cm 7cm 0cm, clip]{\hyrun/maps/elec_s_181_lvopt__Co2L0-3H-T-H-B-I-A-solar+p3-linemaxext10-costs-all_2050.pdf}
            \label{fig:tsc:w-el-w-h2}
        \end{subfigure}
        \begin{subfigure}[t]{0.55\textwidth}
            \centering
            \caption{with power grid reinforcement, without hydrogen network}
            \includegraphics[width=\textwidth, trim=0cm .3cm 7cm 0cm, clip]{\hyrun/maps/elec_s_181_lvopt__Co2L0-3H-T-H-B-I-A-solar+p3-linemaxext10-noH2network-costs-all_2050.pdf}
            \label{fig:tsc:w-el-wo-h2}
        \end{subfigure}
    } \makebox[\textwidth][c]{
        \begin{subfigure}[t]{0.55\textwidth}
            \centering
            \caption{without power grid reinforcement, with hydrogen network}
            \includegraphics[width=\textwidth, trim=0cm .3cm 7cm 0cm, clip]{\hyrun/maps/elec_s_181_lv1.0__Co2L0-3H-T-H-B-I-A-solar+p3-linemaxext10-costs-all_2050.pdf}
            \label{fig:tsc:wo-el-w-h2}
        \end{subfigure}
        \begin{subfigure}[t]{0.55\textwidth}
            \centering
            \caption{without power grid reinforcement, without hydrogen network}
            \includegraphics[width=\textwidth, trim=0cm .3cm 7cm 0cm, clip]{\hyrun/maps/elec_s_181_lv1.0__Co2L0-3H-T-H-B-I-A-solar+p3-linemaxext10-noH2network-costs-all_2050.pdf}
            \label{fig:tsc:wo-el-wo-h2}
        \end{subfigure}
    } \vspace{-.5cm} \makebox[1.1\textwidth][c]{
        \centering
    \includegraphics[width=1.1\textwidth]{color_legend}
    } \caption{ Regional distribution of system costs and electricity grid
    expansion for scenarios with and without electricity or hydrogen network
    expansion. The pie charts depict the annualised system cost alongside the
    shares of the various technologies for each region. The line widths depict
    the level of added grid capacity between two regions, which was capped at 10
    GW.}
    \label{fig:tsc}
\end{figure}

\begin{figure}
    \centering
    \makebox[\textwidth][c]{
        \begin{subfigure}[t]{0.65\textwidth}
            \centering
            \caption{hydrogen infrastructure with power grid reinforcement}
            \includegraphics[width=\textwidth]{\hyrun/maps/elec_s_181_lvopt__Co2L0-3H-T-H-B-I-A-solar+p3-linemaxext10-h2_network_2050.pdf}
            \label{fig:h2-network:w-el}
        \end{subfigure}
        \begin{subfigure}[t]{0.65\textwidth}
            \centering
            \caption{hydrogen infrastructure without power grid reinforcement}
            \includegraphics[width=\textwidth]{\hyrun/maps/elec_s_181_lv1.0__Co2L0-3H-T-H-B-I-A-solar+p3-linemaxext10-h2_network_2050.pdf}
            \label{fig:h2-network:wo-el}
        \end{subfigure}
    }
    \makebox[\textwidth][c]{
    \begin{subfigure}[t]{0.65\textwidth}
        \centering
        \caption{hydrogen flows with power grid reinforcement}
        \includegraphics[width=\textwidth]{\hyrun/elec_s_181_lvopt__Co2L0-3H-T-H-B-I-A-solar+p3-linemaxext10_2050/H2-flow-map-backbone.pdf}
    \end{subfigure}
    \begin{subfigure}[t]{0.65\textwidth}
        \centering
        \caption{hydrogen flows without power grid reinforcement}
        \includegraphics[width=\textwidth]{\hyrun/elec_s_181_lv1.0__Co2L0-3H-T-H-B-I-A-solar+p3-linemaxext10_2050/H2-flow-map-backbone.pdf}
    \end{subfigure}
    } \caption{Optimised hydrogen network, storage, reconversion and production
    sites with and without electricity grid reinforcement. The size of the
    circles depicts the electrolysis and fuel cell capacities in the respective
    region. The line widths depict the optimised hydrogen pipeline capacities.
    The darker shade depicts the share of capacity built from retrofitted gas
    pipelines. The coloring of the regions indicates installed hydrogen storage
    capacities. The second row shows net flow of hydrogen in the network and the
    respective energy balance. Flows larger than 2 TWh are shown with
    arrow sizes proportional to net flow volume.}
    \label{fig:h2-network}
\end{figure}

\section*{Hydrogen network takes over role of bulk energy transport}
\label{sec:energy-moved}
\addcontentsline{toc}{section}{\nameref{sec:energy-moved}}

\begin{figure}
    \centering
    % \makebox[\textwidth][c]{
        \begin{subfigure}[t]{0.49\textwidth}
            \centering
            \caption{transmission capacity built}
            \includegraphics[width=\textwidth]{twkm}
            \label{fig:network-stats:twkm}
        \end{subfigure}
        \begin{subfigure}[t]{0.49\textwidth}
            \centering
            \caption{energy volume transported}
            \includegraphics[width=\textwidth]{ewhkm}
            \label{fig:network-stats:ewhkm}
        \end{subfigure}
    % }
    \caption{Transmission capacity built and energy volume transported for
        various network expansion scenarios. For the hydrogen network, a
        distinction between retrofitted and new pipelines is made. For the
        electricity network, a distinction is made between existing and added
        capacity or how much energy is moved via HVAC or HVDC power lines. Both
        measures weight capacity (TW) or energy (EWh) by the length (km) of the
        network connection.}
    \label{fig:network-stats}
\end{figure}

\cref{fig:network-stats} shows statistics on the total electricity and hydrogen
transmission capacity built as well as how much energy is moved through the
respective networks, while distinguishing between retrofitted and new
capacities.

Depending on the level of power grid expansion, between \mintwkmhydrogen and
\maxtwkmhydrogen~TWkm of hydrogen pipelines are built. The higher value is
obtained when the hydrogen network partially offsets the lack of electricity
grid reinforcement. On the other hand, restricting hydrogen expansion only has a
small effect on cost-optimal levels of power grid expansion. The length-weighted
power grid capacity is more than doubled in the least-cost scenario; without a
hydrogen network, the cost-optimal power grid capacity is \twkmhigher\% higher.

When both hydrogen and electricity grid expansion is allowed, the hydrogen
network transports approximately half the amount of energy transmitted via the
electricity network (\cref{fig:network-stats:ewhkm}). This is striking because
the hydrogen network capacity is little more than a quarter that of the power
grid (\cref{fig:network-stats:twkm}). In consequence, the utilisation rate of
\utilisationHy\% of the hydrogen network is much higher than the
\utilisationAC\% of the electricity grid (\cref{fig:si:grid-utilisation}). One
plausible explanation for this observation is that the buffering of produced
hydrogen in cavern storage allows more coordinated bulk energy tranport in
hydrogen networks, whereas the power grid directly balances the variability of
renewable electricity supply. % and subject to Kirchhoff's Voltage Law.

When electricity grid expansion is restricted, the hydrogen network plays a
dominant role in transporting energy around Europe. In this case, around twice
as much energy is moved in the hydrogen network (\ewhkmhydrogen~EWhkm) than in
the electricity network (\ewhkmelectricity~EWhkm). Between only power grid
expansion and only hydrogen network expansion, the difference in the total
volume of energy transported is only \ewhkmdiff\%.

\section*{New hydrogen network can leverage repurposed natural gas pipelines}
\label{sec:repurposed}
\addcontentsline{toc}{section}{\nameref{sec:repurposed}}

With our assumptions, developing electricity transmission lines is approximately
60\% more expensive than building new hydrogen pipelines. We assume costs for a
new hydrogen pipeline of 250~\euro/MW/km, whereas, for a new high-voltage
transmission line, we assume 400~\euro/MW/km (see~\cref{sec:si:costs}). Despite
higher costs, we observe that electricity grid reinforcements are preferred over
hydrogen pipelines. Part of the reason may be that electricity is more versatile
in our scenarios with high levels of direct electrification. If hydrogen has to
be produced and then re-electrified, the efficiency losses mean additional
generation capacity would be needed to compensate. However, pipelines are
particularly attractive where the end-use is hydrogen-based.

The appeal of a hydrogen network is further spurred by existing natural gas
infrastructure available to be retrofitted. Repurposing a natural gas pipeline
to transport hydrogen instead cost just around half that of building a new
hydrogen pipeline (117 versus 250 \euro/MW/km; see~\cref{sec:si:costs}). For the
capacity retrofit we include costs for required compressor substitutions and
assume that for every unit of gas pipeline decommissioned, 60\% of its capacity
becomes available for hydrogen transport. In consequence, even detours of the
hydrogen network topology may be cost-effective if, through rerouting, more
repurposing potentials can be tapped.

As \cref{fig:h2-network} illustrates, the optimised hydrogen network topology is
highly concentrated in the North West of Europe. Individual pipeline connections
between regions have optimised capacities up to 30 GW. Of the total hydrogen
network volume, between \minretroshare\% and \maxretroshare\% consists of
repurposed gas pipelines. The share is highest when the electricity grid is not
permitted to be reinforced. Up to a quarter of the existing natural gas network
is retrofitted to transport hydrogen instead, leaving large capacities that are
used neither for hydrogen nor methane transport, particularly in Germany,
Poland, Italy and the North Sea. In our scenarios, 29-42\% of retrofittable gas
pipelines fully exhaust their conversion potential to hydrogen. The most notable
corridors for gas pipeline retrofitting are located offshore across the North
Sea and the English Channel and in Great Britain, Germany, Austria, Switzerland,
Northern France and Italy. The most prominent new hydrogen pipelines are built
in the British Isles particularly to connect Ireland, Northern France, the
Netzherlands, and in Spain and Portugal. The sizeable existing natural gas
transmission capacities in Southern Italy and Eastern Europe are largely not
repurposed for hydrogen transport in this self-sufficient scenario for Europe.
However, this picture might change if clean energy import options were
considered. Since most hydrogen is used to produce synthetic fuels and ammonia,
if these were imported, much of the hydrogen demand would fall away, thereby
also reducing the need for hydrogen transport infrastructure. Moreover, direct
hydrogen imports into Europe may alter cost-effective network topologies as new
import locations need to be connected rather than domestic production. For
instance, the networks role might change from distributing energy from North Sea
hydrogen hubs to integrating inbound pipelines from North-Africa with increased
network capacities in Southern Europe.

% reference import scenarios and how they affect hydrogen infrastructure



\section*{Regional imbalance of supply and demand is reinforced by transmission}
\label{sec:imbalance}
\addcontentsline{toc}{section}{\nameref{sec:imbalance}}

\begin{figure}
    \centering
    \makebox[\textwidth][c]{
        \begin{subfigure}[t]{0.6\textwidth}
            \centering
            \caption{with power grid reinforcement, with hydrogen network}
            \includegraphics[width=\textwidth]{\hyrun/elec_s_181_lvopt__Co2L0-3H-T-H-B-I-A-solar+p3-linemaxext10_2050/import-export-total-200.pdf}
            \label{fig:io:w-el-w-h2}
        \end{subfigure}
        \begin{subfigure}[t]{0.6\textwidth}
            \centering
            \caption{with power grid reinforcement, without hydrogen network}
            \includegraphics[width=\textwidth]{\hyrun/elec_s_181_lvopt__Co2L0-3H-T-H-B-I-A-solar+p3-linemaxext10-noH2network_2050/import-export-total-200.pdf}
            \label{fig:io:w-el-wo-h2}
        \end{subfigure}
    } \makebox[\textwidth][c]{
        \begin{subfigure}[t]{0.6\textwidth}
            \centering
            \caption{without power grid reinforcement, with hydrogen network}
            \includegraphics[width=\textwidth]{\hyrun/elec_s_181_lv1.0__Co2L0-3H-T-H-B-I-A-solar+p3-linemaxext10_2050/import-export-total-200.pdf}
            \label{fig:io:wo-el-w-h2}
        \end{subfigure}
        \begin{subfigure}[t]{0.6\textwidth}
            \centering
            \caption{without power grid reinforcement, without hydrogen network}
            \includegraphics[width=\textwidth]{\hyrun/elec_s_181_lv1.0__Co2L0-3H-T-H-B-I-A-solar+p3-linemaxext10-noH2network_2050/import-export-total-200.pdf}
            \label{fig:io:wo-el-wo-h2}
        \end{subfigure}
    } \caption{Regional total energy balances for scenarios with and without
    electricity or hydrogen network expansion, revealing regions with net energy
    surpluses and deficits. The Lorenz curves on the upper left of each map
    depict the regional inequity of electricity, hydrogen, methane and liquid
    hydrocarbon supply relative to demand. Methane and liquid hydrocarbon supply
    can be of fossil, biogenic or synthetic origin. If the annual sums of supply
    and demand are equal in each region, the Lorenz curve resides on the identiy
    line. But the more imbalanced the regional supply is relative to demand, the
    further the curve dents into the bottom right corner of the graph.}
    \label{fig:io}
\end{figure}

\cref{fig:io} shows the net energy surpluses and deficits of each region
alongside so-called Lorenz curves that depict regional imbalances between supply
and demand for each carrier and how they vary among the four network expansion
scenarios.

In line with previously shown capacity expansion plans, energy surplus is found
largely in the windy coastal and sunny Southern regions that supply the inland
regions of Europe, which have high demands but less attractive renewable
potentials. The net energy surplus of individual regions amounts to up to 260
TWh. Examples are Danish offshore wind power exports and large wind-based
production sites for synthetic fuels in Ireland. For Denmark, this surplus is
more than twice as high as its final energy demand, resulting in the situation
that three quarters of Denmark's energy production is exported. Net deficits of
single regions can have similarly high values, close to 200 TWh. Examples are,
in particular, the industrial cluster between Rotterdam and the Ruhr valley.

Energy transport infrastructure fuels the uneven regional distribution of supply
relative to demand. This is illustrated by the Lorenz curves presented in
\cref{fig:io} for different energy carriers and network expansion scenarios. The
Lorenz curves plot the carrier's cumulative share of supply versus the
cumulative share of demand, sorted by the ratio of supply and demand in
ascending order. If the annual sums of supply and demand are equal in each
region, the Lorenz curve resides on the identity line. However, the more unequal
the regional supply is relative to demand, the further the curve dents into the
bottom right corner of the graph.

For the least-cost scenario, \cref{fig:io:w-el-w-h2} highlights that hydrogen
supply is more regionally imbalanced relative to demand than electricity supply.
Roughly 60\% of the hydrogen demand is consumed in regions that produce less
than 11\% of the total hydrogen supply. Conversely, 40\% of the hydrogen supply
is produced in regions that consume less than 12\% of total hydrogen demand.
Naturally, reduced electricity grid expansion causes more evenly distributed
electricity supply (\cref{fig:io:wo-el-w-h2,fig:io:wo-el-wo-h2}).

% If hydrogen
% transport is restricted (\cref{fig:io:w-el-wo-h2,fig:io:wo-el-wo-h2}), the
% production of liquid hydrocarbons is increased in renewable-rich regions because
% they can be transported at low cost. In this case, 70\% of the demand for
% liquid hydrocarbons is consumed in regions that produce less than 1\% of the
% total supply. With full network expansion, 70\% of demand is consumed in regions
% that produce 16\% of the total supply.

