\section*{Common features across four scenarios of European carbon-neutrality}
\label{sec:es}
\addcontentsline{toc}{section}{\nameref{sec:es}}

% TSC

Across all four scenarios, the total system costs are dominated by investments
in generation from wind and solar, and conversion from power to heat (primarily
heat pumps) and to hydrogen and liquid hydrocarbons (for transport fuels and as
a feedstock for the chemicals industry). The composition and variation of total
system costs are shown in \cref{fig:sensitivity-h2}.

The spatial distribution of the investments are shown in \cref{fig:tsc}. In particular, \cref{fig:tsc:w-el-w-h2} shows
the the least-cost solution with full electricity and hydrogen network expansion.

% generation

While solar capacity is spread relatively evenly around the continent with a
slight skew torwards Southern Europe, both onshore and offshore wind are
concentrated around the North Sea and British Isles.

If allowed, new power transmission capacity
is concentrated in regions so that they help the integration of wind in these
regions, and the transport of wind energy to inland locations.
net flow directions: more mixed patterns, balancing rather than net transport over long distances

% hydrogen

Electrolyzer capacities for power-to-hydrogen see a massive scale-up between
1057~GW and 1297~GW depending on the permitted energy transport infrastructure.
The capacities are lowest when the electricity grid can be expanded.
In this case, their locations correlate strongly with wind
capacities ($R^2=0.85$).
average capacity factors of electrolysis 36\% to 40\%, flexible operation
leveraging periods with high wind speeds across Europe (see
\cref{fig:output-ts-1}), buffered in hydrogen cavern storage for stable
production of synthetic hydrocarbons, whereas Fischer-Tropsch runs 87\% to 95\%,
only interrupted in winter periods with low wind speeds and high heat demands
Blue hydrogen production from steam methane reforming with carbon capture
did not occur at large scale in any of the scenarios.

If allowed, a new network transports hydrogen from these sites of production to the rest of
Europe where hydrogen is consumed by industry (for ammonia, high value chemicals
and steel production), for heavy-duty transport and in fuel cells for power and
heat backup.
net flow directions: Ireland and England to Netherlands, Belgium and Western Germany; Scotland via North Sea to all of Germany; France to Southwest Germany; North East Spain via France to Switzerland

a new network
of underground hydrogen storage and pipelines in Europe helps to balance
generation from renewables in time and space.

What drives the hydrogen network?
- industry demand for hydrogen in areas with less attractive renewable potentials
- electricity grid bottlenecks / alternative energy transport
- waste heat of synthetic fuel production can only be leveraged in district heating networks of urban areas
- move produced hydrogen to cavern storage locations, rather than storage in steel tanks
- because gas and oil can be moved freely in the model, the spatial distribution of their demands is not a considered siting factor for synfuel production

hydrogen storage:
- storage capacities? 62-66 TWh with hydrogen network, 32-35 TWh without hydrogen network; similar injection patterns; \cite{caglayanImpactDifferent2019} has 130~TWh of cavern storage
- all storage is cavern storage; no storage in steel tanks unldess neither hydrogen nor electricity network can be built. In this case, 1.3 TWh steel tank (4\%))
- placed Northern Ireland, Denmark, England

Of the huge hydrogen production (2437~TWh/a), most is used to produce
Fischer-Tropsch fuels for organic chemicals and transport fuels (1425~TWh/a), of
which 356~TWh/a of waste heat enter district heating networks. A total of
778~TWh/a is used in shipping (two thirds) and land (one third) transport. The
industry sector consumes 196~TWh/a, and 30~TWh/a are used for methanation.
Around 79~TWh/a of hydrogen are lost in conversion. If the electricity grid
expansion is restricted but hydrogen can be transported, even more hydrogen is
produced to be re-electrified in critical phases and locations of system
operation (100 TWh\el). \todo[]{reference sankey in SI}

% other

Methane production is limited to biogas (346~TWh/a) and some fossil gas
(371~TWh/a), the latter of whose emissions are offset by bioenergy with carbon
capture. Direct air capture with sequestration and significant synthetic methane production
were only observed if both hydrogen and electricity network could not be
expanded. Methane is used for process heat in some industry applications and as
a heating backup for power-to-heat units.

From our model we infer a required carbon price between
\SI{435}{\sieuro\per\tco} with and \SI{513}{\sieuro\per\tco} without network
expansion to achieve carbon-neutrality in Europe.

\section*{Hydrogen network benefit is robust, strongest without grid expansion}
\label{sec:h2}
\addcontentsline{toc}{section}{\nameref{sec:h2}}

In \cref{fig:sensitivity-h2}, we compare the total system costs and their
composition between the four main scenarios which vary in whether or not the
power grid can be expanded beyond today's levels and if a new hydrogen backbone
based on new and retrofitted pipelines can be built.

% overall

Overall, we find that system costs are not overly affected by restrictions on
the development of electricity or hydrogen transmission infrastructure. The
realisable cost savings are small compared to total system costs, and systems
without grid expansion present themselves as equally feasible alternatives. The
combined net benefit of hydrogen and electricity grid expansion is 93 bn\euro/a.
A system without either would then be around 12\% more expensive. This limited
cost increase can be attributed  to the high level of synthetic fuel production,
which is necessary for industry, transport and backup electricity and heating
applications. The option for flexible conversion plant operation, cheap storage
and low-cost exchange between the nodes offers sufficient leeway to manage
electricity and hydrogen transport restrictions.

% grid expansion benefit

The net benefit of power grid expansion is between \euro46-62~billion per
year, against transmission grid expansion costs between \euro11-14~billion per
year. System costs decrease despite the increasing investments in electricity
transmission infrastructure. The benefit is strongest if no hydrogen network can
compensate for the lack of grid capacity to transport energy over long
distances. \cref{sec:si:lv} presents additional intermediate results about
the system cost sensitivity between a doubling of power grid capacity and no
grid expansion. The electricity network enables renewable resources with better
capacity factors to be integrated from further away, resulting in lower capacity
needs for solar and wind. The grid also allows renewable variations to be
smoothed in space, resulting in lower hydrogen demand for balancing power and
heat. A restriction on the level of power grid expansion leads to more local
production from solar photovolatics and increased hydrogen production. When the
electricity grid can be expanded, money spent on solar generation, hydrogen
electrolysis and pipelines reduces, while the generation of offshore wind power
is increased.

\todo[inline]{move explanation of power grid benefit to preceding section}

% hydrogen network benefit  %5.6-7.9

The presence of a new hydrogen backbone can reduce system costs by up to 6\%.
The net benefit between \euro31-46~billion per year (4-6\%) largely exceeds the
cost of the hydrogen network, which are estimated between \euro5-8~billion per
year. The hydrogen backbone's benefit is strongest when the electricity grid is
not expanded. However, even with high levels of grid expansion it is still
strongly beneficial infrastructure.

% combined benefit, nearly additive cost reductions

While grid reinforcements provide slightly higher cost reductions, hydrogen and
electricity transmission infrastructure are strongest together. Approximately
half of the combined benefit of transmission infrastructure can be achieved by
only building a new hydrogen backbone, whereas two thirds of the benefit can be
reached by exclusively reinforcing the electricity transmission system. Compared
to the combined net benefit of 93 bn\euro/a, the individual benefits sum up to a
value that is a mere 16\% higher ($46+62=108$ bn\euro/a). Thus, offered cost
reductions are largely additive.

\todo[inline]{I wonder: Is this maybe due to the restriction
to 10 GW extension per electricity transmission line? Maybe benefit of H2
network would be lower if this constraint were lifted.}

% electricity and hydrogen network: perfect substitutes?

This also means that a hydrogen backbone is not a perfect substitute for power
grid reinforcements. It can only partially compensate for the lack of grid
expansion, yielding roughly 75\% of the electricity grid's benefit. Rather energy
transport as electrons and molecules seem to offer complementary
strengths. Nevertheless, seen from a system-level perspective, differences in
total costs are small. A system build exclusively around hydrogen network
expansion is just around 2\% more expensive than an alternative system that only
allows electricity grid expansion. % (4\% without onshore wind)

\begin{figure}
    \centering
    \includegraphics[width=0.9\textwidth]{sensitivity-h2-new.pdf}
    \caption{Benefits of electricity and hydrogen network infrastructure.
    The figure compares four scenarios with and without expansion of a hydrogen network (left to right) and the electricity grid (top to bottom).
    Each bar depicts the total system cost of one scenario alongside its cost composition.
    Arrows between the bars indicate absolute and relative cost increases
    as network infrastructures are successively restricted.}
    \label{fig:sensitivity-h2}
\end{figure}


\section*{Bulk energy transport through hydrogen backbone}
\label{sec:energy-moved}
\addcontentsline{toc}{section}{\nameref{sec:energy-moved}}

\cref{fig:network-stats} shows statistics on the total electricity and
hydrogen transmission capacity built as well as how much energy is moved through
the respective networks, while distinguising between retrofitted and new
capacities.

% capacity built

Depending on the level of power grid expansion, between 342 and 422~TWkm of
hydrogen pipelines are built. The higher value is obtained when the hydrogen
network partially offsets the lack of electricity grid reinforcement. On the
other side, restricting hydrogen expansion has only a small effect on
cost-optimal levels of grid expansion. While with a hydrogen network, the power
grid capacity is a little more than doubled, without it the cost-optimal power
grid capacities are 10\% higher.

% energy moved

When both hydrogen and electricity grid expansion are allowed, both networks
turn out to transport approximately the same amount of energy
(\cref{fig:network-stats:ewhkm}), even though the hydrogen network capacity is
less than half that of the power grid and rather comparable with the existing
power grid (\cref{fig:network-stats:twkm}). In consequence, the utilisation rate
of the hydrogen network (59\%) is much higher than that of the electricity grid
(35\%). Supplementary figures show the regional distribution of network
loadings. \todo[]{SI figures for network loading}

When electricity grid expansion is restricted, the hydrogen network plays a
dominant role for transporting energy around Europe. In this case, around three
times more energy is moved in the hydrogen network (2.8~EWhkm) than in the
electricity network (1~EWhkm). However, the restriction of grid expansion
affects rather the division of energy flows between hydrogen and electricity
network. The total amount of energy moved as hydrogen or electricity is only
reduced by 22\%.

\section*{New hydrogen backbone can leverage repurposed gas pipelines}
\label{sec:repurposed}
\addcontentsline{toc}{section}{\nameref{sec:repurposed}}

% when is H2 network attractive compared to power grid?

A network of hydrogen pipelines is particularly attractive when the end use is
hydrogen-based product. It is less competitive to supply electricity demands
because additional generation capacity is needed to compensate efficiency losses
in the production and re-electrification of hydrogen. The flexible use of
electricity due to its high exergy levels as well as the fact that because of
high direct electification levels electricity demand exceeds the demand for
hydrogen-based products may explain the higher benefit if electricity
transmission reinforcement despite the higher development costs.

With our assumptions, developing a MWkm of transmission lines is by a factor 1.6
more expensive than building a MWkm of hydrogen pipeline. According to our
assumptions, a new hydrogen pipeline is costed at 250~\euro/MW/km, whereas costs
for a new high-voltage transmission line may be as high as 400~\euro/MW/km. The
attractiveness of a hydrogen network compared to power grid expansion is further
spurred by existing gas infrastructure that can be retrofitted. Repurposing a
gas pipeline to transport hydrogen is assumed to cost around half that of
building a new hydrogen pipeline (117 versus 250 \euro/MW/km). This estimate
includes the cost for compressor substitution. Consequently, detours of the
hydrogen network topology may be cost-effective if through rerouting repurposing
potentials can be tapped.

% repurposing capacities and network topology

As \cref{fig:h2-network} illustrates, the optimised hydrogen network topology is
highly concentrated in North-Western Europe. Individual pipeline connections
between regions have optimised capacities up to 50 GW; as much as thirty
parallel 380~kV transmission lines.

Of the total hydrogen network volume, between 58\% and 66\% consists of
repurposed gas pipelines. The share is highest when the electricity grid is not
permitted to be reinforced. Of the existing gas network, up to a third is
retrofitted to % (measured by TWkm)
transport hydrogen instead. This still leaves large gas network capacities that
are neither used for hydrogen nor methane transport, particularly in Germany,
Poland, Italy and the North Sea. This is demonstrated in supplementary runs with
full gas network resolution in \cref{sec:si:detailed}. In our scenarios, a
little more than 40\% of retrofittable gas pipelines fully exhaust their
conversion potential to hydrogen.

The most notable corridors for gas pipeline retrofitting, are located offshore
across the North Sea and the English Channel, as well as in England, Germany,
Austria and Northern Italy. The large existing gas transmission capacities in
Southern Italy and Eastern Europe are not repurposed for hydrogen transport in
this self-sufficient scenario for Europe. However, this picture would likely
change if impored options were considered in those regions. The largest new
hydrogen pipelines are built on the British Isles, between Denmark and Germany,
inside Belgium, Northern France and the Netherlands, and in the North-East of
Spain.


\section*{Regional imbalance of supply and demand is severe}
\label{sec:imbalance}
\addcontentsline{toc}{section}{\nameref{sec:imbalance}}
% import exports

\cref{fig:io} shows the net energy surpluses and deficits of each region
alongside so-called Lorenz curves that depict regional inequities between supply
and demand for each carrier and the four network expansion scenarios.

In line with previously shown capacity expansion plans, energy surplus is found
largely in the wind-rich coastal and solar-rich most Southern regions that
supply the inland regions of Europe, which have high demands but less attractive
renewable potentials. The net energy surplus of individual regions amounts to up
to 200 TWh. Examples are Danish offshore wind power exports observed in
particular with eletricity grid reinforcement, large production sites for
synthetic fuels in Ireland leveraging favourable local onshore wind potentials,
but also other regions in Spain, Greece, France and Germany. Net energy deficits
of single regions add up to similarly high values close to 200 TWh. Examples are
in particular the metropolitan areas around London and Paris, as well as the
industrial cluster between Rotterdam and the Ruhr valley.

Energy transport infrastructure fuels the uneven regional distribution of supply
relative to demand. This is nicely illustrated by the Lorenz curves presented in
\cref{fig:io} for different energy carriers and network expansion scenarios. The
Lorenz curves plot the carrier's cumulative share of supply versus the
cumulative share of demand, sorted by the ratio of supply and demand in
ascending order. If the annual sums of supply and demand are equal in each
region, the Lorenz curve resides on the identiy line. But the more unequal the
regional supply is relative to demand, the further the curve dents into the
bottom right corner of the graph.

For the least-cost scenario, \cref{fig:io:w-el-w-h2} highlights that hydrogen
supply is more regionally imbalanced relative to demand than electricity supply.
Roughly 60\% of the hydrogen demand is consumed in regions that produce less
than 1\% of total hydrogen supply. Conversely, 40\% of the hydrogen supply is
produced in regions that consume less than 5\% of total hydrogen demand.
Naturally, reduced electricity grid expansion causes more evenly distributed
electricity supply (\cref{fig:io:wo-el-w-h2,fig:io:wo-el-wo-h2}). If hydrogen
transport is restricted (\cref{fig:io:w-el-wo-h2,fig:io:wo-el-wo-h2}), the
production of liquid hydrocarbons is increased in renewable-rich regions because
they can be transported at low cost. In this case, 70\% of the demand for
oil-based products is consumed in regions that produce less than 1\% of total
supply. With full network expansion, 70\% of demand covers 16\% of supply.

% - 17\% of oil from fossil sources
% - 45\% of methane comes from unlocated fossil sources (explains steep end)


\begin{figure}
    \centering
    \makebox[\textwidth][c]{
        \begin{subfigure}[t]{0.55\textwidth}
            \centering
            \caption{energy moved}
            \includegraphics[width=\textwidth]{ewhkm}
            \label{fig:network-stats:ewhkm}
        \end{subfigure}
        \begin{subfigure}[t]{0.55\textwidth}
            \centering
            \caption{transmission capacity built}
            \includegraphics[width=\textwidth]{twkm}
            \label{fig:network-stats:twkm}
        \end{subfigure}
    }
    \caption{Transmission capacity built and energy moved for various network expansion scenarios.
        For the hydrogen network a distinction between retrofitted and new pipelines is made.
        For the electricity network a distinction is made between existing and added capacity
        or how much energy is moved via HVAC or HVDC power lines. Both measures weight capacity
        and energy by the length of the line.}
    \label{fig:network-stats}
\end{figure}

\begin{figure}
    \centering
    \makebox[\textwidth][c]{
        \begin{subfigure}[t]{0.6\textwidth}
            \centering
            \caption{With grid reinforcement, with hydrogen network}
            \includegraphics[width=\textwidth, trim=0cm 0cm 7cm 0cm, clip]{\hyrun/maps/elec_s_181_lvopt__Co2L0-3H-T-H-B-I-A-solar+p3-linemaxext10-costs-all_2030.pdf}
            \label{fig:tsc:w-el-w-h2}
        \end{subfigure}
        \begin{subfigure}[t]{0.6\textwidth}
            \centering
            \caption{With grid reinforcement, without hydrogen network}
            \includegraphics[width=\textwidth, trim=0cm 0cm 7cm 0cm, clip]{\hyrun/maps/elec_s_181_lvopt__Co2L0-3H-T-H-B-I-A-solar+p3-linemaxext10-noH2network-costs-all_2030.pdf}
            \label{fig:tsc:w-el-wo-h2}
        \end{subfigure}
    }
    \makebox[\textwidth][c]{
        \begin{subfigure}[t]{0.6\textwidth}
            \centering
            \caption{Without grid reinforcement, with hydrogen network}
            \includegraphics[width=\textwidth, trim=0cm 0cm 7cm 0cm, clip]{\hyrun/maps/elec_s_181_lv1.0__Co2L0-3H-T-H-B-I-A-solar+p3-linemaxext10-noH2network-costs-all_2030.pdf}
            \label{fig:tsc:wo-el-w-h2}
        \end{subfigure}
        \begin{subfigure}[t]{0.6\textwidth}
            \centering
            \caption{Without grid reinforcement, without hydrogen network}
            \includegraphics[width=\textwidth, trim=0cm 0cm 7cm 0cm, clip]{\hyrun/maps/elec_s_181_lv1.0__Co2L0-3H-T-H-B-I-A-solar+p3-linemaxext10-noH2network-costs-all_2030.pdf}
            \label{fig:tsc:wo-el-wo-h2}
        \end{subfigure}
    } \caption{
        Regional distribution of system costs and electricity grid
    expansion for scenarios with and without electricity or hydrogen network
    expansion. The pie charts depict the annualised system cost alongside the
    shares of the various technologies for each region. The color legend is the
    same as for Figure 1. The line widths depict the level of added grid
    capacity, distinguising between HVAC lines (grey) and HVDC links (purple).
    The added capacity of a connection between two regions was capped to 10 GW.
    }
    \label{fig:tsc}
\end{figure}

\begin{figure}
    \centering
    \makebox[\textwidth][c]{
        \begin{subfigure}[t]{0.6\textwidth}
            \centering
            \caption{with grid reinforcement}
            \includegraphics[width=\textwidth]{\hyrun/maps/elec_s_181_lvopt__Co2L0-3H-T-H-B-I-A-solar+p3-linemaxext10-h2_network_2030.pdf}
            \label{fig:h2-network:w-el}
        \end{subfigure}
        \begin{subfigure}[t]{0.6\textwidth}
            \centering
            \caption{without grid reinforcement}
            \includegraphics[width=\textwidth]{\hyrun/maps/elec_s_181_lv1.0__Co2L0-3H-T-H-B-I-A-solar+p3-linemaxext10-h2_network_2030.pdf}
            \label{fig:h2-network:wo-el}
        \end{subfigure}
    }
    \caption{Optimised hydrogen network and production sites with and without
    electricity grid reinforcement. The size of the circles depicts the
    electrolysis and fuel cell capacities in the respective region. The line
    widths depict the optimised hydrogen pipeline capacities. The darker shade
    depicts the share of capacity that is built from retrofitted gas pipelines.}
    \label{fig:h2-network}
\end{figure}



\begin{figure}
    \centering
    \makebox[\textwidth][c]{
        \begin{subfigure}[t]{0.6\textwidth}
            \centering
            \caption{With grid reinforcement, with hydrogen network}
            \includegraphics[width=\textwidth]{\hyrun/elec_s_181_lvopt__Co2L0-3H-T-H-B-I-A-solar+p3-linemaxext10_2030/import-export-total-200.pdf}
            \label{fig:io:w-el-w-h2}
        \end{subfigure}
        \begin{subfigure}[t]{0.6\textwidth}
            \centering
            \caption{With grid reinforcement, without hydrogen network}
            \includegraphics[width=\textwidth]{\hyrun/elec_s_181_lvopt__Co2L0-3H-T-H-B-I-A-solar+p3-linemaxext10-noH2network_2030/import-export-total-200.pdf}
            \label{fig:io:w-el-wo-h2}
        \end{subfigure}
    }
    \makebox[\textwidth][c]{
        \begin{subfigure}[t]{0.6\textwidth}
            \centering
            \caption{Without grid reinforcement, with hydrogen network}
            \includegraphics[width=\textwidth]{\hyrun/elec_s_181_lv1.0__Co2L0-3H-T-H-B-I-A-solar+p3-linemaxext10_2030/import-export-total-200.pdf}
            \label{fig:io:wo-el-w-h2}
        \end{subfigure}
        \begin{subfigure}[t]{0.6\textwidth}
            \centering
            \caption{Without grid reinforcement, without hydrogen network}
            \includegraphics[width=\textwidth]{\hyrun/elec_s_181_lv1.0__Co2L0-3H-T-H-B-I-A-solar+p3-linemaxext10-noH2network_2030/import-export-total-200.pdf}
            \label{fig:io:wo-el-wo-h2}
        \end{subfigure}
    }
    \caption{Regional total energy balances for scenarios with and without
    electricity or hydrogen network expansion, revealing regions with net energy
    surpluses and deficits. The Lorenz curves on the upper left of each map
    depict the regional inequity of electricity, hydrogen, methane and oil
    supply relative to demand. If the annual sums of supply and demand are equal
    in each region, the Lorenz curve resides on the identiy line. But the more
    unequal the regional supply is relative to demand, the further the curve
    dents into the bottom right corner of the graph.}
    \label{fig:io}
\end{figure}

\section*{Onshore wind restrictions alter hydrogen backbone topology}
\label{sec:onwind}
\addcontentsline{toc}{section}{\nameref{sec:onwind}}

\begin{figure}
    \centering
    \makebox[\textwidth][c]{
        \begin{subfigure}[t]{0.6\textwidth}
            \centering
            \caption{hydrogen network}
            \includegraphics[height=0.35\textheight]{\hyrun/maps/elec_s_181_lv1.0__Co2L0-3H-T-H-B-I-A-solar+p3-linemaxext10-onwind+p0-h2_network_2030.pdf}
            \label{fig:no-onw:h2}
        \end{subfigure}
        \begin{subfigure}[t]{0.6\textwidth}
            \centering
            \caption{energy balance}
            \includegraphics[height=0.35\textheight]{\hyrun/elec_s_181_lv1.0__Co2L0-3H-T-H-B-I-A-solar+p3-linemaxext10-onwind+p0_2030/import-export-total-200.pdf}
            \label{fig:no-onw:io}
        \end{subfigure}
        }
    \begin{subfigure}[t]{0.6\textwidth}
        \centering
        \caption{system cost}
        \includegraphics[height=0.35\textheight]{\hyrun/maps/elec_s_181_lv1.0__Co2L0-3H-T-H-B-I-A-solar+p3-linemaxext10-onwind+p0-costs-all_2030.pdf}
        \label{fig:no-onw:tsc}
    \end{subfigure}
    \caption{Maps of regional energy balance, hydrogen network and production sites, and spatial and technological distribution of system costs for a scenario without onshore wind and without power grid expansion.}
    \label{fig:no-onw}
\end{figure}

% cost impact and overall system

Like building new power transmission lines, the deployment of onshore wind may
not always be socially accepted such that it may not be possible to leverage its
full potential. In the following section and \cref{fig:no-onw}, we explore the
hypothetical impact of restricting the installable potentials of onshore wind
down to zero.

We find that as onshore wind is eliminated, costs rise by \euro~104 bn/a (12\%)
when the electricity grid is fixed to today's capacities but a hydrogen network
can be developed. In comparison to the least-cost solution with full network
expansion, this solution is XX\% more expensive. \cref{sec:si:onw} presents
intermediate results between full and no onshore wind expansion. The model
substitutes onshore wind, particularly in the British Isles, for higher
investment in offshore wind in the North Sea and solar generators in Southern
Europe (\cref{fig:no-onw:tsc}). Because offshore capacities are concentrated
near coastlines, and grid capacity is restricted, total spending on hydrogen
electrolyzers and networks also increases to absorb the increased offshore
generation. Without onshore wind, the potentials for rooftop solar PV and
offshore wind in Europe are largely exhausted, such that in this self-sufficient
scenario for Europe, the effect of installable potentials becomes critical.

% changes in hydrogen infrastructure

Whereas with onshore wind, the  British Isles and North Sea dominate hydrogen
production, Southern Europe becomes a large exporter of solar-based hydrogen if
the development of onshore wind capacities is restricted
(\cref{fig:no-onw:h2,fig:no-onw:io}). This shift in hydrogen
infrastructure also impacts the share of gas pipelines being retrofitted for
hydrogen transport. As the Iberian Peninsula becomes a preferred region for
hydrogen production but has a more sparse gas transmission network, the rate of
retrofitted pipeline capacity reduces from 66\% to 53\%. Many new hydrogen
pipelines are built to connect Spain with France, but also to connect increased
hydrogen production from Danish offshore wind to Germany. Gas pipeline
retrofitting is concentrated in Germany, Austria and Italy.
