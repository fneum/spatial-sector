\section*{Common features across four scenarios of European carbon-neutrality}
% \section*{Hydrogen infrastructure across four scenarios of European
% carbon-neutrality}
\label{sec:es}
\addcontentsline{toc}{section}{\nameref{sec:es}}

\topic{total system cost}

Across all four scenarios, the total system costs are dominated by investments
in generation from wind and solar and conversion from power to heat (primarily
heat pumps) and to hydrogen and liquid hydrocarbons (for transport fuels and as
a feedstock for the chemicals industry). System costs vary between 764 and
857~bn\euro/a, depending on available network expansion options. The composition
and variation of total system costs are shown in \cref{fig:sensitivity-h2}. From
our model, we infer that a carbon price between \SI{435}{\sieuro\per\tco} with
transmission expansion and \SI{513}{\sieuro\per\tco} without would be required
to achieve carbon-neutrality and self-sufficiency in Europe.

\topic{electricity generation and transmission}

The spatial distribution of investments is shown in \cref{fig:tsc}. For
instance, \cref{fig:tsc:w-el-w-h2} shows the the least-cost solution with full
electricity and hydrogen network expansion. Across all scenarios, we see
\SIrange{210}{270}{\giga\watt} offshore wind, \SIrange{1820}{1950}{\giga\watt}
onshore wind, and \SIrange{2380}{3630}{\giga\watt} solar photovoltaics. The
capacities of photovoltaics split roughly into 10\% rooftop PV and 90\%
utility-scale PV. The offshore share of wind generation capacities varies
between 10\% and 13\%. The share is highest when transmission networks can be
fully expanded. While solar capacities are spread relatively evenly around the
continent with a slight skew towards Southern Europe, both onshore and offshore
wind are concentrated around the North Sea and the British Isles. When allowed,
new electricity transmission capacity congregates in regions so that they help
the integration of wind in these regions and the transport of wind energy to
inland locations (see \nameref{sec:h2}).

\topic{hydrogen production capacities}

Furthermore, electrolyser capacities for power-to-hydrogen see a massive
scale-up to between \SIrange{1057}{1297}{\giga\watt} depending on the permitted
energy transport infrastructure. The capacities are lowest when the electricity
grid can be expanded. In this case, their locations correlate strongly with wind
capacities ($R^2=0.85$), as seen in \cref{fig:tsc}. New hydrogen infrastructure
accompanies the buildout of hydrogen production facilities. A system of hydrogen
underground storage and pipelines in Europe helps to balance generation from
renewables in time and space.

\topic{hydrogen network in general}

In space, a new pipeline network transports hydrogen from preferred production
sites to the rest of Europe, where hydrogen is consumed by industry (for
ammonia, high-value chemicals and steel production), heavy-duty transport, and
fuel cells for power and heat backup. Varying in magnitude per scenario, we see
major net flows of hydrogen from Ireland and England to the Benelux Union and
Western Germany; from Scotland through the North Sea to Germany; from France to
South West Germany; and from the North East of Spain to Southern France.
Compared to net flows in the electricity network, which also balances renewable
generation back and forth as weather systems pass the continent, the hydrogen
network more distinctly targets energy transport over long distances.

\todo[inline]{reference subsequent section which discusses in more detail}

\todo[inline]{reference flow pattern plots in SI}

\topic{what drives the hydrogen network?}

The development of a hydrogen backbone is driven by the fact that (i) industry
demand for hydrogen is located in areas with less attractive renewable
potentials, (ii) bottlenecks in the electricity transmission network exist and
give impetus to alternative energy transport options, and (iii) moving produced
hydrogen to cavern storage locations is more cost-effective than local storage
in steel tanks. Another location factor for hydrogen network infrastructure is
linked to the siting of synthetic hydrocarbon production. Because we assume that
waste heat from these processes can only be recovered in district heating
networks, production sites of synthetic fuels are preferred in urban areas to
which additional hydrogen needs to be transferred. Since we assume that oil and
gas can be moved freely in the model, the spatial distribution of their demands
is not a siting factor that is taken into account.

\topic{flexible electrolyser operation supported by cheap hydrogen storage}

The flexible operation of electrolysers further supports the system integration
of variable renewables in time. Hydrogen production can leverage periods with
exceptionally high wind speeds across Europe by running the electrolysis with
relatively low average utilisation rates between 36\% and 40\% (see
\cref{fig:output-ts-1}). The produced hydrogen is then buffered in salt caverns
to allow for a stable subsequent production of synthetic hydrocarbons. For
Fischer-Tropsch plants, we see much higher average utilisation rates between
87\% and 95\% because upfront investment costs for this technology are high.
Their operation is only interrupted in winter periods with low wind speeds and
low ambient temperatures to give way to backup heat and power supply options.
Hydrogen storage is required to benefit from temporal balancing through flexible
electrolyser operation. We find cost-optimal storage capacities between
\SIrange{62}{66}{\twh} with a hydrogen network and \SIrange{32}{35}{\twh}
without a hydrogen network while featuring similar filling level patterns
throughout the year. Almost all hydrogen is stored in salt caverns,
predominantly exploiting geological potentials in Northern Ireland, England and
Denmark. We observe no storage in steel tanks unless neither a hydrogen nor the
electricity network can be expanded. In this case, we see up to 1.3~TWh of steel
tank capacity, which represents 4\% of the respective total hydrogen storage
capacity.

\todo[inline]{why does less hydrogen network lead to less hydrogen storage?}

\topic{hydrogen uses}

Together with the supporting infrastructure, the production of huge amounts of
hydrogen (2437~TWh/a) offers various use cases. Most of the hydrogen is used to
produce Fischer-Tropsch fuels for organic chemicals and transport fuels
(1425~TWh/a), of which 356~TWh/a of waste heat enter district heating networks.
A total of 778~TWh/a is used in shipping (two thirds) and land (one third)
transport. The industry sector consumes 196~TWh/a. Around 79~TWh/a of hydrogen
is lost in conversion. If the electricity grid expansion is restricted, but
hydrogen can be transported, even more hydrogen is produced to be re-electrified
in critical phases and locations of system operation (100 TWh\el).
Only 30~TWh/a is used to produce methane, which is needed for process heat in
some industrial applications and as a heating backup for power-to-heat units.
This is because the model prefers to use biogas (346~TWh/a) and limited amounts
of fossil gas (371~TWh/a), which is offset by sequestering biogenic carbon
dioxide, over synthetic production. Direct air capture with subsequent
sequestration and significant synthetic methane or blue hydrogen production were
not observed.

\todo[inline]{reference sankey in SI}

\section*{Hydrogen network benefit is robust, strongest without grid expansion}
\label{sec:h2}
\addcontentsline{toc}{section}{\nameref{sec:h2}}

\begin{figure}
    \centering
    \includegraphics[width=0.85\textwidth]{sensitivity-h2-new.pdf}
    \caption{Benefits of electricity and hydrogen network infrastructure. The
    figure compares four scenarios with and without expansion of a hydrogen
    network (left to right) and the electricity grid (top to bottom). Each bar
    depicts the total system cost of one scenario alongside its cost
    composition. Arrows between the bars indicate absolute and relative cost
    increases as network infrastructures are successively restricted.}
    \label{fig:sensitivity-h2}
\end{figure}

In \cref{fig:sensitivity-h2}, we compare the total system costs and their
composition between the four main scenarios, which vary in whether or not the
power grid can be expanded beyond today's levels and if a new hydrogen backbone
based on new and retrofitted pipelines can be built.

\topic{high-level results}

Overall, we find that system costs are not overly affected by restrictions on
the development of electricity or hydrogen transmission infrastructure. The
realisable cost savings are small compared to total system costs, and systems
without grid expansion present themselves as equally feasible alternatives. The
combined net benefit of hydrogen and electricity grid expansion is 93~bn\euro/a.
A system without either would then be around 12\% more expensive. This limited
cost increase can be attributed to the high level of synthetic fuel production,
which is necessary for industry, transport, and backup electricity and heating
applications. The option for flexible conversion plant operation, cheap energy
storage and low-cost exchange between regions offer sufficient leeway to manage
electricity and hydrogen transport restrictions effectively.

\topic{grid expansion benefit}

The net benefit of power grid expansion is between \euro46-62~billion per year.
This is contrasted by transmission grid expansion costs between
\euro11-14~billion per year. System costs decrease despite the increasing
investments in electricity transmission infrastructure. The benefit is strongest
if no hydrogen network can compensate for the lack of grid capacity to transport
energy over long distances. \cref{sec:si:lv} presents additional intermediate
results about the system cost sensitivity between a doubling of power grid
capacity and no grid expansion. Electricity grid reinforcement enables renewable
resources with better capacity factors to be integrated from further away,
resulting in lower capacity needs for solar and wind. The grid also allows
renewable variations to be smoothed in space, resulting in lower hydrogen demand
for balancing power and heat. A restriction on the level of power grid expansion
leads to more local production from solar photovoltaics and increased hydrogen
production. When the electricity grid can be expanded, money spent on solar
generation, hydrogen electrolysis, and pipelines reduces, while offshore wind
power generation increases.

\todo[inline]{move explanation of power grid benefit to preceding section}

\topic{hydrogen network benefit}  %5.6-7.9

The presence of a new hydrogen backbone can reduce system costs by up to 6\%.
The net benefit between \euro31-46~billion per year (4-6\%) largely exceeds the
cost of the hydrogen network, which is estimated between \euro5-8~billion per
year. The hydrogen backbone's benefit is strongest when the electricity grid is
not expanded. However, even with high levels of grid expansion, it is still an
immensely beneficial infrastructure.

\topic{combined benefit, nearly additive cost reductions}

While grid reinforcements provide slightly higher cost reductions, hydrogen and
electricity transmission infrastructure are strongest together. Approximately
half of the combined benefit of transmission infrastructure can be achieved by
only building a new hydrogen backbone. In contrast, two-thirds of the benefit
can be reached by exclusively reinforcing the electricity transmission system.
Compared to the combined net benefit of 93 bn\euro/a, the individual benefits
sum up to a value that is only 16\% higher ($46+62=108$ bn\euro/a). Thus,
offered cost reductions are mainly additive.

\todo[inline]{I wonder: Is this maybe due to the restriction to 10 GW extension
per electricity transmission line? Maybe benefit of H2 network would be lower if
this constraint were lifted.}

\topic{electricity and hydrogen network: perfect substitutes?}

This also means that a hydrogen backbone is not a perfect substitute for power
grid reinforcements. It can only partially compensate for the lack of grid
expansion, yielding roughly 75\% of the electricity grid's benefit. Instead,
energy transport as electrons and molecules seem to offer complementary
strengths. Nevertheless, from a system-level perspective, differences in total
costs are small. A system built exclusively around hydrogen network expansion is
just around 2\% more expensive than an alternative system that only allows
electricity grid expansion. % (4\% without onshore wind)

% (high exergy levels)

\section*{Bulk energy transport through hydrogen backbone}
\label{sec:energy-moved}
\addcontentsline{toc}{section}{\nameref{sec:energy-moved}}

\begin{figure}
    \centering
    % \makebox[\textwidth][c]{
        \begin{subfigure}[t]{0.49\textwidth}
            \centering
            \caption{energy moved}
            \includegraphics[width=\textwidth]{ewhkm}
            \label{fig:network-stats:ewhkm}
        \end{subfigure}
        \begin{subfigure}[t]{0.49\textwidth}
            \centering
            \caption{transmission capacity built}
            \includegraphics[width=\textwidth]{twkm}
            \label{fig:network-stats:twkm}
        \end{subfigure}
    % }
    \caption{Transmission capacity built and energy moved for various network
        expansion scenarios. For the hydrogen network, a distinction between
        retrofitted and new pipelines is made. For the electricity network, a
        distinction is made between existing and added capacity or how much
        energy is moved via HVAC or HVDC power lines. Both measures weight
        capacity (TW) and energy (EWh) by the length (km) of the line.}
    \label{fig:network-stats}
\end{figure}

\cref{fig:network-stats} shows statistics on the total electricity and hydrogen
transmission capacity built as well as how much energy is moved through the
respective networks, while distinguishing between retrofitted and new
capacities.

\topic{capacity built}

Depending on the level of power grid expansion, between 342 and 422~TWkm of
hydrogen pipelines are built. The higher value is obtained when the hydrogen
network partially offsets the lack of electricity grid reinforcement. On the
other hand, restricting hydrogen expansion only has a small effect on
cost-optimal levels of power grid expansion. The power grid capacity is a little
more than doubled in the least-cost scenario; without a hydrogen network, the
cost-optimal power grid capacity is 10\% higher.

\topic{energy moved}

When both hydrogen and electricity grid expansion is allowed, both networks
transport approximately the same amount of energy
(\cref{fig:network-stats:ewhkm}). This is striking because the hydrogen network
capacity is less than half that of the power grid and more comparable with the
existing power grid capacities (\cref{fig:network-stats:twkm}). In consequence,
the utilisation rate of the hydrogen network (59\%) is much higher than that of
the electricity grid (35\%). Supplementary figures show the regional
distribution of network loadings.

\todo[inline]{reference SI figures for network loading}

When electricity grid expansion is restricted, the hydrogen network plays a
dominant role in transporting energy around Europe. In this case, around three
times more energy is moved in the hydrogen network (2.8~EWhkm) than in the
electricity network (1~EWhkm). However, grid expansion restrictions mainly
affect the division of energy flows between hydrogen and electricity network.
The total energy moved as hydrogen or electricity is only reduced by 22\%.

\section*{New hydrogen backbone can leverage repurposed gas pipelines}
\label{sec:repurposed}
\addcontentsline{toc}{section}{\nameref{sec:repurposed}}

\begin{figure}
    \centering
    \makebox[\textwidth][c]{
        \begin{subfigure}[t]{0.57\textwidth}
            \centering
            \caption{With grid reinforcement, with hydrogen network}
            \includegraphics[width=\textwidth, trim=0cm 0cm 7cm 0cm, clip]{\hyrun/maps/elec_s_181_lvopt__Co2L0-3H-T-H-B-I-A-solar+p3-linemaxext10-costs-all_2030.pdf}
            \label{fig:tsc:w-el-w-h2}
        \end{subfigure}
        \begin{subfigure}[t]{0.57\textwidth}
            \centering
            \caption{With grid reinforcement, without hydrogen network}
            \includegraphics[width=\textwidth, trim=0cm 0cm 7cm 0cm, clip]{\hyrun/maps/elec_s_181_lvopt__Co2L0-3H-T-H-B-I-A-solar+p3-linemaxext10-noH2network-costs-all_2030.pdf}
            \label{fig:tsc:w-el-wo-h2}
        \end{subfigure}
    } \makebox[\textwidth][c]{
        \begin{subfigure}[t]{0.57\textwidth}
            \centering
            \caption{Without grid reinforcement, with hydrogen network}
            \includegraphics[width=\textwidth, trim=0cm 0cm 7cm 0cm, clip]{\hyrun/maps/elec_s_181_lv1.0__Co2L0-3H-T-H-B-I-A-solar+p3-linemaxext10-noH2network-costs-all_2030.pdf}
            \label{fig:tsc:wo-el-w-h2}
        \end{subfigure}
        \begin{subfigure}[t]{0.55\textwidth}
            \centering
            \caption{Without grid reinforcement, without hydrogen network}
            \includegraphics[width=\textwidth, trim=0cm 0cm 7cm 0cm, clip]{\hyrun/maps/elec_s_181_lv1.0__Co2L0-3H-T-H-B-I-A-solar+p3-linemaxext10-noH2network-costs-all_2030.pdf}
            \label{fig:tsc:wo-el-wo-h2}
        \end{subfigure}
    } \caption{ Regional distribution of system costs and electricity grid
        expansion for scenarios with and without electricity or hydrogen network
        expansion. The pie charts depict the annualised system cost alongside
        the shares of the various technologies for each region. The colour
        legend is the same as in Figure 1. The line widths depict the level of
        added grid capacity, distinguishing between HVAC lines (grey) and HVDC
        links (purple). The added capacity of a connection between two regions
        was capped at 10 GW. }
    \label{fig:tsc}
\end{figure}

\begin{figure}
    \centering
    \makebox[\textwidth][c]{
        \begin{subfigure}[t]{0.6\textwidth}
            \centering
            \caption{with grid reinforcement}
            \includegraphics[width=\textwidth]{\hyrun/maps/elec_s_181_lvopt__Co2L0-3H-T-H-B-I-A-solar+p3-linemaxext10-h2_network_2030.pdf}
            \label{fig:h2-network:w-el}
        \end{subfigure}
        \begin{subfigure}[t]{0.6\textwidth}
            \centering
            \caption{without grid reinforcement}
            \includegraphics[width=\textwidth]{\hyrun/maps/elec_s_181_lv1.0__Co2L0-3H-T-H-B-I-A-solar+p3-linemaxext10-h2_network_2030.pdf}
            \label{fig:h2-network:wo-el}
        \end{subfigure}
    } \caption{Optimised hydrogen network and production sites with and without
    electricity grid reinforcement. The size of the circles depicts the
    electrolysis and fuel cell capacities in the respective region. The line
    widths depict the optimised hydrogen pipeline capacities. The darker shade
    depicts the share of capacity built from retrofitted gas pipelines.}
    \label{fig:h2-network}
\end{figure}

\topic{when is the H2 network attractive compared to the power grid?}

A network of hydrogen pipelines is particularly attractive compared to grid
expansion when the end-use are hydrogen-based products. It is less competitive
in supplying electricity demands with remote generation because additional
generation capacity would be needed to compensate for efficiency losses in the
production and re-electrification of hydrogen. The flexible use of electricity
and the dominance of electricity over hydrogen demand due to high levels of
direct electrification may explain the larger benefit gained by electricity
transmission reinforcement despite the higher development costs.

\topic{why repurposing is be attractive?}

With our assumptions, developing a MWkm of transmission lines is 1.6 times more
expensive than building a MWkm of hydrogen pipeline. We assume costs for a new
hydrogen pipeline of 250~\euro/MW/km, whereas, for a new high-voltage
transmission line, we assume 400~\euro/MW/km. The attractiveness of a hydrogen
network compared to power grid expansion is further spurred by existing gas
infrastructure that can be retrofitted. Repurposing a gas pipeline to transport
hydrogen is assumed to cost around half that of building a new hydrogen pipeline
(117 versus 250 \euro/MW/km). This estimate includes the cost of compressor
substitution. Consequently, even detours of the hydrogen network topology may be
cost-effective if, through rerouting, more repurposing potentials can be tapped.

\topic{repurposing capacities and network topology}

As \cref{fig:h2-network} illustrates, the optimised hydrogen network topology is
highly concentrated in North-Western Europe. Individual pipeline connections
between regions have optimised capacities up to 50 GW; equivalent to thirty
parallel 380~kV transmission lines \cite{}.

Of the total hydrogen network volume, between 58\% and 66\% consists of
repurposed gas pipelines. The share is highest when the electricity grid is not
permitted to be reinforced. Up to a third of the existing gas network is
retrofitted to transport hydrogen instead, leaving large capacities neither used
for hydrogen nor methane transport, particularly in Germany, Poland, Italy and
the North Sea. This is demonstrated in supplementary runs with full gas network
resolution in \cref{sec:si:detailed}. In our scenarios, a little more than 40\%
of retrofittable gas pipelines fully exhaust their conversion potential to
hydrogen.

The most notable corridors for gas pipeline retrofitting are located offshore
across the North Sea and the English Channel and in England, Germany, Austria,
and Northern Italy. The sizeable existing gas transmission capacities in
Southern Italy and Eastern Europe are not repurposed for hydrogen transport in
this self-sufficient scenario for Europe. However, this picture would likely
change if import options were considered in those regions. The most prominent
new hydrogen pipelines are built in the British Isles, between Denmark and
Germany, inside Belgium, Northern France and the Netherlands, and in the
North-East of Spain.


\section*{Regional imbalance of supply and demand is severe}
\label{sec:imbalance}
\addcontentsline{toc}{section}{\nameref{sec:imbalance}}

\begin{figure}
    \centering
    \makebox[\textwidth][c]{
        \begin{subfigure}[t]{0.6\textwidth}
            \centering
            \caption{With grid reinforcement, with hydrogen network}
            \includegraphics[width=\textwidth]{\hyrun/elec_s_181_lvopt__Co2L0-3H-T-H-B-I-A-solar+p3-linemaxext10_2030/import-export-total-200.pdf}
            \label{fig:io:w-el-w-h2}
        \end{subfigure}
        \begin{subfigure}[t]{0.6\textwidth}
            \centering
            \caption{With grid reinforcement, without hydrogen network}
            \includegraphics[width=\textwidth]{\hyrun/elec_s_181_lvopt__Co2L0-3H-T-H-B-I-A-solar+p3-linemaxext10-noH2network_2030/import-export-total-200.pdf}
            \label{fig:io:w-el-wo-h2}
        \end{subfigure}
    }
    \makebox[\textwidth][c]{
        \begin{subfigure}[t]{0.6\textwidth}
            \centering
            \caption{Without grid reinforcement, with hydrogen network}
            \includegraphics[width=\textwidth]{\hyrun/elec_s_181_lv1.0__Co2L0-3H-T-H-B-I-A-solar+p3-linemaxext10_2030/import-export-total-200.pdf}
            \label{fig:io:wo-el-w-h2}
        \end{subfigure}
        \begin{subfigure}[t]{0.6\textwidth}
            \centering
            \caption{Without grid reinforcement, without hydrogen network}
            \includegraphics[width=\textwidth]{\hyrun/elec_s_181_lv1.0__Co2L0-3H-T-H-B-I-A-solar+p3-linemaxext10-noH2network_2030/import-export-total-200.pdf}
            \label{fig:io:wo-el-wo-h2}
        \end{subfigure}
    }
    \caption{Regional total energy balances for scenarios with and without
    electricity or hydrogen network expansion, revealing regions with net energy
    surpluses and deficits. The Lorenz curves on the upper left of each map
    depict the regional inequity of electricity, hydrogen, methane and oil
    supply relative to demand. If the annual sums of supply and demand are equal
    in each region, the Lorenz curve resides on the identiy line. But the more
    unequal the regional supply is relative to demand, the further the curve
    dents into the bottom right corner of the graph.}
    \label{fig:io}
\end{figure}

\cref{fig:io} shows the net energy surpluses and deficits of each region
alongside so-called Lorenz curves that depict regional inequities between supply
and demand for each carrier and how they vary among the four network expansion
scenarios.

In line with previously shown capacity expansion plans, energy surplus is found
largely in the wind-rich coastal and solar-rich most Southern regions that
supply the inland regions of Europe, which have high demands but less attractive
renewable potentials. The net energy surplus of individual regions amounts to up
to 200 TWh. Examples are Danish offshore wind power exports in combination with
grid reinforcement and large production sites for synthetic fuels in Ireland
that exploit favourable local onshore wind potentials. Net energy deficits of
single regions have similarly high values, close to 200 TWh. Examples are, in
particular, the metropolitan areas around London and Paris, as well as the
industrial cluster between Rotterdam and the Ruhr valley.

Energy transport infrastructure fuels the uneven regional distribution of supply
relative to demand. This is nicely illustrated by the Lorenz curves presented in
\cref{fig:io} for different energy carriers and network expansion scenarios. The
Lorenz curves plot the carrier's cumulative share of supply versus the
cumulative share of demand, sorted by the ratio of supply and demand in
ascending order. If the annual sums of supply and demand are equal in each
region, the Lorenz curve resides on the identity line. However, the more unequal
the regional supply is relative to demand, the further the curve dents into the
bottom right corner of the graph.

For the least-cost scenario, \cref{fig:io:w-el-w-h2} highlights that hydrogen
supply is more regionally imbalanced relative to demand than electricity supply.
Roughly 60\% of the hydrogen demand is consumed in regions that produce less
than 1\% of the total hydrogen supply. Conversely, 40\% of the hydrogen supply
is produced in regions that consume less than 5\% of total hydrogen demand.
Naturally, reduced electricity grid expansion causes more evenly distributed
electricity supply (\cref{fig:io:wo-el-w-h2,fig:io:wo-el-wo-h2}). If hydrogen
transport is restricted (\cref{fig:io:w-el-wo-h2,fig:io:wo-el-wo-h2}), the
production of liquid hydrocarbons is increased in renewable-rich regions because
they can be transported at low cost. In this case, 70\% of the demand for
oil-based products is consumed in regions that produce less than 1\% of the
total supply. With full network expansion, 70\% of demand covers 16\% of supply.

% - 17\% of oil from fossil sources
% - 45\% of methane comes from unlocated fossil sources (explains steep end)

\section*{Onshore wind restrictions alter hydrogen backbone topology}
\label{sec:onwind}
\addcontentsline{toc}{section}{\nameref{sec:onwind}}

\begin{figure}
    \centering
    \makebox[\textwidth][c]{
        \begin{subfigure}[t]{0.6\textwidth}
            \centering
            \caption{hydrogen network}
            \includegraphics[height=0.42\textheight]{\hyrun/maps/elec_s_181_lv1.0__Co2L0-3H-T-H-B-I-A-solar+p3-linemaxext10-onwind+p0-h2_network_2030.pdf}
            \label{fig:no-onw:h2}
        \end{subfigure}
        \begin{subfigure}[t]{0.6\textwidth}
            \centering
            \caption{energy balance}
            \includegraphics[height=0.42\textheight]{\hyrun/elec_s_181_lv1.0__Co2L0-3H-T-H-B-I-A-solar+p3-linemaxext10-onwind+p0_2030/import-export-total-200.pdf}
            \label{fig:no-onw:io}
        \end{subfigure}
        }
    \begin{subfigure}[t]{0.6\textwidth}
        \centering
        \caption{system cost}
        \includegraphics[height=0.42\textheight]{\hyrun/maps/elec_s_181_lv1.0__Co2L0-3H-T-H-B-I-A-solar+p3-linemaxext10-onwind+p0-costs-all_2030.pdf}
        \label{fig:no-onw:tsc}
    \end{subfigure}
    \caption{Maps of regional energy balance, hydrogen network and production sites, and spatial and technological distribution of system costs for a scenario without onshore wind and without power grid expansion.}
    \label{fig:no-onw}
\end{figure}

\topic{cost impact and overall system}

Like building new power transmission lines, the deployment of onshore wind may
not always be socially accepted, such that it may not be possible to leverage
its full potential. In the following additional sensitivity analysis, we explore
the hypothetical impact of restricting the installable potentials of onshore
wind down to zero (\cref{fig:no-onw}).

We find that as onshore wind is eliminated, costs rise by \euro~104 bn/a (12\%)
when the electricity grid is fixed to today's capacities, but a hydrogen network
can still be developed. In comparison to the least-cost solution with full
network expansion, this solution is 23\% more expensive. A solution in which
neither a hydrogen network could be developed would be 30\% more expensive.
\cref{sec:si:onw} presents intermediate results between full and no onshore wind
expansion. The model substitutes onshore wind, particularly in the British
Isles, for higher investment in offshore wind in the North Sea and solar
generators in Southern Europe (\cref{fig:no-onw:tsc}). Because offshore
capacities are concentrated near coastlines, and grid capacity is restricted,
total spending on hydrogen electrolysers and networks also increases to absorb
the increased offshore generation. Without onshore wind, the potentials for
rooftop solar PV and offshore wind in Europe are largely exhausted, such that in
this self-sufficient scenario for Europe, the effect of installable potentials
becomes critical.

\topic{changes in hydrogen infrastructure}

Whereas with onshore wind, the British Isles and North Sea dominate hydrogen
production, Southern Europe becomes a large exporter of solar-based hydrogen if
the development of onshore wind capacities is restricted
(\cref{fig:no-onw:h2,fig:no-onw:io}). This shift in hydrogen infrastructure also
impacts the share of gas pipelines being retrofitted for hydrogen transport. As
the Iberian Peninsula becomes a preferred region for hydrogen production but has
a more sparse gas transmission network today, the rate of retrofitted pipeline
capacity reduces from 66\% to 53\%. Many new hydrogen pipelines are built to
connect Spain with France, but also to connect increased hydrogen production
from Danish offshore wind to Germany. Gas pipeline retrofitting is then
concentrated in Germany, Austria and Italy.
