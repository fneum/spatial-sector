\section{Hydrogen}
\label{sec:si:h2}

\subsection{Hydrogen Demand}
\label{sec:si:h2:demand}

Hydrogen is consumed in the industry sector to produce ammonia and direct
reduced iron (DRI) (see \cref{sec:si:industry:steel}). Hydrogen is also consumed
to produce synthetic methane and liquid hydrocarbons (see
\cref{sec:si:methane:supply} and \cref{sec:si:oil:supply}) which have multiple
uses in industry and other sectors. For transport applications, the consumption
of hydrogen is exogenously fixed. It is used in heavy-duty land transport (see
\cref{sec:si:transport:land}) and as liquified hydrogen in the navigation sector
(see \cref{sec:si:transport:shipping}). Furthermore, stationary fuel cells may
re-electrify hydrogen (with waste heat as a byproduct) to balance renewable
fluctuations. The regional distribution of hydrogen demands is shown in
\cref{fig:demand-space:hydrogen}.

\subsection{Hydrogen Supply}
\label{sec:si:h2:supply}

Today, most hydrogen is produced from natural gas by steam methane reforming
(SMR)
\begin{equation}
    \ce{ CH4 + H2O -> CO + 3H2 }
\end{equation}
combined with a water-gas shift reaction
\begin{equation}
    \ce{CO + H2O -> CO2 + H2}.
\end{equation}
We consider this route of production with and without carbon capture (CC),
assuming a capture rate of 90\%. These routes are also referred to as blue and
grey hydrogen. The methane input can be of fossil or synthetic origin.

Furthermore, we consider water electrolysis (green hydrogen) which uses electric
energy to split water into hydrogen and oxygen
\begin{equation}
    \ce{2H2O -> 2 H2 + O2}.
\end{equation}
For the electrolysis, we assume alkaline electrolysers since they have lower
cost \citeS{} and higher cumulative installed capacity \citeS{} than polymer
electrolyte membrane (PEM) electrolysers. Waste heat from electrolysis is not
leveraged in the model.

The split between these three different technology
options and their installed capacities are a result of the optimisation
depending on the techno-economic assumptions listed in \cref{sec:si:costs}.

\subsection{Hydrogen Transport}
\label{sec:si:h2:transport}

Hydrogen can be transported in pipelines. These can be retrofitted natural gas
pipelines or completely new pipelines. The cost of retrofitting a gas pipeline
is about half that of building a new hydrogen pipeline. These costs include the
cost for new compressors but neglect the energy demand for compression.

The endogenous retrofitting of gas pipelines to hydrogen pipelines is
implemented in a way, such that for every unit of gas pipeline decommissioned,
60\% of its nominal capacity are available for hydrogen transport on the
respective route, following assumptions from the European Hydrogen Backbone
report \citeS{EuropeanHydrogen}\todo{check this assumption in EHB}. When the gas network is not resolved, this
value denotes the potential for repurposed hydrogen pipelines.

New pipelines can be built additionally on all routes where there currently is a
gas or electricity network connection. These new pipelines will be built where
no sufficient retrofitting options are available. The capacities of new and
repurposed pipelines are a result of the optimisation.

\subsection{Hydrogen Storage}
\label{sec:si:h2:storage}

\begin{figure}
    \centering
    \makebox[\textwidth][c]{
    \begin{subfigure}[t]{0.6\textwidth}
        \centering
        \includegraphics[width=\textwidth]{caverns.pdf}
    \end{subfigure}
    \begin{subfigure}[t]{0.6\textwidth}
        \centering
        \includegraphics[width=\textwidth]{cavern-potentials-offshore.pdf}
    \end{subfigure}
    }
    \makebox[\textwidth][c]{
    \begin{subfigure}[t]{0.6\textwidth}
        \centering
        \includegraphics[width=\textwidth]{cavern-potentials-onshore.pdf}
    \end{subfigure}
    \begin{subfigure}[t]{0.6\textwidth}
        \centering
        \includegraphics[width=\textwidth]{cavern-potentials-nearshore.pdf}
    \end{subfigure}
    }
    \caption{Cavern storage potentials}
    \label{fig:clustered-caverns}
\end{figure}

Hydrogen can be stored in overground steel tanks or underground salt caverns.
The annuitised cost for cavern storage is around 30 times lower than for storage
in steel tanks including compression. For underground storage potentials for
hydrogen in European salt caverns we take data from Caglayan et
al.~\citeS{caglayanTechnicalPotential2019} and map it to
each of the 181 model regions (\cref{fig:clustered-caverns}). We include only
those caverns that are located on land and within 50 km of the shore
(nearshore). We impose this restriction to circumvent environmental problems
associated with brine water disposal \citeS{caglayanTechnicalPotential2019}. The
storage potential is abundant and the constraining factor is more where they
exist and less how large the energy storage potentials are.

\section{Methane}
\label{sec:si:methane}

\subsection{Methane Demand}
\label{sec:si:methane:demand}

Methane is used in individual and large-scale gas boilers, in CHP plants with
and without carbon capture, in OCGT and CCGT power plants, and in some industry
subsectors for the provision of high temperature heat (see
\cref{sec:si:industry}) Methane is not used in the transport sector because of
engine slippage.  The regional distribution of methane demands is shown in
\cref{fig:demand-space:hydrogen}. However, the results shown in the main body of
the paper relax all methane transmission costs and constraints.

\subsection{Methane Supply}
\label{sec:si:methane:supply}

Besides methane from fossil origins, the model also considers biogenic and
synthetic sources. If gas infrastructure is regionally resolved (see
\cref{sec:si:methane:transport}), fossil gas can enter the system only at
existing and planned LNG terminals, pipeline entry-points, and intra-European
gas extraction sites (see \cref{fig:gas-raw}), which are retrieved from the
SciGRID Gas IGGIELGN dataset \citeS{plutaSciGRIDGas2022} and the GEM Wiki
\citeS{}. Biogas can be upgraded to methane (see \cref{sec:si:bio:potentials}).
Synthetic methane can be produced by processing hydrogen and captures \co in the
Sabatier reaction
\begin{equation}
    \ce{CO2 + 4H2 -> CH4 + 2H2O}.
\end{equation}
Direct power-to-methane conversion with efficient heat integration developed in
the HELMETH project is also an option \citeS{gruberPowertoGasThermal2018}. The
share of synthetic, biogenic and fossil methane is an optimisation result
depending on the techno-economic assumptions listed in \cref{sec:si:costs}.

\subsection{Methane Transport}
\label{sec:si:methane:transport}

\begin{figure}
    \includegraphics[width=1\textwidth,center]{gas_network.pdf}
    \label{fig:gas-raw}
    \caption{Gas network}
\end{figure}

The existing European gas transmission network is represented based on the
SciGRID Gas IGGIELGN dataset \citeS{plutaSciGRIDGas2022}, as shown in
\cref{fig:gas-raw}. This dataset is based on compiled and merged data from the
ENTSOG map \citeS{} and other publicly available data sources. It includes data
on the capacity, diameter, pressure, length, and directionality of pipelines.
Missing capacity data is conservatively inferred from the pipe diameter
following conversion factors derived from \cite{EuropeanHydrogen}. The gas
network is clustered to the model's 181 regions (see
\cref{fig:clustered-networks}). Gas pipelines can be endogenously expanded or
repurposed for hydrogen transport (see \cref{sec:si:h2:transport}). Gas flows
are represented by a lossless transport model.

The results presented in the main body of the article regard the gas
transmission network only to determine the retofitting potentials for hydrogen
pipelines. These assume methane to be transported without cost or capacity
constraints, since future demand is predicted to be low compared to available
transport capacities even if a certain share is repurposed for hydrogen
transport such that no bottlenecks are expected. Selected runs with gas network
infrastructure included are presented in \cref{sec:si:detailed}.

\section{Oil-based Products}
\label{sec:si:oil}

\subsection{Oil-based Product Demand}
\label{sec:si:demand}

Naphtha is used as a feedstock in the chemicals industry (see
\cref{sec:si:industry:chemicals}). Furthermore, kerosene is used as transport
fuel in the aviation sector (see \cref{sec:si:transport:aviation}).
Non-electrified agriculture machinery also consumes gasoline. The regional
distribution of the demand for oil-based products is shown in
\cref{fig:demand-space:oil}. However, this carrier is copperplated in the model.

\subsection{Oil-based Product Supply}
\label{sec:si:oil:supply}

In addition to fossil origins, oil-based products can be synthetically produced
by processing hydrogen and captured \co in Fischer-Tropsch plants
\begin{equation}
    \ce{$n$CO + ($2n$ + 1)H2 -> C_$n$H_{2n+2} + $n$H2O}.
\end{equation}
with costs as included in \cref{sec:si:costs}. The waste heat from the
Fischer-Tropsch process is supplied to district heating networks.

\subsection{Oil-based Product Transport}
\label{sec:si:oil:transport}

Liquid hydrocarbons are assumed to be transported freely among the model region
since future demand is predicted to be low, transport costs for liquids are low
and no bottlenecks are expected.

\section{Biomass}
\label{sec:si:bio}



\subsection{Biomass Supply and Potentials}
\label{sec:si:bio:potentials}

\begin{figure}
    \centering
    \makebox[\textwidth][c]{
    \begin{subfigure}[t]{0.45\textwidth}
        \centering
        % \caption{electricity demand}
        \includegraphics[width=\textwidth]{biomass-solid biomass.pdf}
    \end{subfigure}
    \begin{subfigure}[t]{0.45\textwidth}
        \centering
        % \caption{hydrogen demand}
        \includegraphics[width=\textwidth]{biomass-biogas.pdf}
    \end{subfigure}
    \begin{subfigure}[t]{0.45\textwidth}
        \centering
        % \caption{methane demand}
        \includegraphics[width=\textwidth]{biomass-not included.pdf}
    \end{subfigure}
    }
    \caption{Biomass potentials.}
    \label{fig:biomass-potentials}
\end{figure}

Regional biomass supply potentials are taken from the JRC ENSPRESO database
\citeS{jrcbiomass2015}. This dataset includes various biomass feedstocks at
NUTS2 resolution for low, medium and high availability scenarios.  We use the
medium availability scenario for 2030, assuming no biomass import from outside
Europe. The data for NUTS2 regions is mapped to PyPSA-Eur-Sec model regions in
proportion to the area overlap.

Only residues from agriculture, forestry, and biodegradable municipal
waste are considered as energy feedstocks. Fuel crops are avoided because they
compete with scarce land for food production, while primary wood as well as wood
chips and pellets are avoided because of concerns about sustainability
\citeS{bentsenCarbonDebt2017}.
Manure and sludge waste are available to the model as biogas, whereas other
wastes and residues are classified as solid biomass. Solid biomass resources are
available for combustion in combined-heat-and-power (CHP) plants and for medium
temperature heat (below \SI{500}{\celsius}) applications in industry.
The technical characteristics for the solid biomass CHP are taken from the
Danish Energy Agency Technology Database \citeS{dea2016} assumptions for a
medium-sized back pressure CHP with wood pellet feedstock; this has very similar
costs and efficiencies to CHPs with feedstocks of straw and wood chips.

A summary of which feedstocks are used in the model is shown in
\cref{tab:biomass}; the respective regional distribution of potentials is
included in \cref{fig:biomass-potentials}. In 2015, the EU28 biomass energy
consumption consisted of \SI{180}{\twh} of biogas, \SI{1063}{\twh} of solid
biofuels, \SI{109}{\twh} renewable municipal waste and \SI{159}{\twh} of liquid
biofuels \citeS{}. In comparison, PyPSA-Eur-Sec implies a doubling of biogas
consumption and similar amounts of solid biofuels, but a shift from energy crops
and primary wood to residues and wastes. Zappa et al.
\citeS{zappa100Renewable2019} additionally allowed the use of roundwood chips
and pellets, and grassy, willow and poplar energy crops.

\begin{table}
    \centering
    \small
    \begin{tabularx}{\textwidth}{lXr}
        \toprule
        Application & Source & Potential [\si{\twh\per\year}] \\
        \midrule
        solid biomass & primary agricultural residues; forest energy residue; secondary forestry residues: woodchips, sawdust; forestry residues from landscape care; biodegradable municipal waste & 1186 \\
        biogas & wet and dry manure; biodegradable sludge & 346\\
        not used & energy crops: sugar beet bioethanol, rape seed and other oil crops, starchy crops, grassy, willow, poplar; roundwood fuelwood; roundwood chips and pellets & 1661 \\
        \bottomrule
    \end{tabularx}
    \caption{Use of biomass potentials according to classifications from the JRC \cite{jrcbiomass2015} in the medium availability scenario for 2030.}
    \label{tab:biomass}
\end{table}

\subsection{Biomass Demand}
\label{sec:si:bio:demand}

Solid biomass provides process heat up to \SI{500}{\celsius} in industry and can
also feed CHP plants in district heating networks. As noted in
\cref{sec:si:industry}, solid biomass is used as heat supply in the paper and
pulp and food, beverages and tobacco industries, where required temperatures are
lower \citeS{naeglerQuantificationEuropean2015,rehfeldtBottomupEstimation2018}.
The regional distribution of solid biomass demand is shown in
\cref{fig:demand-space:biomass}.

\subsection{Biomass Transport}
\label{sec:si:bio:transport}

Solid biomass is assumed to be transported freely among the modelled regions.
Biogas can be upgraded and then transported via the methane network. If the
methane network is neglected, also biogas can be moved without cost or
constraints.\todo{Is this still up-to-date? Check!}