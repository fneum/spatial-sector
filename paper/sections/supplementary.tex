\section{Model Overview}
\label{sec:si:model-overview}

\begin{figure}
    \centering
    \includegraphics[]{../graphics/multisector_figure.pdf}
\end{figure}

\begin{figure}
    \centering
\begin{subfigure}[t]{0.49\textwidth}
    \centering
    % \caption{clustered electricity network}
    \includegraphics[width=\textwidth]{electricity-network-today-map.pdf}
\end{subfigure}
\begin{subfigure}[t]{0.49\textwidth}
    \centering
    % \caption{gas network}
    \includegraphics[width=\textwidth]{gas-network-today-map.pdf}
\end{subfigure}
\caption{Clustered electricity and gas transmission networks.}
\label{fig:clustered-networks}
\end{figure}

\begin{figure}
    \centering
    \includegraphics[height=0.25\textheight]{total-annual-demand.pdf}
    \includegraphics[height=0.25\textheight]{ts-demand.pdf}
    \caption{Final energy demand temporal diversity.}
    \label{fig:demand-time}
\end{figure}

\newgeometry{top=0.5cm, bottom=1.5cm}
\begin{figure}
    \centering
    \begin{subfigure}[t]{0.49\textwidth}
        \centering
        \caption{electricity demand}
        \label{fig:demand-space:electricity}
        \includegraphics[width=\textwidth]{demand-map-electricity.pdf}
    \end{subfigure}
    \begin{subfigure}[t]{0.49\textwidth}
        \centering
        \caption{hydrogen demand}
        \label{fig:demand-space:hydrogen}
        \includegraphics[width=\textwidth]{demand-map-H2.pdf}
    \end{subfigure}
    \begin{subfigure}[t]{0.49\textwidth}
        \centering
        \caption{methane demand}
        \label{fig:demand-space:methane}
        \includegraphics[width=\textwidth]{demand-map-gas.pdf}
    \end{subfigure}
    \begin{subfigure}[t]{0.49\textwidth}
        \centering
        \caption{heat demand}
        \label{fig:demand-space:heat}
        \includegraphics[width=\textwidth]{demand-map-heat.pdf}
    \end{subfigure}
    \begin{subfigure}[t]{0.49\textwidth}
        \centering
        \caption{oil demand}
        \label{fig:demand-space:oil}
        \includegraphics[width=\textwidth]{demand-map-oil.pdf}
    \end{subfigure}
    \begin{subfigure}[t]{0.49\textwidth}
        \centering
        \caption{solid biomass demand}
        \label{fig:demand-space:biomass}
        \includegraphics[width=\textwidth]{demand-map-solid biomass.pdf}
    \end{subfigure}
    \caption{Final energy demand spatial diversity.}
    \label{fig:demand-space}
\end{figure}
\restoregeometry



Not all of the sectors are at the full nodal resolution, and some demand for
some sectors is distributed to nodes using heuristics that need to be corrected.
Some networks are copper-plated to reduce computational times.

overnight scenario

no pathway

weather year 2013

High resolution multi-sectoral approach: Developing the most advanced open source modeling system with regard to spatial, sectoral, technological and temporal resolution
- All energy infrastructures in one optimisation problem (electricity, gas, heat, industry)
- Detailed transmission grid representation in EU-wide model
- Detailed representation of demand sectors: Industry, buildings, transport
- so that the variability of demand and variable renewable supply can be represented, and so that existing grid bottlenecks are visible.

Energy sector coupling, storage and conversion is modelled to connect
electricity, heating (individual buildings, district heating and industry),
transport and gas (methane, hydrogen and carbon dioxide) in the different
sectors (buildings, transport and industry).

The spatial resolution of the model can be customised by the user. It can be set
at electricity substation level, based on administrative boundaries such as
NUTS2 or country-level, or to a custom number of nodes to enhance computational
performance. The spatial resolution of the input data varies: conventional power
plant locations are known exactly, as are large industrial facilities; wind and
solar time series are regionalised based on the underlying ERA5 reanalysis data
(around 20km by 20km); heat and transport demand at NUTS3; electricity demand
time series are at TSO control region level.

Electricity can be converted to heat via heat pumps or resistive heaters; to
hydrogen gas, or further to methane and liquid hydrocarbons; or to work in
various demand devices. Methane can be reformed to hydrogen, and most fuels can
be used for electricity generation in turbines or fuel cells.

Energy/material storage can be optimised in PyPSA-Eur-Sec including conventional
pumped hydro storage, electrochemical storage like Lithium ion batteries,
storage of gases including methane, hydrogen and carbon dioxide, storage of
liquid fuels, as well as thermal energy storage in the form of hot water both in
individual buildings and in district heating networks.

compared to electricity, heating and transport are strongly peaked
- heating is strongly seasonal but also with synoptic variations
- transport has strong daily periodicity

\section{Electricity Sector}
\label{sec:si:electricity}

Modelling of electricity supply and demand in Europe largely follows the open
electricity generation and transmission model PyPSA-Eur
\citeS{horschPyPSAEurOpen2018}. PyPSA-Eur processes publicly available data on
the topology of the power transmission network, historical time series of
weather observations and electricity consumption, conventional power plants, and
renewable potentials.

\subsection{Electricity Demand}
\label{sec:si:electricity:demand}

Hourly electricity demand at country-level for the reference year 2013 published
by ENTSO-E is retrieved via the interface of the Open Power System Data (OPSD)
initiative \citeS{}. Existing electrified heating is subtracted from this
demand, so that power-to-heat options can be optimised separately. Furthermore,
current industry electricity demand is subtracted and handled separately
considering further electrification in the industry sector (see XXX).

For the distribution of electricity demand for industry we leverage geographical
data from the industrial database developed within the Hotmaps project \citeS{}.
The remaining electricity demand for households and services is heuristically
distributed inside each country to 40\% proportional to population density and
to 60\% proportional to gross domestic product based on regression performed for
\citeS{horschPyPSAEurOpen2018}. The total spatial distribution of electricity
demands is shown in \cref{fig:demand-space:electricity}.

\subsection{Electricity Supply}

\begin{SCfigure}
    \caption{Powerplants.}
    \label{fig:powerplants}
    \includegraphics[width=0.7\textwidth]{powerplants.pdf}
\end{SCfigure}

For conventional electricity generators, PyPSA-Eur-Sec uses the open
\textit{powerplantmatching} tool, which merges datasets from a variety of
sources \citeS{gotzensPerformingEnergy}. As shown in \cref{fig:powerplants}, it
provides data on the power plants about their location, technology and fuel
type, age, and capacity, inlcuding hard coal, lignite, oil, open and combined
cycle gas turbines (OCGT and CCGT), and nuclear generators. Furthermore,
existing run-of-river, pumped-hydro storage plants, and hydro-electric dams, are
also part of the dataset, for which inflow is modelled based on runoff data from
reanalysis weather data and and scaled hydropower generation statistics (see
XXX). In general, we suppose these to be non-extendable due to assumed
geographical constraints.

Expandable renewable generators include onshore and offshore wind, utility-scale
and rooftop solar photovoltaics, biomass from multiple feedstocks. The model
decides to build new capacities based on available land and on the weather
resource (see XXX). Because the continent-wide avaialbility of data on the
locations of wind and solar installations is fragmentary, we disregard already
existing wind and solar capacities. Moreover, new OCGT and CCGT as well as gas or biomass-fueled combined heat and power (CHP) generators may be built.

back-pressure plants with heat production proportional to electricity output

Specific techno-economic assumptions, like costs, lifetimes and efficiencies are
included in \cref{sec:si:costs}.

\subsection{Electricity Storage}
\label{sec:si:electricity:storage}

Electric energy can be stored in batteries (home, utility-scale, electric
vehicles), existing pumped-hydro storage (PHS), hydrogen storage and other
synthetically produced energy carriers (like methane and oil). For stationary
batteries we distinguish costs for inverters and for storage at home or
utility-scale. With these assumptions, home battery storage is about 40\% more
expensive than utility-scale battery storage (see \cref{sec:si:costs}). The
batteries' energy and power capacities can be independently sized.

To store electricity, hydrogen may be produced by water electrolysis (see \cref{sec:si:h2:supply}),
stored in overground steel tanks or underground salt caverns (see see \cref{sec:si:h2:storage}), and
re-electrified in a utility-scale fuel cell. Synthetic methane can be
re-electrified through an open cycle gas turbine (OCGT) or a combined heat and
power (CHP) plant.

\subsection{Electricity Transport}
\label{sec:si:electricity:transport}

\begin{figure}
    \includegraphics[width=\textwidth, center]{network.pdf}
    \label{fig:base-network}
    \caption{ENTSO-E transmission network}
\end{figure}

% topology

The topology of the European electricity transmission network is represented at
substation level based on maps released in the interactive ENTSO-E map \citeS{}
using a modified version of the GridKit \citeS{} tool. As displayed in
\cref{fig:base-network}, the dataset includes HVAC ines at and above 220 kV
across the mulitple synchronous zones of the ENTSO-E area, but excludes Turkey
and North-African countries which are also synchronised to the continental
European grid, interconnections to Russia, Belarus and Ukraine as well as small
island networks with less than four nodes at transmission level, such as Cyprus,
Crete and Malta. In total, the network encompasses around 3000 substations, 6600
HVAC lines and around 70 HVDC links, some of which are planned projects from the
Ten Year Network Development Plant (TYNDP) that are not yet in operation
\citeS{}.

% unclustered model regions

The transmission network topology determines the basic regions of the
PyPSA-Eur-Sec model. Each substation has an associated Voronoi cell that
describes the region that is closer to the substation than to any other
substation except for country borders, which are kept to retain the integrity of
country totals. Exemplary Voronoi cells are illustrated in \cref{fig:voronoi}.
We use these as geographical catchment area for demands, renewable resource
potentials, and power plants, assuming that supply and demand always connect to
the closest substation. The Voronoi cells are also computed for offshore regions
based on the countries' Exclusive Economic Zones (EEZs) and the adjacent onshore
substations.

\begin{SCfigure}
    \caption{Exemplary Voronoi cells of the transmission network's substations.}
    \includegraphics[width=0.55\textwidth]{voronoi}
    \label{fig:voronoi}
\end{SCfigure}

% physical modelling

Capacities and electrical characteristics of transmission lines and substations,
such as impedances and thermal ratings\todo{summer or winter ratings?}, are
inferred from standard types for each voltage level from \citeS{}. For each HVAC
line, we further restrict line loading to 70\% of the nominal rating to
approximate $N-1$ security, which protects the system against overloading if any
one transmission line fails. This conservative security margin is commonly
applied in the industry \citeS{}. Dynamic line rating is not considered. Power
flow is modelled through lossless linearised power flow equations using an
efficient cycle-based formulation of Kirchhoff's voltage law
\citeS{horschLinearOptimal2018}.

% clustering

Solving the capacity expansion optimisation for the whole European energy system
at full network resolution is to large to be solved in reasonable time.
Therefore, we simplify the network topology by lowering the spatial resolution.
We initially remove the network's radial paths, i.e.~nodes with only one
connection, by linking remote resources to adjacent nodes and transforming the
network to a uniform voltage level of \SI{380}{\kilo\volt}. We also aggregate
generators of the same kind that connect to the same substation. Based on these
initial simplification, the network resolution is further reduced to a variable
number of nodes, in this case to 181 regions, by using a \textit{k-means}
clustering algorithm, which uses regional electricity consumption as weights
\citeS{frysztackiStrongEffect2021, Hoersch2017}. Only substations within the
same country can be aggregated. The equivalent lines connecting the clustered
regions are determined by the aggregated electro-technical characteristics of
original transmission lines. Their weighted cost takes into consideration the
underwater fraction of the lines and adds 25\% to the crow-fly distance to
approximate routing constraints.

% distribution grid

Contrary to the transmission level, the grid topology at the distribition level
(at and below \SI{110}{\kilo\volt}) is not included. Only the total power
exchange capacity between transmission and distribution level is co-optimised.
Costs of \SI{500}{\sieuro\per\kilo\watt} are assumed as well as lossless
distribution. Rooftop PV, heat pumps, resistive heaters, home batteries,
electric vehicles and electricity demands are connected to the low-voltage
level. All other remaining technologies connect directly to the transmission
grid. In this way, distribution grid capacity is developed if it is beneficial
to balance the local mismatch between supply and demand.

\section{Transport Sector}
\label{sec:si:transport}

Transport and mobility comprises light and heavy road, rail, shipping and
aviation transport. Annual energy demands for this sector are derived from the
JRC-IDEES database \citeS{IDEES}.

\subsection{Land Transport}
\label{sec:si:transport:land}

The diffusion of battery electric vehicles (BEV) and fuel cell electric vehicles
(FCEV) in land transport is exogenously defined. For 2050, we assume that 85\%
of land transport is electrified and 15\% uses hydrogen fuel cells. No more
internal combustion engines exist.

The energy savings gained by electrifying road transport, are computed through
country-specific factors that compare the current final energy consumption of
cars per distance travelled (average for Europe
\SI{0.7}{\kilo\watt\hour\per\kilo\metre}, \citeS{}) to the
\SI{0.18}{\kilo\watt\hour\per\kilo\metre} assumed for the battery-to-wheel
efficiency of electric vehicles.

Weekly profiles of distances travelled published by the Germand Federal Highway
Research Institute (BASt) \citeS{} are used to generate hourly time series for
each European country taking into account their local time. Furthermore, a
temperature dependence is included in the time series to account for
heating/cooling demand in transport. For temperatures below \SI{15}{\celsius}
and above \SI{20}{\celsius} temperature coefficients of
\SI{0.98}{\percent\per\celsius} and \SI{0.63}{\percent\per\celsius} are assumed
\citeS{brown2018}

For battery electric vehicles, we assume a storage capacity of
\SI{50}{\kilo\watt\hour}, a charging capacity of \SI{11}{\kilo\watt} and a 90\%
charging efficiency. We assume that half of the BEV fleet can shift their
charging time and participate in vehicle-to-grid (V2G) services to facilitate
system operation. The BEV state of charge is forced to be higher than 75\% at
7am every day to ensure that the batteries are sufficiently charged for the peak
usage in the morning. This also restricts BEV demand to be shifted within a day
and prevent EV batteries from becoming seasonal storage. The percentage of BEV
connected to the grid at any time is inversely proportional to the transport
demand profile, which translates into an average/minimum availability of
80\%/62\% of the time. These values are conservative compared to most of the
literature, where average parking times of the European vehicle fleet is
estimated at 92\% \citeS{}. The battery cost of BEV is not included in the model
since it is assumed that BEV owners buy them to primarily satisfy their mobility
needs.

\subsection{Aviation}
\label{sec:si:transport:aviation}

The aviation sector consumes kerosene that is synthetically produced or of
fossil origin (see XXX).

\subsection{Shipping}
\label{sec:si:transport:shipping}

The shipping sector consumes liquid hydrogen.
The liquefaction costs for hydrogen are taken into account (see XXX).
Other fuel options, like methanol or ammonia, are currently not considered.

\section{Industry Sector}
\label{sec:si:industry}

\citeS{graichenKlimaneutraleIndustrie,rootzenExploringLimits2013}

Industry demand is split into a dozen different sectors, with the major ones
being iron and steel, cement and basic chemicals. The location of existing
industrial facilities is based on the Horizon-2020-funded project hotmaps.
Demand for materials can either be met using existing conventional routes, such
as blast furnaces for iron and rotary kilns for cement, or with new processes,
such as hydrogen direct reduction for iron or kilns equipped with CCS and/or
oxyfuel for cement. Basic chemicals can use green hydrogen as a feedstock rather
than fossil fuels, for example for ammonia production, or using Fischer-Tropsch
naphtha in steam crackers for High Value Chemicals.

\begin{SCfigure}
    \includegraphics[width=0.7\textwidth]{fec_industry_today_tomorrow.pdf}
    \caption{Final consumption of energy and non-energy feedstocks in industry today (left bar) and
    our future scenario in 2050 (right bar)}
    \label{fig:fec-industry}
\end{SCfigure}

\begin{SCfigure}
    \includegraphics[width=0.7\textwidth]{process-emissions.pdf}
    \caption{Process emissions in industry today (top bar) and in 2050 (bottom bar)}
    \label{fig:process-emissions}
\end{SCfigure}

\begin{SCfigure}
    \includegraphics[width=0.7\textwidth]{hotmaps.pdf}
    \caption{Distribution of industries.}
    \label{fig:hotmaps}
\end{SCfigure}

\begin{table}[t]
    \centering
    \setlength{\tabcolsep}{6pt}
    \begin{tabular}{@{} p{5cm}r @{}}
      \toprule
      Material & Production [\si{\mega\tonne\per\year}] \\
      \midrule
      steel & 174 \\
      cement & 186 \\
      glass & 44 \\
      ceramics and other NMM & 360 \\
      ammonia & 18 \\
      other basic chemicals & 74 \\
      other chemicals & 28 \\
      pulp & 39 \\
      paper & 94 \\ \bottomrule
    \end{tabular}
    \caption{Industrial production for main products in 2015.}
    \label{tab:industryproduction}
  \end{table}


emissions
- energy-related
- process-related

For process-related emissions need alternative manufacturing process
or higher rates of recycling such that less virgin material is needed

largest sectors (according to FEC)
- iron and steel
- chemicals
- non-metallic mineral products
- pulp paper printing
- food beverages tobacco
- non-ferrous metals

Demand

For the industry energy demand, we assume the same
production of materials as today, but with process switching, fuel switching to
low-emission alternatives as well as carbon capture for use or sequestration
(CCU/S).

Process switching includes, for example, moving primary production of
steel from blast furnace reduction to direct reduction with hydrogen, or
switching to an oxyfuel process in cement manufacture.

Fuel switching means
replacing mechanical processes and process heat applications that use fossil
fuels with low-emission alternatives such as electricity, biomass or synthetic
fuels such as hydrogen or methane.

Feedstocks for the chemicals industry are
also converted to non-fossil alternatives.

To determine where fuel switching can take place we use the breakdown of energy
usage for each process within each sector provided by the JRC-IDEES database
\citeS{IDEES}.

Where there are fossil and electrified alternatives for the same
process (e.g. in glass manufacture or drying in industry XX) we assume that the
process is completely electrified.

Where process heat is required (steam
process?) our approach depends on the temperature required
\citeS{naeglerQuantificationEuropean2015,rehfeldtBottomupEstimation2018}. Processes that require temperatures below
\SI{500}{\celsius} are supplied with solid biomass, since we assume that residues and
wastes are not suitable for high-temperature applications. We see solid biomass
use primarily in the pulp and paper industry, where it is already widespread,
and in food, beverages and tobacco, where it replaces natural gas. Industries
which require high temperatures (above \SI{500}{\celsius}), such as metals, chemicals
and non-metalic minerals are either electified where processes already exist, or
the heat is provided with synthetic methane.

what is currently supplied with electricity (lighting, air compressors, motor drives, fans, pumps)
and low-enthalpy heat demand are directly added to electricity and heat buses

Industry heat demand can be supplied via DH or is added to services heat demand.
Can be supplied by other processes producing heat as byproduct (DAC, FT)

For EU-28 plus NO, CH, IL Rehfeldt estimated that from 2015 industrial heat demand
45\% is above \SI{500}{\celsius}, 30\% \SIrange{100}{500}{\celsius}, 25\% below \SI{100}{\celsius}

\citeS{naeglerQuantificationEuropean2015} similarly:
48\% is above \SI{400}{\celsius}, 27\% \SIrange{100}{400}{\celsius}, 25\% below \SI{100}{\celsius}

Because of high share of high-temperature process heat demand
- no geothermal
- no solar thermal supply

process heat supplied by methane, biomass, electricity depending on sector

Based on materials demand from JRC-IDEES and other sources such as the USGS for
ammonia.

Industry is split into many sectors, including iron and steel, ammonia, other
basic chemicals, cement, non-metalic minerals, alumuninium, other non-ferrous
metals, pulp, paper and printing, food, beverages and tobacco, and other more
minor sectors.

Inside each country the industrial demand is distributed using the Hotmaps
Industrial Database.

Hotmaps open database includes cement, basic chemicals, glass, iron and steel, non-ferrous metals,
non-metallic minerals, paper, refineries. Enables regional analyses, calculation of
site-specific energy demand, waste heat potentials, emissions, market shares,
process-specific evaluations

calculate energy demands and process emissions per unit of material based on JRC-IDEES

material output per country from JRC IDEES, ammonia production statistics,
Eurostat energy balances, national statistics from Switzerland to calculate total energy demand
and process emissions by sector

material output assumed to stay constant (exception recycling)

Supply

Process switching (e.g. from blast furnaces to direct reduction and electric arc
furnaces for steel) is defined exogenously.

Fuel switching for process heat is mostly also done exogenously.

Solid biomass is used for up to 500 Celsius, mostly in paper and pulp and food
and beverages.

Higher temperatures are met with methane.

\citeS{neuwirthFuturePotential2022}

\subsection{Iron and Steel}
\label{sec:si:industry:steel}

\citeS{toktarovaInteractionElectrified2022, mandovaPossibilitiesCO22018, suopajarviUseBiomass2018,voglPhasingOut2021, bhaskarDecarbonizingPrimary}

70\% from scrap, rest from direct reduction with 1.7 MWh H2 / t steel + electric arc (process emissions 0.03 t COs / t steel)

Two routes today to manufacture steel in Europe
- primary/integrated steelworks (60\% of steel production)
- secondary/electric arc furnaces (40\%) \citeS{lechtenbohmerDecarbonisingEnergy2016}

Primary - Integrated steelworks
- use blast furnaces in which coke is used to reduce iron ore into molten iron
\begin{equation}
    \ce{FE2O3 + 3CO -> 2Fe + 3\co}
\end{equation}
\begin{equation}
    \ce{FeO + 3CO -> Fe + 3\co}
\end{equation}
- this is then converted to steel
- The primary route implies large process emissions (0.22 t\co/t steel)

Secondary - electric arc furnaces (EAF)
- use EAF to melt scrap metal
- this limits \co emissions to burning of graphite electrodes \citeS{Friedrichsen_2018}
- 0.03 t\co/t steel

DRI - Direct Reduced Iron
\begin{equation}
    \ce{FE2O3+H2 -> 2FeO + H2O}
\end{equation}
\begin{equation}
    \ce{FeO + H2 -> Fe + H2O}
\end{equation}
- this is processed in EAF and circumvents associated process emissions
- today, DRI uses methane as reduction agent but we assume this to be substituted with hydrogen
- needs 1.7 MWh H2 / t steel \citeS{voglAssessmentHydrogen2018} and 0.322 MWh elec / t steel \url{https://ssabwebsitecdn.azureedge.net/-/media/hybrit/files/hybrit_brochure.pdf}

share of steel produced by hydrogen-based DRI + EAF is exogenous: 30\%
integrated steelworks: 0\%
scrap + EAF: 70\%

replace integrated steelworks with DRI + EAF

NB: DRI already done with hydrogen in Trinidad and in Abu Dhabi (where \co from
SMR is captured and used for enhanced oil recovery) H2 Future, HYBRIT, SALCOS

\citeS{circular_economy} circular economy practices have potential of expanding
the share of secondary route to 85\% just by increasing the amount and quality of scrap metal

process emissions from lime (added to remove impurity?), graphite anodes AND
from carbon added to get steel alloy  \citeS{voglAssessmentHydrogen2018}

EAF not good for flat steel in automotive

NB: derived gases not included in model, since take from blast furnace top
gases, originating in coke

Why not add CCS to existing plants? Hard for some reason \citeS{kuramochiComparativeAssessment2012}.

For the remaining subprocesses in this sector:
- methane as energy source for smelting process
- activities associated with furnaces, refining and rolling, product finishing are electrified

\subsection{Chemicals Industry}
\label{sec:si:industry:chemicals}

\citeS{elserTakingEuropean, FuturePetrochemicals, nicholsonManufacturingEnergy2021, meysAchievingNetzero2021, thunmanCircularUse2019, introzziDECHEMAGesellschaft, meysCircularEconomy2020, guWastePlastics2017, boulamantiProductionCosts2017}

wide range of diverse industries ranging from:
- basic organics compounds (olefins, alcohols, aromatics)
- basic inorcanic compounds (ammonia, chlorine)
- polymers (plastics)
- end-user products (cosmetics, pharmaceutics)

chemicals industry consumes lots of fossil-fuel based feedstocks \citeS{leviMappingGlobal2018}

basic chemicals: HVC (high-value chemicals), chlorine, methanol and ammonia

Recycling!
- specify reuse, primary production, and mechanical and chemical recycling
fraction of platics

\citeS{circular_economy, kullmannCombiningWorlds2021,kullmannImpactsMaterial,kullmannValueRecycling,carolinaliljenstromDataSeparate, fuhrPlastikatlasDaten2019,conversioMaterialFlow2020}

synthetic naphtha for primary production

Ammonia production data is taken from the USGS

Natural gas dominates in Europe as source for hydrogen for Haber-Bosch process.
Replace with electrolytic/clean hydrogen.


ammonia production from USGS (United States Geological Survey) statistics

Assumptions for existing ammonia energy demand from XX and for electrolytic
process from XX.

hydrogen can be combined with Nitrogen to obtain ammonia in the Haber-Bosch process \citeS{leviMappingGlobal2018}
\begin{equation}
    \ce{N2 + 3H2 -> 2NH3}
\end{equation}
Currently, H2 for ammonia industry in Europe is from SMR, in the model can be SMR, SMR CC, electrolysers
\co often captured from SMR for urea production (20\% of fertilizer in Europe,
but 50-60\% in China and India) and food and beverages
with electrolytic-H2 we have 6.5 MWh \ce{H2} / t \ce{NH3} and 1.17 MWh elec / t \ce{NH3} (Wang, 2018 Joule)

chlorine, aromatics, olefins

ammonia is subtracted from basic chemicals in IDEES database.


ALL of liquid hydrocarbon feedstock from FT naphtha.
- LPG, diesel oil, residual fuel oil

We assume that all of carbon eventually finds its way from product into
atmosphere. But this is wierd, since there is no release from landfill, plastics
are not biodegradable. If they are burned as waste, we should use a
waste-to-energy plant, which we currently don't have in the model.

\citeS{lechtenbohmerDecarbonisingEnergy2016} has 14.8 MWh FT-naphtha per ton of HVC, and 2.7 MWh per
ton for processing; 3.1 t\co/tHVC are captured from flue gases.

TODO: We have much smaller emissions, are we accounting for heating as well as
process emission? I guess with methane we do...

share of synthetic vs fossil-fuel based methane and naphtha is endogenous

transformation of energy-consuming processes:
- FEC in steam processing is converted to methane since requires temperature above \SI{500}{\celsius} \citeS{rehfeldtBottomupEstimation2018}
- remaining processes are electrified using the current efficiency  of microwave for high-enthalpy heat processing,
  electric furnaces, electric process cooling and electric generic processes

process emissions from feedstock in the chemical industry represent 0.369 t\co / t ethylene eq.
- we consider process emissions for all the material output
- conservative, because it assumes that all plastic-embedded \co will eventually be released into the atmosphere
- plastic disposal in landfilling will avoid, or at least delay, associated \co emissions

\subsection{Non-metallic Mineral Products}
\label{sec:si:industry:nmmp}

Cement

\citeS{fennellDecarbonizingCement2021}

waste and solid biomass; capture of \co emissions

cement manufacturing involves large process and energy emissions

calcination of limestone to chemically reactive calcium oxide (known as lime)
involves process emissions of 0.54 t \ce{\co} / t cement
\begin{equation}
    \ce{CaCO3 -> CaO + \co}
\end{equation}

mitigation strategies (new raw materials, recovering unused cement from concrete at end of life \url{https://ec.europa.eu/clima/news/commission-calls-climate-neutral-europe-2050_en})
are at a very early development stage and have not been considered

Keep current share of biomass; replace fossil fuels for process heat with
synthetic methane (exact share determined by the model)

assume that FEC (except electricity, low-temperature heat and biomass consumption)
is supplied by methane which can deliver the required high-temperature heat

NB: CH4 burns differently in kiln, not straightforward, see IEEE paper
\citeS{akhtarCoalNatural2013}

process emissions are captured and, for net-zero emission scenarios,
they need to be compensated by negative emissions

Do post-combustion carbon capture on emissions and account for heat and
electricity for capture with aqueous amine solution. The classic amine process
based on monoethanolamine (MEA) Some suggest need CHP for this. DEA input
assumptions in 2030 are 0.72 MWh/t\co heat and 0.022 MWh/t\co electricity
per outputed tonne of \co assuming a capture rate of 90\%. Additional
electricity for \co compression to 150 bar and dehydration is 0.085
MWh/t\co.

Alternative: use oxyfuel so you get concentrated \co. No heat requirement,
electricity only used for O$_2$ air separation unit (ASU) of 0.08 MWh/t\co
and 0.17 MWh/t\co for post treatment (unlike post-combustion capture, output
is impure, so have to remove O$_2$ and N$_2$ by cryogenic distillation). Can
avoid ASU by using O$_2$ from electrolysis. Electricity use higher than
post-combustion, but don't have big heat requirement.

Beware biomass share in IDEES-Industry is higher for EU28 (2920 ktoe in 2015)
than in IDEES-EnergyBalance (906 ktoe). This is because IDEES is putting waste
(including non-renewable waste) together with biomass under biomass.

\citeS{kuramochiComparativeAssessment2012}: The major difference between centralized power plants and
industrial plants such as cement plants is that the former have large quantities
of low-grade heat that can be used for solvent regeneration, whereas the latter
generally do not [42].


Can use oxyfuel for cement to make CCS easier. Already being done, can retrofit
to older plant
\url{https://engineered.thyssenkrupp.com/en/oxyfuel-a-climate-neutral-cement-production-is-getting-closer/}

NEED oxyfuel, otherwise CCS requirements are too high.

TODO: How much O$_2$ from electrolysis? How much O$_2$ required by cement per
ton? For current combustion materials, around 0.17 tO$_2$/tCement => 29
MtO$_2$/a needed, but have 672 MtO$_2$/a from electrolysis; H2 for steel and
ammonia probably is enough to provide O$_2$ for oxyfuel cement (see private.org
- have plenty of O2 from electrolysis)

NB: for assessments of electricity for oxyfuel cement, deduct unneeded ASU (air
separation unit via cryogenic distillation) since don't need 200-300 kWh/tO$_2$
(consistent with 50-60 EUR/tO$_2$ price).

NB: cement production is used to using solid fuels. Natural gas needs some
aadjustment of firing mechanisms
\citeS{akhtarCoalNatural2013}.

Some people push electric kilns for heat, but considered less mature than gas
and certainly less than solid

CemZero project
\url{https://group.vattenfall.com/what-we-do/roadmap-to-fossil-freedom/industry-decarbonisation/cementa},
apparently poo-pooed here \url{https://www.cementa.se/sv/cemzero}.


\citeS{lechtenbohmerDecarbonisingEnergy2016} has electrification of cement with 0.9~MWh\el/tCLinker
(12\% efficiency improvement in thermal demand compared to 2010)



Ceramics

complete electrification because many already electrified processes:
- microwave drying and sintering of raw materials,
- electric kilns for primary production processes
- electric furnaces for the product finishing
- FEC 0.44 MWh/t ceramics
- process emissions: 0.03 t\co/t ceramic

\citeS{furszyferdelrioDecarbonizingCeramics2022a}

Glass

Electrify everything.
- electric melting tanks
- electric annealing
- electricity demand 2.07 MWh/t of glass \citeS{lechtenbohmerDecarbonisingEnergy2016}

\citeS{furszyferdelrioDecarbonizingGlass2022}


\citeS{lechtenbohmerDecarbonisingEnergy2016} also has big efficiency improvement with 0.85~MWh\el/tGlass.

\subsection{Non-ferrous Metals}
\label{sec:si:industry:nfm}

includes
- base metals (aluminium, copper, lead, zink)
- precious metals (gold, silver)
- technology metals (molybdenum, cobalt, silicon)

Aluminium

80\% recycling, for rest: methane for high-enthalpy heat (bauxite to alumina) followed by electrolysis (process emissions 1.5 t \co / t Al)

more than half of the FEC of this sector

Two alternative production routes today to manufacture aluminium in Europe today
- primary route: 40\%
- secondary route: 60\%
- exogenous: increase by 2050 to 80\% \citeS{Friedrichsen_2018} and \url{https://ec.europa.eu/clima/news/commission-calls-climate-neutral-europe-2050_en}

Primary route:
- two energy-intensive processes
- production of alumina from bauxite (aluminium ore)
- electrolysis to transfrom alumina to aluminium via the Hall-H\'{e}roult process
\begin{equation}
    \ce{2Al2O3 + 3C -> 4Al + 3\co}
\end{equation}
- primary route requires high-enthalpy heat to produce alumina - supplied by methane
- 1.5 t\co/t aluminium
- inert anodes might be commercially available in 2030 avoiding processe emissions \citeS{Friedrichsen_2018} but not considered

Secondary route:
- scrap aluminium is remelted
- energy demand is 10\% of primary route and process emissions are avoided
- assuming all subprocesses in this route electrified: 1.7 MWh elec / t aluminium

Other non-ferrous metals, electrification of entire manufacturing process is assumed
- 3.2 MWh / t lead equivalent

\subsection{Other Industry Subsectors}
\label{sec:si:industry:other}

energy demands and \co emissions for the agriculture, forestry and fishing
sector

pulp, paper, printing:

Already high share of biomass. Keep and add biomass for paper production, since
temperatures required are low.

food, beverages, tobacco \citeS{sovacoolDecarbonizingFood2021}

textiles and leather

machinery equipment


transport equipment

wood and wood products

otherwise

low- and mid-temperature process heat in these industries is assumed to be supplied by biomass
while the remaining processes are electrified

Comparison

\begin{itemize}
    \item synergies paper (still see benefit of transmission, but MUCH bigger electrolysis with industry/aviation/shipping) \citeS{brownSynergiesSector2018}
    \item JRC papers (Herib Blanco etc.) \citeS{blancoPotentialHydrogen2018,blancoPotentialPowertoMethane2018}
    \item FZJ steel paper
    \item PAC (uses solid biomass for non-energy requirements in chemicals industry, only 270 TWh in 2050, because of circular economy; phases out waste incineration) \citeS{caneurope/eebBuildingParis}
    \item LTS from commission \citeS{in-depth_2018}
    \item Material Economics reports \citeS{circular_economy,me2019}
\end{itemize}

\section{Heating Sector}
\label{sec:si:heat}

\subsection{Heat Demand}
\label{sec:si:heat:demand}

Building heating considering space and water heating in the residential and
services sectors is resolved for each region, both for individual buildings and
district heating systems, which include different supply options.

Annual heat demands per country are retrieved from JRC-IDEES \citeS{IDEES} for
the year 2011 and split into space and water heating. The space heating demand
is reduced by retrofitting measures that improve the buildings' thermal
envelopes. This reduction is exogenously fixed at 29\%
\citeS{zeyenMitigatingHeat2021,lombardiWeatherinducedVariability2022}. For space
heating, the annual demands are converted to daily values based on the
population-weighted Heating Degree Day (HDD) using the \textit{atlite} tool
\citeS{}, where space heat demand is proportional to the difference between the
daily average ambient temperature (read from ERA5 \citeS{}) and a threshold
temperature above which space heat demand is zero. A threshold temperature of
\SI{15}{\celsius} is assumed. The daily space heat demand is distributed to the
hours of the day following heat demand profiles from BDEW \citeS{}. These differ
for weekdays and weekends/holidays and between residential and services demand.
Hot water demand is assumed to be constant throughout the year.

For every country, heat demand is split between low and high population density
areas. These country-level totals are then distributed to each region in
proportion to their rural and urban populations respectively. Urban areas with
dense heat demand can be supplied with large-scale district heating systems. We
assume that by 2050, 60\% of urban heat demand is supplied by district heating
networks. Lump-sum losses of 15\% are assumed in district heating systems.

Cooling demand is supplied by electricity and included in the electricity
demand. Cooling demand is assumed to remain at current levels.

As shown in \cref{fig:demand-time}, total heat demand is similar to the total
electricity demand but features much more pronounced seasonal variations. The
total building heating demand adds up to \SI{3084}{\twh\per\year} of which 78\%
occurs in urban areas.

\citeS{kavvadiasDecarbonisingEU2019,fleiterBaselineScenario2017}

\subsection{Heat Supply}
\label{sec:si:heat:supply}

Different supply options are available depending on whether demand is met
centrally through district heating systems or decentrally through appliances in
individual buidlings. Supply options in individual buildings include gas and oil
boilers, air- and ground-sourced heat pumps, resistive heaters, and solar
thermal collectors. For large-scale district heating systems more options are
available: combined heat and power (CHP) plants consuming gas or biomass from
waste and residues with and without CCS, large-scale air-sourced
heat pumps, gas and oil boilers, resistive heaters and fuel cell CHPs.
Additionally, waste heat from the Fischer-Tropsch and Sabatier processes for the
production of synthetic hydrocarbons can supply district heating systems.
Ground-source heat pumps are only allowed in rural areas because of space constraints.
Thus, only air-source heat pumps are allowed in urban areas. This is a conservative
assumption, since there are many possible sources of low-temperature heat that
could be tapped in cities (e.g.~waste water, ground water, or natural bodies of water).
Costs, lifetimes and efficiencies for these technologies are listed in \cref{sec:si:costs}.

% CHPs

CHPs are based on back pressure plants operating with a fixed ratio of
electricity to heat output. The efficiencies of each are given on the back
pressure line, where the back pressure coefficient $c_b$ is the electricity
output divided by the heat output. For biomass CHP, we assume $c_b=0.46$,
whereas for gas CHP, we assume $c_b=1$.

% heat pumps

The coefficient of performance (COP) of air- and ground-sourced heat pumps
depends on the ambient or soil temperature respectively. Hence, the COP is a
time-varying parameter. Generally, the COP will be lower during winter when
temperatures are low. Because the ambient temperature is more volatile than the
soil temperature, the COP of ground-sourced heat pumps is less variable.
Moreover, the COP depends on the difference between the source and sink
temperatures
\begin{equation}
    \Delta T = T_{sink} - T_{source}.
\end{equation}
For the sink water temperature $T_{sink}$ we assume \SI{55}{\celsius} For the
time- and location-dependent source temperatures $T_{source}$, we rely on the
ERA5 reanalysis weather data \citeS{}. The temperature differences are converted
into COP time series using results from a regression analysis performed in
\citeS{staffellReviewDomestic2012}. For air-sourced heat pumps (ASHP), we use
the function
\begin{equation}
    COP(\Delta T) = 6.81 + 0.121 \Delta T + 0.000630 \Delta T^2;
\end{equation}
for ground-sourced heat pumps (GSHP), we use the function
\begin{equation}
    COP(\Delta T) = 8.77 + 0.150 \Delta T + 0.000734 \Delta T^2.
\end{equation}
The resulting time series are displayed in \cref{fig:cfs-ts}.
The spatial diversity of heat pump coefficients is shown in \cref{fig:cfs-maps}.

\citeS{dahlCostSensitivity2019, madedduCOReduction2020}

\subsection{Heat Storage}
\label{sec:si:heat:storage}

Thermal energy storage (TES) is available in large water pits associated with
district heating networks and small water tanks for individual, decentral
applications. A thermal energy density 46.8~kWh$_{\text{th}}$/m$^3$ is assumed,
corresponding to temperature difference of \SI{40}{\kelvin}. The decay of
thermal energy $1-\exp(-\sfrac{1}{24\tau})$ is assumed to have a time constant
of $\tau=180$ days for central TES and $\tau=3$ days for individual TES. The
charging and discharging efficiencies are 90\% due to pipe losses.

\section{Renewables Potentials}

\begin{figure}
    \centering
    % \makebox[\textwidth][c]{
        \begin{subfigure}[t]{0.49\textwidth}
            \centering
        \includegraphics[width=\textwidth]{windspeeds.png}
    \end{subfigure}
    \begin{subfigure}[t]{0.49\textwidth}
        \centering
        \includegraphics[width=\textwidth]{irradiation.png}
    \end{subfigure}
    \begin{subfigure}[t]{0.49\textwidth}
        \centering
        \includegraphics[width=\textwidth]{temperatures.png}
    \end{subfigure}
    \begin{subfigure}[t]{0.49\textwidth}
        \centering
        \includegraphics[width=\textwidth]{runoff.png}
    \end{subfigure}
    % }
    \caption{weather data}
    \label{fig:weather-data}
\end{figure}


\begin{figure}
    \centering
    % \makebox[\textwidth][c]{
    \begin{subfigure}[t]{0.49\textwidth}
        \centering
        \caption{solar}
        \includegraphics[width=\textwidth]{solar-energy-density.pdf}
    \end{subfigure}
    \begin{subfigure}[t]{0.49\textwidth}
        \centering
        \caption{wind}
        \includegraphics[width=\textwidth]{wind-energy-density.pdf}
    \end{subfigure}
    % }
    \caption{Available energy density for wind and utility-scale solar PV energy generation.}
    \label{fig:energy-density}
\end{figure}

\begin{figure}
    \centering
    % \makebox[\textwidth][c]{
        \begin{subfigure}[t]{0.49\textwidth}
            \centering
        \includegraphics[width=\textwidth]{cf-raw-ts-onshore wind.pdf}
    \end{subfigure}
    \begin{subfigure}[t]{0.49\textwidth}
        \centering
        \includegraphics[width=\textwidth]{cf-raw-ts-offshore wind.pdf}
    \end{subfigure}
    \begin{subfigure}[t]{0.49\textwidth}
        \centering
        \includegraphics[width=\textwidth]{cf-raw-ts-solar PV.pdf}
    \end{subfigure}
    \begin{subfigure}[t]{0.49\textwidth}
        \centering
        \includegraphics[width=\textwidth]{cf-raw-ts-run of river.pdf}
    \end{subfigure}
    \begin{subfigure}[t]{0.49\textwidth}
        \centering
        \includegraphics[width=\textwidth]{cop-ts-air-sourced heat pump.pdf}
    \end{subfigure}
    \begin{subfigure}[t]{0.49\textwidth}
        \centering
        \includegraphics[width=\textwidth]{cop-ts-ground-sourced heat pump.pdf}
    \end{subfigure}
    % }
    \caption{Capacity factor time series of renewable energy sources.}
    \label{fig:cfs-ts}
\end{figure}

\begin{figure}
    \centering
    % \makebox[\textwidth][c]{
    \begin{subfigure}[t]{0.49\textwidth}
        \centering
        \includegraphics[width=\textwidth]{cf-onwind.pdf}
    \end{subfigure}
    \begin{subfigure}[t]{0.49\textwidth}
        \centering
        \includegraphics[width=\textwidth]{cf-offwind-dc.pdf}
    \end{subfigure}
    \begin{subfigure}[t]{0.49\textwidth}
        \centering
        \includegraphics[width=\textwidth]{cf-solar.pdf}
    \end{subfigure}
    \begin{subfigure}[t]{0.49\textwidth}
        \centering
        \includegraphics[width=\textwidth]{cf-solar thermal.pdf}
    \end{subfigure}
    \begin{subfigure}[t]{0.49\textwidth}
        \centering
        \includegraphics[width=\textwidth]{cf-air-sourced heat pump.pdf}
    \end{subfigure}
    \begin{subfigure}[t]{0.49\textwidth}
        \centering
        \includegraphics[width=\textwidth]{cf-ground-sourced heat pump.pdf}
    \end{subfigure}
    % }
    \caption{Regional distribution of capacity factors of renewable energy sources.}
    \label{fig:cfs-maps}
\end{figure}


\newgeometry{margin=2cm}
\begin{landscape}

\begin{figure}
    \centering
    % \makebox[\textwidth][c]{
    \begin{subfigure}[t]{0.5\textwidth}
            \centering
        \caption{solar land eligibility}
        \includegraphics[width=\textwidth]{eligibility-solar-250-20.pdf}
    \end{subfigure}
    \begin{subfigure}[t]{0.5\textwidth}
        \centering
        \caption{onshore wind land eligibility}
        \includegraphics[width=\textwidth]{eligibility-onwind-250-20.pdf}
    \end{subfigure}
    \begin{subfigure}[t]{0.5\textwidth}
        \centering
        \caption{offshore wind land eligibility}
        \includegraphics[width=\textwidth]{eligibility-offwind-250-20.pdf}
    \end{subfigure}
    %}
    \caption{Land eligibility.}
    \label{fig:eligibility}
\end{figure}


\end{landscape}
\restoregeometry


\begin{table}
    \caption{Land types considered suitable for every technology from Corine Land Cover database}
    \small
    \begin{tabularx}{\textwidth}{lX}
        \toprule
        Solar PV & artificial surfaces (1-11), agriculture land except for those
        areas already occupied by agriculture with significant natural
        vegetation and agro-forestry areas (12-20), natural grasslands (26), bare rocks (31),
        sparsely vegetated areas (32) \\ \midrule
        Onshore wind & agriculture areas (12-22), forests (23-25), scrubs and herbaceous vegetation associations (26-29), bare rocks (31), sparsely vegetated areas (32) \\ \midrule
        Offshore wind & sea and ocean (44) \\ \bottomrule
    \end{tabularx}
    \label{tab:eligibility}
\end{table}

Eligibile areas for developing renewable infrastructure are calculated per technology and substation's Voronoi cell using the \textit{atlite} tool \citeS{}.

The land available for wind and utility-scale solar PV capacities in a particular region is constrained by eligible codes of the CORINE land use database \citeS{} and is further restricted by distance criteria
and the natural protection areas specified in the Natura 2000 dataset \citeS{}. These criteria are summarised in XXX.

Moreover, offshore wind farms may not be built at sea depths exceeding \SI{50}{\metre}, as indicated by the GEBCO bathymetry dataset \citeS{}.
This currently disreagards the possibility of floating wind turbines \citeS{lerchSensitivityAnalysis2018,lauraLifecycleCost2014,myhrLevelisedCost2014,kauscheFloatingOffshore2018,castro-santosEconomicFeasibility2016}.

To express the potential in terms of installable capacities, the available areas are multiplied with allowed deployment densities,
which we consider to be a fraction of the technology's technical deployment density to preempt public acceptance issues.

\subsection{Onshore Wind Potentials}

X MW/sqkm conversion factor

20\% of the available land define potential

\citeS{mckennaHighresolutionLargescale2022,Ryberg2018}

\subsection{Offshore Wind Potentials}

Offshore:

- near-shore (distance less than 30km) AC connection, far-shore, DC connected
- costs include AC-DC conversion

wake effects


\subsection{Solar Potentials}

utility PV

X MW/sqkm

9\% of the available land define potential

Installable potentials for rooftop PV are included with an assumption of 1 kWp
per person (0.1 kW/m2 and 10 m2/person)

\citeS{bodisHighresolutionGeospatial2019}

\section{Renewables Time Series}

The location-dependent renewables availability time series are generated based
on two historical weather datasets. We retrieve wind sepeeds at
\SI{100}{\metre}, surface roughness, soil and air temperatures, and surface
run-off from rainfall or melting snow from the global ERA5 reanalysis dataset
provided by the ECMWF \citeS{}. It provides hourly values for each of these
parameters since 1950 on a $0.25^{\circ} \times 0.25^{\circ}$ grid. In Germany,
such a weather cell expands approximately \SI{20}{\kilo\metre} from east to west
and \SI{31}{km} from north to south. For the direct and diffuse solar
irradiance, we use the satellite-aided SARAH-2 dataset, which assesses cloud
cover in more detail than the ERA5 dataset. It features values from 1983 to 2015
at an even higher resolution with a $0.05^{\circ} \times 0.05^{\circ}$ grid and
30-minute intervals \citeS{}. In general, the reference weather year can be
freely chosen for the optimisation, but in this contribution all analyses are
based on the year 2013, which is regarded as characteristic year for both wind
and solar resources (e.g. \citeS{}).

Models for wind turbines, solar panels, heat pumps and the inflow into hydro
basins convert the weather data to hourly time series for capacity factors and
performance coefficients. Using power curves of typical wind turbines\todo{which
wind turbine types offshore onshore}, wind speeds scaled to the according hub
height are mapped to power outputs. The solar photovoltaic panels' output is
calculated based on the incidence angle of solar irradiation, the panel's tilt
angle, and conversion efficiency. The creation of heat pump time series follows
regression analyses that map soil or air temperatures to the coefficient of
performance (COP) \citeS{nouvelEuropeanMapping2015, staffellReviewDomestic2012}.
The open-source library \textit{atlite} \citeS{} provides functionality to
perform all these calculations efficiently. Finally, the obtained time series
are aggregated to each region heuristically in proportion to each grid cell's
mean capacity factor. This assumes a capacity layout proportional to mean
capacity factors.

Hydroelectric inflow time series are derived from run-off data.

Solar thermal

In combination with the capacity potentials derived from the assumed land use restrictions,
the time-averaged capacity factors are used to display in \cref{fig:energy-density} the energy that could be produced from wind and solar energy in the different regions of Europe.

\section{Hydrogen}
\label{sec:si:h2}

\subsection{Hydrogen Demand}
\label{sec:si:h2:demand}

Hydrogen is consumed in the industry sector to produced ammonia and direct
reduced iron (DRI) (see \cref{sec:si:industry:steel}). Hydrogen is also consumed
to produce synthetic methane and liquid hydrocarbons (see
\cref{sec:si:methane:supply} and \cref{sec:si:oil:supply}) which have multiple
uses in industry and other sectors. For transport applications, the consumption
of hydrogen is exogenously fixed. It is used in heavy-duty land transport (see
\cref{sec:si:transport:land}) and as liquified hydrogen in the navigation sector
(see \cref{sec:si:transport:shipping}). Furthermore, stationary fuel cells may
re-electrify hydrogen (with waste heat as a byproduct) to balance renewable
fluctuations.

\subsection{Hydrogen Supply}
\label{sec:si:h2:supply}

Today, most hydrogen is produced from natural gas by steam methane reforming
(SMR)
\begin{equation}
    \ce{ CH4 + H2O -> CO + 3H2 }
\end{equation}
combined with a water-gas shift reaction
\begin{equation}
    \ce{CO + H2O -> CO2 + H2}.
\end{equation}
We consider this route of production with and without carbon capture (CC),
assuming a capture rate of 90\%. These routes are also referred to as blue and
grey hydrogen. The methane input can be of fossil or synthetic origin.

Furthermore, we consider water electrolysis (green hydrogen) which uses electric
energy to split water into hydrogen and oxygen
\begin{equation}
    \ce{2H2O -> 2 H2 + O2}.
\end{equation}
For the electrolysis, we assume alkaline electrolysers since they have lower
cost \citeS{} and higher cumulative installed capacity \citeS{} than polymer
electrolyte membrane (PEM) electrolysers. Waste heat from electrolysis is not
leveraged in the model.

The split between these three different technology
options and their installed capacities are a result of the optimisation
depending on the techno-economic assumptions listed in \cref{sec:si:costs}.

\subsection{Hydrogen Transport}
\label{sec:si:h2:transport}

Hydrogen can be transported in pipelines. These can be retrofitted natural gas
pipelines or completely new pipelines. The cost of retrofitting a gas pipeline
is about half that of building a new hydrogen pipeline. These costs include the
cost for new compressors but neglect the energy demand for compression.

The endogenous retrofitting of gas pipelines to hydrogen pipelines is
implemented in a way, such that for every unit of gas pipeline decommissioned,
60\% of its nominal capacity are available for hydrogen transport on the
respective route, following assumptions from the European Hydrogen Backbone
report \citeS{EuropeanHydrogen}\todo{check this assumption in EHB}. When the gas network is not resolved, this
value denotes the potential for repurposed hydrogen pipelines.

New pipelines can be built additionally on all routes where there currently is a
gas or electricity network connection. These new pipelines will be built where
no sufficient retrofitting options are available. The capacities of new and
repurposed pipelines are a result of the optimisation.

\subsection{Hydrogen Storage}
\label{sec:si:h2:storage}

\begin{SCfigure}
    \centering
    \includegraphics[width=0.7\textwidth]{caverns.pdf}
    \caption{Caverns}
    \label{fig:caverns}
\end{SCfigure}

\begin{figure}
    \centering
    \makebox[\textwidth][c]{
    \begin{subfigure}[t]{0.45\textwidth}
        \centering
        \includegraphics[width=\textwidth]{cavern-potentials-nearshore.pdf}
    \end{subfigure}
    \begin{subfigure}[t]{0.45\textwidth}
        \centering
        \includegraphics[width=\textwidth]{cavern-potentials-onshore.pdf}
    \end{subfigure}
    \begin{subfigure}[t]{0.45\textwidth}
        \centering
        \includegraphics[width=\textwidth]{cavern-potentials-offshore.pdf}
    \end{subfigure}
    }
    \caption{Cavern storage potentials}
    \label{fig:clustered-caverns}
\end{figure}

Hydrogen can be stored in overground steel tanks or underground salt caverns.
The annuitised cost for cavern storage is around 30 times lower than for storage
in steel tanks including compression. For underground storage potentials for
hydrogen in European salt caverns we take data from Caglayan et
al.~\citeS{caglayanTechnicalPotential2019} (\cref{fig:caverns} and map it to
each of the 181 model regions (\cref{fig:clustered-caverns}). We include only
those caverns that are located on land and within 50 km of the shore
(nearshore). We impose this restriction to circumvent environmental problems
associated with brine water disposal \citeS{caglayanTechnicalPotential2019}. The
storage potential is abundant and the constraining factor is more where they
exist and less how large the energy storage potentials are.

\section{Methane}
\label{sec:si:methane}

\subsection{Methane Demand}
\label{sec:si:methane:demand}

Methane is used in individual and large-scale gas boilers, in CHP plants with
and without carbon capture, in OCGT and CCGT power plants, and in some industry
subsectors for the provision of high temperature heat (see
\cref{sec:si:industry}) Methane is not used in the transport sector because of
engine slippage.

\subsection{Methane Supply}
\label{sec:si:methane:supply}

Besides methane from fossil origins, the model also considers biogenic and
synthetic sources. If gas infrastructure is regionally resolved (see
\cref{sec:si:methane:transport}), fossil gas can enter the system only at
existing and planned LNG terminals, pipeline entry-points, and intra-European
gas extraction sites (see \cref{fig:gas-raw}), which are retrieved from the
SciGRID Gas IGGIELGN dataset \citeS{plutaSciGRIDGas2022} and the GEM Wiki
\citeS{}. Biogas can be upgraded to methane (see \cref{sec:si:bio:potentials}).
Synthetic methane can be produced by processing hydrogen and captures \co in the
Sabatier reaction
\begin{equation}
    \ce{CO2 + 4H2 -> CH4 + 2H2O}.
\end{equation}
Direct power-to-methane conversion with efficient heat integration developed in
the HELMETH project is also an option \citeS{gruberPowertoGasThermal2018}. The
share of synthetic, biogenic and fossil methane is an optimisation result
depending on the techno-economic assumptions listed in \cref{sec:si:costs}.

\subsection{Methane Transport}
\label{sec:si:methane:transport}

\begin{figure}
    \includegraphics[width=1\textwidth,center]{gas_network.pdf}
    \label{fig:gas-raw}
    \caption{Gas network}
\end{figure}

The existing European gas transmission network is represented based on the
SciGRID Gas IGGIELGN dataset \citeS{plutaSciGRIDGas2022}, as shown in
\cref{fig:gas-raw}. This dataset is based on compiled and merged data from the
ENTSOG map \citeS{} and other publicly available data sources. It includes data
on the capacity, diameter, pressure, length, and directionality of pipelines.
Missing capacity data is conservatively inferred from the pipe diameter
following conversion factors derived from \cite{EuropeanHydrogen}. The gas
network is clustered to the model's 181 regions (see
\cref{fig:clustered-networks}). Gas pipelines can be endogenously expanded or
repurposed for hydrogen transport (see \cref{sec:si:h2:transport}). Gas flows
are represented by a lossless transport model.

The results presented in the main body of the article regard the gas
transmission network only to determine the retofitting potentials for hydrogen
pipelines. These assume methane to be transported without cost or capacity
constraints, since future demand is predicted to be low compared to available
transport capacities even if a certain share is repurposed for hydrogen
transport such that no bottlenecks are expected. Selected runs with gas network
infrastructure included are presented in \cref{sec:si:detailed}.

\section{Oil-based Products}
\label{sec:si:oil}

\subsection{Oil-based Product Demand}
\label{sec:si:demand}

Naphtha is used as a feedstock in the chemicals industry (see
\cref{sec:si:industry:chemicals}). Furthermore, kerosene is used as transport
fuel in the aviation sector (see \cref{sec:si:transport:aviation}).
Non-electrified agriculture machinery also consumes gasoline.

\subsection{Oil-based Product Supply}
\label{sec:si:oil:supply}

In addition to fossil origins, oil-based products can be synthetically produced
by processing hydrogen and captured \co in Fischer-Tropsch plants
\begin{equation}
    \ce{$n$CO + ($2n$ + 1)H2 -> C_$n$H_{2n+2} + $n$H2O}.
\end{equation}
with costs as included in \cref{sec:si:costs}. The waste heat from the
Fischer-Tropsch process is supplied to district heating networks.

\subsection{Oil-based Product Transport}
\label{sec:si:oil:transport}

Liquid hydrocarbons are assumed to be transported freely among the model region
since future demand is predicted to be low, transport costs for liquids are low
and no bottlenecks are expected.

\section{Biomass}
\label{sec:si:bio}

Biomass resources are available for different potential assessments (low,
medium, high) for many different feedstocks and NUTS2 resolution based on the
JRC’s ENSPRESO database. They can be used in electricity generation with and
without CCS, as well as to provide low- to medium-temperature process heat in
industry.

\subsection{Biomass Potentials}
\label{sec:si:bio:potentials}

\begin{figure}
    \centering
    \makebox[\textwidth][c]{
    \begin{subfigure}[t]{0.45\textwidth}
        \centering
        % \caption{electricity demand}
        \includegraphics[width=\textwidth]{biomass-solid biomass.pdf}
    \end{subfigure}
    \begin{subfigure}[t]{0.45\textwidth}
        \centering
        % \caption{hydrogen demand}
        \includegraphics[width=\textwidth]{biomass-biogas.pdf}
    \end{subfigure}
    \begin{subfigure}[t]{0.45\textwidth}
        \centering
        % \caption{methane demand}
        \includegraphics[width=\textwidth]{biomass-not included.pdf}
    \end{subfigure}
    }
    \caption{Biomass potentials.}
    \label{fig:biomass-potentials}
\end{figure}

Only wastes and residues from the JRC ENSPRESO biomass dataset.

\cref{fig:biomass-potentials}

nodal where biomass potential is regionally disaggregated

Use JRC ENSPRESO database to spatially disaggregate biomass potentials to
PyPSA-Eur regions based on overlaps with NUTS2 regions from ENSPRESO
(proportional to area)

Biomass supply potentials for each European country are taken from the database
of the Joint Research Centre (JRC) of the European Commission
\citeS{jrcbiomass2015}.

Only residues from agriculture and forestry as well as
biodegradable municipal waste are considered as energy feedstocks. Fuel crops
are avoided because they compete with scarce land for food production, while
primary wood as well as wood chips and pellets are avoided because of concerns
about sustainability \citeS{bentsenCarbonDebt2017}.

The JRC provides potentials in low,
medium and high availability scenarios, which depend on supply and competition
with other uses of each feedstock. The medium availability scenario for
2030 is used, assuming no biomass import from outside Europe.

Manure and sludge waste are available to the model as biogas (that is upgraded to
biomethane), while other wastes and residues are classified as solid biomass and
available for combustion in combined-heat-and-power plants (CHP) and for medium
temperature heat (lower than \SI{500}{\celsius}) applications in industry.

The technical
characteristics for the solid biomass CHP are taken from the Danish Energy
Agency Technology Database \citeS{dea2016} assumptions for a medium-sized back
pressure CHP with wood pellet feedstock; this has very similar costs and
efficiencies to CHPs with feedstocks of straw and wood chips.

A summary of the feedstocks and use in the model is shown in \cref{tab:biomass}; the respective regional distribution of potentials in \cref{fig:biomass-potentials}.

In 2015 the EU28 energy
usage was 180~TWh of biogas, 1063~TWh of solid biofuels, 109~TWh renewable
municipal waste and 159~TWh of liquid biofuels. Our model contains roughly a
doubling of the biogas production from 2015 and similar amounts of solid
biofuels, but a shift from energy crops and primary wood to residues and wastes.

Since most liquid biofuels come from energy crops today, these do not appear in
PyPSA-Eur-Sec.

Carbon capture and sequestration of bioenergy \co emissions, results in net negative emissions.
Carbon capture and use of bioenergy \co emissions results in net neutral emissions.

\begin{table}
    \centering
    \small
    \begin{tabularx}{\textwidth}{lXr}
        \toprule
        Application & Source & Potential [\si{\twh\per\year}] \\
        \midrule
        solid biomass & primary agricultural residues; forest energy residue; secondary forestry residues: woodchips, sawdust; forestry residues from landscape care; biodegradable municipal waste & 1186 \\
        biogas & wet and dry manure; biodegradable sludge & 346\\
        not used & energy crops: sugar beet bioethanol, rape seed and other oil crops, starchy crops, grassy, willow, poplar; roundwood fuelwood; roundwood chips and pellets & 1661 \\
        \bottomrule
    \end{tabularx}
    \caption{Use of biomass potentials according to classifications from the JRC \cite{jrcbiomass2015} in the medium availability scenario for 2030.}
    \label{tab:biomass}
\end{table}


\subsection{Biomass Demand}
\label{sec:si:bio:demand}

Solid biomass provides process heat up to \SI{500}{\celsius} in industry and can
also feed CHP plants in district heating networks. As noted in
\cref{sec:si:industry}, solid biomass is used as heat supply in the paper and
pulp and food, beverages and tobacco industries, where required temperatures are
lower \citeS{naeglerQuantificationEuropean2015,rehfeldtBottomupEstimation2018}.

\subsection{Biomass Transport}
\label{sec:si:bio:transport}

Solid biomass is assumed to be transported freely among the modelled regions.
Biogas can be upgraded and then transported via the methane network. If the
methane network is neglected, also biogas can be moved without cost or
constraints.

\section{Carbon dioxide capture, usage and sequestration (CCU/S)}
\label{sec:si:carbon-management}

Carbon management becomes important in net-zero scenarios. PyPSA-Eur-Sec
includes carbon capture from electricity generators and industrial facilities,
carbon dioxide storage and transport, the usage of carbon dioxide in synthetic
hydrocarbons, as well as the ultimate sequestration of carbon dioxide
underground.


direct air capture DAC

synthetic methane and liquid hydrocarbons

transport and sequestration is \SI{20}{\sieuro\per\tco}

yearly sequestration limit to 200 Mt\co / a

carbon cycle

carbon is tracked through system

Carbon capture is needed in the model both to capture and sequester process
emissions with a fossil origin, such as those from calcination of fossil
limestone in the cement industry, as well as to provide carbon for the
production of hydrocarbons for dense transport fuels and as a chemical
feedstock, for example for the plastics industry.

Carbon dioxide can be captured from industry process emissions, emissions
related to industry process heat (methane and biomass), combined heat and power plants, and directly
from the air wit direct air capture (DAC).

DAC includes
- adsorption phase with inputs electricity and heat to assist adsorption process and regenerate adsorbent
- compression of \co prior to storage which consumes electricity and rejects heat

process emissions captured at 90\% capture rate at cost as in cement industry

SMR, CHP, biomass and methane demand in industry the model can decide w/wo carbon capture

these capacities are co-optimised

Carbon dioxide can be used as an input for methanation and Fischer-Tropsch
fuels/naphtha, or it can be sequestered underground.

Can also be sequestered:
- 20 \euro/t\co for transport and sequestration
- 2-14 USD/t\co for pipe transport IEA
- 10 USD/t\co for underground sequestration
- limit is 200 Mt (rather conservative but enough to capture process emissions)

unavoidable process emissions

\co: single node for Europe, but a transport and storage cost is added for sequestered \co. Optionally: nodal, with \co transport via pipelines.

Why net-zero target for \co? Since don't include LULUCF or non-\co (waste management and agriculture), which balance each other in EU analysis

CCU/S needed for synthetic fuels AND to deal with process emissions (from e.g.
cement)

need for feedstocks in chemicals industry and dense hydrocarbon fuels for aviation

we have capture on

- industrial process emissions, using same assumptions as the cement kiln
example described above from DEA  (90\% capture rate, 0.72 MWh/t\co heat and
0.1 MWh/t\co electricity, including compression and dehydration) - heat is
taken from urban heat buses, electricity from public grid - steam methane
reforming - biomass CHPs, using DEA assumptions for post-combustion capture on a
small CHP in 2030 (90\% capture rate, 0.72 MWh/t\co heat and 0.11 MWh/t\co
electricity, including compression and dehydration) - heat and electricity is
taken from CHP output - direct air capture (2 MWh/t\co heat, 0.47 MWh/t\co
electricity including compression and dehydration) - heat is taken from urban
heat buses (even though T is below \SI{100}{\celsius}), electricity from public grid; waste
heat from compression is used for amine washing

TODO: as of 210202: electricity and heat demand of process emission CC ignored
(only capital costs are used); also for fuel-based emissions, simple 10\% of
fuel is taken

127 Mt\co/a fossil-origin process emissions in industry (limestone for cement,
soda ash and graphite electrodes); need sequestration for this otherwise not
net-zero.

For FT-fuels demand need 0.27*1391 = 376 Mt\co/a. This comes from BECCU and
industry CCU with synfuels. Could also come from waste + CCU.


Comparison of heat/electricity requirements for capture:

\citeS{kuramochiComparativeAssessment2012} 5.2: Can integrate low-grade steam from power plants with
\co capture from cement

``In the case of post-combustion \co capture, a CHP plant will likely be built
together with the \co capture unit because this is the only way to generate
steam efficiently for \co capture solvent regeneration.''

Beware: need to take account of heat need for regenerating \co capture solvent.
In power plants, can use waste heat, but in industrial plants there is less
low-grade heat (need ~110 C). \citeS{kuramochiComparativeAssessment2012}

Need around 3.9 GJ/t\co heat for regeneration of aqueous MEA ~ 1 MWh/t\co
\citeS{zhangParametricStudy2016}. Solid biomass has 0.3 t\co/MWh\th => need 0.3 MWh heat for
each MWh\th of solid biomass => heat output is hugely reduced!!! Unless we have
oxyfuel biomass combustion...

Breyer \citeS{breyerCarbonDioxide2020}
has 1.2 MWh/t\co heat at \SI{100}{\celsius} and 0.2 MWh\el/t\co electricity for DAC.
Dittmeyer has energy requirements twice as high... DEA is closer to Dittmeyer

\citeS{fasihiTechnoeconomicAssessment2019, martin-robertsCarbonCapture2021}


\section{Mathematical Model Formulation}
\label{sec:si:math}

\section{Cost of Grid Reinforcement and Onshore Wind Potential Restrictions}
\label{sec:si:sensitivity-lv-onw}

\begin{figure}
    \centering
    \begin{subfigure}[t]{\textwidth}
        \centering
        \caption{Sensitivity of electricity transmission grid expansion limits}
        \includegraphics[width=\textwidth]{lv-sensitivity.pdf}
    \end{subfigure}
    \begin{subfigure}[t]{\textwidth}
        \centering
        \caption{Sensitivity of onshore wind expansion limits}
        \includegraphics[width=\textwidth]{onw-sensitivity.pdf}
    \end{subfigure}
    \caption{Sensitivity of electricity transmission grid expansion limits and onshore wind restrictions.}
    \label{fig:lv-onw-restriction}
\end{figure}

\begin{SCfigure}
    \centering
    \includegraphics[width=0.8\textwidth]{sensitivity-h2.pdf}
    \caption{Sensitivity of hydrogen network infrastructure.}
    \label{fig:h2-restriction}
\end{SCfigure}

\subsection{Cost of Electricity Grid Reinforcement Restrictions}
\label{sec:si:lv}

The volume limit is given in fractions of today's grid volume: a
line volume limit of 1.0 means no new capacity is allowed beyond today's grid
(since the model cannot remove existing lines); a limit of 1.25 means the total
grid capacity can grow by 25\% (25\% is similar to the planned extra capacity in
the European network operators' Ten Year Network Development Plan (TYNDP)\cite{TYNDP2016})

Figure shows the composition of total yearly system costs
(including all investment and operational costs) as we vary the allowed
grid expansion, from no expansion (only today's grid) to
a doubling of today's grid capacities (the model optimises where new
capacity is placed).

The total costs are dominated by investments in generation from wind
and solar, and conversion from power to heat (primarily heat pumps)
and to hydrogen and liquid hydrocarbons (for transport fuels and as a
feedstock for the chemicals industry). Vehicle costs are not included
here. The costs of today's system are \euro700~billion per year,
computed with the same assumptions, which is only slightly less than
these net-zero emission systems.

As the grid is expanded, total costs decrease
slightly, despite the increasing costs of the grid. The grid enables
renewable resources with better capacity factors to be integrated from
further away, resulting in lower capacity needs for solar and
wind. The grid also allows renewable variations to be smoothed in space,
resulting in lower hydrogen demand for balancing power and heat.  The
total cost benefit of a doubling of grid capacity is around
\euro44~billion per year, but over half of the benefit (\euro26~billion per year) is available already at a 25\% expansion (similar to the level seen in the TYNDP \cite{TYNDP2016}).

Figure shows the spatial distribution of the investments
for the case of a 25\% expansion. While solar capacity is spread
relatively evenly around the continent, both onshore and offshore wind
are concentrated around the North Sea and British Isles. New
transmission capacity is concentrated in HVDC lines that help the
integration of wind in these regions, and the transport of wind energy
to inland locations.

Electrolyzer capacities for power-to-hydrogen see a massive scale-up,
from several dozen MW today to 1048~GW in this scenario (note that
several 100~MW electrolyzer facilities are already in planning for the
2020s). Their locations correlate strongly with wind capacities,
particularly offshore wind. A new network
transports hydrogen from these sites of production to the rest of
Europe where hydrogen is consumed by industry (for ammonia, organic
chemicals and steel production), for heavy-duty transport and in fuel
cells for power and heat backup. Of the huge hydrogen production
(2949~TWh/a), most of it (1722~TWh/a) goes to Fischer-Tropsch fuels
for organic chemicals and transport fuels, 615~TWh/a to fuel cell CHPs
in district heating networks, and the rest to industry and transport.
The hydrogen network plays a dominant role transporting energy around
Europe when grid expansion is restricted: more energy is moved further
in the hydrogen network (209~TWhkm/h) than either the HVAC
(99~TWhkm/h) or HVDC (3.3~TWhkm/h) networks.

Methane production is limited to biogas (352~TWh/a) and some fossil
gas (388~TWh/a), the latter of whose emissions are offset by bioenergy
with carbon capture and direct air capture with sequestration.
Methane is used for process heat in some industry applications and as
a heating backup for power-to-heat units. The model has the option of
efficient power-to-methane conversion (methanation integrated with a
solid oxide electrolyzer) but does not choose to build it because of
its high capital costs compared to the direct hydrogen route.

50\% more power grid volume (expansion of 162 TWkm) and 260 TWkm of hydrogen grid

25\% more power grid volume (expansion of 81 TWkm) and 282 TWkm of hydrogen grid

no more power grid volume and 308 TWkm of hydrogen grid

direct system costs bit higher than today's system

systems without grid expansion are feasible, but more costly

as grid is expanded, costs reduce from solar, PtX and H2 network, more offshore wind

total cost benefit of extra grid: 47 billion euro per year

over half of the benefit is available at 25\% grid expansion (like TYNDP)







\subsection{Cost of Onshore Wind Potential Restrictions}
\label{sec:si:onw}

If we take the case with no grid expansion and restrict the installable
potentials of onshore down to zero, costs rise by an additional \euro~42~billion
per year. The model substitutes onshore wind for
higher investment in offshore and solar generators. Because offshore capacities
are concentrated near coastlines, and grid capacity is restricted, total
spending on hydrogen electrolyzers and networks also increases to absorb the
offshore generation.

Just as in the case of restricted line volumes, the rise in costs is non-linear:
if we restrict to 25\% of the onshore potential (around 100~GW for Germany),
costs rise by only \euro~14~billion per year. 25\% of the onshore wind potential
represents a possible social compromise between total system cost, and public
acceptance concerns about onshore wind development.

Without onshore wind, solar rooftop and offshore potentials are maxxed out
- If all sectors included and Europe self-sufficient, effect of installable potentials is critical

Without onshore wind, hydrogen network looks much different:
- with: British Isles and North Sea dominate hydrogen production
- without: Southern Europe becomes much larger exporter of hydrogen

Technical potentials for onshore wind respect land usage, but do not represent
sociallly-acceptable potentials
- technical potential of 480 GW in Germany is unlikely to be built
- costs rise by 122 billion per year as we eliminate onshore wind
- rise is only 45 billion per year if we allow a quarter of technical potential (120 GW in Germany)

\section{Detailed Results of Least-Cost Solution with Gas Network}
\label{sec:si:detailed}

competition of H2 and gas flow? can gas network be completely removed?

no loss modelling in gas networks

amount of re-electrified hydrogen/gas

How many pipelines are used in both directions?

onshore wind sensitivities



\begin{figure}
    \centering
    % \makebox[\textwidth][c]{
    \begin{subfigure}[t]{\textwidth}
        \centering
        \caption{electricity}
        \includegraphics[width=\textwidth]{20211218-181-lv/elec_s_181_lv1.0__Co2L0-3H-T-H-B-I-A-solar+p3-linemaxext10_2030/ts-balance-total electricity-D-2013.pdf}
    \end{subfigure}
    \begin{subfigure}[t]{\textwidth}
        \centering
        \caption{heat}
        \includegraphics[width=\textwidth]{20211218-181-lv/elec_s_181_lv1.0__Co2L0-3H-T-H-B-I-A-solar+p3-linemaxext10_2030/ts-balance-total heat-D-2013.pdf}
    \end{subfigure}
    \begin{subfigure}[t]{\textwidth}
        \centering
        \caption{hydrogen}
        \includegraphics[width=\textwidth]{20211218-181-lv/elec_s_181_lv1.0__Co2L0-3H-T-H-B-I-A-solar+p3-linemaxext10_2030/ts-balance-H2-D-2013.pdf}
    \end{subfigure}
    %}
    \caption{Daily time series}
    \label{fig:output-ts-1}
\end{figure}

\begin{figure}
    \centering
    % \makebox[\textwidth][c]{
    \begin{subfigure}[t]{\textwidth}
        \centering
        \caption{methane}
        \includegraphics[width=\textwidth]{20211218-181-lv/elec_s_181_lv1.0__Co2L0-3H-T-H-B-I-A-solar+p3-linemaxext10_2030/ts-balance-gas-D-2013.pdf}
    \end{subfigure}
    \begin{subfigure}[t]{\textwidth}
        \centering
        \caption{oil}
        \includegraphics[width=\textwidth]{20211218-181-lv/elec_s_181_lv1.0__Co2L0-3H-T-H-B-I-A-solar+p3-linemaxext10_2030/ts-balance-oil-D-2013.pdf}
    \end{subfigure}
    \begin{subfigure}[t]{\textwidth}
        \centering
        \caption{stored \co}
        \includegraphics[width=\textwidth]{20211218-181-lv/elec_s_181_lv1.0__Co2L0-3H-T-H-B-I-A-solar+p3-linemaxext10_2030/ts-balance-co2 stored-D-2013.pdf}
    \end{subfigure}
    %}
    \caption{Daily time series}
    \label{fig:output-ts-2}
\end{figure}


\begin{figure}
    \centering
    % \makebox[\textwidth][c]{
    \begin{subfigure}[t]{\textwidth}
        \centering
        \caption{electricity}
        \includegraphics[width=\textwidth]{20211218-181-lv/elec_s_181_lv1.0__Co2L0-3H-T-H-B-I-A-solar+p3-linemaxext10_2030/ts-balance-total electricity--2013-02.pdf}
    \end{subfigure}
    \begin{subfigure}[t]{\textwidth}
        \centering
        \caption{heat}
        \includegraphics[width=\textwidth]{20211218-181-lv/elec_s_181_lv1.0__Co2L0-3H-T-H-B-I-A-solar+p3-linemaxext10_2030/ts-balance-total heat--2013-02.pdf}
    \end{subfigure}
    \begin{subfigure}[t]{\textwidth}
        \centering
        \caption{hydrogen}
        \includegraphics[width=\textwidth]{20211218-181-lv/elec_s_181_lv1.0__Co2L0-3H-T-H-B-I-A-solar+p3-linemaxext10_2030/ts-balance-H2--2013-02.pdf}
    \end{subfigure}
    %}
    \caption{hourly time series}
    \label{fig:output-ts-3}
\end{figure}

\begin{figure}
    \centering
    % \makebox[\textwidth][c]{
    \begin{subfigure}[t]{\textwidth}
        \centering
        \caption{methane}
        \includegraphics[width=\textwidth]{20211218-181-lv/elec_s_181_lv1.0__Co2L0-3H-T-H-B-I-A-solar+p3-linemaxext10_2030/ts-balance-gas--2013-02.pdf}
    \end{subfigure}
    \begin{subfigure}[t]{\textwidth}
        \centering
        \caption{oil}
        \includegraphics[width=\textwidth]{20211218-181-lv/elec_s_181_lv1.0__Co2L0-3H-T-H-B-I-A-solar+p3-linemaxext10_2030/ts-balance-oil--2013-02.pdf}
    \end{subfigure}
    \begin{subfigure}[t]{\textwidth}
        \centering
        \caption{stored \co}
        \includegraphics[width=\textwidth]{20211218-181-lv/elec_s_181_lv1.0__Co2L0-3H-T-H-B-I-A-solar+p3-linemaxext10_2030/ts-balance-co2 stored--2013-02.pdf}
    \end{subfigure}
    %}
    \caption{hourly time series}
    \label{fig:output-ts-4}
\end{figure}

\newgeometry{margin=2cm}
\begin{landscape}



\section{Techno-Economic Assumptions}
\label{sec:si:costs}

Net-zero expected in 2035-2060, so take 2030 costs to be conservative, and also because of pathway (if want in 2050, much of infrastructure is built 2030-2050). NB: Even though total cost may be high, we're mostly focused on comparative study.

\begin{small}
\begin{longtable}{p{4cm}p{4cm}rp{3cm}p{10cm}}
\toprule
                      &            &        value &                          unit &                                                                                                                                                                                                                                                                                                                               source \\
technology & parameter &              &                               &                                                                                                                                                                                                                                                                                                                                      \\
\midrule
\endfirsthead

\toprule
                      &            &        value &                          unit &                                                                                                                                                                                                                                                                                                                               source \\
technology & parameter &              &                               &                                                                                                                                                                                                                                                                                                                                      \\
\midrule
\endhead
\midrule
\multicolumn{5}{r}{{Continued on next page}} \\
\midrule
\endfoot

\bottomrule
\endlastfoot
AC grid connection (station) & overnight investment &       250.00 &               \euro/kW$_{el}$ &                                                                                                                                                                                                                                                            DEA https://ens.dk/en/our-services/projections-and-models/technology-data \\
AC grid connection (submarine) & overnight investment &     2,685.00 &                   \euro/MW/km &                                                                                                                                                                                                                                                            DEA https://ens.dk/en/our-services/projections-and-models/technology-data \\
AC grid connection (underground) & overnight investment &     1,342.00 &                   \euro/MW/km &                                                                                                                                                                                                                                                            DEA https://ens.dk/en/our-services/projections-and-models/technology-data \\
CCGT & Cb coefficient &         2.00 &  50$^{\circ}$C/100$^{\circ}$C &                                                                                                                                                                                                                                                                        Danish Energy Agency, technology\_data\_for\_el\_and\_dh.xlsx \\
                      & Cv coefficient &         0.15 &  50$^{\circ}$C/100$^{\circ}$C &                                                                                                                                                                                                                                                                        Danish Energy Agency, technology\_data\_for\_el\_and\_dh.xlsx \\
                      & FOM &         3.35 &                       \%/year &                                                                                                                                                                                                                                                                        Danish Energy Agency, technology\_data\_for\_el\_and\_dh.xlsx \\
                      & VOM &         4.20 &                     \euro/MWh &                                                                                                                                                                                                                                                                        Danish Energy Agency, technology\_data\_for\_el\_and\_dh.xlsx \\
                      & efficiency &         0.58 &                      per unit &                                                                                                                                                                                                                                                                        Danish Energy Agency, technology\_data\_for\_el\_and\_dh.xlsx \\
                      & lifetime &        25.00 &                         years &                                                                                                                                                                                                                                                                        Danish Energy Agency, technology\_data\_for\_el\_and\_dh.xlsx \\
                      & overnight investment &       830.00 &                      \euro/kW &                                                                                                                                                                                                                                                                        Danish Energy Agency, technology\_data\_for\_el\_and\_dh.xlsx \\
CHP (biomass with carbon capture) & FOM &         3.00 &                       \%/year &                                                                                                                                                                                                                                                    Danish Energy Agency, technology\_data\_for\_industrial\_process\_heat\_0002.xlsx \\
                      & carbon capture rate &         0.90 &                      per unit &                                                                                                                                                                                                                                                    Danish Energy Agency, technology\_data\_for\_industrial\_process\_heat\_0002.xlsx \\
                      & electricity input &         0.08 &            MWh/t$_{\ce{CO2}}$ &                                                                                                                                                                                                                                                    Danish Energy Agency, technology\_data\_for\_industrial\_process\_heat\_0002.xlsx \\
                      & electricity input &         0.02 &            MWh/t$_{\ce{CO2}}$ &                                                                                                                                                                                                                                                    Danish Energy Agency, technology\_data\_for\_industrial\_process\_heat\_0002.xlsx \\
                      & heat input &         0.72 &            MWh/t$_{\ce{CO2}}$ &                                                                                                                                                                                                                                                    Danish Energy Agency, technology\_data\_for\_industrial\_process\_heat\_0002.xlsx \\
                      & heat output &         0.14 &            MWh/t$_{\ce{CO2}}$ &                                                                                                                                                                                                                                                    Danish Energy Agency, technology\_data\_for\_industrial\_process\_heat\_0002.xlsx \\
                      & heat output &         0.72 &            MWh/t$_{\ce{CO2}}$ &                                                                                                                                                                                                                                                    Danish Energy Agency, technology\_data\_for\_industrial\_process\_heat\_0002.xlsx \\
                      & lifetime &        25.00 &                         years &                                                                                                                                                                                                                                                    Danish Energy Agency, technology\_data\_for\_industrial\_process\_heat\_0002.xlsx \\
                      & overnight investment & 2,700,000.00 &      \euro/(t$_{\ce{CO2}}$/h) &                                                                                                                                                                                                                                                    Danish Energy Agency, technology\_data\_for\_industrial\_process\_heat\_0002.xlsx \\
CHP (biomass) & Cb coefficient &         0.46 &                     40°C/80°C &                                                                                                                                                                                                                                                                        Danish Energy Agency, technology\_data\_for\_el\_and\_dh.xlsx \\
                      & Cv coefficient &         1.00 &                     40°C/80°C &                                                                                                                                                                                                                                                                        Danish Energy Agency, technology\_data\_for\_el\_and\_dh.xlsx \\
                      & FOM &         3.58 &                       \%/year &                                                                                                                                                                                                                                                                        Danish Energy Agency, technology\_data\_for\_el\_and\_dh.xlsx \\
                      & VOM &         2.10 &              \euro/MWh$_{el}$ &                                                                                                                                                                                                                                                                        Danish Energy Agency, technology\_data\_for\_el\_and\_dh.xlsx \\
                      & efficiency &         0.30 &                      per unit &                                                                                                                                                                                                                                                                        Danish Energy Agency, technology\_data\_for\_el\_and\_dh.xlsx \\
                      & efficiency (heat) &         0.71 &                      per unit &                                                                                                                                                                                                                                                                        Danish Energy Agency, technology\_data\_for\_el\_and\_dh.xlsx \\
                      & lifetime &        25.00 &                         years &                                                                                                                                                                                                                                                                        Danish Energy Agency, technology\_data\_for\_el\_and\_dh.xlsx \\
                      & overnight investment &     3,210.28 &               \euro/kW$_{el}$ &                                                                                                                                                                                                                                                                        Danish Energy Agency, technology\_data\_for\_el\_and\_dh.xlsx \\
CHP (decentral) & FOM &         3.00 &                       \%/year &                                                                                                                                                                                                                                                                                                                                   HP \\
                      & discount rate &         0.04 &                      per unit &                                                                                                                                                                                                                                                                                                                        Palzer thesis \\
                      & lifetime &        25.00 &                         years &                                                                                                                                                                                                                                                                                                                                   HP \\
                      & overnight investment &     1,400.00 &               \euro/kW$_{el}$ &                                                                                                                                                                                                                                                                                                                                   HP \\
CHP (gas, central) & Cb coefficient &         1.00 &  50$^{\circ}$C/100$^{\circ}$C &                                                                                                                                                                                                                                                                        Danish Energy Agency, technology\_data\_for\_el\_and\_dh.xlsx \\
                      & Cv coefficient &         0.17 &                      per unit &                                                                                                                                                                                                                                                                                               DEA (loss of fuel for additional heat) \\
                      & FOM &         3.32 &                       \%/year &                                                                                                                                                                                                                                                                        Danish Energy Agency, technology\_data\_for\_el\_and\_dh.xlsx \\
                      & VOM &         4.20 &                     \euro/MWh &                                                                                                                                                                                                                                                                        Danish Energy Agency, technology\_data\_for\_el\_and\_dh.xlsx \\
                      & efficiency &         0.41 &                      per unit &                                                                                                                                                                                                                                                                        Danish Energy Agency, technology\_data\_for\_el\_and\_dh.xlsx \\
                      & lifetime &        25.00 &                         years &                                                                                                                                                                                                                                                                        Danish Energy Agency, technology\_data\_for\_el\_and\_dh.xlsx \\
                      & overnight investment &       560.00 &                      \euro/kW &                                                                                                                                                                                                                                                                        Danish Energy Agency, technology\_data\_for\_el\_and\_dh.xlsx \\
CHP (solid biomass, central) & Cb coefficient &         0.46 &                     40°C/80°C &                                                                                                                                                                                                                                                                        Danish Energy Agency, technology\_data\_for\_el\_and\_dh.xlsx \\
                      & Cv coefficient &         1.00 &                     40°C/80°C &                                                                                                                                                                                                                                                                        Danish Energy Agency, technology\_data\_for\_el\_and\_dh.xlsx \\
                      & FOM &         4.10 &                       \%/year &                                                                                                                                                                                                                                                                        Danish Energy Agency, technology\_data\_for\_el\_and\_dh.xlsx \\
                      & VOM &         1.85 &              \euro/MWh$_{el}$ &                                                                                                                                                                                                                                                                        Danish Energy Agency, technology\_data\_for\_el\_and\_dh.xlsx \\
                      & efficiency &         0.29 &                      per unit &                                                                                                                                                                                                                                                                        Danish Energy Agency, technology\_data\_for\_el\_and\_dh.xlsx \\
                      & efficiency (heat) &         0.69 &                      per unit &                                                                                                                                                                                                                                                                        Danish Energy Agency, technology\_data\_for\_el\_and\_dh.xlsx \\
                      & lifetime &        25.00 &                         years &                                                                                                                                                                                                                                                                        Danish Energy Agency, technology\_data\_for\_el\_and\_dh.xlsx \\
                      & overnight investment &     2,851.41 &               \euro/kW$_{el}$ &                                                                                                                                                                                                                                                                        Danish Energy Agency, technology\_data\_for\_el\_and\_dh.xlsx \\
DC grid connection (station) & overnight investment &       400.00 &               \euro/kW$_{el}$ &                                                                                                                                                                                                                                                   Haertel 2017; assuming one onshore and one offshore node + 13\% learning reduction \\
DC grid connection (submarine) & overnight investment &     2,000.00 &                   \euro/MW/km &                                                                                                                                                                                                                          DTU report based on Fig 34 of https://ec.europa.eu/energy/sites/ener/files/documents/2014\_nsog\_report.pdf \\
DC grid connection (underground) & overnight investment &     1,000.00 &                   \euro/MW/km &                                                                                                                                                                                                                                                                                      Haertel 2017; average + 13\% learning reduction \\
Fischer-Tropsch & FOM &         3.00 &                       \%/year &                                                                                                                                                                                                                                                                                                                doi:10.3390/su9020306 \\
                      & VOM &         4.20 &              \euro/MWh$_{FT}$ &                                                                                                                                                                                                                                                                       Danish Energy Agency, data\_sheets\_for\_renewable\_fuels.xlsx \\
                      & efficiency &         0.70 &                      per unit &                                                                                                                                                                                                                                                                       Danish Energy Agency, data\_sheets\_for\_renewable\_fuels.xlsx \\
                      & lifetime &        25.00 &                         years &                                                                                                                                                                                                                                                                       Danish Energy Agency, data\_sheets\_for\_renewable\_fuels.xlsx \\
                      & overnight investment &     1,600.00 &          \euro/kW$_{FT}$/year &                                                                                                                                                                                                                                                                       Danish Energy Agency, data\_sheets\_for\_renewable\_fuels.xlsx \\
HELMETH (direct power-to-methane) & FOM &         3.00 &                       \%/year &                                                                                                                                                                                                                                                                                                                            no source \\
                      & efficiency &         0.80 &                      per unit &                                                                                                                                                                                                                                                                                                                HELMETH press release \\
                      & lifetime &        25.00 &                         years &                                                                                                                                                                                                                                                                                                                            no source \\
                      & overnight investment &     2,000.00 &                      \euro/kW &                                                                                                                                                                                                                                                                                                                            no source \\
HVAC transmission line (overhead) & FOM &         2.00 &                       \%/year &                                                                                                                                                                                                                                                                                                                             Hagspiel \\
                      & lifetime &        40.00 &                         years &                                                                                                                                                                                                                                                                                                                             Hagspiel \\
                      & overnight investment &       400.00 &                   \euro/MW/km &                                                                                                                                                                                                                                                                                                                             Hagspiel \\
HVDC inverter pair & FOM &         2.00 &                       \%/year &                                                                                                                                                                                                                                                                                                                             Hagspiel \\
                      & lifetime &        40.00 &                         years &                                                                                                                                                                                                                                                                                                                             Hagspiel \\
                      & overnight investment &   150,000.00 &                      \euro/MW &                                                                                                                                                                                                                                                                                                                             Hagspiel \\
HVDC transmission line (overhead) & FOM &         2.00 &                       \%/year &                                                                                                                                                                                                                                                                                                                             Hagspiel \\
                      & lifetime &        40.00 &                         years &                                                                                                                                                                                                                                                                                                                             Hagspiel \\
                      & overnight investment &       400.00 &                   \euro/MW/km &                                                                                                                                                                                                                                                                                                                             Hagspiel \\
HVDC transmission line (submarine) & FOM &         0.35 &                       \%/year &                                                                                                                                                                                                                                                               Purvins et al. (2018): https://doi.org/10.1016/j.jclepro.2018.03.095 . \\
                      & lifetime &        40.00 &                         years &                                                                                                                                                                                                                                                               Purvins et al. (2018): https://doi.org/10.1016/j.jclepro.2018.03.095 . \\
                      & overnight investment &       471.16 &                   \euro/MW/km &                                                                                                                                                                                                                                                               Purvins et al. (2018): https://doi.org/10.1016/j.jclepro.2018.03.095 . \\
OCGT & FOM &         1.78 &                       \%/year &                                                                                                                                                                                                                                                                        Danish Energy Agency, technology\_data\_for\_el\_and\_dh.xlsx \\
                      & VOM &         4.50 &                     \euro/MWh &                                                                                                                                                                                                                                                                        Danish Energy Agency, technology\_data\_for\_el\_and\_dh.xlsx \\
                      & efficiency &         0.41 &                      per unit &                                                                                                                                                                                                                                                                        Danish Energy Agency, technology\_data\_for\_el\_and\_dh.xlsx \\
                      & lifetime &        25.00 &                         years &                                                                                                                                                                                                                                                                        Danish Energy Agency, technology\_data\_for\_el\_and\_dh.xlsx \\
                      & overnight investment &       435.24 &                      \euro/kW &                                                                                                                                                                                                                                                                        Danish Energy Agency, technology\_data\_for\_el\_and\_dh.xlsx \\
battery inverter & FOM &         0.34 &                       \%/year &                                                                                                                                                                                                                                                         Danish Energy Agency, technology\_data\_catalogue\_for\_energy\_storage.xlsx \\
                      & efficiency &         0.96 &                      per unit &                                                                                                                                                                                                                                                         Danish Energy Agency, technology\_data\_catalogue\_for\_energy\_storage.xlsx \\
                      & lifetime &        10.00 &                         years &                                                                                                                                                                                                                                                Danish Energy Agency, technology\_data\_catalogue\_for\_energy\_storage.xlsx, Note K. \\
                      & overnight investment &       160.00 &                      \euro/kW &                                                                                                                                                                                                                                                         Danish Energy Agency, technology\_data\_catalogue\_for\_energy\_storage.xlsx \\
battery storage & lifetime &        25.00 &                         years &                                                                                                                                                                                                                                                         Danish Energy Agency, technology\_data\_catalogue\_for\_energy\_storage.xlsx \\
                      & overnight investment &       142.00 &                     \euro/kWh &                                                                                                                                                                                                                                                         Danish Energy Agency, technology\_data\_catalogue\_for\_energy\_storage.xlsx \\
biogas & fuel &        59.00 &              \euro/MWh$_{th}$ &                                                                                                                                                                                                                                                                                                                        JRC and Zappa \\
biogas upgrading & FOM &         2.49 &                       \%/year &                                                                                                                                                                                                                                                                       Danish Energy Agency, data\_sheets\_for\_renewable\_fuels.xlsx \\
                      & VOM &         3.18 &               \euro/MWh input &                                                                                                                                                                                                                                                                       Danish Energy Agency, data\_sheets\_for\_renewable\_fuels.xlsx \\
                      & lifetime &        15.00 &                         years &                                                                                                                                                                                                                                                                       Danish Energy Agency, data\_sheets\_for\_renewable\_fuels.xlsx \\
                      & overnight investment &       381.00 &                \euro/kW input &                                                                                                                                                                                                                                                                       Danish Energy Agency, data\_sheets\_for\_renewable\_fuels.xlsx \\
biomass & FOM &         4.53 &                       \%/year &                                                                                                                                                                                                                                                                                        DIW DataDoc http://hdl.handle.net/10419/80348 \\
                      & efficiency &         0.47 &                      per unit &                                                                                                                                                                                                                                                                                        DIW DataDoc http://hdl.handle.net/10419/80348 \\
                      & fuel &         7.00 &              \euro/MWh$_{th}$ &                                                                                                                                                                                                                                                                                                                             IEA2011b \\
                      & lifetime &        30.00 &                         years &                                                                                                                                                                                                                                                                             ECF2010 in DIW DataDoc http://hdl.handle.net/10419/80348 \\
                      & overnight investment &     2,209.00 &               \euro/kW$_{el}$ &                                                                                                                                                                                                                                                                                        DIW DataDoc http://hdl.handle.net/10419/80348 \\
cement capture & FOM &         3.00 &                       \%/year &                                                                                                                                                                                                                                                    Danish Energy Agency, technology\_data\_for\_industrial\_process\_heat\_0002.xlsx \\
                      & carbon capture rate &         0.90 &                      per unit &                                                                                                                                                                                                                                                    Danish Energy Agency, technology\_data\_for\_industrial\_process\_heat\_0002.xlsx \\
                      & electricity input &         0.08 &            MWh/t$_{\ce{CO2}}$ &                                                                                                                                                                                                                                                    Danish Energy Agency, technology\_data\_for\_industrial\_process\_heat\_0002.xlsx \\
                      & electricity input &         0.02 &            MWh/t$_{\ce{CO2}}$ &                                                                                                                                                                                                                                                    Danish Energy Agency, technology\_data\_for\_industrial\_process\_heat\_0002.xlsx \\
                      & heat input &         0.72 &            MWh/t$_{\ce{CO2}}$ &                                                                                                                                                                                                                                                    Danish Energy Agency, technology\_data\_for\_industrial\_process\_heat\_0002.xlsx \\
                      & heat output &         0.14 &            MWh/t$_{\ce{CO2}}$ &                                                                                                                                                                                                                                                    Danish Energy Agency, technology\_data\_for\_industrial\_process\_heat\_0002.xlsx \\
                      & heat output &         1.54 &            MWh/t$_{\ce{CO2}}$ &                                                                                                                                                                                                                                                    Danish Energy Agency, technology\_data\_for\_industrial\_process\_heat\_0002.xlsx \\
                      & lifetime &        25.00 &                         years &                                                                                                                                                                                                                                                    Danish Energy Agency, technology\_data\_for\_industrial\_process\_heat\_0002.xlsx \\
                      & overnight investment & 2,600,000.00 &      \euro/(t$_{\ce{CO2}}$/h) &                                                                                                                                                                                                                                                    Danish Energy Agency, technology\_data\_for\_industrial\_process\_heat\_0002.xlsx \\
coal & FOM &         1.60 &                       \%/year &                                                                                                                                                                                                                                                                            Lazard s Levelized Cost of Energy Analysis - Version 13.0 \\
                      & VOM &         3.50 &              \euro/MWh$_{el}$ &                                                                                                                                                                                                                                                                            Lazard s Levelized Cost of Energy Analysis - Version 13.0 \\
                      & carbon intensity &         0.34 &     t$_{\ce{CO2}}$/MWh$_{th}$ &                                                                                                                                                                                                                                Entwicklung der spezifischen Kohlendioxid-Emissionen des deutschen Strommix in den Jahren 1990 - 2018 \\
                      & efficiency &         0.33 &                      per unit &                                                                                                                                                                                                                                                                            Lazard s Levelized Cost of Energy Analysis - Version 13.0 \\
                      & fuel &         8.15 &              \euro/MWh$_{th}$ &                                                                                                                                                                                                                                                                                                                              BP 2019 \\
                      & lifetime &        40.00 &                         years &                                                                                                                                                                                                                                                                            Lazard s Levelized Cost of Energy Analysis - Version 13.0 \\
                      & overnight investment &     3,845.51 &               \euro/kW$_{el}$ &                                                                                                                                                                                                                                                                            Lazard s Levelized Cost of Energy Analysis - Version 13.0 \\
decentral oil boiler & FOM &         2.00 &                       \%/year &                                                                                                                                                                                       Palzer thesis (https://energiesysteme-zukunft.de/fileadmin/user\_upload/Publikationen/PDFs/ESYS\_Materialien\_Optimierungsmodell\_REMod-D.pdf) \\
                      & efficiency &         0.90 &                      per unit &                                                                                                                                                                                       Palzer thesis (https://energiesysteme-zukunft.de/fileadmin/user\_upload/Publikationen/PDFs/ESYS\_Materialien\_Optimierungsmodell\_REMod-D.pdf) \\
                      & lifetime &        20.00 &                         years &                                                                                                                                                                                       Palzer thesis (https://energiesysteme-zukunft.de/fileadmin/user\_upload/Publikationen/PDFs/ESYS\_Materialien\_Optimierungsmodell\_REMod-D.pdf) \\
                      & overnight investment &       156.01 &               \euro/kW$_{th}$ &                                                                                                                                                                  Palzer thesis (https://energiesysteme-zukunft.de/fileadmin/user\_upload/Publikationen/PDFs/ESYS\_Materialien\_Optimierungsmodell\_REMod-D.pdf) (+eigene Berechnung) \\
direct air capture (DAC) & FOM &         4.95 &                       \%/year &                                                                                                                                                                                                                                                    Danish Energy Agency, technology\_data\_for\_industrial\_process\_heat\_0002.xlsx \\
                      & electricity input &         0.15 &            MWh/t$_{\ce{CO2}}$ &                                                                                                                                                                                                                                                    Danish Energy Agency, technology\_data\_for\_industrial\_process\_heat\_0002.xlsx \\
                      & electricity input &         0.32 &            MWh/t$_{\ce{CO2}}$ &                                                                                                                                                                                                                                                    Danish Energy Agency, technology\_data\_for\_industrial\_process\_heat\_0002.xlsx \\
                      & heat input &         2.00 &            MWh/t$_{\ce{CO2}}$ &                                                                                                                                                                                                                                                    Danish Energy Agency, technology\_data\_for\_industrial\_process\_heat\_0002.xlsx \\
                      & heat output &         0.20 &            MWh/t$_{\ce{CO2}}$ &                                                                                                                                                                                                                                                    Danish Energy Agency, technology\_data\_for\_industrial\_process\_heat\_0002.xlsx \\
                      & heat output &         1.00 &            MWh/t$_{\ce{CO2}}$ &                                                                                                                                                                                                                                                    Danish Energy Agency, technology\_data\_for\_industrial\_process\_heat\_0002.xlsx \\
                      & lifetime &        20.00 &                         years &                                                                                                                                                                                                                                                    Danish Energy Agency, technology\_data\_for\_industrial\_process\_heat\_0002.xlsx \\
                      & overnight investment & 6,000,000.00 &      \euro/(t$_{\ce{CO2}}$/h) &                                                                                                                                                                                                                                                    Danish Energy Agency, technology\_data\_for\_industrial\_process\_heat\_0002.xlsx \\
electricity distribution grid & FOM &         2.00 &                       \%/year &                                                                                                                                                                                                                                                                                                                                 TODO \\
                      & lifetime &        40.00 &                         years &                                                                                                                                                                                                                                                                                                                                 TODO \\
                      & overnight investment &       500.00 &                      \euro/kW &                                                                                                                                                                                                                                                                                                                                 TODO \\
electricity grid connection & FOM &         2.00 &                       \%/year &                                                                                                                                                                                                                                                                                                                                 TODO \\
                      & lifetime &        40.00 &                         years &                                                                                                                                                                                                                                                                                                                                 TODO \\
                      & overnight investment &       140.00 &                      \euro/kW &                                                                                                                                                                                                                                                                                                                                  DEA \\
electrolysis & FOM &         2.00 &                       \%/year &                                                                                                                                                                                                                                                                       Danish Energy Agency, data\_sheets\_for\_renewable\_fuels.xlsx \\
                      & efficiency &         0.68 &                      per unit &                                                                                                                                                                                                                                                                       Danish Energy Agency, data\_sheets\_for\_renewable\_fuels.xlsx \\
                      & lifetime &        30.00 &                         years &                                                                                                                                                                                                                                                                       Danish Energy Agency, data\_sheets\_for\_renewable\_fuels.xlsx \\
                      & overnight investment &       450.00 &               \euro/kW$_{el}$ &                                                                                                                                                                                                                                                                       Danish Energy Agency, data\_sheets\_for\_renewable\_fuels.xlsx \\
fossil gas & carbon intensity &         0.20 &     t$_{\ce{CO2}}$/MWh$_{th}$ &                                                                                                                                                                                                                                Entwicklung der spezifischen Kohlendioxid-Emissionen des deutschen Strommix in den Jahren 1990 - 2018 \\
                      & fuel &        20.10 &              \euro/MWh$_{th}$ &                                                                                                                                                                                                                                                                                                                              BP 2019 \\
fossil oil & FOM &         2.46 &                       \%/year &                                                                                                                                                                                                                                                                        Danish Energy Agency, technology\_data\_for\_el\_and\_dh.xlsx \\
                      & VOM &         6.00 &                     \euro/MWh &                                                                                                                                                                                                                                                                        Danish Energy Agency, technology\_data\_for\_el\_and\_dh.xlsx \\
                      & carbon intensity &         0.27 &     t$_{\ce{CO2}}$/MWh$_{th}$ &                                                                                                                                                                                                                                Entwicklung der spezifischen Kohlendioxid-Emissionen des deutschen Strommix in den Jahren 1990 - 2018 \\
                      & efficiency &         0.35 &                      per unit &                                                                                                                                                                                                                                                                        Danish Energy Agency, technology\_data\_for\_el\_and\_dh.xlsx \\
                      & fuel &        50.00 &              \euro/MWh$_{th}$ &                                                                                                                                                                                                                                     IEA WEM2017 97USD/boe = http://www.iea.org/media/weowebsite/2017/WEM\_Documentation\_WEO2017.pdf \\
                      & lifetime &        25.00 &                         years &                                                                                                                                                                                                                                                                        Danish Energy Agency, technology\_data\_for\_el\_and\_dh.xlsx \\
                      & overnight investment &       343.00 &                      \euro/kW &                                                                                                                                                                                                                                                                        Danish Energy Agency, technology\_data\_for\_el\_and\_dh.xlsx \\
fuel cell & Cb coefficient &         1.25 &  50$^{\circ}$C/100$^{\circ}$C &                                                                                                                                                                                                                                                                        Danish Energy Agency, technology\_data\_for\_el\_and\_dh.xlsx \\
                      & FOM &         5.00 &                       \%/year &                                                                                                                                                                                                                                                                        Danish Energy Agency, technology\_data\_for\_el\_and\_dh.xlsx \\
                      & efficiency &         0.50 &                      per unit &                                                                                                                                                                                                                                                                        Danish Energy Agency, technology\_data\_for\_el\_and\_dh.xlsx \\
                      & lifetime &        10.00 &                         years &                                                                                                                                                                                                                                                                        Danish Energy Agency, technology\_data\_for\_el\_and\_dh.xlsx \\
                      & overnight investment &     1,100.00 &               \euro/kW$_{el}$ &                                                                                                                                                                                                                                                                        Danish Energy Agency, technology\_data\_for\_el\_and\_dh.xlsx \\
gas boiler (central) & FOM &         3.80 &                       \%/year &                                                                                                                                                                                                                                                                        Danish Energy Agency, technology\_data\_for\_el\_and\_dh.xlsx \\
                      & VOM &         1.00 &              \euro/MWh$_{th}$ &                                                                                                                                                                                                                                                                        Danish Energy Agency, technology\_data\_for\_el\_and\_dh.xlsx \\
                      & efficiency &         1.04 &                      per unit &                                                                                                                                                                                                                                                                        Danish Energy Agency, technology\_data\_for\_el\_and\_dh.xlsx \\
                      & lifetime &        25.00 &                         years &                                                                                                                                                                                                                                                                        Danish Energy Agency, technology\_data\_for\_el\_and\_dh.xlsx \\
                      & overnight investment &        50.00 &               \euro/kW$_{th}$ &                                                                                                                                                                                                                                                                        Danish Energy Agency, technology\_data\_for\_el\_and\_dh.xlsx \\
gas boiler (decentral) & FOM &         6.69 &                       \%/year &                                                                                                                                                                                                                                                    Danish Energy Agency, technologydatafor\_heating\_installations\_marts\_2018.xlsx \\
                      & discount rate &         0.04 &                      per unit &                                                                                                                                                                                                                                                                                                                        Palzer thesis \\
                      & efficiency &         0.98 &                      per unit &                                                                                                                                                                                                                                                    Danish Energy Agency, technologydatafor\_heating\_installations\_marts\_2018.xlsx \\
                      & lifetime &        20.00 &                         years &                                                                                                                                                                                                                                                    Danish Energy Agency, technologydatafor\_heating\_installations\_marts\_2018.xlsx \\
                      & overnight investment &       296.82 &               \euro/kW$_{th}$ &                                                                                                                                                                                                                                                    Danish Energy Agency, technologydatafor\_heating\_installations\_marts\_2018.xlsx \\
heat pump (air-sourced, central) & FOM &         0.23 &                       \%/year &                                                                                                                                                                                                                                                                        Danish Energy Agency, technology\_data\_for\_el\_and\_dh.xlsx \\
                      & VOM &         2.51 &              \euro/MWh$_{th}$ &                                                                                                                                                                                                                                                                        Danish Energy Agency, technology\_data\_for\_el\_and\_dh.xlsx \\
                      & efficiency &         3.60 &                      per unit &                                                                                                                                                                                                                                                                        Danish Energy Agency, technology\_data\_for\_el\_and\_dh.xlsx \\
                      & lifetime &        25.00 &                         years &                                                                                                                                                                                                                                                                        Danish Energy Agency, technology\_data\_for\_el\_and\_dh.xlsx \\
                      & overnight investment &       856.25 &               \euro/kW$_{th}$ &                                                                                                                                                                                                                                                                        Danish Energy Agency, technology\_data\_for\_el\_and\_dh.xlsx \\
heat pump (air-sourced, decentral) & FOM &         3.00 &                       \%/year &                                                                                                                                                                                                                                                    Danish Energy Agency, technologydatafor\_heating\_installations\_marts\_2018.xlsx \\
                      & discount rate &         0.04 &                      per unit &                                                                                                                                                                                                                                                                                                                        Palzer thesis \\
                      & efficiency &         3.60 &                      per unit &                                                                                                                                                                                                                                                    Danish Energy Agency, technologydatafor\_heating\_installations\_marts\_2018.xlsx \\
                      & lifetime &        18.00 &                         years &                                                                                                                                                                                                                                                    Danish Energy Agency, technologydatafor\_heating\_installations\_marts\_2018.xlsx \\
                      & overnight investment &       850.00 &               \euro/kW$_{th}$ &                                                                                                                                                                                                                                                    Danish Energy Agency, technologydatafor\_heating\_installations\_marts\_2018.xlsx \\
heat pump (ground-sourced, central) & FOM &         0.39 &                       \%/year &                                                                                                                                                                                                                                                                        Danish Energy Agency, technology\_data\_for\_el\_and\_dh.xlsx \\
                      & VOM &         1.25 &              \euro/MWh$_{th}$ &                                                                                                                                                                                                                                                                        Danish Energy Agency, technology\_data\_for\_el\_and\_dh.xlsx \\
                      & efficiency &         1.73 &                      per unit &                                                                                                                                                                                                                                                                        Danish Energy Agency, technology\_data\_for\_el\_and\_dh.xlsx \\
                      & lifetime &        25.00 &                         years &                                                                                                                                                                                                                                                                        Danish Energy Agency, technology\_data\_for\_el\_and\_dh.xlsx \\
                      & overnight investment &       507.60 &              \euro/kW$_{th}$  &                                                                                                                                                                                                                                                                        Danish Energy Agency, technology\_data\_for\_el\_and\_dh.xlsx \\
heat pump (ground-sourced, decentral) & FOM &         1.82 &                       \%/year &                                                                                                                                                                                                                                                    Danish Energy Agency, technologydatafor\_heating\_installations\_marts\_2018.xlsx \\
                      & discount rate &         0.04 &                      per unit &                                                                                                                                                                                                                                                                                                                        Palzer thesis \\
                      & efficiency &         3.90 &                      per unit &                                                                                                                                                                                                                                                    Danish Energy Agency, technologydatafor\_heating\_installations\_marts\_2018.xlsx \\
                      & lifetime &        20.00 &                         years &                                                                                                                                                                                                                                                    Danish Energy Agency, technologydatafor\_heating\_installations\_marts\_2018.xlsx \\
                      & overnight investment &     1,400.00 &               \euro/kW$_{th}$ &                                                                                                                                                                                                                                                    Danish Energy Agency, technologydatafor\_heating\_installations\_marts\_2018.xlsx \\
home battery inverter & FOM &         0.34 &                       \%/year &                                                                                                                                                            Global Energy System based on 100\% Renewable Energy, Energywatchgroup/LTU University, 2019, Danish Energy Agency, technology\_data\_catalogue\_for\_energy\_storage.xlsx \\
                      & efficiency &         0.96 &                      per unit &                                                                                                                                                            Global Energy System based on 100\% Renewable Energy, Energywatchgroup/LTU University, 2019, Danish Energy Agency, technology\_data\_catalogue\_for\_energy\_storage.xlsx \\
                      & lifetime &        10.00 &                         years &                                                                                                                                                   Global Energy System based on 100\% Renewable Energy, Energywatchgroup/LTU University, 2019, Danish Energy Agency, technology\_data\_catalogue\_for\_energy\_storage.xlsx, Note K. \\
                      & overnight investment &       228.06 &                      \euro/kW &                                                                                                                                                            Global Energy System based on 100\% Renewable Energy, Energywatchgroup/LTU University, 2019, Danish Energy Agency, technology\_data\_catalogue\_for\_energy\_storage.xlsx \\
home battery storage & lifetime &        25.00 &                         years &                                                                                                                                                            Global Energy System based on 100\% Renewable Energy, Energywatchgroup/LTU University, 2019, Danish Energy Agency, technology\_data\_catalogue\_for\_energy\_storage.xlsx \\
                      & overnight investment &       202.90 &                     \euro/kWh &                                                                                                                                                            Global Energy System based on 100\% Renewable Energy, Energywatchgroup/LTU University, 2019, Danish Energy Agency, technology\_data\_catalogue\_for\_energy\_storage.xlsx \\
hydrogen liquefaction & FOM &         8.00 &                       \%/year &                                                                                                                                                                                                                                   Reuß et al 2017: https://doi.org/10.1016/j.apenergy.2017.05.050 , Table 9 and equation in sec 3.0. \\
                      & lifetime &        20.00 &                         years &                                                                                                                                                                                                                                   Reuß et al 2017: https://doi.org/10.1016/j.apenergy.2017.05.050 , Table 9 and equation in sec 3.0. \\
                      & overnight investment & 1,497,967.32 &          \euro/MW$_{\ce{H2}}$ &                                                                                                                                                                                                                                   Reuß et al 2017: https://doi.org/10.1016/j.apenergy.2017.05.050 , Table 9 and equation in sec 3.0. \\
hydrogen pipeline & FOM &         3.17 &                       \%/year &                                                                                                                                                                                                                                          Danish Energy Agency, Technology Data for Energy Transport (2021), Excel datasheet: H2 140. \\
                      & lifetime &        50.00 &                         years &                                                                                                                                                                                                                                          Danish Energy Agency, Technology Data for Energy Transport (2021), Excel datasheet: H2 140. \\
                      & overnight investment &       226.47 &                   \euro/MW/km &                                                                                                                                             European Hydrogen Backbone Report (June 2021): https://gasforclimate2050.eu/wp-content/uploads/2021/06/EHB\_Analysing-the-future-demand-supply-and-transport-of-hydrogen\_June-2021.pdf. \\
hydrogen pipeline (repurposed) & FOM &         3.17 &                       \%/year &                                                                                                                                                                                                                                          Danish Energy Agency, Technology Data for Energy Transport (2021), Excel datasheet: H2 140. \\
                      & lifetime &        50.00 &                         years &                                                                                                                                                                                                                                          Danish Energy Agency, Technology Data for Energy Transport (2021), Excel datasheet: H2 140. \\
                      & overnight investment &       105.88 &                   \euro/MW/km &                                                                                                                                             European Hydrogen Backbone Report (June 2021): https://gasforclimate2050.eu/wp-content/uploads/2021/06/EHB\_Analysing-the-future-demand-supply-and-transport-of-hydrogen\_June-2021.pdf. \\
hydrogen pipeline (submarine) & FOM &         3.00 &                       \%/year &                                                                                                                                                                                                                                                                                       Assume same as for CH4 (g) submarine pipeline. \\
                      & lifetime &        30.00 &                         years &                                                                                                                                                                                                                                                                                       Assume same as for CH4 (g) submarine pipeline. \\
                      & overnight investment &       329.37 &                   \euro/MW/km &  Assume similar cost as for CH4 (g) submarine pipeline but with the same factor as between onland CH4 (g) pipeline and H2 (g) pipeline (2.86). This estimate is comparable to a 36in diameter pipeline calaculated based on d’Amore-Domenech et al (2021): 10.1016/j.apenergy.2021.116625 , supplementary material (=251 EUR/MW/km). \\
hydrogen storage (steel tank) & FOM &         1.11 &                       \%/year &                                                                                                                                                                                                                                                         Danish Energy Agency, technology\_data\_catalogue\_for\_energy\_storage.xlsx \\
                      & lifetime &        30.00 &                         years &                                                                                                                                                                                                                                                         Danish Energy Agency, technology\_data\_catalogue\_for\_energy\_storage.xlsx \\
                      & overnight investment &        44.91 &                     \euro/kWh &                                                                                                                                                                                                                                                         Danish Energy Agency, technology\_data\_catalogue\_for\_energy\_storage.xlsx \\
hydrogen storage (underground) & FOM &         0.00 &                       \%/year &                                                                                                                                                                                                                                                         Danish Energy Agency, technology\_data\_catalogue\_for\_energy\_storage.xlsx \\
                      & VOM &         0.00 &                     \euro/MWh &                                                                                                                                                                                                                                                         Danish Energy Agency, technology\_data\_catalogue\_for\_energy\_storage.xlsx \\
                      & lifetime &       100.00 &                         years &                                                                                                                                                                                                                                                         Danish Energy Agency, technology\_data\_catalogue\_for\_energy\_storage.xlsx \\
                      & overnight investment &         2.00 &                     \euro/kWh &                                                                                                                                                                                                                                                         Danish Energy Agency, technology\_data\_catalogue\_for\_energy\_storage.xlsx \\
lignite & FOM &         1.60 &                       \%/year &                                                                                                                                                                                                                                                                            Lazard s Levelized Cost of Energy Analysis - Version 13.0 \\
                      & VOM &         3.50 &              \euro/MWh$_{el}$ &                                                                                                                                                                                                                                                                            Lazard s Levelized Cost of Energy Analysis - Version 13.0 \\
                      & carbon intensity &         0.41 &     t$_{\ce{CO2}}$/MWh$_{th}$ &                                                                                                                                                                                                                                Entwicklung der spezifischen Kohlendioxid-Emissionen des deutschen Strommix in den Jahren 1990 - 2018 \\
                      & efficiency &         0.33 &                      per unit &                                                                                                                                                                                                                                                                            Lazard s Levelized Cost of Energy Analysis - Version 13.0 \\
                      & fuel &         2.90 &              \euro/MWh$_{th}$ &                                                                                                                                                                                                                                                                                                                                  DIW \\
                      & lifetime &        40.00 &                         years &                                                                                                                                                                                                                                                                            Lazard s Levelized Cost of Energy Analysis - Version 13.0 \\
                      & overnight investment &     3,845.51 &               \euro/kW$_{el}$ &                                                                                                                                                                                                                                                                            Lazard s Levelized Cost of Energy Analysis - Version 13.0 \\
methanation & FOM &         4.00 &                       \%/year &                                                                                                                                                                                                                                                                   Fasihi et al 2017, table 1, https://www.mdpi.com/2071-1050/9/2/306 \\
                      & efficiency &         0.80 &                      per unit &                                                                                                                                                                                                                                                                                                            Palzer and Schaber thesis \\
                      & lifetime &        30.00 &                         years &                                                                                                                                                                                                                                                                   Fasihi et al 2017, table 1, https://www.mdpi.com/2071-1050/9/2/306 \\
                      & overnight investment &       278.00 &         \euro/kW$_{\ce{CH4}}$ &                                                                                                                                                                                                                                                                   Fasihi et al 2017, table 1, https://www.mdpi.com/2071-1050/9/2/306 \\
natural gas pipeline & FOM &         1.50 &                       \%/year &                                                                                                                                                                                                                                                      Assume same as for H2 (g) pipeline in 2050 (CH4 pipeline as mature technology). \\
                      & lifetime &        50.00 &                         years &                                                                                                                                                                                                                                                      Assume same as for H2 (g) pipeline in 2050 (CH4 pipeline as mature technology). \\
                      & overnight investment &        79.00 &                   \euro/MW/km &                                                                                                                                                                                                                                                                                                                         Guesstimate. \\
natural gas pipeline (submarine) & FOM &         3.00 &                       \%/year &                                                                                                                                                                                                                                              d’Amore-Domenech et al (2021): 10.1016/j.apenergy.2021.116625 , supplementary material. \\
                      & lifetime &        30.00 &                         years &                                                                                                                                                                                                                                              d’Amore-Domenech et al (2021): 10.1016/j.apenergy.2021.116625 , supplementary material. \\
                      & overnight investment &       114.89 &                   \euro/MW/km &                                                                                                                                                                                                                                                                                        Kaiser (2017): 10.1016/j.marpol.2017.05.003 . \\
offshore wind & FOM &         2.32 &                       \%/year &                                                                                                                                                                                                                                                                        Danish Energy Agency, technology\_data\_for\_el\_and\_dh.xlsx \\
                      & VOM &         3.89 &        \euro/MWh$_{el}$, 2020 &                                                                                                                                                                                                                                                                        Danish Energy Agency, technology\_data\_for\_el\_and\_dh.xlsx \\
                      & lifetime &        30.00 &                         years &                                                                                                                                                                                                                                                                        Danish Energy Agency, technology\_data\_for\_el\_and\_dh.xlsx \\
                      & overnight investment &     1,523.55 &         \euro/kW$_{el}$, 2020 &                                                                                                                                                                                                                                                                        Danish Energy Agency, technology\_data\_for\_el\_and\_dh.xlsx \\
onshore wind & FOM &         1.22 &                       \%/year &                                                                                                                                                                                                                                                                        Danish Energy Agency, technology\_data\_for\_el\_and\_dh.xlsx \\
                      & VOM &         1.35 &                     \euro/MWh &                                                                                                                                                                                                                                                                        Danish Energy Agency, technology\_data\_for\_el\_and\_dh.xlsx \\
                      & lifetime &        30.00 &                         years &                                                                                                                                                                                                                                                                        Danish Energy Agency, technology\_data\_for\_el\_and\_dh.xlsx \\
                      & overnight investment &     1,035.56 &                      \euro/kW &                                                                                                                                                                                                                                                                        Danish Energy Agency, technology\_data\_for\_el\_and\_dh.xlsx \\
pumped hydro storage & FOM &         1.00 &                       \%/year &                                                                                                                                                                                                                                                                                        DIW DataDoc http://hdl.handle.net/10419/80348 \\
                      & efficiency &         0.75 &                      per unit &                                                                                                                                                                                                                                                                                        DIW DataDoc http://hdl.handle.net/10419/80348 \\
                      & lifetime &        80.00 &                         years &                                                                                                                                                                                                                                                                                                                              IEA2010 \\
                      & overnight investment &     2,208.16 &               \euro/kW$_{el}$ &                                                                                                                                                                                                                                                                                        DIW DataDoc http://hdl.handle.net/10419/80348 \\
reservoir hydro & FOM &         1.00 &                       \%/year &                                                                                                                                                                                                                                                                                        DIW DataDoc http://hdl.handle.net/10419/80348 \\
                      & efficiency &         0.90 &                      per unit &                                                                                                                                                                                                                                                                                        DIW DataDoc http://hdl.handle.net/10419/80348 \\
                      & lifetime &        80.00 &                         years &                                                                                                                                                                                                                                                                                                                              IEA2010 \\
                      & overnight investment &     2,208.16 &               \euro/kW$_{el}$ &                                                                                                                                                                                                                                                                                        DIW DataDoc http://hdl.handle.net/10419/80348 \\
resistive heater (central) & FOM &         1.70 &                       \%/year &                                                                                                                                                                                                                                                                        Danish Energy Agency, technology\_data\_for\_el\_and\_dh.xlsx \\
                      & VOM &         1.00 &              \euro/MWh$_{th}$ &                                                                                                                                                                                                                                                                        Danish Energy Agency, technology\_data\_for\_el\_and\_dh.xlsx \\
                      & efficiency &         0.99 &                      per unit &                                                                                                                                                                                                                                                                        Danish Energy Agency, technology\_data\_for\_el\_and\_dh.xlsx \\
                      & lifetime &        20.00 &                         years &                                                                                                                                                                                                                                                                        Danish Energy Agency, technology\_data\_for\_el\_and\_dh.xlsx \\
                      & overnight investment &        60.00 &               \euro/kW$_{th}$ &                                                                                                                                                                                                                                                                        Danish Energy Agency, technology\_data\_for\_el\_and\_dh.xlsx \\
resistive heater (decentral) & FOM &         2.00 &                       \%/year &                                                                                                                                                                                                                                                                                                                       Schaber thesis \\
                      & discount rate &         0.04 &                      per unit &                                                                                                                                                                                                                                                                                                                        Palzer thesis \\
                      & efficiency &         0.90 &                      per unit &                                                                                                                                                                                                                                                                                                                       Schaber thesis \\
                      & lifetime &        20.00 &                         years &                                                                                                                                                                                                                                                                                                                       Schaber thesis \\
                      & overnight investment &       100.00 &                   \euro/kWhth &                                                                                                                                                                                                                                                                                                                       Schaber thesis \\
run of river & FOM &         2.00 &                       \%/year &                                                                                                                                                                                                                                                                                        DIW DataDoc http://hdl.handle.net/10419/80348 \\
                      & efficiency &         0.90 &                      per unit &                                                                                                                                                                                                                                                                                        DIW DataDoc http://hdl.handle.net/10419/80348 \\
                      & lifetime &        80.00 &                         years &                                                                                                                                                                                                                                                                                                                              IEA2010 \\
                      & overnight investment &     3,312.24 &               \euro/kW$_{el}$ &                                                                                                                                                                                                                                                                                        DIW DataDoc http://hdl.handle.net/10419/80348 \\
solar PV & FOM &         1.95 &                       \%/year &                                                                                                                                                                                                                                                                        Danish Energy Agency, technology\_data\_for\_el\_and\_dh.xlsx \\
                      & VOM &         0.01 &                   \euro/MWhel &                                                                                                                                                                                                                                                                                           RES costs made up to fix curtailment order \\
                      & lifetime &        40.00 &                         years &                                                                                                                                                                                                                                                                        Danish Energy Agency, technology\_data\_for\_el\_and\_dh.xlsx \\
                      & overnight investment &       492.11 &               \euro/kW$_{el}$ &                                                                                                                                                                                                                                                                        Danish Energy Agency, technology\_data\_for\_el\_and\_dh.xlsx \\
solar PV (rooftop) & FOM &         1.42 &                       \%/year &                                                                                                                                                                                                                                                                        Danish Energy Agency, technology\_data\_for\_el\_and\_dh.xlsx \\
                      & discount rate &         0.04 &                      per unit &                                                                                                                                                                                                                                                                                                               standard for decentral \\
                      & lifetime &        40.00 &                         years &                                                                                                                                                                                                                                                                        Danish Energy Agency, technology\_data\_for\_el\_and\_dh.xlsx \\
                      & overnight investment &       636.66 &               \euro/kW$_{el}$ &                                                                                                                                                                                                                                                                        Danish Energy Agency, technology\_data\_for\_el\_and\_dh.xlsx \\
solar PV (utility-scale) & FOM &         2.48 &                       \%/year &                                                                                                                                                                                                                                                                        Danish Energy Agency, technology\_data\_for\_el\_and\_dh.xlsx \\
                      & lifetime &        40.00 &                         years &                                                                                                                                                                                                                                                                        Danish Energy Agency, technology\_data\_for\_el\_and\_dh.xlsx \\
                      & overnight investment &       347.56 &               \euro/kW$_{el}$ &                                                                                                                                                                                                                                                                        Danish Energy Agency, technology\_data\_for\_el\_and\_dh.xlsx \\
solar thermal (central) & FOM &         1.40 &                       \%/year &                                                                                                                                                                                                                                                                                                                                   HP \\
                      & lifetime &        20.00 &                         years &                                                                                                                                                                                                                                                                                                                                   HP \\
                      & overnight investment &   140,000.00 &               \euro/1000m$^2$ &                                                                                                                                                                                                                                                                                                                                   HP \\
solar thermal (decentral) & FOM &         1.30 &                       \%/year &                                                                                                                                                                                                                                                                                                                                   HP \\
                      & discount rate &         0.04 &                      per unit &                                                                                                                                                                                                                                                                                                                        Palzer thesis \\
                      & lifetime &        20.00 &                         years &                                                                                                                                                                                                                                                                                                                                   HP \\
                      & overnight investment &   270,000.00 &               \euro/1000m$^2$ &                                                                                                                                                                                                                                                                                                                                   HP \\
solid biomass & carbon intensity &         0.30 &     t$_{\ce{CO2}}$/MWh$_{th}$ &                                                                                                                                                                                                                                                                                                                                 TODO \\
                      & fuel &        25.20 &              \euro/MWh$_{th}$ &                                                                                                                                                                                                                                                                         Is a 100\% renewable European power system feasible by 2050? \\
steam methane reforming & FOM &         5.00 &                       \%/year &                                                                                                                                                                                                                                                                                                                 Danish Energy Agency \\
                      & efficiency &         0.76 &             per unit (in LHV) &                                                                    IEA Global average levelised cost of hydrogen production by energy source and technology, 2019 and 2050 (2020), https://www.iea.org/data-and-statistics/charts/global-average-levelised-cost-of-hydrogen-production-by-energy-source-and-technology-2019-and-2050 \\
                      & lifetime &        30.00 &                         years &                                                                    IEA Global average levelised cost of hydrogen production by energy source and technology, 2019 and 2050 (2020), https://www.iea.org/data-and-statistics/charts/global-average-levelised-cost-of-hydrogen-production-by-energy-source-and-technology-2019-and-2050 \\
                      & overnight investment &   493,470.40 &         \euro/MW$_{\ce{CH4}}$ &                                                                                                                                                                                                                                                                                                                 Danish Energy Agency \\
steam methane reforming with carbon capture & FOM &         5.00 &                       \%/year &                                                                                                                                                                                                                                                                                                                 Danish Energy Agency \\
                      & carbon capture rate &         0.90 &         \euro/MW$_{\ce{CH4}}$ &                                                                    IEA Global average levelised cost of hydrogen production by energy source and technology, 2019 and 2050 (2020), https://www.iea.org/data-and-statistics/charts/global-average-levelised-cost-of-hydrogen-production-by-energy-source-and-technology-2019-and-2050 \\
                      & efficiency &         0.69 &             per unit (in LHV) &                                                                    IEA Global average levelised cost of hydrogen production by energy source and technology, 2019 and 2050 (2020), https://www.iea.org/data-and-statistics/charts/global-average-levelised-cost-of-hydrogen-production-by-energy-source-and-technology-2019-and-2050 \\
                      & lifetime &        30.00 &                         years &                                                                    IEA Global average levelised cost of hydrogen production by energy source and technology, 2019 and 2050 (2020), https://www.iea.org/data-and-statistics/charts/global-average-levelised-cost-of-hydrogen-production-by-energy-source-and-technology-2019-and-2050 \\
                      & overnight investment &   572,425.66 &         \euro/MW$_{\ce{CH4}}$ &                                                                                                                                                                                                                                                                                                                 Danish Energy Agency \\
thermal storage (water tank, central) & FOM &         0.55 &                       \%/year &                                                                                                                                                                                                                                                         Danish Energy Agency, technology\_data\_catalogue\_for\_energy\_storage.xlsx \\
                      & lifetime &        25.00 &                         years &                                                                                                                                                                                                                                                         Danish Energy Agency, technology\_data\_catalogue\_for\_energy\_storage.xlsx \\
                      & overnight investment &         0.54 &                     \euro/kWh &                                                                                                                                                                                                                                                         Danish Energy Agency, technology\_data\_catalogue\_for\_energy\_storage.xlsx \\
thermal storage (water tank, decentral) & FOM &         1.00 &                       \%/year &                                                                                                                                                                                                                                                                                                                                   HP \\
                      & discount rate &         0.04 &                      per unit &                                                                                                                                                                                                                                                                                                                        Palzer thesis \\
                      & lifetime &        20.00 &                         years &                                                                                                                                                                                                                                                                                                                                   HP \\
                      & overnight investment &        18.38 &                     \euro/kWh &                                                                                                                                                                                                                                                                                                                     IWES Interaktion \\
water tank charger & efficiency &         0.84 &                      per unit &                                                                                                                                                                                                                                                         Danish Energy Agency, technology\_data\_catalogue\_for\_energy\_storage.xlsx \\
water tank discharger & efficiency &         0.84 &                      per unit &                                                                                                                                                                                                                                                         Danish Energy Agency, technology\_data\_catalogue\_for\_energy\_storage.xlsx \\
\end{longtable}

\end{small}

\end{landscape}

\restoregeometry