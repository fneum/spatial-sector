\section{Model Overview}

Not all of the sectors are at the full nodal resolution, and some demand for
some sectors is distributed to nodes using heuristics that need to be corrected.
Some networks are copper-plated to reduce computational times.

\section{Electricity Sector}

Electricity supply and demand follows the electricity generation and
transmission model PyPSA-Eur, except that hydrogen storage is integrated into
the hydrogen supply, demand and network, and PyPSA-Eur-Sec includes CHPs.


\subsection{Electricity Demand}

distribution electricity demand for
industry uses the geographical data from
the Hotmaps Industrial Database.

subtracts existing electrified heating from
the existing electricity demand, so that power-to-heat can be optimised
separately.
Building heating demand: nodal, distributed in each country based on population.

The remaining electricity demand for households and services is distributed
inside each country proportional to GDP and population.

\subsection{Electricity Supply}

OCGT CCGT

hydro

run of river

\subsection{Electricity Storage}

battery

Distinguish costs for home battery storage and inverter from utility-scale battery costs.

pumped-hydro

\subsection{Electricity Transport}

clusters down the electricity transmission substations in each European country
based on the k-means algorithm

ENTSO-E

TYNDP

\section{Transport Sector}

\subsection{Land Transport}

Land transport is separated by energy carrier (fossil, hydrogen fuel cell
electric vehicle, and electric vehicle), but still needs to be separated into
heavy and light vehicles (the data is there, just not the code yet).

\subsection{Aviation}

kerosene

\subsection{Shipping}

hydrogen liquefaction costs for hydrogen demand in shipping

\section{Industry Sector}

Demand

Based on materials demand from JRC-IDEES and other sources such as the USGS for ammonia.

Industry is split into many sectors, including iron and steel, ammonia, other basic chemicals, cement, non-metalic minerals, alumuninium, other non-ferrous metals, pulp, paper and printing, food, beverages and tobacco, and other more minor sectors.

Inside each country the industrial demand is distributed using the Hotmaps Industrial Database.

Supply

Process switching (e.g. from blast furnaces to direct reduction and electric arc furnaces for steel) is defined exogenously.

Fuel switching for process heat is mostly also done exogenously.

Solid biomass is used for up to 500 Celsius, mostly in paper and pulp and food and beverages.

Higher temperatures are met with methane.

\subsection{Iron and Steel}

\subsection{Chemicals Industry}

basic chemicals: HVC (high-value chemicals), chlorine, methanol and ammonia

specify reuse, primary production, and mechanical and chemical recycling fraction of platics

Ammonia production data is taken from the USGS

\subsection{Non-metallic Mineral Products}

Cement

Ceramics

Glass

\subsection{Non-ferrous Metals}

Aluminium

\subsection{Other Industry Subsectors}

energy demands and CO2 emissions for the agriculture, forestry and fishing sector

\section{Heating Sector}

\subsection{Heating Demand}

Heat demand is split into:

urban central: large-scale district heating networks in urban areas with dense
heat demand

residential/services urban decentral: heating for individual buildings in urban
areas

residential/services rural: heating for individual buildings in rural areas,
agriculture heat uses

Building heating demand: nodal, distributed in each country based on population.


\subsection{Heating Supply}

Oil and gas boilers

Heat pumps

Either air-to-water or ground-to-water heat pumps are implemented.

They have coefficient of performance (COP) based on either the external air or the soil hourly temperature.

Ground-source heat pumps are only allowed in rural areas because of space constraints.

Only air-source heat pumps are allowed in urban areas. This is a conservative
assumption, since there are many possible sources of low-temperature heat that
could be tapped in cities (waste water, rivers, lakes, seas, etc.).

Resistive heaters

Large Combined Heat and Power (CHP) plants

https://doi.org/10.1016/j.energy.2018.10.044

PyPSA-Eur-Sec includes CHP plants fuelled by methane, hydrogen and solid biomass from waste and residues.

Hydrogen CHPs are fuel cells.

Methane and biomass CHPs are based on back pressure plants operating with a
fixed ratio of electricity to heat output. The methane CHP is modelled on the
Danish Energy Agency (DEA) “Gas turbine simple cycle (large)” while the solid
biomass CHP is based on the DEA’s “09b Wood Pellets Medium”.

The efficiencies of each are given on the back pressure line, where the back
pressure coefficient $c_b$ is the electricity output divided by the heat output.
The plants are not allowed to deviate from the back pressure line and are
implement as Link objects with a fixed ratio of heat to electricity output.

Micro-CHP for individual buildings

Waste heat from Fuel Cells, Methanation and Fischer-Tropsch plants

Solar thermal collectors

District heating!

\subsection{Heat Storage}

Thermal energy storage using hot water tanks
Small for decentral applications.
Big water pit storage for district heating.

\section{Wind}

\subsection{Wind Potentials}

\subsection{Wind Time Series}

\section{Solar}

\subsection{Solar Potentials}

utility PV

Installable potentials for rooftop PV are included with an assumption of 1 kWp
per person.

Solar thermal

\subsection{Solar Time series}

\citeS{zappa2019}

\section{Hydrogen}

\subsection{Hydrogen Demand}

Stationary fuel cell CHP.

Transport applications.

Industry (ammonia, precursor to hydrocarbons for chemicals and iron/steel).

\subsection{Hydrogen Supply}

Steam Methane Reforming (SMR), SMR+CCS, electrolysers.

\subsection{Hydrogen Transport}

retrofitting

new pipelines

\subsection{Hydrogen Storage}

cavern storage

steel tanks

\section{Methane}

\subsection{Methane Demand}

Can be used in boilers, in CHPs, in industry for high temperature heat, in OCGT.

Not used in transport because of engine slippage.

\subsection{Methane Supply}

Fossil, biogas, Sabatier (hydrogen to methane), HELMETH (directly power to
methane with efficient heat integration).

\subsection{Methane Transport}

Scigrid Gas dataset

single node for Europe, since future demand is so low and no bottlenecks are expected.

\section{Oil-based Products}

\subsection{Oil-based Product Demand}

Transport fuels, agriculture machinery and naphtha as a feedstock for the
chemicals industry.

\subsection{Oil-based Product Supply}

Fossil or Fischer-Tropsch.

\subsection{Oil-based Product Transport}

Liquid hydrocarbons: single node for Europe, since transport costs for liquids are low.


\section{Biomass}

\subsection{Biomass Potentials}

Only wastes and residues from the JRC ENSPRESO biomass dataset.

nodal where biomass potential is regionally disaggregated 

Use JRC ENSPRESO database to spatially disaggregate biomass potentials to
PyPSA-Eur regions based on overlaps with NUTS2 regions from ENSPRESO
(proportional to area)

\subsection{Biomass Demand}

Solid biomass provides process heat up to 500 Celsius in industry, as well as
feeding CHP plants in district heating networks.

solid biomass is used in the paper and pulp as well as food, beverages and
tobacco industries, where required temperatures are lower (see
DOI:10.1002/er.3436 and DOI:10.1007/s12053-017-9571-y).

\subsection{Biomass Transport}

solid biomass has to be consumed locally

biogas can be upgraded and then transported via methane network


\section{Carbon dioxide capture, usage and sequestration (CCU/S)}

Carbon dioxide can be captured from industry process emissions, emissions
related to industry process heat, combined heat and power plants, and directly
from the air (DAC).

Carbon dioxide can be used as an input for methanation and Fischer-Tropsch
fuels, or it can be sequestered underground.

CO2: single node for Europe, but a transport and storage cost is added for sequestered CO2. Optionally: nodal, with CO2 transport via pipelines.

\section{Mathematical Model Formulation}

\section{Techno-Economic Assumptions}
