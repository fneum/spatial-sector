\section{Model Overview}

\begin{figure}
    \centering
    \includegraphics[]{../graphics/multisector_figure.pdf}
\end{figure}

\begin{figure}
    \centering
    \begin{subfigure}[t]{0.49\textwidth}
        \centering
        % \caption{clustered electricity network}
        \includegraphics[width=\textwidth]{electricity-network-today-map.pdf}
    \end{subfigure}
    \begin{subfigure}[t]{0.49\textwidth}
        \centering
        % \caption{gas network}
        \includegraphics[width=\textwidth]{gas-network-today-map.pdf}
    \end{subfigure}
    \begin{subfigure}[t]{0.49\textwidth}
        \centering
        % \caption{electricity demand}
        \includegraphics[width=\textwidth]{demand-map-electricity.pdf}
    \end{subfigure}
    \begin{subfigure}[t]{0.49\textwidth}
        \centering
        % \caption{hydrogen demand}
        \includegraphics[width=\textwidth]{demand-map-H2.pdf}
    \end{subfigure}
    \begin{subfigure}[t]{0.49\textwidth}
        \centering
        % \caption{methane demand}
        \includegraphics[width=\textwidth]{demand-map-gas.pdf}
    \end{subfigure}
    \begin{subfigure}[t]{0.49\textwidth}
        \centering
        % \caption{heat demand}
        \includegraphics[width=\textwidth]{demand-map-heat.pdf}
    \end{subfigure}
    \begin{subfigure}[t]{\textwidth}
        \centering
        \includegraphics[height=0.25\textheight]{total-annual-demand.pdf}
        \includegraphics[height=0.25\textheight]{ts-demand.pdf}
    \end{subfigure}
    \caption{Final energy demand diversity.}
    \label{fig:demand-space}
\end{figure}

Not all of the sectors are at the full nodal resolution, and some demand for
some sectors is distributed to nodes using heuristics that need to be corrected.
Some networks are copper-plated to reduce computational times.

overnight scenario

no pathway

weather year 2013

High resolution multi-sectoral approach: Developing the most advanced open source modeling system with regard to spatial, sectoral, technological and temporal resolution
- All energy infrastructures in one optimisation problem (electricity, gas, heat, industry)
- Detailed transmission grid representation in EU-wide model
- Detailed representation of demand sectors: Industry, buildings, transport
- so that the variability of demand and variable renewable supply can be represented, and so that existing grid bottlenecks are visible.

Energy sector coupling, storage and conversion is modelled to connect
electricity, heating (individual buildings, district heating and industry),
transport and gas (methane, hydrogen and carbon dioxide) in the different
sectors (buildings, transport and industry).

The spatial resolution of the model can be customised by the user. It can be set
at electricity substation level, based on administrative boundaries such as
NUTS2 or country-level, or to a custom number of nodes to enhance computational
performance. The spatial resolution of the input data varies: conventional power
plant locations are known exactly, as are large industrial facilities; wind and
solar time series are regionalised based on the underlying ERA5 reanalysis data
(around 20km by 20km); heat and transport demand at NUTS3; electricity demand
time series are at TSO control region level.

The electricity transmission
network is represented at substation level based on maps released by the
European Network of Transmission System Operators for Electricity (ENTSO-E). It
includes all existing high-voltage line capacities between substations, as well
as planned projects in the Ten Year Network Development Plan (TYNDP). In total
around 3650 substations and 6000 transmission lines are modelled.

Electricity can be converted to heat via heat pumps or resistive heaters; to
hydrogen gas, or further to methane and liquid hydrocarbons; or to work in
various demand devices. Methane can be reformed to hydrogen, and most fuels can
be used for electricity generation in turbines, fuel cells or engines.

Energy/material storage can be optimised in PyPSA-Eur-Sec including conventional
pumped hydro storage, electrochemical storage like Lithium ion batteries,
storage of gases including methane, hydrogen and carbon dioxide, storage of
liquid fuels, as well as thermal energy storage in the form of hot water both in
individual buildings and in district heating networks.

compared to electricity, heating and transport are strongly peaked
- heating is strongly seasonal but also with synoptic variations
- transport has strong daily periodicity

\section{Electricity Sector}

Electricity supply and demand follows the electricity generation and
transmission model PyPSA-Eur, except that hydrogen storage is integrated into
the hydrogen supply, demand and network, and PyPSA-Eur-Sec includes CHPs.


\subsection{Electricity Demand}

distribution electricity demand for
industry uses the geographical data from
the Hotmaps Industrial Database.

subtracts existing electrified heating from
the existing electricity demand, so that power-to-heat can be optimised
separately.
Building heating demand: nodal, distributed in each country based on population.

The remaining electricity demand for households and services is distributed
inside each country proportional to GDP and population.

hourly electricity demand for every country from ENTSO-E via OPSD interface

\subsection{Electricity Supply}

\begin{SCfigure}
    \includegraphics[width=0.7\textwidth]{powerplants.pdf}
    \caption{Powerplants.}
\end{SCfigure}

OCGT CCGT

hydro dams, PHS and run of river
- capacities are fixed at 2015 values
- inflow modelled based on runoff data from ERA5 and scaled using EIA annual hydropower generation statistics

solar photovoltaics (rooftop and utility-scale installations)

onshore and offshore wind

CHP (with gas or biomass)
- back-pressure plants with heat production proportional to electricity output

costs, lifetimes and efficiencies in SX.

powerplantmatching for existing power plants \citeS{gotzensPerformingEnergy}

For conventional electricity generators PyPSA-Eur-Sec uses the open
powerplantmatching database comprising all existing conventional electricity
generators for coal, gas, oil, nuclear and hydroelectricity. Exact locations,
fuel types, capacities, build dates and planned decommission dates are included.

Renewable generators include onshore and offshore wind, utility-scale and
rooftop solar PV, biomass (multiple feedstocks), hydroelectricity (dams,
run-of-river and pumped storage) as well as solar thermal collectors for
building heating. For countries where open data is available, the existing
capacities in each region are known. The model also allows to build new
capacities based on available land and on the weather resource. Time series for
wind, solar and precipitation availability are derived from the ERA5 weather
reanalysis dataset. They are converted to power plant data using the open source
atlite library. Hydroelectric inflow time series are derived from run-off data.

\subsection{Electricity Storage}

battery
- distinguish costs for home battery storage and inverter from utility-scale battery costs.

pumped-hydro (PHS)
- with fixed capacities

hydrogen storage

\subsection{Electricity Transport}

\begin{figure}
    \includegraphics[width=\textwidth, center]{network.pdf}
    \label{fig:network}
    \caption{ENTSO-E transmission network}
\end{figure}

ENTSO-E
- high-voltage transmission lines at and above 220 kV
- HVDC links

Ten Year Network Development Plant (TYNDP)
- planned projections

Clustering
- clusters down the electricity transmission substations in each European country
- based on the k-means algorithm to a variable number of nodes (here: 181 regions)
- move all to 380 kV
- remove stubs
- 125\% length factor
- weighted cost takes into account fraction that is underwater

To approximate $N-1$ contingency constraints (to protect against overloading if any one
transmission line fails) lines may only be loaded up to 70\% of their nominal rating
- no dynamic line rating included

linearised power flow equations are considered (KCL and KVL) using an
efficient cycle-based formulation of KVL

distribution grid topology is not included
- only the total exchange capacity between distribution and transmission level
- rooftop PV, heat pupms, resistive heater, home batteries and electric vehicles connect to low-voltage level
- everything else connects directly to the transmission grid

\section{Transport Sector}

annual energy demands from road and rail transport, aviation and navigation/shipping form JRC IDEES \citeS{IDEES}

Transport and mobility comprises light and heavy road, rail, shipping and
aviation transport. For road and rail, electrification, fuel cell vehicles and
liquid fuels are available. For shipping and aviation, only hydrogen and liquid
fuels are available. Battery electric vehicles for passenger transport can be
enables with demand response as well as vehicle-to-grid capabilities.

\subsection{Land Transport}

Land transport is separated by energy carrier (fossil, hydrogen fuel cell
electric vehicle, and electric vehicle), but still needs to be separated into
heavy and light vehicles (the data is there, just not the code yet).

Split between 85\% electric and 15\% FCEV (not differentiated between light and heavy).

for electrified land transport, country-specific factors are computed by comparing
current car FEC per km (average for Europe 0.7 kWh/km) to the 0.18 kWh/km assumed for
battery-to-wheels efficiency in electric vehicles

weekly profiles for km driven provided by the Germand Federal Highway Research Institute (BASt)
are used to obtain hourly time series for each European country taking into account local time
Furthermore, a temperature dependence is included in the time series to account for heating/cooling
demand in transport. For temperatures below/above 15C/20C temperature coefficients of .98\%/C and 0.63\%/C are assumed \cite{brown2018}

assume
- battery of 50 kWh
- charging capacity of 11 kW
- 90\% charging efficiency

V2G
- half of BEV can shift dis-/charging time to facilitate system operation
- these BEVs can also provide vehicle-to-grid (V2G) services

The BEV state of charge is forced to be higher than 75\% at 7am every day
to ensure that the batteries are close to full in the morning peak usage.
This also restricts BEV demand to be shifted within a day and prevent EV
batteries from becoming seasonal storage

The percentage of BEV connected to the grid at any time is inversely proportional
to the transport demand profile, which translates into an average/minimum availability of 80\%/62\%.
(conservative compared to literature: average parking times of European fleet of vehicles at 92\%

battery cost of BEV is not included in the model since it is assumed that EV owners buy
them to satisfy their mobility needs

\subsection{Aviation}

kerosene from Fischer-Tropsch (synthetic or fossil origin)

\subsection{Shipping}

hydrogen liquefaction costs for hydrogen demand in shipping

Assuming liquid hydrogen.

Should switch to methanol or ammonia

`` This estimate is based on an estimate of international navigation emissions
based on bunker fuels sold in the EU, comparable to the memo item as reported in
the EU greenhouse gas inventory reported to the UNFCCC.''


\section{Industry Sector}

\citeS{graichenKlimaneutraleIndustrie,rootzenExploringLimits2013}

Industry demand is split into a dozen different sectors, with the major ones
being iron and steel, cement and basic chemicals. The location of existing
industrial facilities is based on the Horizon-2020-funded project hotmaps.
Demand for materials can either be met using existing conventional routes, such
as blast furnaces for iron and rotary kilns for cement, or with new processes,
such as hydrogen direct reduction for iron or kilns equipped with CCS and/or
oxyfuel for cement. Basic chemicals can use green hydrogen as a feedstock rather
than fossil fuels, for example for ammonia production, or using Fischer-Tropsch
naphtha in steam crackers for High Value Chemicals.

\begin{SCfigure}
    \includegraphics[width=0.7\textwidth]{fec_industry_today_tomorrow.pdf}
    \caption{Final consumption of energy and non-energy feedstocks in industry today (left bar) and
    our future scenario in 2050 (right bar)}
\end{SCfigure}

\begin{SCfigure}
    \includegraphics[width=0.7\textwidth]{process-emissions.pdf}
    \caption{Process emissions in industry today (left bar) and in 2050 (right bar)}
\end{SCfigure}

\begin{SCfigure}
    \includegraphics[width=0.7\textwidth]{hotmaps.pdf}
    \caption{Distribution of industries.}
\end{SCfigure}

\begin{table}[t]
    \centering
    \setlength{\tabcolsep}{6pt}
    \begin{tabular}{@{} p{5cm}r @{}}
      \toprule
      Material & Production [\si{\mega\tonne\per\year}] \\
      \midrule
      steel & 174 \\
      cement & 186 \\
      glass & 44 \\
      ceramics and other NMM & 360 \\
      ammonia & 18 \\
      other basic chemicals & 74 \\
      other chemicals & 28 \\
      pulp & 39 \\
      paper & 94 \\ \bottomrule
    \end{tabular}
    \caption{Industrial production for main products in 2015.}
    \label{tab:industryproduction}
  \end{table}


emissions
- energy-related
- process-related

For process-related emissions need alternative manufacturing process
or higher rates of recycling such that less virgin material is needed

largest sectors (according to FEC)
- iron and steel
- chemicals
- non-metallic mineral products
- pulp paper printing
- food beverages tobacco
- non-ferrous metals

Demand

For the industry energy demand, we assume the same
production of materials as today, but with process switching, fuel switching to
low-emission alternatives as well as carbon capture for use or sequestration
(CCU/S).

Process switching includes, for example, moving primary production of
steel from blast furnace reduction to direct reduction with hydrogen, or
switching to an oxyfuel process in cement manufacture.

Fuel switching means
replacing mechanical processes and process heat applications that use fossil
fuels with low-emission alternatives such as electricity, biomass or synthetic
fuels such as hydrogen or methane.

Feedstocks for the chemicals industry are
also converted to non-fossil alternatives.

To determine where fuel switching can take place we use the breakdown of energy
usage for each process within each sector provided by the JRC-IDEES database
\citeS{IDEES}.

Where there are fossil and electrified alternatives for the same
process (e.g. in glass manufacture or drying in industry XX) we assume that the
process is completely electrified.

Where process heat is required (steam
process?) our approach depends on the temperature required
\citeS{naeglerQuantificationEuropean2015,rehfeldtBottomupEstimation2018}. Processes that require temperatures below
\SI{500}{\celsius} are supplied with solid biomass, since we assume that residues and
wastes are not suitable for high-temperature applications. We see solid biomass
use primarily in the pulp and paper industry, where it is already widespread,
and in food, beverages and tobacco, where it replaces natural gas. Industries
which require high temperatures (above \SI{500}{\celsius}), such as metals, chemicals
and non-metalic minerals are either electified where processes already exist, or
the heat is provided with synthetic methane.

what is currently supplied with electricity (lighting, air compressors, motor drives, fans, pumps)
and low-enthalpy heat demand are directly added to electricity and heat buses

Industry heat demand can be supplied via DH or is added to services heat demand.
Can be supplied by other processes producing heat as byproduct (DAC, FT)

For EU-28 plus NO, CH, IL Rehfeldt estimated that from 2015 industrial heat demand
45\% is above \SI{500}{\celsius}, 30\% \SIrange{100}{500}{\celsius}, 25\% below \SI{100}{\celsius}

\citeS{naeglerQuantificationEuropean2015} similarly:
48\% is above \SI{400}{\celsius}, 27\% \SIrange{100}{400}{\celsius}, 25\% below \SI{100}{\celsius}

Because of high share of high-temperature process heat demand
- no geothermal
- no solar thermal supply

process heat supplied by methane, biomass, electricity depending on sector

Based on materials demand from JRC-IDEES and other sources such as the USGS for
ammonia.

Industry is split into many sectors, including iron and steel, ammonia, other
basic chemicals, cement, non-metalic minerals, alumuninium, other non-ferrous
metals, pulp, paper and printing, food, beverages and tobacco, and other more
minor sectors.

Inside each country the industrial demand is distributed using the Hotmaps
Industrial Database.

Hotmaps open database includes cement, basic chemicals, glass, iron and steel, non-ferrous metals,
non-metallic minerals, paper, refineries. Enables regional analyses, calculation of
site-specific energy demand, waste heat potentials, emissions, market shares,
process-specific evaluations

calculate energy demands and process emissions per unit of material based on JRC-IDEES

material output per country from JRC IDEES, ammonia production statistics,
Eurostat energy balances, national statistics from Switzerland to calculate total energy demand
and process emissions by sector

material output assumed to stay constant (exception recycling)

Supply

Process switching (e.g. from blast furnaces to direct reduction and electric arc
furnaces for steel) is defined exogenously.

Fuel switching for process heat is mostly also done exogenously.

Solid biomass is used for up to 500 Celsius, mostly in paper and pulp and food
and beverages.

Higher temperatures are met with methane.

\citeS{neuwirthFuturePotential2022}

\subsection{Iron and Steel}

\citeS{toktarovaInteractionElectrified2022, mandovaPossibilitiesCO22018, suopajarviUseBiomass2018,voglPhasingOut2021, bhaskarDecarbonizingPrimary}

70\% from scrap, rest from direct reduction with 1.7 MWh H2 / t steel + electric arc (process emissions 0.03 t COs / t steel)

Two routes today to manufacture steel in Europe
- primary/integrated steelworks (60\% of steel production)
- secondary/electric arc furnaces (40\%) \citeS{lechtenbohmerDecarbonisingEnergy2016}

Primary - Integrated steelworks
- use blast furnaces in which coke is used to reduce iron ore into molten iron
\begin{equation}
    \ce{FE2O3 + 3CO -> 2Fe + 3\co}
\end{equation}
\begin{equation}
    \ce{FeO + 3CO -> Fe + 3\co}
\end{equation}
- this is then converted to steel
- The primary route implies large process emissions (0.22 t\co/t steel)

Secondary - electric arc furnaces (EAF)
- use EAF to melt scrap metal
- this limits \co emissions to burning of graphite electrodes \citeS{Friedrichsen_2018}
- 0.03 t\co/t steel

DRI - Direct Reduced Iron
\begin{equation}
    \ce{FE2O3+H2 -> 2FeO + H2O}
\end{equation}
\begin{equation}
    \ce{FeO + H2 -> Fe + H2O}
\end{equation}
- this is processed in EAF and circumvents associated process emissions
- today, DRI uses methane as reduction agent but we assume this to be substituted with hydrogen
- needs 1.7 MWh H2 / t steel \citeS{voglAssessmentHydrogen2018} and 0.322 MWh elec / t steel \url{https://ssabwebsitecdn.azureedge.net/-/media/hybrit/files/hybrit_brochure.pdf}

share of steel produced by hydrogen-based DRI + EAF is exogenous: 30\%
integrated steelworks: 0\%
scrap + EAF: 70\%

replace integrated steelworks with DRI + EAF

NB: DRI already done with hydrogen in Trinidad and in Abu Dhabi (where \co from
SMR is captured and used for enhanced oil recovery) H2 Future, HYBRIT, SALCOS

\citeS{circular_economy} circular economy practices have potential of expanding
the share of secondary route to 85\% just by increasing the amount and quality of scrap metal

process emissions from lime (added to remove impurity?), graphite anodes AND
from carbon added to get steel alloy  \citeS{voglAssessmentHydrogen2018}

EAF not good for flat steel in automotive

NB: derived gases not included in model, since take from blast furnace top
gases, originating in coke

Why not add CCS to existing plants? Hard for some reason \citeS{kuramochiComparativeAssessment2012}.

For the remaining subprocesses in this sector:
- methane as energy source for smelting process
- activities associated with furnaces, refining and rolling, product finishing are electrified

\subsection{Chemicals Industry}

\citeS{elserTakingEuropean, FuturePetrochemicals, nicholsonManufacturingEnergy2021, meysAchievingNetzero2021, thunmanCircularUse2019, introzziDECHEMAGesellschaft, meysCircularEconomy2020, guWastePlastics2017, boulamantiProductionCosts2017}

wide range of diverse industries ranging from:
- basic organics compounds (olefins, alcohols, aromatics)
- basic inorcanic compounds (ammonia, chlorine)
- polymers (plastics)
- end-user products (cosmetics, pharmaceutics)

chemicals industry consumes lots of fossil-fuel based feedstocks \citeS{leviMappingGlobal2018}

basic chemicals: HVC (high-value chemicals), chlorine, methanol and ammonia

Recycling!
- specify reuse, primary production, and mechanical and chemical recycling
fraction of platics

\citeS{circular_economy, kullmannCombiningWorlds2021,kullmannImpactsMaterial,kullmannValueRecycling,carolinaliljenstromDataSeparate, fuhrPlastikatlasDaten2019,conversioMaterialFlow2020}

synthetic naphtha for primary production

Ammonia production data is taken from the USGS

Natural gas dominates in Europe as source for hydrogen for Haber-Bosch process.
Replace with electrolytic/clean hydrogen.


ammonia production from USGS (United States Geological Survey) statistics

Assumptions for existing ammonia energy demand from XX and for electrolytic
process from XX.

hydrogen can be combined with Nitrogen to obtain ammonia in the Haber-Bosch process \citeS{leviMappingGlobal2018}
\begin{equation}
    \ce{N2 + 3H2 -> 2NH3}
\end{equation}
Currently, H2 for ammonia industry in Europe is from SMR, in the model can be SMR, SMR CC, electrolysers
\co often captured from SMR for urea production (20\% of fertilizer in Europe,
but 50-60\% in China and India) and food and beverages
with electrolytic-H2 we have 6.5 MWh \ce{H2} / t \ce{NH3} and 1.17 MWh elec / t \ce{NH3} (Wang, 2018 Joule)

chlorine, aromatics, olefins

ammonia is subtracted from basic chemicals in IDEES database.


ALL of liquid hydrocarbon feedstock from FT naphtha.
- LPG, diesel oil, residual fuel oil

We assume that all of carbon eventually finds its way from product into
atmosphere. But this is wierd, since there is no release from landfill, plastics
are not biodegradable. If they are burned as waste, we should use a
waste-to-energy plant, which we currently don't have in the model.

\citeS{lechtenbohmerDecarbonisingEnergy2016} has 14.8 MWh FT-naphtha per ton of HVC, and 2.7 MWh per
ton for processing; 3.1 t\co/tHVC are captured from flue gases.

TODO: We have much smaller emissions, are we accounting for heating as well as
process emission? I guess with methane we do...

share of synthetic vs fossil-fuel based methane and naphtha is endogenous

transformation of energy-consuming processes:
- FEC in steam processing is converted to methane since requires temperature above \SI{500}{\celsius} \citeS{rehfeldtBottomupEstimation2018}
- remaining processes are electrified using the current efficiency  of microwave for high-enthalpy heat processing,
  electric furnaces, electric process cooling and electric generic processes

process emissions from feedstock in the chemical industry represent 0.369 t\co / t ethylene eq.
- we consider process emissions for all the material output
- conservative, because it assumes that all plastic-embedded \co will eventually be released into the atmosphere
- plastic disposal in landfilling will avoid, or at least delay, associated \co emissions

\subsection{Non-metallic Mineral Products}

Cement

\citeS{fennellDecarbonizingCement2021}

waste and solid biomass; capture of \co emissions

cement manufacturing involves large process and energy emissions

calcination of limestone to chemically reactive calcium oxide (known as lime)
involves process emissions of 0.54 t \ce{\co} / t cement
\begin{equation}
    \ce{CaCO3 -> CaO + \co}
\end{equation}

mitigation strategies (new raw materials, recovering unused cement from concrete at end of life \url{https://ec.europa.eu/clima/news/commission-calls-climate-neutral-europe-2050_en})
are at a very early development stage and have not been considered

Keep current share of biomass; replace fossil fuels for process heat with
synthetic methane (exact share determined by the model)

assume that FEC (except electricity, low-temperature heat and biomass consumption)
is supplied by methane which can deliver the required high-temperature heat

NB: CH4 burns differently in kiln, not straightforward, see IEEE paper
\citeS{akhtarCoalNatural2013}

process emissions are captured and, for net-zero emission scenarios,
they need to be compensated by negative emissions

Do post-combustion carbon capture on emissions and account for heat and
electricity for capture with aqueous amine solution. The classic amine process
based on monoethanolamine (MEA) Some suggest need CHP for this. DEA input
assumptions in 2030 are 0.72 MWh/t\co heat and 0.022 MWh/t\co electricity
per outputed tonne of \co assuming a capture rate of 90\%. Additional
electricity for \co compression to 150 bar and dehydration is 0.085
MWh/t\co.

Alternative: use oxyfuel so you get concentrated \co. No heat requirement,
electricity only used for O$_2$ air separation unit (ASU) of 0.08 MWh/t\co
and 0.17 MWh/t\co for post treatment (unlike post-combustion capture, output
is impure, so have to remove O$_2$ and N$_2$ by cryogenic distillation). Can
avoid ASU by using O$_2$ from electrolysis. Electricity use higher than
post-combustion, but don't have big heat requirement.

Beware biomass share in IDEES-Industry is higher for EU28 (2920 ktoe in 2015)
than in IDEES-EnergyBalance (906 ktoe). This is because IDEES is putting waste
(including non-renewable waste) together with biomass under biomass.

\citeS{kuramochiComparativeAssessment2012}: The major difference between centralized power plants and
industrial plants such as cement plants is that the former have large quantities
of low-grade heat that can be used for solvent regeneration, whereas the latter
generally do not [42].


Can use oxyfuel for cement to make CCS easier. Already being done, can retrofit
to older plant
\url{https://engineered.thyssenkrupp.com/en/oxyfuel-a-climate-neutral-cement-production-is-getting-closer/}

NEED oxyfuel, otherwise CCS requirements are too high.

TODO: How much O$_2$ from electrolysis? How much O$_2$ required by cement per
ton? For current combustion materials, around 0.17 tO$_2$/tCement => 29
MtO$_2$/a needed, but have 672 MtO$_2$/a from electrolysis; H2 for steel and
ammonia probably is enough to provide O$_2$ for oxyfuel cement (see private.org
- have plenty of O2 from electrolysis)

NB: for assessments of electricity for oxyfuel cement, deduct unneeded ASU (air
separation unit via cryogenic distillation) since don't need 200-300 kWh/tO$_2$
(consistent with 50-60 EUR/tO$_2$ price).

NB: cement production is used to using solid fuels. Natural gas needs some
aadjustment of firing mechanisms
\citeS{akhtarCoalNatural2013}.

Some people push electric kilns for heat, but considered less mature than gas
and certainly less than solid

CemZero project
\url{https://group.vattenfall.com/what-we-do/roadmap-to-fossil-freedom/industry-decarbonisation/cementa},
apparently poo-pooed here \url{https://www.cementa.se/sv/cemzero}.


\citeS{lechtenbohmerDecarbonisingEnergy2016} has electrification of cement with 0.9~MWh\el/tCLinker
(12\% efficiency improvement in thermal demand compared to 2010)



Ceramics

complete electrification because many already electrified processes:
- microwave drying and sintering of raw materials,
- electric kilns for primary production processes
- electric furnaces for the product finishing
- FEC 0.44 MWh/t ceramics
- process emissions: 0.03 t\co/t ceramic

\cite{furszyferdelrioDecarbonizingCeramics2022a}

Glass

Electrify everything.
- electric melting tanks
- electric annealing
- electricity demand 2.07 MWh/t of glass \citeS{lechtenbohmerDecarbonisingEnergy2016}

\cite{furszyferdelrioDecarbonizingGlass2022}


\citeS{lechtenbohmerDecarbonisingEnergy2016} also has big efficiency improvement with 0.85~MWh\el/tGlass.

\subsection{Non-ferrous Metals}

includes
- base metals (aluminium, copper, lead, zink)
- precious metals (gold, silver)
- technology metals (molybdenum, cobalt, silicon)

Aluminium

80\% recycling, for rest: methane for high-enthalpy heat (bauxite to alumina) followed by electrolysis (process emissions 1.5 t \co / t Al)

more than half of the FEC of this sector

Two alternative production routes today to manufacture aluminium in Europe today
- primary route: 40\%
- secondary route: 60\%
- exogenous: increase by 2050 to 80\% \citeS{Friedrichsen_2018} and \url{https://ec.europa.eu/clima/news/commission-calls-climate-neutral-europe-2050_en}

Primary route:
- two energy-intensive processes
- production of alumina from bauxite (aluminium ore)
- electrolysis to transfrom alumina to aluminium via the Hall-H\'{e}roult process
\begin{equation}
    \ce{2Al2O3 + 3C -> 4Al + 3\co}
\end{equation}
- primary route requires high-enthalpy heat to produce alumina - supplied by methane
- 1.5 t\co/t aluminium
- inert anodes might be commercially available in 2030 avoiding processe emissions \citeS{Friedrichsen_2018} but not considered

Secondary route:
- scrap aluminium is remelted
- energy demand is 10\% of primary route and process emissions are avoided
- assuming all subprocesses in this route electrified: 1.7 MWh elec / t aluminium

Other non-ferrous metals, electrification of entire manufacturing process is assumed
- 3.2 MWh / t lead equivalent

\subsection{Other Industry Subsectors}

energy demands and \co emissions for the agriculture, forestry and fishing
sector

pulp, paper, printing:

Already high share of biomass. Keep and add biomass for paper production, since
temperatures required are low.

food, beverages, tobacco \citeS{sovacoolDecarbonizingFood2021}

textiles and leather

machinery equipment


transport equipment

wood and wood products

otherwise

low- and mid-temperature process heat in these industries is assumed to be supplied by biomass
while the remaining processes are electrified

Comparison

\begin{itemize}
    \item synergies paper (still see benefit of transmission, but MUCH bigger electrolysis with industry/aviation/shipping) \citeS{brownSynergiesSector2018}
    \item JRC papers (Herib Blanco etc.) \citeS{blancoPotentialHydrogen2018,blancoPotentialPowertoMethane2018}
    \item FZJ steel paper
    \item PAC (uses solid biomass for non-energy requirements in chemicals industry, only 270 TWh in 2050, because of circular economy; phases out waste incineration) \citeS{caneurope/eebBuildingParis}
    \item LTS from commission \citeS{in-depth_2018}
    \item Material Economics reports \citeS{circular_economy,me2019}
\end{itemize}

\section{Heating Sector}

\subsection{Heating Demand}

building heating is aggregated at each node both for individual buildings and district heating.
Demand time series are based on a degree-day representation and improvements to
the thermal envelopes of buildings to reduce demand can also be endogenously
optimised. Supply options in individual buildings include gas and oil boilers,
air- and ground-sourced heat pumps, resistive heaters, solar thermal collectors
and small thermal energy storage. At district heating level more options are
available: combined heat and power (CHP) plants consuming gas, coal and biomass
with and without CCS, large-scale heat pumps, pit thermal storage, gas and oil
boilers, resistive heating and fuel cells.

\citeS{kavvadiasDecarbonisingEU2019,fleiterBaselineScenario2017}

all space and water heating in the residential and services sectors is considered

Heat demand is split into:

urban central: large-scale district heating networks in urban areas with dense
heat demand

residential/services urban decentral: heating for individual buildings in urban
areas

residential/services rural: heating for individual buildings in rural areas,
agriculture heat uses

Building heating demand: nodal, distributed in each country based on population.

annual heat demands per country retrieved from JRC-IDEES

converted to daily heat demand based on population weighted Heating Degree Day (HDD)
- heating proportional to difference between ambient temperature (from ERA5) and threshold temperature of \SI{15}{\celsius}
- daily space heat is distributed to the hours of the day following BDEW heat demand profiles
- these differ for weekdays and weekends and between residential and services demand
- hot water demand constant throughout the day

total heating demand X TWhth/a

pronounced seasonal variation

For each country
- heat demand split between low and high population density areas
- urban areas can be supplied with district heating (DH) systems
- 60\% of urban heat demand covered by DH
- 15\% pauschal district heating system losses

Cooling demand
- supplied by electricity and included in electricity demand

space heating demand reduced by building retrofitting
- exogenously fixed at 29\% reduction

\subsection{Heating Supply}

Oil and gas boilers

Heat pumps

Either air-to-water or ground-to-water heat pumps are implemented.

They have coefficient of performance (COP) based on either the external air or the soil hourly temperature.

Ground-source heat pumps are only allowed in rural areas because of space constraints.

Only air-source heat pumps are allowed in urban areas. This is a conservative
assumption, since there are many possible sources of low-temperature heat that
could be tapped in cities (waste water, rivers, lakes, seas, etc.).

Resistive heaters

Large Combined Heat and Power (CHP) plants

\citeS{dahlCostSensitivity2019, madedduCOReduction2020}

PyPSA-Eur-Sec includes CHP plants fuelled by methane, hydrogen and solid biomass from waste and residues.

Hydrogen CHPs are fuel cells.

Methane and biomass CHPs are based on back pressure plants operating with a
fixed ratio of electricity to heat output. The methane CHP is modelled on the
Danish Energy Agency (DEA) “Gas turbine simple cycle (large)” while the solid
biomass CHP is based on the DEA’s “09b Wood Pellets Medium”.

The efficiencies of each are given on the back pressure line, where the back
pressure coefficient $c_b$ is the electricity output divided by the heat output.
The plants are not allowed to deviate from the back pressure line and are
implement with a fixed ratio of heat to electricity output.

Micro-CHP for individual buildings

Waste heat from Fuel Cells, Methanation and Fischer-Tropsch plants

Solar thermal collectors

District heating!

no individual solar thermal?

CHP with CCS, mostly solid biomass

waste heat from industry?

for efficiency improvements, use Lisa thermal envelope retrofitting, currently
determined exogenously with 29\% reduction compared to 2015 (check) XX cite
Lisa's paper \citeS{zeyenMitigatingHeat2021,lombardiWeatherinducedVariability2022}

urban central (DH)
- central air-sourced heat pumps
- heat resistors
- gas boilers
- solar thermal collectors
- CHP units using methane or solid biomass (back pressure)
- waste heat from fuel cells, methanation, Fischer-Tropsch

decentral
- individual ground- and air-sourced heat pumps
- heat resistors
- gas boilers

only air-sourced heat pumps is conservative assumption since many possible sources of
low-temperature heat in cities (wastewater, rivers, lakes, seas, ground)

Coefficient of performance (COP)
- depends on ambient or ground temparature
- difference between source and sink temperatures

\begin{equation}
    \Delta T = T_{sink} - T_{source}
\end{equation}

Air-sourced heat pumps (ASHP)
\begin{equation}
    COP = 6.81 + 0.121 \Delta T + 0.000630 \Delta T^2
\end{equation}

Ground-sourced heat pumps (GSHP)
\begin{equation}
    COP = 8.77 + 0.150 \Delta T + 0.000734 \Delta T^2
\end{equation}

Sink water temperature $T_{sink}$ is \SI{55}{\celsius}

Source temperature from ERA5

\subsection{Heat Storage}

Thermal energy storage (TES) using hot water tanks
- Small water tanks for decentral applications
- Big water pit storage for district heating.

thermal energy density 46.8 kWhth/m3 corresponding to temperature difference of 40 K

decay of thermal energy $1-\exp(-\sfrac{1}{24\tau})$

with time constant of $\tau=180$ days for central TES and $\tau=3$ days for individual TES

charging and discharging efficiencies are 90\% due to pipe losses

\section{Renewables Potentials}

\begin{figure}
    \centering
    % \makebox[\textwidth][c]{
        \begin{subfigure}[t]{0.49\textwidth}
            \centering
        \includegraphics[width=\textwidth]{windspeeds.png}
    \end{subfigure}
    \begin{subfigure}[t]{0.49\textwidth}
        \centering
        \includegraphics[width=\textwidth]{irradiation.png}
    \end{subfigure}
    \begin{subfigure}[t]{0.49\textwidth}
        \centering
        \includegraphics[width=\textwidth]{temperatures.png}
    \end{subfigure}
    \begin{subfigure}[t]{0.49\textwidth}
        \centering
        \includegraphics[width=\textwidth]{runoff.png}
    \end{subfigure}
    % }
    \caption{weather data}
    \label{fig:weather-data}
\end{figure}

\begin{figure}
    \centering
    % \makebox[\textwidth][c]{
        \begin{subfigure}[t]{0.49\textwidth}
            \centering
        \includegraphics[width=\textwidth]{cf-raw-ts-onshore wind.pdf}
    \end{subfigure}
    \begin{subfigure}[t]{0.49\textwidth}
        \centering
        \includegraphics[width=\textwidth]{cf-raw-ts-offshore wind.pdf}
    \end{subfigure}
    \begin{subfigure}[t]{0.49\textwidth}
        \centering
        \includegraphics[width=\textwidth]{cf-raw-ts-solar PV.pdf}
    \end{subfigure}
    \begin{subfigure}[t]{0.49\textwidth}
        \centering
        \includegraphics[width=\textwidth]{cf-raw-ts-run of river.pdf}
    \end{subfigure}
    \begin{subfigure}[t]{0.49\textwidth}
        \centering
        \includegraphics[width=\textwidth]{cop-ts-air-sourced heat pump.pdf}
    \end{subfigure}
    \begin{subfigure}[t]{0.49\textwidth}
        \centering
        \includegraphics[width=\textwidth]{cop-ts-ground-sourced heat pump.pdf}
    \end{subfigure}
    % }
    \caption{weather data}
    \label{fig:weather-data}
\end{figure}

\begin{figure}
    \centering
    % \makebox[\textwidth][c]{
    \begin{subfigure}[t]{0.49\textwidth}
        \centering
        \includegraphics[width=\textwidth]{cf-onwind.pdf}
    \end{subfigure}
    \begin{subfigure}[t]{0.49\textwidth}
        \centering
        \includegraphics[width=\textwidth]{cf-offwind-dc.pdf}
    \end{subfigure}
    \begin{subfigure}[t]{0.49\textwidth}
        \centering
        \includegraphics[width=\textwidth]{cf-solar.pdf}
    \end{subfigure}
    \begin{subfigure}[t]{0.49\textwidth}
        \centering
        \includegraphics[width=\textwidth]{cf-solar thermal.pdf}
    \end{subfigure}
    \begin{subfigure}[t]{0.49\textwidth}
        \centering
        \includegraphics[width=\textwidth]{cf-air-sourced heat pump.pdf}
    \end{subfigure}
    \begin{subfigure}[t]{0.49\textwidth}
        \centering
        \includegraphics[width=\textwidth]{cf-ground-sourced heat pump.pdf}
    \end{subfigure}
    % }
    \caption{weather data}
    \label{fig:weather-data}
\end{figure}

\begin{figure}
    \centering
    % \makebox[\textwidth][c]{
    \begin{subfigure}[t]{0.49\textwidth}
        \centering
        \caption{solar}
        \includegraphics[width=\textwidth]{solar-energy-density.pdf}
    \end{subfigure}
    \begin{subfigure}[t]{0.49\textwidth}
        \centering
        \caption{wind}
        \includegraphics[width=\textwidth]{wind-energy-density.pdf}
    \end{subfigure}
    % }
    \caption{Available energy density}
    \label{fig:energy-density}
\end{figure}

\newgeometry{margin=2cm}
\begin{landscape}

\begin{figure}
    \centering
    % \makebox[\textwidth][c]{
    \begin{subfigure}[t]{0.5\textwidth}
            \centering
        \caption{solar land eligibility}
        \includegraphics[width=\textwidth]{eligibility-solar-250-20.pdf}
    \end{subfigure}
    \begin{subfigure}[t]{0.5\textwidth}
        \centering
        \caption{onshore wind land eligibility}
        \includegraphics[width=\textwidth]{eligibility-onwind-250-20.pdf}
    \end{subfigure}
    \begin{subfigure}[t]{0.5\textwidth}
        \centering
        \caption{offshore wind land eligibility}
        \includegraphics[width=\textwidth]{eligibility-offwind-250-20.pdf}
    \end{subfigure}
    %}
    \caption{Land eligibility}
    \label{fig:eligibility}
\end{figure}


\end{landscape}
\restoregeometry

\subsection{Onshore Wind Potentials}

limited potential:

available land in every region

according to Corine Land Cover (CLC) database (Table which ones are allowed)

excluding Natura 2000 protected areas



X MW/sqkm conversion factor

20\% of the available land define potential

\cite{mckennaHighresolutionLargescale2022,Ryberg2018}

\subsection{Offshore Wind Potentials}

Offshore:

- bathymetry dataset GEBCO
- areas with sea depth below 50 m (no floating offshore wind)
- near-shore (distance less than 30km) AC connection, far-shore, DC connected
- costs include AC-DC conversion

floating
\citeS{lerchSensitivityAnalysis2018,lauraLifecycleCost2014,myhrLevelisedCost2014,kauscheFloatingOffshore2018,castro-santosEconomicFeasibility2016}

\subsection{Solar Potentials}

utility PV

X MW/sqkm

9\% of the available land define potential

Installable potentials for rooftop PV are included with an assumption of 1 kWp
per person (0.1 kW/m2 and 10 m2/person)

\citeS{bodisHighresolutionGeospatial2019}

Solar thermal
- from reanalysis ERA5

\section{Renewables Time Series}

\subsection{Wind Time Series}

capacity factor time series modelled by converting wind speeds from ERA5 reanalysis into wind
generation, following the methodology described in

ERA5: hourly resolution, 30 km spatial resolution

atlite: open-source tool

capacity layout proportional to mean capacity factors is assumed for every region


\subsection{Solar Time Series}

from satellite irradiance data (SARAH-2)

\begin{table}
    \caption{Land types considered suitable for every technology from Corine Land Cover database}
    \small
    \begin{tabularx}{\textwidth}{lX}
        \toprule
        Solar-PV & artificial surfaces (1-11), agriculture land except for those
        areas already occupied by agriculture with significant natural
        vegetation and agro-forestry areas (12-20), natural grasslands (26), bare rocks (31),
        sparsely vegetated areas (32) \\ \midrule
        Onshore wind & agriculture areas (12-22), forests (23-25), scrubs and herbaceous vegetation associations (26-29), bare rocks (31), sparsely vegetated areas (32) \\ \midrule
        Offshore wind & sea and ocean (44) \\ \bottomrule
    \end{tabularx}
\end{table}

\subsection{Heat Pump Time Series}

\citeS{nouvelEuropeanMapping2015, staffellReviewDomestic2012}

\subsection{Hydro Time Series}

\section{Hydrogen}

\subsection{Hydrogen Demand}

Stationary fuel cell CHP.
- to convert electricity to hydrogen (and heat as byproduct)
- to balance renewable fluctuations

Transport applications (shipping, heavy-duty transport)
- exogenously fixed

Industry (ammonia, precursor to hydrocarbons for chemicals and iron/steel direct reduced iron (DRI)).

produce synthetic methane and hydrocarbons

\subsection{Hydrogen Supply}

Steam Methane Reforming (SMR)
\begin{equation}
    \ce{ CH4 + H2O -> CO + 3H2 }
\end{equation}
combined with water-gas shift reaction
\begin{equation}
    \ce{CO + H2O -> \co + H2}
\end{equation}


SMR+CCS

electrolysers
- assuming alkaline electrolysers
- lower cost, higher installed capacity
- compared to polymer electrolyte membrane (PEM) electrolysis

Split between the different options is optimised.


\subsection{Hydrogen Transport}

retrofitting
- methane network

new pipelines
- based on routes of electricity and methane transmission networks
- capacities built if cost-effective (e.g. no suitable retrofitting option available)

\subsection{Hydrogen Storage}

\begin{SCfigure}
    \centering
    \includegraphics[width=0.7\textwidth]{caverns.pdf}
    \label{fig:caverns}
    \caption{Caverns}
\end{SCfigure}

\begin{figure}
    \centering
    \makebox[\textwidth][c]{
    \begin{subfigure}[t]{0.45\textwidth}
        \centering
        \includegraphics[width=\textwidth]{cavern-potentials-nearshore.pdf}
    \end{subfigure}
    \begin{subfigure}[t]{0.45\textwidth}
        \centering
        \includegraphics[width=\textwidth]{cavern-potentials-onshore.pdf}
    \end{subfigure}
    \begin{subfigure}[t]{0.45\textwidth}
        \centering
        \includegraphics[width=\textwidth]{cavern-potentials-offshore.pdf}
    \end{subfigure}
    }
    \caption{Cavern storage potentials}
    \label{fig:clustered-caverns}
\end{figure}

hydrogen can be stored

overground steel tanks

underground cavern storage
- potentials taken from Caglayan
- mapped onto each region
- exclude caverns in the sea or within 50 km of the shore
- to avoid environmental problems associated with brine solution disposal

\citeS{caglayanTechnicalPotential2019}

underground storage is around x times cheaper than steel tanks

\section{Methane}

\subsection{Methane Demand}

Can be used in gas boilers, in CHPs (w/wo carbon capture), in industry for high temperature heat, in OCGT/CCGT power plants.

Not used in transport because of engine slippage.

\subsection{Methane Supply}

Fossil
- LNG terminals (existing and planned)
- entry-points (Russian pipelines)
- gas extraction sites (e.g. Norway)

biogas
- upgrad to methane

Sabatier
- hydrogen and captured \co to methane
- methanation
\begin{equation}
    \ce{\co + 4H2 -> CH4 + 2H2O}
\end{equation}

HELMETH
- direct power to methane with efficient heat integration
- \citeS{gruberPowertoGasThermal2018}

The share of synthetic, bio-based and fossil-based methane is an optimisation result

\subsection{Methane Transport}

\begin{figure}
    \includegraphics[width=1\textwidth,center]{gas_network.pdf}
    \label{fig:gas-raw}
    \caption{Gas network}
\end{figure}

Scigrid Gas dataset
- direction of pipelines taken into account

single node for Europe
- assumed to be transported without cost
- since future demand is predicted to be low and no bottlenecks are expected
- nevertheless, capacities and routes of gas transmission network used as options to retrofit to hydrogen pielines

selected runs with gas network infrastructure
- abundant transport capacities even if H2 retrofitting to demand

Gas transport has recently been included in the model. Existing natural gas
networks are represented based on the Scigrid-Gas dataset, which is in turn
based on the ENTSOG map and other open sources. This network can be expanded, or
converted for hydrogen usage. Hydrogen networks can either reuse existing
methane pipelines or new hydrogen pipelines can be built. For carbon dioxide
transport, a pipeline network is also planned. The following graphics show the
existing gas network in the model, as well as an optimal newly-built hydrogen
network.

\section{Oil-based Products}

\subsection{Oil-based Product Demand}

Transport fuel in aviation (kerosene)

agriculture machinery

naphtha as a feedstock for the chemicals industry

\subsection{Oil-based Product Supply}

Fossil oil

Fischer-Tropsch process
- with hydrogen and captured \co
- produces oil and waste heat
\begin{equation}
    \ce{$n$CO + ($2n$ + 1)H2 -> C_$n$H_{2n+2} + $n$H2O}
\end{equation}


\subsection{Oil-based Product Transport}

Liquid hydrocarbons:
- single node for Europe
- since transport costs for liquids are low


\section{Biomass}

Biomass resources are available for different potential assessments (low,
medium, high) for many different feedstocks and NUTS2 resolution based on the
JRC’s ENSPRESO database. They can be used in electricity generation with and
without CCS, as well as to provide low- to medium-temperature process heat in
industry.

\subsection{Biomass Potentials}

\begin{figure}
    \centering
    \makebox[\textwidth][c]{
    \begin{subfigure}[t]{0.45\textwidth}
        \centering
        % \caption{electricity demand}
        \includegraphics[width=\textwidth]{biomass-solid biomass.pdf}
    \end{subfigure}
    \begin{subfigure}[t]{0.45\textwidth}
        \centering
        % \caption{hydrogen demand}
        \includegraphics[width=\textwidth]{biomass-biogas.pdf}
    \end{subfigure}
    \begin{subfigure}[t]{0.45\textwidth}
        \centering
        % \caption{methane demand}
        \includegraphics[width=\textwidth]{biomass-not included.pdf}
    \end{subfigure}
    }
    \caption{Biomass potentials.}
    \label{fig:biomass-potentials}
\end{figure}

Only wastes and residues from the JRC ENSPRESO biomass dataset.

nodal where biomass potential is regionally disaggregated

Use JRC ENSPRESO database to spatially disaggregate biomass potentials to
PyPSA-Eur regions based on overlaps with NUTS2 regions from ENSPRESO
(proportional to area)

Biomass supply potentials for each European country are taken from the database
of the Joint Research Centre (JRC) of the European Commission
\citeS{jrcbiomass2015}.

Only residues from agriculture and forestry as well as
biodegradable municipal waste are considered as energy feedstocks. Fuel crops
are avoided because they compete with scarce land for food production, while
primary wood as well as wood chips and pellets are avoided because of concerns
about sustainability \citeS{bentsenCarbonDebt2017}.

The JRC provides potentials in low,
medium and high availability scenarios, which depend on supply and competition
with other uses of each feedstock. In PyPSA the medium availability scenario for
2030 is used, assuming no biomass import from outside Europe.

Manure and sludge waste are available to the model as biogas (upgraded to
biomethane), while other wastes and residues are classified as solid biomass and
available for combustion in combined-heat-and-power plants (CHP) and for medium
temperature heat (lower than \SI{500}{\celsius}) applications in industry.

The technical
characteristics for the solid biomass CHP are taken from the Danish Energy
Agency Technology Database \citeS{dea2016} assumptions for a medium-sized back
pressure CHP with wood pellet feedstock; this has very similar costs and
efficiencies to CHPs with feedstocks of straw and wood chips.

In 2015 the EU28 energy
usage was 180~TWh of biogas, 1063~TWh of solid biofuels, 109~TWh renewable
municipal waste and 159~TWh of liquid biofuels. Our model contains roughly a
doubling of the biogas production from 2015 and similar amounts of solid
biofuels, but a shift from energy crops and primary wood to residues and wastes.

Since most liquid biofuels come from energy crops today, these do not appear in
PyPSA-Eur-Sec. Zappa et al (2019) \citeS{zappa100Renewable2019} uses the same JRC database
but in addition to the feedstocks we use, they also allow roundwood chips and
pellets, as well as grassy, willow and poplar energy crops.

Don't have biomass gasification, but do allow carbon capture for usage or
sequestration from bioenergy plants.

\begin{table}[t]
    \centering
    \small
    \begin{tabular}{lllr}
      \toprule
      Source & JRC code & use in model & Potential [\si{\peta\joule\per\year}] \\
      \midrule
      Energy crop: sugar beet bioethanol & MINBIOCRP21 & not used & 882.5 \\
      Energy crop: rape seed and other oil crops & MINBIORPS1 & not used & 949.3 \\
      Energy crop: starchy crops & MINBIOCRP11 & not used & 288.0 \\
      Energy crop: grassy & MINBIOCRP31 & not used & 1777.6 \\
      Energy crop: willow & MINBIOCRP41 & not used & 317.7\\
      Energy crop: poplar & MINBIOCRP41a & not used & 96.6 \\
      Wet and dry manure & MINBIOGAS1 & biogas  & 1237.6 \\
      Primary agricultural residues & MINBIOAGRW1 & solid biomass &  1120.6\\
      Roundwood fuelwood & MINBIOWOO & not used & 273.2 \\
      Roundwood chips and pellets & MINBIOWOOa & not used & 2087.1\\
      Forestry energy residue & MINBIOFSR1 & solid biomass & 1955.0 \\
      Secondary forestry residues: woodchips & MINBIOWOO1 & solid biomass & 432.2 \\
      Secondary forestry residues: sawdust & MINBIOWOO1a & solid biomass & 152.0 \\
      Forestry residues from landscape care & MINBIOFSR1a & solid biomass & 320.5\\
      Biodegradable municipal waste  & MINBIOMUN1 & solid biomass & 565.1 \\
      Biodegradable sludge & MINBIOSLU1 & biogas  & 34.9 \\ \bottomrule
    \end{tabular}
    \caption{Biomass classification from the JRC \cite{jrcbiomass2015}, usage in PyPSA-Eur-Sec, and potentials in JRC's medium availability scenario for 2030. XX Move to Appendix.}
    \label{tab:biomass}
  \end{table}


\begin{table}[t]
    \centering
    \small
    \begin{tabular}{lr}
      \toprule
      Source & Potential [\si{\twh\per\year}] \\
      \midrule
      not used & 1853 \\
      biogas & 353 \\
      solid biomass & 1263 \\ \bottomrule
    \end{tabular}
    \caption{Total biomass potentials and usage in model.}
    \label{tab:biomasstwh}
  \end{table}

\subsection{Biomass Demand}

Solid biomass provides process heat up to \SI{500}{\celsius} in industry, as well as
feeding CHP plants in district heating networks.

solid biomass is used as heat supply in the paper and pulp as well as food, beverages and
tobacco industries, where required temperatures are lower (see
\citeS{naeglerQuantificationEuropean2015} and \citeS{rehfeldtBottomupEstimation2018}).

\subsection{Biomass Transport}

solid biomass has to be consumed locally (check!)

biogas can be upgraded and then transported via methane network

neglect of transport not expected to have a large impact since most countries have biomass demand
for industry lower than internal potential


\section{Carbon dioxide capture, usage and sequestration (CCU/S)}

Carbon management becomes important in net-zero scenarios. PyPSA-Eur-Sec
includes carbon capture from electricity generators and industrial facilities,
carbon dioxide storage and transport, the usage of carbon dioxide in synthetic
hydrocarbons, as well as the ultimate sequestration of carbon dioxide
underground. The major depleted hydrocarbon fields and saline aquifers in Europe
have recently been included in the model.

carbon is tracked through system: up to 90\% of industrial emissions can be captured

direct air capture DAC

synthetic methane and liquid hydrocarbons

transport and sequestration is \SI{20}{\sieuro\per\tco}

yearly sequestration limit to 200 Mt\co / a

management of the carbon cycle

Carbon capture is needed in the model both to capture and sequester process
emissions with a fossil origin, such as those from calcination of fossil
limestone in the cement industry, as well as to provide carbon for the
production of hydrocarbons for dense transport fuels and as a chemical
feedstock, for example for the plastics industry.

Carbon dioxide can be captured from industry process emissions, emissions
related to industry process heat (methane and biomass), combined heat and power plants, and directly
from the air wit direct air capture (DAC).

DAC includes
- adsorption phase with inputs electricity and heat to assist adsorption process and regenerate adsorbent
- compression of \co prior to storage which consumes electricity and rejects heat

process emissions captured at 95\% capture rate at cost as in cement industry

SMR, CHP, biomass and methane demand in industry the model can decide w/wo carbon capture

these capacities are co-optimised

Carbon dioxide can be used as an input for methanation and Fischer-Tropsch
fuels/naphtha, or it can be sequestered underground.

Can also be sequestered:
- 20 \euro/t\co for transport and sequestration
- 2-14 USD/t\co for pipe transport IEA
- 10 USD/t\co for underground sequestration
- limit is 200 Mt (rather conservative but enough to capture process emissions)

% plot
unavoidable process emissions

\co: single node for Europe, but a transport and storage cost is added for sequestered \co. Optionally: nodal, with \co transport via pipelines.

Why net-zero target for \co? Since don't include LULUCF or non-\co (waste management and agriculture), which balance each other in EU analysis

CCU/S needed for synthetic fuels AND to deal with process emissions (from e.g.
cement)



need for feedstocks in chemicals industry and dense hydrocarbon fuels for aviation

we have capture on

- industrial process emissions, using same assumptions as the cement kiln
example described above from DEA  (90\% capture rate, 0.72 MWh/t\co heat and
0.1 MWh/t\co electricity, including compression and dehydration) - heat is
taken from urban heat buses, electricity from public grid - steam methane
reforming - biomass CHPs, using DEA assumptions for post-combustion capture on a
small CHP in 2030 (90\% capture rate, 0.72 MWh/t\co heat and 0.11 MWh/t\co
electricity, including compression and dehydration) - heat and electricity is
taken from CHP output - direct air capture (2 MWh/t\co heat, 0.47 MWh/t\co
electricity including compression and dehydration) - heat is taken from urban
heat buses (even though T is below \SI{100}{\celsius}), electricity from public grid; waste
heat from compression is used for amine washing

TODO: as of 210202: electricity and heat demand of process emission CC ignored
(only capital costs are used); also for fuel-based emissions, simple 10\% of
fuel is taken

127 Mt\co/a fossil-origin process emissions in industry (limestone for cement,
soda ash and graphite electrodes); need sequestration for this otherwise not
net-zero.

For FT-fuels demand need 0.27*1391 = 376 Mt\co/a. This comes from BECCU and
industry CCU with synfuels. Could also come from waste + CCU.


Comparison of heat/electricity requirements for capture:

\citeS{kuramochiComparativeAssessment2012} 5.2: Can integrate low-grade steam from power plants with
\co capture from cement

``In the case of post-combustion \co capture, a CHP plant will likely be built
together with the \co capture unit because this is the only way to generate
steam efficiently for \co capture solvent regeneration.''

Beware: need to take account of heat need for regenerating \co capture solvent.
In power plants, can use waste heat, but in industrial plants there is less
low-grade heat (need ~110 C). \citeS{kuramochiComparativeAssessment2012}

Need around 3.9 GJ/t\co heat for regeneration of aqueous MEA ~ 1 MWh/t\co
\citeS{zhangParametricStudy2016}. Solid biomass has 0.3 t\co/MWh\th => need 0.3 MWh heat for
each MWh\th of solid biomass => heat output is hugely reduced!!! Unless we have
oxyfuel biomass combustion...

Breyer \citeS{breyerCarbonDioxide2020}
has 1.2 MWh/t\co heat at \SI{100}{\celsius} and 0.2 MWh\el/t\co electricity for DAC.
Dittmeyer has energy requirements twice as high... DEA is closer to Dittmeyer

\citeS{fasihiTechnoeconomicAssessment2019, martin-robertsCarbonCapture2021}


\section{Mathematical Model Formulation}

\section{Other}

competition of H2 and gas flow? can gas network be completely removed?

no loss modelling in gas networks

amount of re-electrified hydrogen/gas

How many pipelines are used in both directions?

onshore wind sensitivities



\begin{figure}
    \centering
    % \makebox[\textwidth][c]{
    \begin{subfigure}[t]{\textwidth}
        \centering
        \caption{electricity}
        \includegraphics[width=\textwidth]{20211218-181-lv/elec_s_181_lv1.0__Co2L0-3H-T-H-B-I-A-solar+p3-linemaxext10_2030/ts-balance-total electricity-D-2013.pdf}
    \end{subfigure}
    \begin{subfigure}[t]{\textwidth}
        \centering
        \caption{heat}
        \includegraphics[width=\textwidth]{20211218-181-lv/elec_s_181_lv1.0__Co2L0-3H-T-H-B-I-A-solar+p3-linemaxext10_2030/ts-balance-total heat-D-2013.pdf}
    \end{subfigure}
    \begin{subfigure}[t]{\textwidth}
        \centering
        \caption{hydrogen}
        \includegraphics[width=\textwidth]{20211218-181-lv/elec_s_181_lv1.0__Co2L0-3H-T-H-B-I-A-solar+p3-linemaxext10_2030/ts-balance-H2-D-2013.pdf}
    \end{subfigure}
    %}
    \caption{Daily time series}
    \label{fig:eligibility}
\end{figure}

\begin{figure}
    \centering
    % \makebox[\textwidth][c]{
    \begin{subfigure}[t]{\textwidth}
        \centering
        \caption{methane}
        \includegraphics[width=\textwidth]{20211218-181-lv/elec_s_181_lv1.0__Co2L0-3H-T-H-B-I-A-solar+p3-linemaxext10_2030/ts-balance-gas-D-2013.pdf}
    \end{subfigure}
    \begin{subfigure}[t]{\textwidth}
        \centering
        \caption{oil}
        \includegraphics[width=\textwidth]{20211218-181-lv/elec_s_181_lv1.0__Co2L0-3H-T-H-B-I-A-solar+p3-linemaxext10_2030/ts-balance-oil-D-2013.pdf}
    \end{subfigure}
    \begin{subfigure}[t]{\textwidth}
        \centering
        \caption{stored \co}
        \includegraphics[width=\textwidth]{20211218-181-lv/elec_s_181_lv1.0__Co2L0-3H-T-H-B-I-A-solar+p3-linemaxext10_2030/ts-balance-co2 stored-D-2013.pdf}
    \end{subfigure}
    %}
    \caption{Daily time series}
    \label{fig:eligibility}
\end{figure}


\begin{figure}
    \centering
    % \makebox[\textwidth][c]{
    \begin{subfigure}[t]{\textwidth}
        \centering
        \caption{electricity}
        \includegraphics[width=\textwidth]{20211218-181-lv/elec_s_181_lv1.0__Co2L0-3H-T-H-B-I-A-solar+p3-linemaxext10_2030/ts-balance-total electricity--2013-02.pdf}
    \end{subfigure}
    \begin{subfigure}[t]{\textwidth}
        \centering
        \caption{heat}
        \includegraphics[width=\textwidth]{20211218-181-lv/elec_s_181_lv1.0__Co2L0-3H-T-H-B-I-A-solar+p3-linemaxext10_2030/ts-balance-total heat--2013-02.pdf}
    \end{subfigure}
    \begin{subfigure}[t]{\textwidth}
        \centering
        \caption{hydrogen}
        \includegraphics[width=\textwidth]{20211218-181-lv/elec_s_181_lv1.0__Co2L0-3H-T-H-B-I-A-solar+p3-linemaxext10_2030/ts-balance-H2--2013-02.pdf}
    \end{subfigure}
    %}
    \caption{hourly time series}
    \label{fig:eligibility}
\end{figure}

\begin{figure}
    \centering
    % \makebox[\textwidth][c]{
    \begin{subfigure}[t]{\textwidth}
        \centering
        \caption{methane}
        \includegraphics[width=\textwidth]{20211218-181-lv/elec_s_181_lv1.0__Co2L0-3H-T-H-B-I-A-solar+p3-linemaxext10_2030/ts-balance-gas--2013-02.pdf}
    \end{subfigure}
    \begin{subfigure}[t]{\textwidth}
        \centering
        \caption{oil}
        \includegraphics[width=\textwidth]{20211218-181-lv/elec_s_181_lv1.0__Co2L0-3H-T-H-B-I-A-solar+p3-linemaxext10_2030/ts-balance-oil--2013-02.pdf}
    \end{subfigure}
    \begin{subfigure}[t]{\textwidth}
        \centering
        \caption{stored \co}
        \includegraphics[width=\textwidth]{20211218-181-lv/elec_s_181_lv1.0__Co2L0-3H-T-H-B-I-A-solar+p3-linemaxext10_2030/ts-balance-co2 stored--2013-02.pdf}
    \end{subfigure}
    %}
    \caption{hourly time series}
    \label{fig:eligibility}
\end{figure}

\newgeometry{margin=2cm}
\begin{landscape}



\section{Techno-Economic Assumptions}

Net-zero expected in 2035-2060, so take 2030 costs to be conservative, and also because of pathway (if want in 2050, much of infrastructure is built 2030-2050). NB: Even though total cost may be high, we're mostly focused on comparative study.

\begin{small}
\begin{longtable}{p{4cm}p{4cm}rp{3cm}p{10cm}}
\toprule
                      &            &        value &                          unit &                                                                                                                                                                                                                                                                                                                               source \\
technology & parameter &              &                               &                                                                                                                                                                                                                                                                                                                                      \\
\midrule
\endfirsthead

\toprule
                      &            &        value &                          unit &                                                                                                                                                                                                                                                                                                                               source \\
technology & parameter &              &                               &                                                                                                                                                                                                                                                                                                                                      \\
\midrule
\endhead
\midrule
\multicolumn{5}{r}{{Continued on next page}} \\
\midrule
\endfoot

\bottomrule
\endlastfoot
AC grid connection (station) & overnight investment &       250.00 &               \euro/kW$_{el}$ &                                                                                                                                                                                                                                                            DEA https://ens.dk/en/our-services/projections-and-models/technology-data \\
AC grid connection (submarine) & overnight investment &     2,685.00 &                   \euro/MW/km &                                                                                                                                                                                                                                                            DEA https://ens.dk/en/our-services/projections-and-models/technology-data \\
AC grid connection (underground) & overnight investment &     1,342.00 &                   \euro/MW/km &                                                                                                                                                                                                                                                            DEA https://ens.dk/en/our-services/projections-and-models/technology-data \\
CCGT & Cb coefficient &         2.00 &  50$^{\circ}$C/100$^{\circ}$C &                                                                                                                                                                                                                                                                        Danish Energy Agency, technology\_data\_for\_el\_and\_dh.xlsx \\
                      & Cv coefficient &         0.15 &  50$^{\circ}$C/100$^{\circ}$C &                                                                                                                                                                                                                                                                        Danish Energy Agency, technology\_data\_for\_el\_and\_dh.xlsx \\
                      & FOM &         3.35 &                       \%/year &                                                                                                                                                                                                                                                                        Danish Energy Agency, technology\_data\_for\_el\_and\_dh.xlsx \\
                      & VOM &         4.20 &                     \euro/MWh &                                                                                                                                                                                                                                                                        Danish Energy Agency, technology\_data\_for\_el\_and\_dh.xlsx \\
                      & efficiency &         0.58 &                      per unit &                                                                                                                                                                                                                                                                        Danish Energy Agency, technology\_data\_for\_el\_and\_dh.xlsx \\
                      & lifetime &        25.00 &                         years &                                                                                                                                                                                                                                                                        Danish Energy Agency, technology\_data\_for\_el\_and\_dh.xlsx \\
                      & overnight investment &       830.00 &                      \euro/kW &                                                                                                                                                                                                                                                                        Danish Energy Agency, technology\_data\_for\_el\_and\_dh.xlsx \\
CHP (biomass with carbon capture) & FOM &         3.00 &                       \%/year &                                                                                                                                                                                                                                                    Danish Energy Agency, technology\_data\_for\_industrial\_process\_heat\_0002.xlsx \\
                      & carbon capture rate &         0.90 &                      per unit &                                                                                                                                                                                                                                                    Danish Energy Agency, technology\_data\_for\_industrial\_process\_heat\_0002.xlsx \\
                      & electricity input &         0.08 &            MWh/t$_{\ce{CO2}}$ &                                                                                                                                                                                                                                                    Danish Energy Agency, technology\_data\_for\_industrial\_process\_heat\_0002.xlsx \\
                      & electricity input &         0.02 &            MWh/t$_{\ce{CO2}}$ &                                                                                                                                                                                                                                                    Danish Energy Agency, technology\_data\_for\_industrial\_process\_heat\_0002.xlsx \\
                      & heat input &         0.72 &            MWh/t$_{\ce{CO2}}$ &                                                                                                                                                                                                                                                    Danish Energy Agency, technology\_data\_for\_industrial\_process\_heat\_0002.xlsx \\
                      & heat output &         0.14 &            MWh/t$_{\ce{CO2}}$ &                                                                                                                                                                                                                                                    Danish Energy Agency, technology\_data\_for\_industrial\_process\_heat\_0002.xlsx \\
                      & heat output &         0.72 &            MWh/t$_{\ce{CO2}}$ &                                                                                                                                                                                                                                                    Danish Energy Agency, technology\_data\_for\_industrial\_process\_heat\_0002.xlsx \\
                      & lifetime &        25.00 &                         years &                                                                                                                                                                                                                                                    Danish Energy Agency, technology\_data\_for\_industrial\_process\_heat\_0002.xlsx \\
                      & overnight investment & 2,700,000.00 &      \euro/(t$_{\ce{CO2}}$/h) &                                                                                                                                                                                                                                                    Danish Energy Agency, technology\_data\_for\_industrial\_process\_heat\_0002.xlsx \\
CHP (biomass) & Cb coefficient &         0.46 &                     40°C/80°C &                                                                                                                                                                                                                                                                        Danish Energy Agency, technology\_data\_for\_el\_and\_dh.xlsx \\
                      & Cv coefficient &         1.00 &                     40°C/80°C &                                                                                                                                                                                                                                                                        Danish Energy Agency, technology\_data\_for\_el\_and\_dh.xlsx \\
                      & FOM &         3.58 &                       \%/year &                                                                                                                                                                                                                                                                        Danish Energy Agency, technology\_data\_for\_el\_and\_dh.xlsx \\
                      & VOM &         2.10 &              \euro/MWh$_{el}$ &                                                                                                                                                                                                                                                                        Danish Energy Agency, technology\_data\_for\_el\_and\_dh.xlsx \\
                      & efficiency &         0.30 &                      per unit &                                                                                                                                                                                                                                                                        Danish Energy Agency, technology\_data\_for\_el\_and\_dh.xlsx \\
                      & efficiency (heat) &         0.71 &                      per unit &                                                                                                                                                                                                                                                                        Danish Energy Agency, technology\_data\_for\_el\_and\_dh.xlsx \\
                      & lifetime &        25.00 &                         years &                                                                                                                                                                                                                                                                        Danish Energy Agency, technology\_data\_for\_el\_and\_dh.xlsx \\
                      & overnight investment &     3,210.28 &               \euro/kW$_{el}$ &                                                                                                                                                                                                                                                                        Danish Energy Agency, technology\_data\_for\_el\_and\_dh.xlsx \\
CHP (decentral) & FOM &         3.00 &                       \%/year &                                                                                                                                                                                                                                                                                                                                   HP \\
                      & discount rate &         0.04 &                      per unit &                                                                                                                                                                                                                                                                                                                        Palzer thesis \\
                      & lifetime &        25.00 &                         years &                                                                                                                                                                                                                                                                                                                                   HP \\
                      & overnight investment &     1,400.00 &               \euro/kW$_{el}$ &                                                                                                                                                                                                                                                                                                                                   HP \\
CHP (gas, central) & Cb coefficient &         1.00 &  50$^{\circ}$C/100$^{\circ}$C &                                                                                                                                                                                                                                                                        Danish Energy Agency, technology\_data\_for\_el\_and\_dh.xlsx \\
                      & Cv coefficient &         0.17 &                      per unit &                                                                                                                                                                                                                                                                                               DEA (loss of fuel for additional heat) \\
                      & FOM &         3.32 &                       \%/year &                                                                                                                                                                                                                                                                        Danish Energy Agency, technology\_data\_for\_el\_and\_dh.xlsx \\
                      & VOM &         4.20 &                     \euro/MWh &                                                                                                                                                                                                                                                                        Danish Energy Agency, technology\_data\_for\_el\_and\_dh.xlsx \\
                      & efficiency &         0.41 &                      per unit &                                                                                                                                                                                                                                                                        Danish Energy Agency, technology\_data\_for\_el\_and\_dh.xlsx \\
                      & lifetime &        25.00 &                         years &                                                                                                                                                                                                                                                                        Danish Energy Agency, technology\_data\_for\_el\_and\_dh.xlsx \\
                      & overnight investment &       560.00 &                      \euro/kW &                                                                                                                                                                                                                                                                        Danish Energy Agency, technology\_data\_for\_el\_and\_dh.xlsx \\
CHP (solid biomass, central) & Cb coefficient &         0.46 &                     40°C/80°C &                                                                                                                                                                                                                                                                        Danish Energy Agency, technology\_data\_for\_el\_and\_dh.xlsx \\
                      & Cv coefficient &         1.00 &                     40°C/80°C &                                                                                                                                                                                                                                                                        Danish Energy Agency, technology\_data\_for\_el\_and\_dh.xlsx \\
                      & FOM &         4.10 &                       \%/year &                                                                                                                                                                                                                                                                        Danish Energy Agency, technology\_data\_for\_el\_and\_dh.xlsx \\
                      & VOM &         1.85 &              \euro/MWh$_{el}$ &                                                                                                                                                                                                                                                                        Danish Energy Agency, technology\_data\_for\_el\_and\_dh.xlsx \\
                      & efficiency &         0.29 &                      per unit &                                                                                                                                                                                                                                                                        Danish Energy Agency, technology\_data\_for\_el\_and\_dh.xlsx \\
                      & efficiency (heat) &         0.69 &                      per unit &                                                                                                                                                                                                                                                                        Danish Energy Agency, technology\_data\_for\_el\_and\_dh.xlsx \\
                      & lifetime &        25.00 &                         years &                                                                                                                                                                                                                                                                        Danish Energy Agency, technology\_data\_for\_el\_and\_dh.xlsx \\
                      & overnight investment &     2,851.41 &               \euro/kW$_{el}$ &                                                                                                                                                                                                                                                                        Danish Energy Agency, technology\_data\_for\_el\_and\_dh.xlsx \\
DC grid connection (station) & overnight investment &       400.00 &               \euro/kW$_{el}$ &                                                                                                                                                                                                                                                   Haertel 2017; assuming one onshore and one offshore node + 13\% learning reduction \\
DC grid connection (submarine) & overnight investment &     2,000.00 &                   \euro/MW/km &                                                                                                                                                                                                                          DTU report based on Fig 34 of https://ec.europa.eu/energy/sites/ener/files/documents/2014\_nsog\_report.pdf \\
DC grid connection (underground) & overnight investment &     1,000.00 &                   \euro/MW/km &                                                                                                                                                                                                                                                                                      Haertel 2017; average + 13\% learning reduction \\
Fischer-Tropsch & FOM &         3.00 &                       \%/year &                                                                                                                                                                                                                                                                                                                doi:10.3390/su9020306 \\
                      & VOM &         4.20 &              \euro/MWh$_{FT}$ &                                                                                                                                                                                                                                                                       Danish Energy Agency, data\_sheets\_for\_renewable\_fuels.xlsx \\
                      & efficiency &         0.70 &                      per unit &                                                                                                                                                                                                                                                                       Danish Energy Agency, data\_sheets\_for\_renewable\_fuels.xlsx \\
                      & lifetime &        25.00 &                         years &                                                                                                                                                                                                                                                                       Danish Energy Agency, data\_sheets\_for\_renewable\_fuels.xlsx \\
                      & overnight investment &     1,600.00 &          \euro/kW$_{FT}$/year &                                                                                                                                                                                                                                                                       Danish Energy Agency, data\_sheets\_for\_renewable\_fuels.xlsx \\
HELMETH (direct power-to-methane) & FOM &         3.00 &                       \%/year &                                                                                                                                                                                                                                                                                                                            no source \\
                      & efficiency &         0.80 &                      per unit &                                                                                                                                                                                                                                                                                                                HELMETH press release \\
                      & lifetime &        25.00 &                         years &                                                                                                                                                                                                                                                                                                                            no source \\
                      & overnight investment &     2,000.00 &                      \euro/kW &                                                                                                                                                                                                                                                                                                                            no source \\
HVAC transmission line (overhead) & FOM &         2.00 &                       \%/year &                                                                                                                                                                                                                                                                                                                             Hagspiel \\
                      & lifetime &        40.00 &                         years &                                                                                                                                                                                                                                                                                                                             Hagspiel \\
                      & overnight investment &       400.00 &                   \euro/MW/km &                                                                                                                                                                                                                                                                                                                             Hagspiel \\
HVDC inverter pair & FOM &         2.00 &                       \%/year &                                                                                                                                                                                                                                                                                                                             Hagspiel \\
                      & lifetime &        40.00 &                         years &                                                                                                                                                                                                                                                                                                                             Hagspiel \\
                      & overnight investment &   150,000.00 &                      \euro/MW &                                                                                                                                                                                                                                                                                                                             Hagspiel \\
HVDC transmission line (overhead) & FOM &         2.00 &                       \%/year &                                                                                                                                                                                                                                                                                                                             Hagspiel \\
                      & lifetime &        40.00 &                         years &                                                                                                                                                                                                                                                                                                                             Hagspiel \\
                      & overnight investment &       400.00 &                   \euro/MW/km &                                                                                                                                                                                                                                                                                                                             Hagspiel \\
HVDC transmission line (submarine) & FOM &         0.35 &                       \%/year &                                                                                                                                                                                                                                                               Purvins et al. (2018): https://doi.org/10.1016/j.jclepro.2018.03.095 . \\
                      & lifetime &        40.00 &                         years &                                                                                                                                                                                                                                                               Purvins et al. (2018): https://doi.org/10.1016/j.jclepro.2018.03.095 . \\
                      & overnight investment &       471.16 &                   \euro/MW/km &                                                                                                                                                                                                                                                               Purvins et al. (2018): https://doi.org/10.1016/j.jclepro.2018.03.095 . \\
OCGT & FOM &         1.78 &                       \%/year &                                                                                                                                                                                                                                                                        Danish Energy Agency, technology\_data\_for\_el\_and\_dh.xlsx \\
                      & VOM &         4.50 &                     \euro/MWh &                                                                                                                                                                                                                                                                        Danish Energy Agency, technology\_data\_for\_el\_and\_dh.xlsx \\
                      & efficiency &         0.41 &                      per unit &                                                                                                                                                                                                                                                                        Danish Energy Agency, technology\_data\_for\_el\_and\_dh.xlsx \\
                      & lifetime &        25.00 &                         years &                                                                                                                                                                                                                                                                        Danish Energy Agency, technology\_data\_for\_el\_and\_dh.xlsx \\
                      & overnight investment &       435.24 &                      \euro/kW &                                                                                                                                                                                                                                                                        Danish Energy Agency, technology\_data\_for\_el\_and\_dh.xlsx \\
battery inverter & FOM &         0.34 &                       \%/year &                                                                                                                                                                                                                                                         Danish Energy Agency, technology\_data\_catalogue\_for\_energy\_storage.xlsx \\
                      & efficiency &         0.96 &                      per unit &                                                                                                                                                                                                                                                         Danish Energy Agency, technology\_data\_catalogue\_for\_energy\_storage.xlsx \\
                      & lifetime &        10.00 &                         years &                                                                                                                                                                                                                                                Danish Energy Agency, technology\_data\_catalogue\_for\_energy\_storage.xlsx, Note K. \\
                      & overnight investment &       160.00 &                      \euro/kW &                                                                                                                                                                                                                                                         Danish Energy Agency, technology\_data\_catalogue\_for\_energy\_storage.xlsx \\
battery storage & lifetime &        25.00 &                         years &                                                                                                                                                                                                                                                         Danish Energy Agency, technology\_data\_catalogue\_for\_energy\_storage.xlsx \\
                      & overnight investment &       142.00 &                     \euro/kWh &                                                                                                                                                                                                                                                         Danish Energy Agency, technology\_data\_catalogue\_for\_energy\_storage.xlsx \\
biogas & fuel &        59.00 &              \euro/MWh$_{th}$ &                                                                                                                                                                                                                                                                                                                        JRC and Zappa \\
biogas upgrading & FOM &         2.49 &                       \%/year &                                                                                                                                                                                                                                                                       Danish Energy Agency, data\_sheets\_for\_renewable\_fuels.xlsx \\
                      & VOM &         3.18 &               \euro/MWh input &                                                                                                                                                                                                                                                                       Danish Energy Agency, data\_sheets\_for\_renewable\_fuels.xlsx \\
                      & lifetime &        15.00 &                         years &                                                                                                                                                                                                                                                                       Danish Energy Agency, data\_sheets\_for\_renewable\_fuels.xlsx \\
                      & overnight investment &       381.00 &                \euro/kW input &                                                                                                                                                                                                                                                                       Danish Energy Agency, data\_sheets\_for\_renewable\_fuels.xlsx \\
biomass & FOM &         4.53 &                       \%/year &                                                                                                                                                                                                                                                                                        DIW DataDoc http://hdl.handle.net/10419/80348 \\
                      & efficiency &         0.47 &                      per unit &                                                                                                                                                                                                                                                                                        DIW DataDoc http://hdl.handle.net/10419/80348 \\
                      & fuel &         7.00 &              \euro/MWh$_{th}$ &                                                                                                                                                                                                                                                                                                                             IEA2011b \\
                      & lifetime &        30.00 &                         years &                                                                                                                                                                                                                                                                             ECF2010 in DIW DataDoc http://hdl.handle.net/10419/80348 \\
                      & overnight investment &     2,209.00 &               \euro/kW$_{el}$ &                                                                                                                                                                                                                                                                                        DIW DataDoc http://hdl.handle.net/10419/80348 \\
cement capture & FOM &         3.00 &                       \%/year &                                                                                                                                                                                                                                                    Danish Energy Agency, technology\_data\_for\_industrial\_process\_heat\_0002.xlsx \\
                      & carbon capture rate &         0.90 &                      per unit &                                                                                                                                                                                                                                                    Danish Energy Agency, technology\_data\_for\_industrial\_process\_heat\_0002.xlsx \\
                      & electricity input &         0.08 &            MWh/t$_{\ce{CO2}}$ &                                                                                                                                                                                                                                                    Danish Energy Agency, technology\_data\_for\_industrial\_process\_heat\_0002.xlsx \\
                      & electricity input &         0.02 &            MWh/t$_{\ce{CO2}}$ &                                                                                                                                                                                                                                                    Danish Energy Agency, technology\_data\_for\_industrial\_process\_heat\_0002.xlsx \\
                      & heat input &         0.72 &            MWh/t$_{\ce{CO2}}$ &                                                                                                                                                                                                                                                    Danish Energy Agency, technology\_data\_for\_industrial\_process\_heat\_0002.xlsx \\
                      & heat output &         0.14 &            MWh/t$_{\ce{CO2}}$ &                                                                                                                                                                                                                                                    Danish Energy Agency, technology\_data\_for\_industrial\_process\_heat\_0002.xlsx \\
                      & heat output &         1.54 &            MWh/t$_{\ce{CO2}}$ &                                                                                                                                                                                                                                                    Danish Energy Agency, technology\_data\_for\_industrial\_process\_heat\_0002.xlsx \\
                      & lifetime &        25.00 &                         years &                                                                                                                                                                                                                                                    Danish Energy Agency, technology\_data\_for\_industrial\_process\_heat\_0002.xlsx \\
                      & overnight investment & 2,600,000.00 &      \euro/(t$_{\ce{CO2}}$/h) &                                                                                                                                                                                                                                                    Danish Energy Agency, technology\_data\_for\_industrial\_process\_heat\_0002.xlsx \\
coal & FOM &         1.60 &                       \%/year &                                                                                                                                                                                                                                                                            Lazard s Levelized Cost of Energy Analysis - Version 13.0 \\
                      & VOM &         3.50 &              \euro/MWh$_{el}$ &                                                                                                                                                                                                                                                                            Lazard s Levelized Cost of Energy Analysis - Version 13.0 \\
                      & carbon intensity &         0.34 &     t$_{\ce{CO2}}$/MWh$_{th}$ &                                                                                                                                                                                                                                Entwicklung der spezifischen Kohlendioxid-Emissionen des deutschen Strommix in den Jahren 1990 - 2018 \\
                      & efficiency &         0.33 &                      per unit &                                                                                                                                                                                                                                                                            Lazard s Levelized Cost of Energy Analysis - Version 13.0 \\
                      & fuel &         8.15 &              \euro/MWh$_{th}$ &                                                                                                                                                                                                                                                                                                                              BP 2019 \\
                      & lifetime &        40.00 &                         years &                                                                                                                                                                                                                                                                            Lazard s Levelized Cost of Energy Analysis - Version 13.0 \\
                      & overnight investment &     3,845.51 &               \euro/kW$_{el}$ &                                                                                                                                                                                                                                                                            Lazard s Levelized Cost of Energy Analysis - Version 13.0 \\
decentral oil boiler & FOM &         2.00 &                       \%/year &                                                                                                                                                                                       Palzer thesis (https://energiesysteme-zukunft.de/fileadmin/user\_upload/Publikationen/PDFs/ESYS\_Materialien\_Optimierungsmodell\_REMod-D.pdf) \\
                      & efficiency &         0.90 &                      per unit &                                                                                                                                                                                       Palzer thesis (https://energiesysteme-zukunft.de/fileadmin/user\_upload/Publikationen/PDFs/ESYS\_Materialien\_Optimierungsmodell\_REMod-D.pdf) \\
                      & lifetime &        20.00 &                         years &                                                                                                                                                                                       Palzer thesis (https://energiesysteme-zukunft.de/fileadmin/user\_upload/Publikationen/PDFs/ESYS\_Materialien\_Optimierungsmodell\_REMod-D.pdf) \\
                      & overnight investment &       156.01 &               \euro/kW$_{th}$ &                                                                                                                                                                  Palzer thesis (https://energiesysteme-zukunft.de/fileadmin/user\_upload/Publikationen/PDFs/ESYS\_Materialien\_Optimierungsmodell\_REMod-D.pdf) (+eigene Berechnung) \\
direct air capture (DAC) & FOM &         4.95 &                       \%/year &                                                                                                                                                                                                                                                    Danish Energy Agency, technology\_data\_for\_industrial\_process\_heat\_0002.xlsx \\
                      & electricity input &         0.15 &            MWh/t$_{\ce{CO2}}$ &                                                                                                                                                                                                                                                    Danish Energy Agency, technology\_data\_for\_industrial\_process\_heat\_0002.xlsx \\
                      & electricity input &         0.32 &            MWh/t$_{\ce{CO2}}$ &                                                                                                                                                                                                                                                    Danish Energy Agency, technology\_data\_for\_industrial\_process\_heat\_0002.xlsx \\
                      & heat input &         2.00 &            MWh/t$_{\ce{CO2}}$ &                                                                                                                                                                                                                                                    Danish Energy Agency, technology\_data\_for\_industrial\_process\_heat\_0002.xlsx \\
                      & heat output &         0.20 &            MWh/t$_{\ce{CO2}}$ &                                                                                                                                                                                                                                                    Danish Energy Agency, technology\_data\_for\_industrial\_process\_heat\_0002.xlsx \\
                      & heat output &         1.00 &            MWh/t$_{\ce{CO2}}$ &                                                                                                                                                                                                                                                    Danish Energy Agency, technology\_data\_for\_industrial\_process\_heat\_0002.xlsx \\
                      & lifetime &        20.00 &                         years &                                                                                                                                                                                                                                                    Danish Energy Agency, technology\_data\_for\_industrial\_process\_heat\_0002.xlsx \\
                      & overnight investment & 6,000,000.00 &      \euro/(t$_{\ce{CO2}}$/h) &                                                                                                                                                                                                                                                    Danish Energy Agency, technology\_data\_for\_industrial\_process\_heat\_0002.xlsx \\
electricity distribution grid & FOM &         2.00 &                       \%/year &                                                                                                                                                                                                                                                                                                                                 TODO \\
                      & lifetime &        40.00 &                         years &                                                                                                                                                                                                                                                                                                                                 TODO \\
                      & overnight investment &       500.00 &                      \euro/kW &                                                                                                                                                                                                                                                                                                                                 TODO \\
electricity grid connection & FOM &         2.00 &                       \%/year &                                                                                                                                                                                                                                                                                                                                 TODO \\
                      & lifetime &        40.00 &                         years &                                                                                                                                                                                                                                                                                                                                 TODO \\
                      & overnight investment &       140.00 &                      \euro/kW &                                                                                                                                                                                                                                                                                                                                  DEA \\
electrolysis & FOM &         2.00 &                       \%/year &                                                                                                                                                                                                                                                                       Danish Energy Agency, data\_sheets\_for\_renewable\_fuels.xlsx \\
                      & efficiency &         0.68 &                      per unit &                                                                                                                                                                                                                                                                       Danish Energy Agency, data\_sheets\_for\_renewable\_fuels.xlsx \\
                      & lifetime &        30.00 &                         years &                                                                                                                                                                                                                                                                       Danish Energy Agency, data\_sheets\_for\_renewable\_fuels.xlsx \\
                      & overnight investment &       450.00 &               \euro/kW$_{el}$ &                                                                                                                                                                                                                                                                       Danish Energy Agency, data\_sheets\_for\_renewable\_fuels.xlsx \\
fossil gas & carbon intensity &         0.20 &     t$_{\ce{CO2}}$/MWh$_{th}$ &                                                                                                                                                                                                                                Entwicklung der spezifischen Kohlendioxid-Emissionen des deutschen Strommix in den Jahren 1990 - 2018 \\
                      & fuel &        20.10 &              \euro/MWh$_{th}$ &                                                                                                                                                                                                                                                                                                                              BP 2019 \\
fossil oil & FOM &         2.46 &                       \%/year &                                                                                                                                                                                                                                                                        Danish Energy Agency, technology\_data\_for\_el\_and\_dh.xlsx \\
                      & VOM &         6.00 &                     \euro/MWh &                                                                                                                                                                                                                                                                        Danish Energy Agency, technology\_data\_for\_el\_and\_dh.xlsx \\
                      & carbon intensity &         0.27 &     t$_{\ce{CO2}}$/MWh$_{th}$ &                                                                                                                                                                                                                                Entwicklung der spezifischen Kohlendioxid-Emissionen des deutschen Strommix in den Jahren 1990 - 2018 \\
                      & efficiency &         0.35 &                      per unit &                                                                                                                                                                                                                                                                        Danish Energy Agency, technology\_data\_for\_el\_and\_dh.xlsx \\
                      & fuel &        50.00 &              \euro/MWh$_{th}$ &                                                                                                                                                                                                                                     IEA WEM2017 97USD/boe = http://www.iea.org/media/weowebsite/2017/WEM\_Documentation\_WEO2017.pdf \\
                      & lifetime &        25.00 &                         years &                                                                                                                                                                                                                                                                        Danish Energy Agency, technology\_data\_for\_el\_and\_dh.xlsx \\
                      & overnight investment &       343.00 &                      \euro/kW &                                                                                                                                                                                                                                                                        Danish Energy Agency, technology\_data\_for\_el\_and\_dh.xlsx \\
fuel cell & Cb coefficient &         1.25 &  50$^{\circ}$C/100$^{\circ}$C &                                                                                                                                                                                                                                                                        Danish Energy Agency, technology\_data\_for\_el\_and\_dh.xlsx \\
                      & FOM &         5.00 &                       \%/year &                                                                                                                                                                                                                                                                        Danish Energy Agency, technology\_data\_for\_el\_and\_dh.xlsx \\
                      & efficiency &         0.50 &                      per unit &                                                                                                                                                                                                                                                                        Danish Energy Agency, technology\_data\_for\_el\_and\_dh.xlsx \\
                      & lifetime &        10.00 &                         years &                                                                                                                                                                                                                                                                        Danish Energy Agency, technology\_data\_for\_el\_and\_dh.xlsx \\
                      & overnight investment &     1,100.00 &               \euro/kW$_{el}$ &                                                                                                                                                                                                                                                                        Danish Energy Agency, technology\_data\_for\_el\_and\_dh.xlsx \\
gas boiler (central) & FOM &         3.80 &                       \%/year &                                                                                                                                                                                                                                                                        Danish Energy Agency, technology\_data\_for\_el\_and\_dh.xlsx \\
                      & VOM &         1.00 &              \euro/MWh$_{th}$ &                                                                                                                                                                                                                                                                        Danish Energy Agency, technology\_data\_for\_el\_and\_dh.xlsx \\
                      & efficiency &         1.04 &                      per unit &                                                                                                                                                                                                                                                                        Danish Energy Agency, technology\_data\_for\_el\_and\_dh.xlsx \\
                      & lifetime &        25.00 &                         years &                                                                                                                                                                                                                                                                        Danish Energy Agency, technology\_data\_for\_el\_and\_dh.xlsx \\
                      & overnight investment &        50.00 &               \euro/kW$_{th}$ &                                                                                                                                                                                                                                                                        Danish Energy Agency, technology\_data\_for\_el\_and\_dh.xlsx \\
gas boiler (decentral) & FOM &         6.69 &                       \%/year &                                                                                                                                                                                                                                                    Danish Energy Agency, technologydatafor\_heating\_installations\_marts\_2018.xlsx \\
                      & discount rate &         0.04 &                      per unit &                                                                                                                                                                                                                                                                                                                        Palzer thesis \\
                      & efficiency &         0.98 &                      per unit &                                                                                                                                                                                                                                                    Danish Energy Agency, technologydatafor\_heating\_installations\_marts\_2018.xlsx \\
                      & lifetime &        20.00 &                         years &                                                                                                                                                                                                                                                    Danish Energy Agency, technologydatafor\_heating\_installations\_marts\_2018.xlsx \\
                      & overnight investment &       296.82 &               \euro/kW$_{th}$ &                                                                                                                                                                                                                                                    Danish Energy Agency, technologydatafor\_heating\_installations\_marts\_2018.xlsx \\
heat pump (air-sourced, central) & FOM &         0.23 &                       \%/year &                                                                                                                                                                                                                                                                        Danish Energy Agency, technology\_data\_for\_el\_and\_dh.xlsx \\
                      & VOM &         2.51 &              \euro/MWh$_{th}$ &                                                                                                                                                                                                                                                                        Danish Energy Agency, technology\_data\_for\_el\_and\_dh.xlsx \\
                      & efficiency &         3.60 &                      per unit &                                                                                                                                                                                                                                                                        Danish Energy Agency, technology\_data\_for\_el\_and\_dh.xlsx \\
                      & lifetime &        25.00 &                         years &                                                                                                                                                                                                                                                                        Danish Energy Agency, technology\_data\_for\_el\_and\_dh.xlsx \\
                      & overnight investment &       856.25 &               \euro/kW$_{th}$ &                                                                                                                                                                                                                                                                        Danish Energy Agency, technology\_data\_for\_el\_and\_dh.xlsx \\
heat pump (air-sourced, decentral) & FOM &         3.00 &                       \%/year &                                                                                                                                                                                                                                                    Danish Energy Agency, technologydatafor\_heating\_installations\_marts\_2018.xlsx \\
                      & discount rate &         0.04 &                      per unit &                                                                                                                                                                                                                                                                                                                        Palzer thesis \\
                      & efficiency &         3.60 &                      per unit &                                                                                                                                                                                                                                                    Danish Energy Agency, technologydatafor\_heating\_installations\_marts\_2018.xlsx \\
                      & lifetime &        18.00 &                         years &                                                                                                                                                                                                                                                    Danish Energy Agency, technologydatafor\_heating\_installations\_marts\_2018.xlsx \\
                      & overnight investment &       850.00 &               \euro/kW$_{th}$ &                                                                                                                                                                                                                                                    Danish Energy Agency, technologydatafor\_heating\_installations\_marts\_2018.xlsx \\
heat pump (ground-sourced, central) & FOM &         0.39 &                       \%/year &                                                                                                                                                                                                                                                                        Danish Energy Agency, technology\_data\_for\_el\_and\_dh.xlsx \\
                      & VOM &         1.25 &              \euro/MWh$_{th}$ &                                                                                                                                                                                                                                                                        Danish Energy Agency, technology\_data\_for\_el\_and\_dh.xlsx \\
                      & efficiency &         1.73 &                      per unit &                                                                                                                                                                                                                                                                        Danish Energy Agency, technology\_data\_for\_el\_and\_dh.xlsx \\
                      & lifetime &        25.00 &                         years &                                                                                                                                                                                                                                                                        Danish Energy Agency, technology\_data\_for\_el\_and\_dh.xlsx \\
                      & overnight investment &       507.60 &              \euro/kW$_{th}$  &                                                                                                                                                                                                                                                                        Danish Energy Agency, technology\_data\_for\_el\_and\_dh.xlsx \\
heat pump (ground-sourced, decentral) & FOM &         1.82 &                       \%/year &                                                                                                                                                                                                                                                    Danish Energy Agency, technologydatafor\_heating\_installations\_marts\_2018.xlsx \\
                      & discount rate &         0.04 &                      per unit &                                                                                                                                                                                                                                                                                                                        Palzer thesis \\
                      & efficiency &         3.90 &                      per unit &                                                                                                                                                                                                                                                    Danish Energy Agency, technologydatafor\_heating\_installations\_marts\_2018.xlsx \\
                      & lifetime &        20.00 &                         years &                                                                                                                                                                                                                                                    Danish Energy Agency, technologydatafor\_heating\_installations\_marts\_2018.xlsx \\
                      & overnight investment &     1,400.00 &               \euro/kW$_{th}$ &                                                                                                                                                                                                                                                    Danish Energy Agency, technologydatafor\_heating\_installations\_marts\_2018.xlsx \\
home battery inverter & FOM &         0.34 &                       \%/year &                                                                                                                                                            Global Energy System based on 100\% Renewable Energy, Energywatchgroup/LTU University, 2019, Danish Energy Agency, technology\_data\_catalogue\_for\_energy\_storage.xlsx \\
                      & efficiency &         0.96 &                      per unit &                                                                                                                                                            Global Energy System based on 100\% Renewable Energy, Energywatchgroup/LTU University, 2019, Danish Energy Agency, technology\_data\_catalogue\_for\_energy\_storage.xlsx \\
                      & lifetime &        10.00 &                         years &                                                                                                                                                   Global Energy System based on 100\% Renewable Energy, Energywatchgroup/LTU University, 2019, Danish Energy Agency, technology\_data\_catalogue\_for\_energy\_storage.xlsx, Note K. \\
                      & overnight investment &       228.06 &                      \euro/kW &                                                                                                                                                            Global Energy System based on 100\% Renewable Energy, Energywatchgroup/LTU University, 2019, Danish Energy Agency, technology\_data\_catalogue\_for\_energy\_storage.xlsx \\
home battery storage & lifetime &        25.00 &                         years &                                                                                                                                                            Global Energy System based on 100\% Renewable Energy, Energywatchgroup/LTU University, 2019, Danish Energy Agency, technology\_data\_catalogue\_for\_energy\_storage.xlsx \\
                      & overnight investment &       202.90 &                     \euro/kWh &                                                                                                                                                            Global Energy System based on 100\% Renewable Energy, Energywatchgroup/LTU University, 2019, Danish Energy Agency, technology\_data\_catalogue\_for\_energy\_storage.xlsx \\
hydrogen liquefaction & FOM &         8.00 &                       \%/year &                                                                                                                                                                                                                                   Reuß et al 2017: https://doi.org/10.1016/j.apenergy.2017.05.050 , Table 9 and equation in sec 3.0. \\
                      & lifetime &        20.00 &                         years &                                                                                                                                                                                                                                   Reuß et al 2017: https://doi.org/10.1016/j.apenergy.2017.05.050 , Table 9 and equation in sec 3.0. \\
                      & overnight investment & 1,497,967.32 &          \euro/MW$_{\ce{H2}}$ &                                                                                                                                                                                                                                   Reuß et al 2017: https://doi.org/10.1016/j.apenergy.2017.05.050 , Table 9 and equation in sec 3.0. \\
hydrogen pipeline & FOM &         3.17 &                       \%/year &                                                                                                                                                                                                                                          Danish Energy Agency, Technology Data for Energy Transport (2021), Excel datasheet: H2 140. \\
                      & lifetime &        50.00 &                         years &                                                                                                                                                                                                                                          Danish Energy Agency, Technology Data for Energy Transport (2021), Excel datasheet: H2 140. \\
                      & overnight investment &       226.47 &                   \euro/MW/km &                                                                                                                                             European Hydrogen Backbone Report (June 2021): https://gasforclimate2050.eu/wp-content/uploads/2021/06/EHB\_Analysing-the-future-demand-supply-and-transport-of-hydrogen\_June-2021.pdf. \\
hydrogen pipeline (repurposed) & FOM &         3.17 &                       \%/year &                                                                                                                                                                                                                                          Danish Energy Agency, Technology Data for Energy Transport (2021), Excel datasheet: H2 140. \\
                      & lifetime &        50.00 &                         years &                                                                                                                                                                                                                                          Danish Energy Agency, Technology Data for Energy Transport (2021), Excel datasheet: H2 140. \\
                      & overnight investment &       105.88 &                   \euro/MW/km &                                                                                                                                             European Hydrogen Backbone Report (June 2021): https://gasforclimate2050.eu/wp-content/uploads/2021/06/EHB\_Analysing-the-future-demand-supply-and-transport-of-hydrogen\_June-2021.pdf. \\
hydrogen pipeline (submarine) & FOM &         3.00 &                       \%/year &                                                                                                                                                                                                                                                                                       Assume same as for CH4 (g) submarine pipeline. \\
                      & lifetime &        30.00 &                         years &                                                                                                                                                                                                                                                                                       Assume same as for CH4 (g) submarine pipeline. \\
                      & overnight investment &       329.37 &                   \euro/MW/km &  Assume similar cost as for CH4 (g) submarine pipeline but with the same factor as between onland CH4 (g) pipeline and H2 (g) pipeline (2.86). This estimate is comparable to a 36in diameter pipeline calaculated based on d’Amore-Domenech et al (2021): 10.1016/j.apenergy.2021.116625 , supplementary material (=251 EUR/MW/km). \\
hydrogen storage (steel tank) & FOM &         1.11 &                       \%/year &                                                                                                                                                                                                                                                         Danish Energy Agency, technology\_data\_catalogue\_for\_energy\_storage.xlsx \\
                      & lifetime &        30.00 &                         years &                                                                                                                                                                                                                                                         Danish Energy Agency, technology\_data\_catalogue\_for\_energy\_storage.xlsx \\
                      & overnight investment &        44.91 &                     \euro/kWh &                                                                                                                                                                                                                                                         Danish Energy Agency, technology\_data\_catalogue\_for\_energy\_storage.xlsx \\
hydrogen storage (underground) & FOM &         0.00 &                       \%/year &                                                                                                                                                                                                                                                         Danish Energy Agency, technology\_data\_catalogue\_for\_energy\_storage.xlsx \\
                      & VOM &         0.00 &                     \euro/MWh &                                                                                                                                                                                                                                                         Danish Energy Agency, technology\_data\_catalogue\_for\_energy\_storage.xlsx \\
                      & lifetime &       100.00 &                         years &                                                                                                                                                                                                                                                         Danish Energy Agency, technology\_data\_catalogue\_for\_energy\_storage.xlsx \\
                      & overnight investment &         2.00 &                     \euro/kWh &                                                                                                                                                                                                                                                         Danish Energy Agency, technology\_data\_catalogue\_for\_energy\_storage.xlsx \\
lignite & FOM &         1.60 &                       \%/year &                                                                                                                                                                                                                                                                            Lazard s Levelized Cost of Energy Analysis - Version 13.0 \\
                      & VOM &         3.50 &              \euro/MWh$_{el}$ &                                                                                                                                                                                                                                                                            Lazard s Levelized Cost of Energy Analysis - Version 13.0 \\
                      & carbon intensity &         0.41 &     t$_{\ce{CO2}}$/MWh$_{th}$ &                                                                                                                                                                                                                                Entwicklung der spezifischen Kohlendioxid-Emissionen des deutschen Strommix in den Jahren 1990 - 2018 \\
                      & efficiency &         0.33 &                      per unit &                                                                                                                                                                                                                                                                            Lazard s Levelized Cost of Energy Analysis - Version 13.0 \\
                      & fuel &         2.90 &              \euro/MWh$_{th}$ &                                                                                                                                                                                                                                                                                                                                  DIW \\
                      & lifetime &        40.00 &                         years &                                                                                                                                                                                                                                                                            Lazard s Levelized Cost of Energy Analysis - Version 13.0 \\
                      & overnight investment &     3,845.51 &               \euro/kW$_{el}$ &                                                                                                                                                                                                                                                                            Lazard s Levelized Cost of Energy Analysis - Version 13.0 \\
methanation & FOM &         4.00 &                       \%/year &                                                                                                                                                                                                                                                                   Fasihi et al 2017, table 1, https://www.mdpi.com/2071-1050/9/2/306 \\
                      & efficiency &         0.80 &                      per unit &                                                                                                                                                                                                                                                                                                            Palzer and Schaber thesis \\
                      & lifetime &        30.00 &                         years &                                                                                                                                                                                                                                                                   Fasihi et al 2017, table 1, https://www.mdpi.com/2071-1050/9/2/306 \\
                      & overnight investment &       278.00 &         \euro/kW$_{\ce{CH4}}$ &                                                                                                                                                                                                                                                                   Fasihi et al 2017, table 1, https://www.mdpi.com/2071-1050/9/2/306 \\
natural gas pipeline & FOM &         1.50 &                       \%/year &                                                                                                                                                                                                                                                      Assume same as for H2 (g) pipeline in 2050 (CH4 pipeline as mature technology). \\
                      & lifetime &        50.00 &                         years &                                                                                                                                                                                                                                                      Assume same as for H2 (g) pipeline in 2050 (CH4 pipeline as mature technology). \\
                      & overnight investment &        79.00 &                   \euro/MW/km &                                                                                                                                                                                                                                                                                                                         Guesstimate. \\
natural gas pipeline (submarine) & FOM &         3.00 &                       \%/year &                                                                                                                                                                                                                                              d’Amore-Domenech et al (2021): 10.1016/j.apenergy.2021.116625 , supplementary material. \\
                      & lifetime &        30.00 &                         years &                                                                                                                                                                                                                                              d’Amore-Domenech et al (2021): 10.1016/j.apenergy.2021.116625 , supplementary material. \\
                      & overnight investment &       114.89 &                   \euro/MW/km &                                                                                                                                                                                                                                                                                        Kaiser (2017): 10.1016/j.marpol.2017.05.003 . \\
offshore wind & FOM &         2.32 &                       \%/year &                                                                                                                                                                                                                                                                        Danish Energy Agency, technology\_data\_for\_el\_and\_dh.xlsx \\
                      & VOM &         3.89 &        \euro/MWh$_{el}$, 2020 &                                                                                                                                                                                                                                                                        Danish Energy Agency, technology\_data\_for\_el\_and\_dh.xlsx \\
                      & lifetime &        30.00 &                         years &                                                                                                                                                                                                                                                                        Danish Energy Agency, technology\_data\_for\_el\_and\_dh.xlsx \\
                      & overnight investment &     1,523.55 &         \euro/kW$_{el}$, 2020 &                                                                                                                                                                                                                                                                        Danish Energy Agency, technology\_data\_for\_el\_and\_dh.xlsx \\
onshore wind & FOM &         1.22 &                       \%/year &                                                                                                                                                                                                                                                                        Danish Energy Agency, technology\_data\_for\_el\_and\_dh.xlsx \\
                      & VOM &         1.35 &                     \euro/MWh &                                                                                                                                                                                                                                                                        Danish Energy Agency, technology\_data\_for\_el\_and\_dh.xlsx \\
                      & lifetime &        30.00 &                         years &                                                                                                                                                                                                                                                                        Danish Energy Agency, technology\_data\_for\_el\_and\_dh.xlsx \\
                      & overnight investment &     1,035.56 &                      \euro/kW &                                                                                                                                                                                                                                                                        Danish Energy Agency, technology\_data\_for\_el\_and\_dh.xlsx \\
pumped hydro storage & FOM &         1.00 &                       \%/year &                                                                                                                                                                                                                                                                                        DIW DataDoc http://hdl.handle.net/10419/80348 \\
                      & efficiency &         0.75 &                      per unit &                                                                                                                                                                                                                                                                                        DIW DataDoc http://hdl.handle.net/10419/80348 \\
                      & lifetime &        80.00 &                         years &                                                                                                                                                                                                                                                                                                                              IEA2010 \\
                      & overnight investment &     2,208.16 &               \euro/kW$_{el}$ &                                                                                                                                                                                                                                                                                        DIW DataDoc http://hdl.handle.net/10419/80348 \\
reservoir hydro & FOM &         1.00 &                       \%/year &                                                                                                                                                                                                                                                                                        DIW DataDoc http://hdl.handle.net/10419/80348 \\
                      & efficiency &         0.90 &                      per unit &                                                                                                                                                                                                                                                                                        DIW DataDoc http://hdl.handle.net/10419/80348 \\
                      & lifetime &        80.00 &                         years &                                                                                                                                                                                                                                                                                                                              IEA2010 \\
                      & overnight investment &     2,208.16 &               \euro/kW$_{el}$ &                                                                                                                                                                                                                                                                                        DIW DataDoc http://hdl.handle.net/10419/80348 \\
resistive heater (central) & FOM &         1.70 &                       \%/year &                                                                                                                                                                                                                                                                        Danish Energy Agency, technology\_data\_for\_el\_and\_dh.xlsx \\
                      & VOM &         1.00 &              \euro/MWh$_{th}$ &                                                                                                                                                                                                                                                                        Danish Energy Agency, technology\_data\_for\_el\_and\_dh.xlsx \\
                      & efficiency &         0.99 &                      per unit &                                                                                                                                                                                                                                                                        Danish Energy Agency, technology\_data\_for\_el\_and\_dh.xlsx \\
                      & lifetime &        20.00 &                         years &                                                                                                                                                                                                                                                                        Danish Energy Agency, technology\_data\_for\_el\_and\_dh.xlsx \\
                      & overnight investment &        60.00 &               \euro/kW$_{th}$ &                                                                                                                                                                                                                                                                        Danish Energy Agency, technology\_data\_for\_el\_and\_dh.xlsx \\
resistive heater (decentral) & FOM &         2.00 &                       \%/year &                                                                                                                                                                                                                                                                                                                       Schaber thesis \\
                      & discount rate &         0.04 &                      per unit &                                                                                                                                                                                                                                                                                                                        Palzer thesis \\
                      & efficiency &         0.90 &                      per unit &                                                                                                                                                                                                                                                                                                                       Schaber thesis \\
                      & lifetime &        20.00 &                         years &                                                                                                                                                                                                                                                                                                                       Schaber thesis \\
                      & overnight investment &       100.00 &                   \euro/kWhth &                                                                                                                                                                                                                                                                                                                       Schaber thesis \\
run of river & FOM &         2.00 &                       \%/year &                                                                                                                                                                                                                                                                                        DIW DataDoc http://hdl.handle.net/10419/80348 \\
                      & efficiency &         0.90 &                      per unit &                                                                                                                                                                                                                                                                                        DIW DataDoc http://hdl.handle.net/10419/80348 \\
                      & lifetime &        80.00 &                         years &                                                                                                                                                                                                                                                                                                                              IEA2010 \\
                      & overnight investment &     3,312.24 &               \euro/kW$_{el}$ &                                                                                                                                                                                                                                                                                        DIW DataDoc http://hdl.handle.net/10419/80348 \\
solar PV & FOM &         1.95 &                       \%/year &                                                                                                                                                                                                                                                                        Danish Energy Agency, technology\_data\_for\_el\_and\_dh.xlsx \\
                      & VOM &         0.01 &                   \euro/MWhel &                                                                                                                                                                                                                                                                                           RES costs made up to fix curtailment order \\
                      & lifetime &        40.00 &                         years &                                                                                                                                                                                                                                                                        Danish Energy Agency, technology\_data\_for\_el\_and\_dh.xlsx \\
                      & overnight investment &       492.11 &               \euro/kW$_{el}$ &                                                                                                                                                                                                                                                                        Danish Energy Agency, technology\_data\_for\_el\_and\_dh.xlsx \\
solar PV (rooftop) & FOM &         1.42 &                       \%/year &                                                                                                                                                                                                                                                                        Danish Energy Agency, technology\_data\_for\_el\_and\_dh.xlsx \\
                      & discount rate &         0.04 &                      per unit &                                                                                                                                                                                                                                                                                                               standard for decentral \\
                      & lifetime &        40.00 &                         years &                                                                                                                                                                                                                                                                        Danish Energy Agency, technology\_data\_for\_el\_and\_dh.xlsx \\
                      & overnight investment &       636.66 &               \euro/kW$_{el}$ &                                                                                                                                                                                                                                                                        Danish Energy Agency, technology\_data\_for\_el\_and\_dh.xlsx \\
solar PV (utility-scale) & FOM &         2.48 &                       \%/year &                                                                                                                                                                                                                                                                        Danish Energy Agency, technology\_data\_for\_el\_and\_dh.xlsx \\
                      & lifetime &        40.00 &                         years &                                                                                                                                                                                                                                                                        Danish Energy Agency, technology\_data\_for\_el\_and\_dh.xlsx \\
                      & overnight investment &       347.56 &               \euro/kW$_{el}$ &                                                                                                                                                                                                                                                                        Danish Energy Agency, technology\_data\_for\_el\_and\_dh.xlsx \\
solar thermal (central) & FOM &         1.40 &                       \%/year &                                                                                                                                                                                                                                                                                                                                   HP \\
                      & lifetime &        20.00 &                         years &                                                                                                                                                                                                                                                                                                                                   HP \\
                      & overnight investment &   140,000.00 &               \euro/1000m$^2$ &                                                                                                                                                                                                                                                                                                                                   HP \\
solar thermal (decentral) & FOM &         1.30 &                       \%/year &                                                                                                                                                                                                                                                                                                                                   HP \\
                      & discount rate &         0.04 &                      per unit &                                                                                                                                                                                                                                                                                                                        Palzer thesis \\
                      & lifetime &        20.00 &                         years &                                                                                                                                                                                                                                                                                                                                   HP \\
                      & overnight investment &   270,000.00 &               \euro/1000m$^2$ &                                                                                                                                                                                                                                                                                                                                   HP \\
solid biomass & carbon intensity &         0.30 &     t$_{\ce{CO2}}$/MWh$_{th}$ &                                                                                                                                                                                                                                                                                                                                 TODO \\
                      & fuel &        25.20 &              \euro/MWh$_{th}$ &                                                                                                                                                                                                                                                                         Is a 100\% renewable European power system feasible by 2050? \\
steam methane reforming & FOM &         5.00 &                       \%/year &                                                                                                                                                                                                                                                                                                                 Danish Energy Agency \\
                      & efficiency &         0.76 &             per unit (in LHV) &                                                                    IEA Global average levelised cost of hydrogen production by energy source and technology, 2019 and 2050 (2020), https://www.iea.org/data-and-statistics/charts/global-average-levelised-cost-of-hydrogen-production-by-energy-source-and-technology-2019-and-2050 \\
                      & lifetime &        30.00 &                         years &                                                                    IEA Global average levelised cost of hydrogen production by energy source and technology, 2019 and 2050 (2020), https://www.iea.org/data-and-statistics/charts/global-average-levelised-cost-of-hydrogen-production-by-energy-source-and-technology-2019-and-2050 \\
                      & overnight investment &   493,470.40 &         \euro/MW$_{\ce{CH4}}$ &                                                                                                                                                                                                                                                                                                                 Danish Energy Agency \\
steam methane reforming with carbon capture & FOM &         5.00 &                       \%/year &                                                                                                                                                                                                                                                                                                                 Danish Energy Agency \\
                      & carbon capture rate &         0.90 &         \euro/MW$_{\ce{CH4}}$ &                                                                    IEA Global average levelised cost of hydrogen production by energy source and technology, 2019 and 2050 (2020), https://www.iea.org/data-and-statistics/charts/global-average-levelised-cost-of-hydrogen-production-by-energy-source-and-technology-2019-and-2050 \\
                      & efficiency &         0.69 &             per unit (in LHV) &                                                                    IEA Global average levelised cost of hydrogen production by energy source and technology, 2019 and 2050 (2020), https://www.iea.org/data-and-statistics/charts/global-average-levelised-cost-of-hydrogen-production-by-energy-source-and-technology-2019-and-2050 \\
                      & lifetime &        30.00 &                         years &                                                                    IEA Global average levelised cost of hydrogen production by energy source and technology, 2019 and 2050 (2020), https://www.iea.org/data-and-statistics/charts/global-average-levelised-cost-of-hydrogen-production-by-energy-source-and-technology-2019-and-2050 \\
                      & overnight investment &   572,425.66 &         \euro/MW$_{\ce{CH4}}$ &                                                                                                                                                                                                                                                                                                                 Danish Energy Agency \\
thermal storage (water tank, central) & FOM &         0.55 &                       \%/year &                                                                                                                                                                                                                                                         Danish Energy Agency, technology\_data\_catalogue\_for\_energy\_storage.xlsx \\
                      & lifetime &        25.00 &                         years &                                                                                                                                                                                                                                                         Danish Energy Agency, technology\_data\_catalogue\_for\_energy\_storage.xlsx \\
                      & overnight investment &         0.54 &                     \euro/kWh &                                                                                                                                                                                                                                                         Danish Energy Agency, technology\_data\_catalogue\_for\_energy\_storage.xlsx \\
thermal storage (water tank, decentral) & FOM &         1.00 &                       \%/year &                                                                                                                                                                                                                                                                                                                                   HP \\
                      & discount rate &         0.04 &                      per unit &                                                                                                                                                                                                                                                                                                                        Palzer thesis \\
                      & lifetime &        20.00 &                         years &                                                                                                                                                                                                                                                                                                                                   HP \\
                      & overnight investment &        18.38 &                     \euro/kWh &                                                                                                                                                                                                                                                                                                                     IWES Interaktion \\
water tank charger & efficiency &         0.84 &                      per unit &                                                                                                                                                                                                                                                         Danish Energy Agency, technology\_data\_catalogue\_for\_energy\_storage.xlsx \\
water tank discharger & efficiency &         0.84 &                      per unit &                                                                                                                                                                                                                                                         Danish Energy Agency, technology\_data\_catalogue\_for\_energy\_storage.xlsx \\
\end{longtable}

\end{small}

\end{landscape}

\restoregeometry