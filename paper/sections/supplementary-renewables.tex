
\section{Renewables}
\label{sec:si:renewables}

\subsection{Potentials}
\label{sec:si:renewable-potentials}


\begin{figure}
    \centering
    \begin{subfigure}[t]{0.48\textwidth}
            \centering
        \caption{solar land eligibility}
        \includegraphics[width=\textwidth]{eligibility-solar-250-20.pdf}
    \end{subfigure}
    \begin{subfigure}[t]{0.48\textwidth}
        \centering
        \caption{onshore wind land eligibility}
        \includegraphics[width=\textwidth]{eligibility-onwind-250-20.pdf}
    \end{subfigure}
    \begin{subfigure}[t]{0.48\textwidth}
        \centering
        \vspace{.5cm}
        \caption{offshore wind land eligibility}
        \includegraphics[width=\textwidth]{eligibility-offwind-250-20.pdf}
    \end{subfigure}
    \caption{Land eligibility for the development of renewable generation capacities. Green color indicates
    areas eligible to build wind or utility-scale solar parks based on suitable land types, natural protection areas, and water depths.}
    \label{fig:eligibility}
\end{figure}

\begin{table}
    \caption{Land types considered suitable for every technology from Corine Land Cover database. Land type codes are referenced in brackets.}
    \small
    \begin{tabularx}{\textwidth}{lX}
        \toprule
        Solar PV & artificial surfaces (1-11), agriculture land except for those
        areas already occupied by agriculture with significant natural
        vegetation and agro-forestry areas (12-20), natural grasslands (26), bare rocks (31),
        sparsely vegetated areas (32) \\ \midrule
        Onshore wind & agriculture areas (12-22), forests (23-25), scrubs and herbaceous vegetation associations (26-29), bare rocks (31), sparsely vegetated areas (32) \\ \midrule
        Offshore wind & sea and ocean (44) \\ \bottomrule
    \end{tabularx}
    \label{tab:eligibility}
\end{table}

Eligibile areas for developing renewable infrastructure are calculated per
technology and substation's Voronoi cell using the
\textit{atlite}\citeS{hofmannAtliteLightweight2021} tool and shown in
\cref{fig:eligibility}.

The land available for wind and utility-scale solar PV capacities in a
particular region is constrained by eligible codes of the
CORINE\citeS{europeanenvironmentagencyeeaCorineLand} land use database (100m
resolution)  and is further restricted by distance criteria and the natural
protection areas specified in the Natura
2000\citeS{europeanenvironmentagencyeeaNatura2000} dataset. These criteria are
summarised in \cref{tab:eligibility}. The installable potentials for rooftop PV
are included with an assumption of 1 kWp per person (0.1 kW/m$^2$ and 10
m$^2$/person). A more sophisticated potential estimate can be found in Bódis et
al.~\citeS{bodisHighresolutionGeospatial2019}. Moreover, offshore wind farms may
not be built at sea depths exceeding \SI{50}{\metre}, as indicated by the GEBCO\citeS{gebcoGEBCO2014}
bathymetry dataset. This currently disreagards the
possibility of floating wind turbines.
\citeS{lerchSensitivityAnalysis2018,lauraLifecycleCost2014,myhrLevelisedCost2014,kauscheFloatingOffshore2018,castro-santosEconomicFeasibility2016}
For near-shore locations (less than 30 km off the shore) AC connections are
considered, whereas for far-shore locations, DC connections including AC-DC
converter costs are assumed. Reservoir hydropower and run-of-river capacities
are exogenously fixed at current values and not expandable.

To express the potential in terms of installable capacities, the available areas
are multiplied with allowed deployment densities, which we consider to be a
fraction of the technology's technical deployment density to preempt public
acceptance issues. These densities are 3 MW/m$^ 2$ for onshore wind, 2 MW/m$^2$
for offshore wind, 5.1 MW/m$^2$ for utility-scale solar. For a review of
alternative potential wind potential assessments, see McKenna et
al.~\citeS{mckennaHighresolutionLargescale2022} and Ryberg et
al.~\citeS{rybergFutureEuropean2019}.


\subsection{Time Series}
\label{sec:si:renewable-ts}

\begin{figure}
    \centering
        \begin{subfigure}[t]{0.49\textwidth}
            \centering
        \includegraphics[width=\textwidth]{windspeeds.png}
    \end{subfigure}
    \begin{subfigure}[t]{0.49\textwidth}
        \centering
        \includegraphics[width=\textwidth]{irradiation.png}
    \end{subfigure}
    \begin{subfigure}[t]{0.49\textwidth}
        \centering
        \includegraphics[width=\textwidth]{temperatures.png}
    \end{subfigure}
    \begin{subfigure}[t]{0.49\textwidth}
        \centering
        \includegraphics[width=\textwidth]{runoff.png}
    \end{subfigure}
    \caption{Extract of used weather data variables from ERA5 and SARAH-2, including mean wind speeds, ambient and soil temperatures, surface runoff and solar irradiation.}
    \label{fig:weather-data}
\end{figure}

\begin{figure}
    \centering
    \begin{subfigure}[t]{0.49\textwidth}
        \centering
        \caption{solar}
        \includegraphics[width=\textwidth]{solar-energy-density.pdf}
    \end{subfigure}
    \begin{subfigure}[t]{0.49\textwidth}
        \centering
        \caption{wind}
        \includegraphics[width=\textwidth]{wind-energy-density.pdf}
    \end{subfigure}
    \caption{Available energy density for wind and utility-scale solar PV power generation.}
    \label{fig:energy-density}
\end{figure}

\begin{figure}
    \centering
        \begin{subfigure}[t]{0.49\textwidth}
            \centering
        \includegraphics[width=\textwidth]{cf-raw-ts-onshore wind.pdf}
    \end{subfigure}
    \begin{subfigure}[t]{0.49\textwidth}
        \centering
        \includegraphics[width=\textwidth]{cf-raw-ts-offshore wind.pdf}
    \end{subfigure}
    \begin{subfigure}[t]{0.49\textwidth}
        \centering
        \includegraphics[width=\textwidth]{cf-raw-ts-solar PV.pdf}
    \end{subfigure}
    \begin{subfigure}[t]{0.49\textwidth}
        \centering
        \includegraphics[width=\textwidth]{cf-raw-ts-run of river.pdf}
    \end{subfigure}
    \begin{subfigure}[t]{0.49\textwidth}
        \centering
        \includegraphics[width=\textwidth]{cop-ts-air-sourced heat pump.pdf}
    \end{subfigure}
    \begin{subfigure}[t]{0.49\textwidth}
        \centering
        \includegraphics[width=\textwidth]{cop-ts-ground-sourced heat pump.pdf}
    \end{subfigure}
    \caption{Spatially aggregated capacity factor time series of renewable energy sources.}
    \label{fig:cfs-ts}
\end{figure}

\begin{figure}
    \centering
    \begin{subfigure}[t]{0.49\textwidth}
        \centering
        \caption{onshore wind}
        \includegraphics[width=\textwidth]{cf-onwind.pdf}
    \end{subfigure}
    \begin{subfigure}[t]{0.49\textwidth}
        \centering
        \caption{solar photovoltaics}
        \includegraphics[width=\textwidth]{cf-solar.pdf}
    \end{subfigure}
    \begin{subfigure}[t]{0.49\textwidth}
        \centering
        \caption{offshore wind (DC-connected)}
        \includegraphics[width=\textwidth]{cf-offwind-dc.pdf}
    \end{subfigure}
    \begin{subfigure}[t]{0.49\textwidth}
        \centering
        \caption{offshore wind (AC-connected)}
        \includegraphics[width=\textwidth]{cf-offwind-ac.pdf}
    \end{subfigure}
    \begin{subfigure}[t]{0.49\textwidth}
        \centering
        \caption{air-sourced heat pump}
        \includegraphics[width=\textwidth]{cf-air-sourced heat pump.pdf}
    \end{subfigure}
    \begin{subfigure}[t]{0.49\textwidth}
        \centering
        \caption{ground-sourced heat pump}
        \includegraphics[width=\textwidth]{cf-ground-sourced heat pump.pdf}
    \end{subfigure}
    \caption{Regional distribution of average capacity factors of renewable energy sources.}
    \label{fig:cfs-maps}
\end{figure}




The location-dependent renewables availability time series are generated based
on two gridded historical weather datasets (\cref{fig:weather-data}). We
retrieve wind sepeeds at \SI{100}{\metre}, surface roughness, soil and air
temperatures, and surface run-off from rainfall or melting snow from the global
ERA5 reanalysis dataset provided by the ECMWF \citeS{ecmwf}. It provides hourly
values for each of these parameters since 1950 on a $0.25^{\circ} \times
0.25^{\circ}$ grid. In Germany, such a weather cell expands approximately
\SI{20}{\kilo\metre} from east to west and \SI{31}{km} from north to south. For
the direct and diffuse solar irradiance, we use the satellite-aided SARAH-2
dataset \citeS{SARAH}, which assesses cloud cover in more detail than the ERA5 dataset. It
features values from 1983 to 2015 at an even higher resolution with a
$0.05^{\circ} \times 0.05^{\circ}$ grid and 30-minute intervals \citeS{SARAH}. In
general, the reference weather year can be freely chosen for the optimisation,
but in this contribution all analyses are based on the year 2013, which is
regarded as characteristic year for both wind and solar resources (e.g.
\citeS{cokerInterannualWeather2020}).

Models for wind turbines, solar panels, heat pumps and the inflow into hydro
basins convert the weather data to hourly time series for capacity factors and
performance coefficients. Using power curves of selected wind turbines types
(Vestas V112 for onshore, NREL 5MW for offshore), wind speeds scaled to the
according hub height are mapped to power outputs. For offshore wind, we
additionally take into account wake effects by applying a uniform correction
factor of 88.55\% to the capacity factors \citeS{boschTemporallyExplicit2018}.
The solar photovoltaic panels' output is calculated based on the incidence angle
of solar irradiation, the panel's tilt angle, and conversion efficiency.
Similarly, solar thermal generation is determined based on collector orientation
and a clear-sky model based on \citeS{henningComprehensiveModel2014a}. The
creation of heat pump time series follows regression analyses that map soil or
air temperatures to the coefficient of performance (COP)
\citeS{nouvelEuropeanMapping2015, staffellReviewDomestic2012}. Hydroelectric
inflow time series are derived from run-off data from ERA5 and scaled using EIA
annual hydropower generation statistics
\citeS{u.s.energyinformationadministrationHydroelectricityNet2022}. The
open-source library \textit{atlite} \citeS{hofmannAtliteLightweight2021}
provides functionality to perform all these calculations efficiently. Finally,
the obtained time series are aggregated to each region heuristically in
proportion to each grid cell's mean capacity factor. This assumes a capacity
layout proportional to mean capacity factors. The resulting spatial and temporal
variability of capacity factors are shown in
\cref{fig:cfs-ts,fig:cfs-maps}.

In combination with the capacity potentials derived from the assumed land use
restrictions, the time-averaged capacity factors are used to display in
\cref{fig:energy-density} the energy that could be produced from wind and solar
energy in the different regions of Europe.
