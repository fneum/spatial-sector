\begin{SCfigure}
    \includegraphics[width=0.7\textwidth]{fec_industry_today_tomorrow.pdf}
    \caption{Final consumption of energy and non-energy feedstocks in industry today (left bar) and
    our future scenario in 2050 (right bar)}
    \label{fig:fec-industry}
\end{SCfigure}

\begin{SCfigure}
    \includegraphics[width=0.7\textwidth]{process-emissions.pdf}
    \caption{Process emissions in industry today (top bar) and in 2050 without carbon capture (bottom bar)}
    \label{fig:process-emissions}
\end{SCfigure}

% \begin{table}[h]
%     \centering
%     \small
%     \setlength{\tabcolsep}{6pt}
%     \begin{tabular}{@{} p{5cm}r @{}}
%       \toprule
%       Material & Production [\si{\mega\tonne\per\year}] \\
%       \midrule
%       steel & 174 \\
%       cement & 186 \\
%       glass & 44 \\
%       ceramics and other NMM & 360 \\
%       ammonia & 18 \\
%       chlorine & 10 \\
%       methanol & 2 \\
%       high-value chemicals (HVC) & 47 \\
%       other chemicals & 28 \\
%       pulp & 39 \\
%       paper & 94 \\ \bottomrule
%     \end{tabular}
%     \caption{Industrial production for main products in 2015.}
%     \label{tab:industryproduction}
%   \end{table}

\begin{SCfigure}
    \includegraphics[width=0.7\textwidth]{hotmaps.pdf}
    \caption{Distribution of industries according to emissions data from the Hotmaps industrial sites database.}
    \label{fig:hotmaps}
\end{SCfigure}


\section{Industry Sector}
\label{sec:si:industry}

Industry demand is split into a dozen different sectors with specific energy
demands, process emissions of carbon dioxide, as well as existing and
prospoective mitigation strategies. \cref{sec:si:industry:overview} provides a
general description of the modelling approach for the industry sector in
PyPSA-Eur-Sec. The following subsections describe the current energy demands,
available mitigation strategies, and whether mitigation is exogenously fixed or
co-optimised with the other components of the model for each industry subsector
in more detail. In 2015, those subsectors with the larges final energy
consumption in Europe were iron and steel, chemicals industry, non-metallic
mineral products, pulp, paper and printing, food, beverages and tobacco, and
non-ferrous metals \cite{IDEES}.

\subsection{Overview}
\label{sec:si:industry:overview}

Greenhouse gas emissions associated with industry can be classified into
energy-related and process-related emissions. Today, fossil fuels are used for
process heat energy in the chemicals industry, but also as a non-energy
feedstock for chemicals like ammonia (\ce{NH3}), ethylene (\ce{C2H4}) and
methanol (\ce{CH3OH}). Energy-related emissions can be curbed by using
low-emission energy sources. The only option to reduce process-related emissions
is by using an alternative manufacturing process or by assuming a certain rate
of recyling so that a lower amount of virgin material is needed.

The overarching modelling procedure can be described as follows. First, the
energy demands and process emissions for every unit of material output are
estimated based on data from the JRC-IDEES database \citeS{IDEES} and the fuel
and process switching described in the subsequent sections. Second, the 2050
energy demands and process emissions are calculated using the
per-unit-of-material ratios based on the industry transformations and the
country-level material production in 2015 \citeS{IDEES}, assuming constant
material demand as shown in \cref{tab:industryproduction}. Missing or too
coarsely aggregated data in the JRC-IDEES database \citeS{IDEES} is supplemented
with additional datasets: Eurostat energy-balances \citeS{}, USGS \citeS{usgs}
for ammonia production, DECHEMA \citeS{introzziDECHEMAGesellschaft} for methanol
and chlorine, and national statistics from Switzerland \citeS{}.

Where there are fossil and electrified alternatives for the same process (e.g.
in glass manufacture or drying) we assume that the process is completely
electrified. Current electricity demands (lighting, air compressors, motor
drives, fans, pumps) will remain electric. Where process heat is required our
approach depends on the temperature required
\citeS{naeglerQuantificationEuropean2015,rehfeldtBottomupEstimation2018}.
Processes that require temperatures below \SI{500}{\celsius} are supplied with
solid biomass, since we assume that residues and wastes are not suitable for
high-temperature applications (cf. \cref{sec:si:heat:supply}). We see solid
biomass use primarily in the pulp and paper industry, where it is already
widespread, and in food, beverages and tobacco, where it replaces natural gas.
Industries which require high temperatures (above \SI{500}{\celsius}), such as
metals, chemicals and non-metalic minerals are either electified where processes
already exist, or the heat is provided with synthetic methane. For Europe,
Rehfeldt et al. \citeS{rehfeldtBottomupEstimation2018} estimated that, from 2015
industrial heat demand, 45\% is above \SI{500}{\celsius}, 30\% within
\SIrange{100}{500}{\celsius}, 25\% below \SI{100}{\celsius}. Similarly, Naegler
et al. \citeS{naeglerQuantificationEuropean2015} estimate that 48\% is above
\SI{400}{\celsius}, 27\% within \SIrange{100}{400}{\celsius}, 25\% below
\SI{100}{\celsius}. Due to the high share of high-temperature process heat
demand, we disregard geothermal and solar thermal energy as source for process
heat. The final consumption of energy and non-energy feedstocks in industry
today in comparison to our future scenario in 2050 are presented in
\cref{fig:fec-industry}.

Inside each country the industrial demand is then distributed using the Hotmaps
Industrial Database, which is illustrated in \cref{fig:hotmaps} \citeS{piamanztobiasfleiterGeoreferencedIndustrial2018}. This
open database includes georeferenced industrial sites of energy-intensive
industry sectors in EU28, including cement, basic chemicals, glass, iron and
steel, non-ferrous metals, non-metallic minerals, paper, refineries subsectors.
The use of this spatial dataset enables the calculation of calculation of
regional and process specific energy demands.

% Demand for materials can either be met
% using existing conventional routes, such as blast furnaces for iron and rotary
% kilns for cement, or with new processes, such as hydrogen direct reduction for
% iron or kilns equipped with CCS. Basic chemicals can use green hydrogen as a
% feedstock rather than fossil fuels, for example for ammonia production, or using
% Fischer-Tropsch naphtha in steam crackers for the production of High Value
% Chemicals (HVCs).

% For the industry energy demand, we assume the same
% production of materials as today, but with process switching, fuel switching to
% low-emission alternatives as well as carbon capture for use or sequestration
% (CCU/S).

% Process switching (e.g. from blast furnaces to direct reduction and electric arc
% furnaces for steel) is defined exogenously. Fuel switching for process heat is
% predominantly also specified exogenously.

% Fuel switching means replacing mechanical processes and process heat
% applications that use fossil fuels with low-emission alternatives such as
% electricity, biomass or synthetic fuels such as hydrogen or methane. Feedstocks
% for the chemicals industry can also be converted to non-fossil alternatives. To
% determine where fuel switching can take place we use the breakdown of energy
% usage for each process within each sector provided by the JRC-IDEES database
% \citeS{IDEES}.

% material output per country from JRC IDEES, Eurostat energy balances, national
% statistics from Switzerland, and specific production data for methanol, ammonia,
% and chlroine to calculate total energy demand and process emissions by sector

% calculate energy demands and process emissions per unit of material based on JRC-IDEES

\citeS{neuwirthFuturePotential2022,graichenKlimaneutraleIndustrie,rootzenExploringLimits2013}

\subsection{Iron and Steel}
\label{sec:si:industry:steel}

Two alternative routes are used today to manufacture steel in Europe. The
primary route (integrated steelworks) represents 60\% of steel production, while
the secondary route (electric arc furnaces), represents the other 40\%
\citeS{lechtenbohmerDecarbonisingEnergy2016}.

The primary route uses blast furnaces in which coke is used to reduce iron ore
into molten iron.
\begin{align}
    \ce{CO2 + C &-> 2CO}, \\
    \ce{3Fe2O3 + CO &-> 2Fe3O4 + CO}, \\
    \ce{Fe3O4 + CO &-> 3FeO + CO2}, \\
    \ce{FeO + CO &-> Fe + CO2}.
\end{align}
% \begin{equation}
%     \ce{2Fe2O3 + 3C -> 4Fe + 3CO2}.
% \end{equation}
which is then converted to steel. The primary route of steelmaking implies large
process emissions of \SI{0.22}{\tco\per\tonne} of steel.

In the secondary route, electric arc furnaces (EAF) are used to melt scrap
metal. This limits the \co emissions to the burning of graphite electrodes
\citeS{Friedrichsen_2018}, and reduces process emissions to
\SI{0.03}{\tco\per\tonne} of steel.

Integrated steelworks can be replaced by direct reduced iron (DRI) and subsequent processing in an electric arc furnace (EAF)
\begin{align}
    \ce{3Fe2O3 + H2 &-> 2Fe3O4 + H2O}, \\
    \ce{Fe3O4 + H2 &-> 3FeO + H2O}, \\
    \ce{FeO + H2 &-> Fe + H2O}.
\end{align}
This circumvents the process emissions associated with the use of coke. For
hydrogen-based DRI we assume energy requirements of 1.7 MWh$_{H_2}$/t steel
\citeS{voglAssessmentHydrogen2018} and 0.322 MWh$_{el}$/t steel
\citeS{}.

The shares of steel produced via each of the three routes by 2050 is exogenously
set in the model. We assume that hydrogen-based DRI plus EAF replaces integrated
steelworks for primary production completely, representing 30\% of total steel
production (down from 60\%). The remaining 70\% (up from 40\%) are manufactured
through the secondary route using scrap metal in EAF. According to
\citeS{circular_economy}, circular economy practices even have the potential to
expand the share of the secondary route to 85\% by increasing the amount and
quality of scrap metal collected.

For the remaining subprocesses in this sector, the following transformations are
assumed. Methane is used as energy source for the smelting process. Activities
associated with furnaces, refining and rolling, product finishing are
electrified assuming the current efficiency values for these cases.
These transformations result in changes in process emissions as outlined in \cref{fig:process-emissions}.

\citeS{toktarovaInteractionElectrified2022, mandovaPossibilitiesCO22018, suopajarviUseBiomass2018,voglPhasingOut2021, bhaskarDecarbonizingPrimary}

\subsection{Chemicals Industry}
\label{sec:si:industry:chemicals}

The chemicals industry includes a wide range of diverse industries ranging from
the production of basic organic compounds (olefins, alcohols, aromatics), basic
inorcanic compounds (ammonia, chlorine), polymers (plastics), end-user products
(cosmetics, pharmaceutics).

The chemicals industry consumes large amounts of fossil-fuel based feedstocks
\citeS{leviMappingGlobal2018}, which can also be produced from renewables as
outlined for hydrogen in \cref{sec:si:h2:supply}, for methane in
\cref{sec:si:methane:supply}, and for oil-based products in
\cref{sec:si:oil:supply}. The ratio between synthetic and fossil-based fuels
used in the industry is an endogenous result of the optimisation.

The basic chemicals consumption data from the JRC IDEES \cite{IDEES} database
comprises high-value chemicals (ethylene, propylene and BTX), chlorine, methanol
and ammonia. However, it is necessary to separate out these chemicals because
their current and future production routes are different.

Statistics for the production of ammonia, which is commonly used as a
fertiliser, are taken from statistics provided by the United States Geological
Survey (USGS) for every country \citeS{}. Ammonia can be made from hydrogen and
nitrogen using the Haber-Bosch process \citeS{leviMappingGlobal2018}.
\begin{equation}
    \ce{N2 + 3H2 -> 2NH3}
\end{equation}
The Haber-Bosch process is not explicitly represented in the model, such that
demand for ammonia enters the model as a demand for hydrogen (6.5
MWh\ce{H2}/t\ce{NH3}) and electricity (1.17 MWh$_{el}$/t\ce{NH3})
\citeS{Wang2018Joule}. Today, natural gas dominates in Europe as the source for
the hydrogen used in the Haber-Bosch process, but the model can choose among the
various hydrogen supply options described in
\cref{sec:si:h2:supply}

The total production and specific energy consumption of chlorine and methanol is
taken from DECHEMA \citeS{introzziDECHEMAGesellschaft}. According to this
source, the production of chlorine amounts to 9.58 Mt$_{Cl}$/a, which is assumed
to require electricity at 3.6 MWh$_{el}$/t of chlorine and yield hydrogen at
0.9372 MWh$_{H_2}$/t of chlorine in the chloralkali process. The production of
methanol adds up to 1.5 Mt$_{MeOH}$/a, requiring electricity at 0.167 MWh$_{el}$/t
of methanol and methane at 10.25 MWh$_{CH_4}$/t of methanol.

The production of ammonia, methanol, and chlorine production is deducted from
the JRC IDEES basic chemicals, leaving the production totals of high-value
chemicals. For this, we assume that the liquid hydrocarbon feedstock comes from
synthetic or fossil-origin naphtha (14 MWh$_{naphtha}$/t of HVC, similar to
\citeS{lechtenbohmerDecarbonisingEnergy2016}), ignoring the methanol-to-olefin
route. Furthermore, we assume the following transformations of the
energy-consuming processes in the production of plastics: the final energy
consumption in steam processing is converted to methane since requires
temperature above \SI{500}{\celsius} (4.1 MWh$_{CH_4}$/t of HVC)
\citeS{rehfeldtBottomupEstimation2018}; and the remaining processes are
electrified using the current efficiency of microwave for high-enthalpy heat
processing, electric furnaces, electric process cooling and electric generic
processes (2.85 MWh$_{el}$/t of HVC).


The process emissions from feedstock in the chemical industry are as high as
\SI{0.369}{\tco\per\tonne} of ethylene equivalent. We consider process emissions
for all the material output, which is a conservative approach since it assumes
that all plastic-embedded \co will eventually be released into the atmosphere.
However, plastic disposal in landfilling will avoid, or at least delay,
associated \co emissions.

Circular economy practices reduce the amount of primary feedstock needed for the
production of plastics in the model and, consequently, also the level of process
emissions (\cref{fig:process-emissions}). We assume that 30\% of plastics are
mechanically recycled requiring 0.547 MWh$_{el}$/t of HVC
\citeS{meysCircularEconomy2020}, 15\% of plastics are chemically recycled
requiring 6.9 MWh$_{el}$/t of HVC based on pyrolysis and electric steam cracking
\citeS{materialeconomicsIndustrialTransformation2019}, and 10\% of plastics are
reused (equivalent to reduction in demand). The remaining 45\% need to be
produced from primary feedstock. In comparison, Material Economics
\citeS{circular_economy} presents a scenario with circular economy scenario with
27\% primary production, 18\% mechanical recycling, 28\% chemical recycling, and
27\% reuse. Another new-processes scenario has 33\% primary production, 14\%
mechanical recycling, 40\% chemical recycling, and 13\% reuse.


\citeS{circular_economy, kullmannCombiningWorlds2021,kullmannImpactsMaterial,kullmannValueRecycling,carolinaliljenstromDataSeparate, fuhrPlastikatlasDaten2019,conversioMaterialFlow2020,elserTakingEuropean, FuturePetrochemicals, nicholsonManufacturingEnergy2021, meysAchievingNetzero2021, thunmanCircularUse2019, introzziDECHEMAGesellschaft, meysCircularEconomy2020, guWastePlastics2017, boulamantiProductionCosts2017}

\subsection{Non-metallic Mineral Products}
\label{sec:si:industry:nmmp}

This subsector includes the manufacturing of cement, ceramics, and glass.

\subsubsection*{Cement}

Cement is used in construction to make concrete. The production of cement
involves high energy consumption and large process emissions. The calcination of
limestone to chemically reactive calcium oxide, also known as lime, involves
process emissions of \SI{0.54}{\tco\per\tonne} cement.
\begin{equation}
    \ce{CaCO3 -> CaO + \co}
\end{equation}
Additionally, \co is emitted from the combustion of fossil fuels to provide
process heat. Thereby, cement constitutes the biggest source of industry
process emissions in Europe (\cref{fig:process-emissions}).

Cement process emissions can be captured assuming a capture rate of 90\%
\citeS{}. Whether emissions are captured is decided by the model taking into
account the capital costs of carbon capture modules. The electricity and heat
demand of process emission carbon capture is currently ignored. For net-zero
emission scenarios, the remaining process emissions need to be compansated by
negative emissions.

With the exception of electricity demand and biomass demand for low-temperature
heat, the final energy consumption of this subsector is assumed to be supplied
by methane, which is capable of delivering the required high-temperature heat.
This implies a switch from burning solid fuels to burning gas which will require
adjustments of the kilns \citeS{akhtarCoalNatural2013}.

Other mitigation strategies to reduce energy consumption or process emissions
(using new raw materials, recovering unused cement from concrete at end of life,
oxyfuel cement production to facilitate carbon sequestration, electric kilns for
heat provision) are at a early development stage and have therefore not been
considered.

\citeS{fennellDecarbonizingCement2021, kuramochiComparativeAssessment2012, lechtenbohmerDecarbonisingEnergy2016}

% \citeS{lechtenbohmerDecarbonisingEnergy2016} has electrification of cement with 0.9~MWh\el/tClinker

\subsubsection*{Ceramics}

The ceramics sector is assumed to be fully electrified based on the current
efficiency of already electrified processes which include microwave drying and
sintering of raw materials, electric kilns for primary production processes,
electric furnaces for the product finishing. In total, the final electricity
consumption is 0.44 MWh/t of ceramic. The manufacturing of ceramics includes
process emissions of \SI{0.03}{\tco\per\tonne} of ceramic.

\citeS{furszyferdelrioDecarbonizingCeramics2022a}

\subsubsection*{Glass}

The production of glass is assumed to be fully electrified based on the current
efficiency of electric melting tanks and electric annealing which adds up to an
electricity demand of 2.07 MWh\el/t of glass
\citeS{lechtenbohmerDecarbonisingEnergy2016}. The manufacturing of glass incurs
process emissions of \SI{0.1}{\tco\per\tonne} of glass. Potential efficiency
improvements, which according to \citeS{lechtenbohmerDecarbonisingEnergy2016}
could reduce energy demands to 0.85~MWh\el/t of glass, have not been considered.

\citeS{furszyferdelrioDecarbonizingGlass2022}


\subsection{Non-ferrous Metals}
\label{sec:si:industry:nfm}

The non-ferrous metal subsector includes the manufacturing of base metals
(aluminium, copper, lead, zink), precious metals (gold, silver), and technology
metals (molybdenum, cobalt, silicon).

The manufacturing of aluminium accounts for more than half of the final energy
consumption of this subsector \citeS{}. Two alternative processing routes are
used today to manufacture aluminium in Europe. The primary route represents 40\%
of the aluminium production, while the secondary route represents the remaining
60\%.

The primary route involves two energy-intensive processes: the production of
alumina from bauxite (aluminium ore) and the electrolysis to transform alumina
into aluminium via the  Hall-H\'{e}roult process
\begin{equation}
    \ce{2Al2O3 + 3C -> 4Al + 3\co}.
\end{equation}
The primary route requires high-enthalpy heat (2.3 MWh/t) to produce alumina
which is supplied by methane and causes process emissions of
\SI{1.5}{\tco\per\tonne} aluminium. According to  \citeS{Friedrichsen_2018},
inert anodes might become commercially available by 2030 that would eliminate
the process emissions. However, they have not been considered in this study.
Assuming all subprocesses are electrified, the primary route requires 15.4
MWh$_{el}$/t of aluminium.

In the secondary route, scrap aluminium is remelted. The energy demand for this
process is only 10\% of the primary route and there are no associated process
emissions. Assuming all subprocesses are electrified, the secondary route
requires 1.7 MWh/t of aluminium. Following \citeS{Friedrichsen_2018}, we assume
a share of recycled aluminium of 80\% by 2050.

For the other non-ferrous metals, we assume the electrification of the entire
manufacturing process with an average electricity demand of 3.2 MWh\el/t lead
equivalent.

\subsection{Other Industry Subsectors}
\label{sec:si:industry:other}

The remaining industry subsectors include
(a) pulp, paper, printing,
(b) food, beverages, tobacco ,
(c) textiles and leather,
(d) machinery equipment,
(e) transport equipment,
(f) wood and wood products,
(g) others.
Low- and mid-temperature process heat in these industries is assumed to be supplied by biomass,
while the remaining processes are electrified.
None of the subsectors involve process emissions.

Energy demands for the agriculture, forestry and fishing sector per country are
taken from the JRC IDEES database \citeS{} (missing countries filled with
eurostat data \citeS{}) and are split into electricity (lighting, ventilation,
specific electricity uses, electric pumping devices), heat (specific heat uses,
low enthalpy heat) machinery oil (motor drives, farming machine drives,
diesel-fueled pumping devices). Heat demand is assigned at “services rural heat”
buses. Time series for demands are assumed to be constant and distributed inside
countries by population

\citeS{sovacoolDecarbonizingFood2021}


% Comparison

% \begin{itemize}
%     \item synergies paper (still see benefit of transmission, but MUCH bigger electrolysis with industry/aviation/shipping) \citeS{brownSynergiesSector2018}
%     \item JRC papers (Herib Blanco etc.) \citeS{blancoPotentialHydrogen2018,blancoPotentialPowertoMethane2018}
%     \item FZJ steel paper
%     \item PAC (uses solid biomass for non-energy requirements in chemicals industry, only 270 TWh in 2050, because of circular economy; phases out waste incineration) \citeS{caneurope/eebBuildingParis}
%     \item LTS from commission \citeS{in-depth_2018}
%     \item Material Economics reports \citeS{circular_economy,me2019}
% \end{itemize}