\begin{SCfigure}
    \includegraphics[width=0.66\textwidth]{fec_industry_today_tomorrow.pdf}
    \caption{Final consumption of energy and non-energy feedstocks in industry today (top bar) and
    our scenario for net-zero emissions by mid-century (bottom bar)}
    \label{fig:fec-industry}
\end{SCfigure}

\begin{SCfigure}
    \includegraphics[width=0.66\textwidth]{process-emissions.pdf}
    \caption{Process emissions in industry today (top bar) and mid-century without carbon capture (bottom bar)}
    \label{fig:process-emissions}
\end{SCfigure}


\begin{SCfigure}
    \includegraphics[width=0.66\textwidth]{hotmaps.pdf}
    \caption{Distribution of industries according to emissions data from the Hotmaps industrial sites database. Marker size is proportional to the industrial site's reported emission levels.}
    \label{fig:hotmaps}
\end{SCfigure}

\begin{SCfigure}
    \includegraphics[width=0.56\textwidth]{demand-map-process emission}
    \caption{Spatial distribution of industrial process emissions.}
    \label{fig:spatial-process-emissions}
\end{SCfigure}

\section{Industry Sector}
\label{sec:si:industry}

Industry demand is split into a dozen different sectors with specific energy
demands, process emissions of carbon dioxide, as well as existing and
prospective mitigation strategies. \cref{sec:si:industry:overview} provides a
general description of the modelling approach for the industry sector in
PyPSA-Eur-Sec. The following subsections describe the current energy demands,
available mitigation strategies, and whether mitigation is exogenously fixed or
co-optimised with the other components of the model for each industry subsector
in more detail. In 2015, those subsectors with the larges final energy
consumption in Europe were iron and steel, chemicals industry, non-metallic
mineral products, pulp, paper and printing, food, beverages and tobacco, and
non-ferrous metals \citeS{europeancommission.jointresearchcentre.JRCIDEESIntegrated2017}.

\subsection{Overview}
\label{sec:si:industry:overview}

Greenhouse gas emissions associated with industry can be classified into
energy-related and process-related emissions (for the spatial distribution of
European process emissions see \cref{fig:spatial-process-emissions}). Today,
fossil fuels are used for process heat energy in the chemicals industry, but
also as a non-energy feedstock for chemicals like ammonia (\ce{NH3}), ethylene
(\ce{C2H4}) and methanol (\ce{CH3OH}). Energy-related emissions can be curbed by
using low-emission energy sources. The only option to reduce process-related
emissions is by using an alternative manufacturing process or by assuming a
certain rate of recyling so that a lower amount of virgin material is needed.

The overarching modelling procedure can be described as follows. First, the
energy demands and process emissions for every unit of material output are
estimated based on data from the JRC-IDEES
database \citeS{europeancommission.jointresearchcentre.JRCIDEESIntegrated2017}
and the fuel and process switching described in the subsequent sections. Second,
energy demands and process emissions for a climate-neutral Europe by mid-century
are calculated using the per-unit-of-material ratios based on the industry
transformations and the country-level material production in
2015 \citeS{europeancommission.jointresearchcentre.JRCIDEESIntegrated2017},
assuming constant material demand. Missing or too coarsely aggregated data in
the JRC-IDEES
database \citeS{europeancommission.jointresearchcentre.JRCIDEESIntegrated2017} is
supplemented with additional datasets: Eurostat energy
balances \citeS{eurostatEnergyBalances2021}, USGS for ammonia
production \citeS{unitedstatesgeologicalsurveyAmmoniaProduction2021}, DECHEMA for
methanol and chlorine \citeS{bazzanellaLowCarbon2017}, and national statistics from
Switzerland \citeS{bundesamtfurenergieEnergieverbrauchIndustrie2021}.

Where there are fossil and electrified alternatives for the same process (e.g.
in glass manufacture or drying) we assume that the process is completely
electrified. Current electricity demands (lighting, air compressors, motor
drives, fans, pumps) will remain electric. Where process heat is required our
approach depends on the temperature required
\citeS{naeglerQuantificationEuropean2015,rehfeldtBottomupEstimation2018}.
Processes that require temperatures below \SI{500}{\celsius} are supplied with
solid biomass, since we assume that residues and wastes are not suitable for
high-temperature applications (\cref{sec:si:heat:supply}). We see solid biomass
use primarily in the pulp and paper industry, where it is already widespread,
and in food, beverages and tobacco, where it replaces natural gas. Industries
which require high temperatures (above \SI{500}{\celsius}), such as metals,
chemicals and non-metalic minerals are either electified where suitable
processes already exist, or the heat is provided with synthetic methane.
Hydrogen for high-temperature process heat was not considered in our
scenarios \citeS{neuwirthFuturePotential2022}. For Europe, Rehfeldt et al.
\citeS{rehfeldtBottomupEstimation2018} estimated that, from 2015 industrial heat
demand, 45\% is above \SI{500}{\celsius}, 30\% within
\SIrange{100}{500}{\celsius}, 25\% below \SI{100}{\celsius}. Similarly, Naegler
et al. \citeS{naeglerQuantificationEuropean2015} estimate that 48\% is above
\SI{400}{\celsius}, 27\% within \SIrange{100}{400}{\celsius}, 25\% below
\SI{100}{\celsius}. Due to the high share of high-temperature process heat
demand, we disregard geothermal and solar thermal energy as source for process
heat. The final consumption of energy and non-energy feedstocks in industry
today in comparison to our scenarios for net-zero emissions by mid-century are
presented in \cref{fig:fec-industry}.

Inside each country the industrial demand is then distributed using the Hotmaps
Industrial Database, which is illustrated in \cref{fig:hotmaps} \citeS{manzGeoreferencedIndustrial2018}. This open database includes
georeferenced industrial sites of energy-intensive industry sectors in EU28,
including cement, basic chemicals, glass, iron and steel, non-ferrous metals,
non-metallic minerals, paper, refineries subsectors. The use of this spatial
dataset enables the calculation of regional and process specific energy demands.
This approach assumes that there will be no significant migration of
energy-intensive industries like, for instance, studied by Toktarova et
al.~\citeS{toktarovaInteractionElectrified2022} for the steel industry.

\subsection{Iron and Steel}
\label{sec:si:industry:steel}

Two alternative routes are used today to manufacture steel in Europe. The
primary route (integrated steelworks) represents 60\% of steel production, while
the secondary route (electric arc furnaces), represents the other 40\%
\citeS{lechtenbohmerDecarbonisingEnergy2016}.

The primary route uses blast furnaces in which coke is used to reduce iron ore
into molten iron.
\begin{align}
    \ce{CO2 + C &-> 2CO}, \\
    \ce{3Fe2O3 + CO &-> 2Fe3O4 + CO}, \\
    \ce{Fe3O4 + CO &-> 3FeO + CO2}, \\
    \ce{FeO + CO &-> Fe + CO2}.
\end{align}
which is then converted to steel. The primary route of steelmaking implies large
process emissions of \SI{0.22}{\tco\per\tonne} of steel, amounting to 7\% of
global greenhouse gas emissions \citeS{voglPhasingOut2021}.

In the secondary route, electric arc furnaces (EAF) are used to melt scrap
metal. This limits the \co emissions to the burning of graphite electrodes
\citeS{Friedrichsen_2018}, and reduces process emissions to
\SI{0.03}{\tco\per\tonne} of steel.

Integrated steelworks can be replaced by direct reduced iron (DRI) and subsequent processing in an electric arc furnace (EAF)
\begin{align}
    \ce{3Fe2O3 + H2 &-> 2Fe3O4 + H2O}, \\
    \ce{Fe3O4 + H2 &-> 3FeO + H2O}, \\
    \ce{FeO + H2 &-> Fe + H2O}.
\end{align}
This circumvents the process emissions associated with the use of coke. For
hydrogen-based DRI we assume energy requirements of 1.7 MWh$_{H_2}$/t steel
\citeS{voglAssessmentHydrogen2018} and 0.322 MWh$_{\text{el}}$/t steel
\citeS{hybritSummaryFindings2021}.

The shares of steel produced via each of the three routes by mid-century is exogenously
set in the model. We assume that hydrogen-based DRI plus EAF replaces integrated
steelworks for primary production completely, representing 30\% of total steel
production (down from 60\%). The remaining 70\% (up from 40\%) are manufactured
through the secondary route using scrap metal in EAF. According to
a Material Economics report \citeS{circular_economy}. circular economy practices even have the potential to
expand the share of the secondary route to 85\% by increasing the amount and
quality of scrap metal collected. Bioenergy as alternative to coke in blast
furnaces has not been considered \citeS{mandovaPossibilitiesCO22018,suopajarviUseBiomass2018}.

For the remaining subprocesses in this sector, the following transformations are
assumed. Methane is used as energy source for the smelting process. Activities
associated with furnaces, refining and rolling, product finishing are
electrified assuming the current efficiency values for these cases.
These transformations result in changes in process emissions as outlined in \cref{fig:process-emissions}.

\subsection{Chemicals Industry}
\label{sec:si:industry:chemicals}

The chemicals industry includes a wide range of diverse industries ranging from
the production of basic organic compounds (olefins, alcohols, aromatics), basic
inorcanic compounds (ammonia, chlorine), polymers (plastics), end-user products
(cosmetics, pharmaceutics).

The chemicals industry consumes large amounts of fossil-fuel based feedstocks
\citeS{leviMappingGlobal2018}, which can also be produced from renewables as
outlined for hydrogen in \cref{sec:si:h2:supply}, for methane in
\cref{sec:si:methane:supply}, and for oil-based products in
\cref{sec:si:oil:supply}. The ratio between synthetic and fossil-based fuels
used in the industry is an endogenous result of the optimisation.

The basic chemicals consumption data from the JRC IDEES \citeS{europeancommission.jointresearchcentre.JRCIDEESIntegrated2017} database
comprises high-value chemicals (ethylene, propylene and BTX), chlorine, methanol
and ammonia. However, it is necessary to separate out these chemicals because
their current and future production routes are different.

Statistics for the production of ammonia, which is commonly used as a
fertiliser, are taken from the United States Geological Survey (USGS) for every
country \citeS{unitedstatesgeologicalsurveyAmmoniaProduction2021}. Ammonia can
be made from hydrogen and nitrogen using the Haber-Bosch process \citeS{leviMappingGlobal2018}.
\begin{equation}
    \ce{N2 + 3H2 -> 2NH3}
\end{equation}
The Haber-Bosch process is not explicitly represented in the model, such that
demand for ammonia enters the model as a demand for hydrogen (6.5
MWh$_{\ce{H2}}$/t$_{\ce{NH3}}$) and electricity (1.17 MWh$_{\text{el}}$/t$_{\ce{NH3}}$)
\citeS{wangGreeningAmmonia2018}. Today, natural gas dominates in Europe as the source for
the hydrogen used in the Haber-Bosch process, but the model can choose among the
various hydrogen supply options described in
\cref{sec:si:h2:supply}

The total production and specific energy consumption of chlorine and methanol is
taken from a DECHEMA report \citeS{bazzanellaLowCarbon2017}. According to this
source, the production of chlorine amounts to 9.58 Mt$_{\ce{Cl}}$/a, which is assumed
to require electricity at 3.6 MWh$_{\text{el}}$/t of chlorine and yield hydrogen at
0.937 MWh$_{\ce{H2}}$/t of chlorine in the chloralkali process. The production of
methanol adds up to 1.5 Mt$_{\ce{MeOH}}$/a, requiring electricity at 0.167 MWh$_{\text{el}}$/t
of methanol and methane at 10.25 MWh$_{\ce{CH4}}$/t of methanol.

The production of ammonia, methanol, and chlorine production is deducted from
the JRC IDEES basic chemicals, leaving the production totals of high-value
chemicals. For this, we assume that the liquid hydrocarbon feedstock comes from
synthetic or fossil-origin naphtha (14 MWh$_{\text{naphtha}}$/t of HVC, similar
to Lechtenböhmer et al.~\citeS{lechtenbohmerDecarbonisingEnergy2016}), ignoring
the methanol-to-olefin route. Furthermore, we assume the following
transformations of the energy-consuming processes in the production of plastics:
the final energy consumption in steam processing is converted to methane since
requires temperature above \SI{500}{\celsius} (4.1 MWh$_{\ce{CH4}}$/t of
HVC) \citeS{rehfeldtBottomupEstimation2018}; and the remaining processes are
electrified using the current efficiency of microwave for high-enthalpy heat
processing, electric furnaces, electric process cooling and electric generic
processes (2.85 MWh$_{\text{el}}$/t of HVC).

The process emissions from feedstock in the chemical industry are as high as
\SI{0.369}{\tco\per\tonne} of ethylene equivalent. We consider process emissions
for all the material output, which is a conservative approach since it assumes
that all plastic-embedded \co will eventually be released into the atmosphere.
However, plastic disposal in landfilling will avoid, or at least delay,
associated \co emissions.

Circular economy practices drastically reduce the amount of primary feedstock
needed for the production of plastics in the model
\citeS{kullmannValueRecycling2022,meysAchievingNetzero2021,meysCircularEconomy2020,guWastePlastics2017} and, consequently, also the energy demands
and level of process emissions \citeS{nicholsonManufacturingEnergy2021} (see
\cref{fig:process-emissions}). We assume that 30\% of plastics are mechanically
recycled requiring 0.547 MWh$_{\text{el}}$/t of HVC
\citeS{meysCircularEconomy2020}, 15\% of plastics are chemically recycled
requiring 6.9 MWh$_{\text{el}}$/t of HVC based on pyrolysis and electric steam
cracking \citeS{materialeconomicsIndustrialTransformation2019}, and 10\% of
plastics are reused (equivalent to reduction in demand). The remaining 45\% need
to be produced from primary feedstock. In comparison, Material Economics
\citeS{circular_economy} presents a scenario with circular economy scenario with
27\% primary production, 18\% mechanical recycling, 28\% chemical recycling, and
27\% reuse. Another new-processes scenario has 33\% primary production, 14\%
mechanical recycling, 40\% chemical recycling, and 13\% reuse.

\subsection{Non-metallic Mineral Products}
\label{sec:si:industry:nmmp}

This subsector includes the manufacturing of cement, ceramics, and glass.

\subsubsection*{Cement}

Cement is used in construction to make concrete. The production of cement
involves high energy consumption and large process emissions. The calcination of
limestone to chemically reactive calcium oxide, also known as lime, involves
process emissions of \SI{0.54}{\tco\per\tonne} cement \citeS{fennellDecarbonizingCement2021}.
\begin{equation}
    \ce{CaCO3 -> CaO + \co}
\end{equation}
Additionally, \co is emitted from the combustion of fossil fuels to provide
process heat. Thereby, cement constitutes the biggest source of industry
process emissions in Europe (\cref{fig:process-emissions}).

Cement process emissions can be captured assuming a capture rate of 90\%
\citeS{DEA}. Whether emissions are captured is decided by the model taking
into account the capital costs of carbon capture modules. The electricity and
heat demand of process emission carbon capture is currently ignored. For
net-zero emission scenarios, the remaining process emissions need to be
compansated by negative emissions.

With the exception of electricity demand and biomass demand for low-temperature
heat (0.06~MWh/t and 0.2~MWh/t), the final energy consumption of this subsector is assumed to be supplied
by methane (0.52 MWh/t), which is capable of delivering the required high-temperature heat.
This implies a switch from burning solid fuels to burning gas which will require
adjustments of the kilns \citeS{akhtarCoalNatural2013}.

Other mitigation strategies to reduce energy consumption or process emissions
(using new raw materials, recovering unused cement from concrete at end of life,
oxyfuel cement production to facilitate carbon
sequestration, electric kilns for heat
provision) are at a early development stage and have therefore not been
considered \citeS{kuramochiComparativeAssessment2012}.

\subsubsection*{Ceramics}

The ceramics sector is assumed to be fully electrified based on the current
efficiency of already electrified processes which include microwave drying and
sintering of raw materials, electric kilns for primary production processes,
electric furnaces for the product finishing \citeS{europeancommission.jointresearchcentre.JRCIDEESIntegrated2017}. In total, the final
electricity consumption is 0.44 MWh/t of ceramic. The manufacturing of ceramics
includes process emissions of \SI{0.03}{\tco\per\tonne} of ceramic. For a
detailed overview of the ceramics industry sector see Furszyfer Del Rio et
al. \citeS{furszyferdelrioDecarbonizingCeramics2022a}.

\subsubsection*{Glass}

The production of glass is assumed to be fully electrified based on the current
efficiency of electric melting tanks and electric annealing which adds up to an
electricity demand of 2.07 MWh\el/t of glass
\citeS{lechtenbohmerDecarbonisingEnergy2016}. The manufacturing of glass incurs
process emissions of \SI{0.1}{\tco\per\tonne} of glass. Potential efficiency
improvements, which according to Lechtenböhmer et al.~\citeS{lechtenbohmerDecarbonisingEnergy2016}
could reduce energy demands to 0.85~MWh\el/t of glass, have not been considered.
For a detailed overview of the glass industry sector see Furszyfer Del Rio et
al.~\citeS{furszyferdelrioDecarbonizingGlass2022}

\subsection{Non-ferrous Metals}
\label{sec:si:industry:nfm}

The non-ferrous metal subsector includes the manufacturing of base metals
(aluminium, copper, lead, zink), precious metals (gold, silver), and technology
metals (molybdenum, cobalt, silicon).

The manufacturing of aluminium accounts for more than half of the final energy
consumption of this subsector. Two alternative processing routes are
used today to manufacture aluminium in Europe. The primary route represents 40\%
of the aluminium production, while the secondary route represents the remaining
60\%.

The primary route involves two energy-intensive processes: the production of
alumina from bauxite (aluminium ore) and the electrolysis to transform alumina
into aluminium via the  Hall-H\'{e}roult process
\begin{equation}
    \ce{2Al2O3 + 3C -> 4Al + 3\co}.
\end{equation}
The primary route requires high-enthalpy heat (2.3 MWh/t) to produce alumina
which is supplied by methane and causes process emissions of
\SI{1.5}{\tco\per\tonne} aluminium. According to Friedrichsen et al. \citeS{Friedrichsen_2018},
inert anodes might become commercially available by 2030 that would eliminate
the process emissions. However, they have not been considered in this study.
Assuming all subprocesses are electrified, the primary route requires 15.4
MWh$_{\text{el}}$/t of aluminium.

In the secondary route, scrap aluminium is remelted. The energy demand for this
process is only 10\% of the primary route and there are no associated process
emissions. Assuming all subprocesses are electrified, the secondary route
requires 1.7 MWh/t of aluminium. Following Friedrichsen et al. \citeS{Friedrichsen_2018}, we assume
a share of recycled aluminium of 80\% by mid-century.

For the other non-ferrous metals, we assume the electrification of the entire
manufacturing process with an average electricity demand of 3.2 MWh\el/t lead
equivalent.

\subsection{Other Industry Subsectors}
\label{sec:si:industry:other}

The remaining industry subsectors include (a) pulp, paper, printing, (b) food,
beverages, tobacco, (c) textiles and leather, (d) machinery equipment, (e)
transport equipment, (f) wood and wood products, (g) others. Low- and
mid-temperature process heat in these industries is assumed to be supplied by
biomass \citeS{sovacoolDecarbonizingFood2021}, while the remaining processes are
electrified. None of the subsectors involve process emissions.

Energy demands for the agriculture, forestry and fishing sector per country are
taken from the JRC IDEES database \citeS{europeancommission.jointresearchcentre.JRCIDEESIntegrated2017}. Missing countries are filled
with eurostat data \citeS{eurostatEnergyBalances2021}. Agricultural energy
demands are split into electricity (lighting, ventilation, specific electricity
uses, electric pumping devices), heat (specific heat uses, low enthalpy heat)
machinery oil (motor drives, farming machine drives, diesel-fueled pumping
devices). Heat demand is for this sector is classified as services rural heat.
Time series for demands are assumed to be constant and distributed inside
countries in proportion to population.

