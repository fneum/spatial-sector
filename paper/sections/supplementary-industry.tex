\section{Industry Sector}
\label{sec:si:industry}

\citeS{graichenKlimaneutraleIndustrie,rootzenExploringLimits2013}

Industry demand is split into a dozen different sectors, with the major ones
being iron and steel, cement and basic chemicals. The location of existing
industrial facilities is based on the Horizon-2020-funded project hotmaps.
Demand for materials can either be met using existing conventional routes, such
as blast furnaces for iron and rotary kilns for cement, or with new processes,
such as hydrogen direct reduction for iron or kilns equipped with CCS and/or
oxyfuel for cement. Basic chemicals can use green hydrogen as a feedstock rather
than fossil fuels, for example for ammonia production, or using Fischer-Tropsch
naphtha in steam crackers for High Value Chemicals.

Fossil fuels are used for process heat in the chemicals industry, but also as a feedstock for chemicals like
ammonia (NH3), ethylene (C2H4) and methanol (CH3OH)

\begin{SCfigure}
    \includegraphics[width=0.7\textwidth]{fec_industry_today_tomorrow.pdf}
    \caption{Final consumption of energy and non-energy feedstocks in industry today (left bar) and
    our future scenario in 2050 (right bar)}
    \label{fig:fec-industry}
\end{SCfigure}

\begin{SCfigure}
    \includegraphics[width=0.7\textwidth]{process-emissions.pdf}
    \caption{Process emissions in industry today (top bar) and in 2050 (bottom bar)}
    \label{fig:process-emissions}
\end{SCfigure}

\begin{SCfigure}
    \includegraphics[width=0.7\textwidth]{hotmaps.pdf}
    \caption{Distribution of industries.}
    \label{fig:hotmaps}
\end{SCfigure}

\begin{table}[t]
    \centering
    \setlength{\tabcolsep}{6pt}
    \begin{tabular}{@{} p{5cm}r @{}}
      \toprule
      Material & Production [\si{\mega\tonne\per\year}] \\
      \midrule
      steel & 174 \\
      cement & 186 \\
      glass & 44 \\
      ceramics and other NMM & 360 \\
      ammonia & 18 \\
      other basic chemicals & 74 \\
      other chemicals & 28 \\
      pulp & 39 \\
      paper & 94 \\ \bottomrule
    \end{tabular}
    \caption{Industrial production for main products in 2015.}
    \label{tab:industryproduction}
  \end{table}


emissions
- energy-related
- process-related

For process-related emissions need alternative manufacturing process
or higher rates of recycling such that less virgin material is needed

largest sectors (according to FEC)
- iron and steel
- chemicals
- non-metallic mineral products
- pulp paper printing
- food beverages tobacco
- non-ferrous metals

Demand

For the industry energy demand, we assume the same
production of materials as today, but with process switching, fuel switching to
low-emission alternatives as well as carbon capture for use or sequestration
(CCU/S).

Process switching includes, for example, moving primary production of
steel from blast furnace reduction to direct reduction with hydrogen, or
switching to an oxyfuel process in cement manufacture.

Fuel switching means
replacing mechanical processes and process heat applications that use fossil
fuels with low-emission alternatives such as electricity, biomass or synthetic
fuels such as hydrogen or methane.

Feedstocks for the chemicals industry are
also converted to non-fossil alternatives.

To determine where fuel switching can take place we use the breakdown of energy
usage for each process within each sector provided by the JRC-IDEES database
\citeS{IDEES}.

Where there are fossil and electrified alternatives for the same
process (e.g. in glass manufacture or drying in industry XX) we assume that the
process is completely electrified.

Where process heat is required (steam
process?) our approach depends on the temperature required
\citeS{naeglerQuantificationEuropean2015,rehfeldtBottomupEstimation2018}. Processes that require temperatures below
\SI{500}{\celsius} are supplied with solid biomass, since we assume that residues and
wastes are not suitable for high-temperature applications. We see solid biomass
use primarily in the pulp and paper industry, where it is already widespread,
and in food, beverages and tobacco, where it replaces natural gas. Industries
which require high temperatures (above \SI{500}{\celsius}), such as metals, chemicals
and non-metalic minerals are either electified where processes already exist, or
the heat is provided with synthetic methane.

what is currently supplied with electricity (lighting, air compressors, motor drives, fans, pumps)
and low-enthalpy heat demand are directly added to electricity and heat buses

Industry heat demand can be supplied via DH or is added to services heat demand.
Can be supplied by other processes producing heat as byproduct (DAC, FT)

For EU-28 plus NO, CH, IL Rehfeldt estimated that from 2015 industrial heat demand
45\% is above \SI{500}{\celsius}, 30\% \SIrange{100}{500}{\celsius}, 25\% below \SI{100}{\celsius}

\citeS{naeglerQuantificationEuropean2015} similarly:
48\% is above \SI{400}{\celsius}, 27\% \SIrange{100}{400}{\celsius}, 25\% below \SI{100}{\celsius}

Because of high share of high-temperature process heat demand
- no geothermal
- no solar thermal supply

process heat supplied by methane, biomass, electricity depending on sector

Based on materials demand from JRC-IDEES and other sources such as the USGS for
ammonia.

Industry is split into many sectors, including iron and steel, ammonia, other
basic chemicals, cement, non-metalic minerals, alumuninium, other non-ferrous
metals, pulp, paper and printing, food, beverages and tobacco, and other more
minor sectors.

Inside each country the industrial demand is distributed using the Hotmaps
Industrial Database.

Hotmaps open database includes cement, basic chemicals, glass, iron and steel, non-ferrous metals,
non-metallic minerals, paper, refineries. Enables regional analyses, calculation of
site-specific energy demand, waste heat potentials, emissions, market shares,
process-specific evaluations

calculate energy demands and process emissions per unit of material based on JRC-IDEES

material output per country from JRC IDEES, ammonia production statistics,
Eurostat energy balances, national statistics from Switzerland to calculate total energy demand
and process emissions by sector

material output assumed to stay constant (exception recycling)

Supply

Process switching (e.g. from blast furnaces to direct reduction and electric arc
furnaces for steel) is defined exogenously.

Fuel switching for process heat is mostly also done exogenously.

Solid biomass is used for up to 500 Celsius, mostly in paper and pulp and food
and beverages.

Higher temperatures are met with methane.

\citeS{neuwirthFuturePotential2022}

\subsection{Iron and Steel}
\label{sec:si:industry:steel}

\citeS{toktarovaInteractionElectrified2022, mandovaPossibilitiesCO22018, suopajarviUseBiomass2018,voglPhasingOut2021, bhaskarDecarbonizingPrimary}

70\% from scrap, rest from direct reduction with 1.7 MWh H2 / t steel + electric arc (process emissions 0.03 t COs / t steel)

Two routes today to manufacture steel in Europe
- primary/integrated steelworks (60\% of steel production)
- secondary/electric arc furnaces (40\%) \citeS{lechtenbohmerDecarbonisingEnergy2016}

Primary - Integrated steelworks
- use blast furnaces in which coke is used to reduce iron ore into molten iron
- coke is used as a reducing agent in blast furnaces for smelting iron ore
\begin{equation}
    \ce{Fe2O3 + 3CO -> 2Fe + 3CO2}
\end{equation}
\begin{equation}
    \ce{FeO + 3CO -> Fe + 3CO2}
\end{equation}
\begin{equation}
    \ce{2Fe2O3 + 3C -> 4Fe + 3CO2}
\end{equation}
- this is then converted to steel
- The primary route implies large process emissions (0.22 t\co/t steel)

Secondary - electric arc furnaces (EAF)
- use EAF to melt scrap metal
- this limits \co emissions to burning of graphite electrodes \citeS{Friedrichsen_2018}
- 0.03 t\co/t steel

DRI - Direct Reduced Iron
\begin{equation}
    \ce{FE2O3+H2 -> 2FeO + H2O}
\end{equation}
\begin{equation}
    \ce{FeO + H2 -> Fe + H2O}
\end{equation}
- this is processed in EAF and circumvents associated process emissions
- today, DRI uses methane as reduction agent but we assume this to be substituted with hydrogen
- needs 1.7 MWh H2 / t steel \citeS{voglAssessmentHydrogen2018} and 0.322 MWh elec / t steel \url{https://ssabwebsitecdn.azureedge.net/-/media/hybrit/files/hybrit_brochure.pdf}

share of steel produced by hydrogen-based DRI + EAF is exogenous: 30\%
integrated steelworks: 0\%
scrap + EAF: 70\%

replace integrated steelworks with DRI + EAF

NB: DRI already done with hydrogen in Trinidad and in Abu Dhabi (where \co from
SMR is captured and used for enhanced oil recovery) H2 Future, HYBRIT, SALCOS

\citeS{circular_economy} circular economy practices have potential of expanding
the share of secondary route to 85\% just by increasing the amount and quality of scrap metal

process emissions from lime (added to remove impurity?), graphite anodes AND
from carbon added to get steel alloy  \citeS{voglAssessmentHydrogen2018}

EAF not good for flat steel in automotive

NB: derived gases not included in model, since take from blast furnace top
gases, originating in coke

Why not add CCS to existing plants? Hard for some reason \citeS{kuramochiComparativeAssessment2012}.

For the remaining subprocesses in this sector:
- methane as energy source for smelting process
- activities associated with furnaces, refining and rolling, product finishing are electrified

\subsection{Chemicals Industry}
\label{sec:si:industry:chemicals}

\citeS{elserTakingEuropean, FuturePetrochemicals, nicholsonManufacturingEnergy2021, meysAchievingNetzero2021, thunmanCircularUse2019, introzziDECHEMAGesellschaft, meysCircularEconomy2020, guWastePlastics2017, boulamantiProductionCosts2017}

wide range of diverse industries ranging from:
- basic organics compounds (olefins, alcohols, aromatics)
- basic inorcanic compounds (ammonia, chlorine)
- polymers (plastics)
- end-user products (cosmetics, pharmaceutics)

chemicals industry consumes lots of fossil-fuel based feedstocks \citeS{leviMappingGlobal2018}

basic chemicals: HVC (high-value chemicals), chlorine, methanol and ammonia

Recycling!
- specify reuse, primary production, and mechanical and chemical recycling
fraction of platics

\citeS{circular_economy, kullmannCombiningWorlds2021,kullmannImpactsMaterial,kullmannValueRecycling,carolinaliljenstromDataSeparate, fuhrPlastikatlasDaten2019,conversioMaterialFlow2020}

synthetic naphtha for primary production


Ammonia production data is taken from the USGS

Natural gas dominates in Europe as source for hydrogen for Haber-Bosch process.
Replace with electrolytic/clean hydrogen.


ammonia production from USGS (United States Geological Survey) statistics

Assumptions for existing ammonia energy demand from XX and for electrolytic
process from XX.

Ammonia, used for fertiliser, can be made from hydrogen and nytrogen using the Haber-Bosch-Process
hydrogen can be combined with Nitrogen to obtain ammonia in the Haber-Bosch process \citeS{leviMappingGlobal2018}
\begin{equation}
    \ce{N2 + 3H2 -> 2NH3}
\end{equation}
Currently, H2 for ammonia industry in Europe is from SMR, in the model can be SMR, SMR CC, electrolysers
\co often captured from SMR for urea production (20\% of fertilizer in Europe,
but 50-60\% in China and India) and food and beverages
with electrolytic-H2 we have 6.5 MWh \ce{H2} / t \ce{NH3} and 1.17 MWh elec / t \ce{NH3} (Wang, 2018 Joule)

chlorine, aromatics, olefins

ammonia is subtracted from basic chemicals in IDEES database.


ALL of liquid hydrocarbon feedstock from FT naphtha.
- LPG, diesel oil, residual fuel oil

Ethylene, used for plastics, can be made by steam cracking from naphtha or ethane or via methanol-to-olefin processes

We assume that all of carbon eventually finds its way from product into
atmosphere. But this is wierd, since there is no release from landfill, plastics
are not biodegradable. If they are burned as waste, we should use a
waste-to-energy plant, which we currently don't have in the model.

\citeS{lechtenbohmerDecarbonisingEnergy2016} has 14.8 MWh FT-naphtha per ton of HVC, and 2.7 MWh per
ton for processing; 3.1 t\co/tHVC are captured from flue gases.

TODO: We have much smaller emissions, are we accounting for heating as well as
process emission? I guess with methane we do...

share of synthetic vs fossil-fuel based methane and naphtha is endogenous

transformation of energy-consuming processes:
- FEC in steam processing is converted to methane since requires temperature above \SI{500}{\celsius} \citeS{rehfeldtBottomupEstimation2018}
- remaining processes are electrified using the current efficiency  of microwave for high-enthalpy heat processing,
  electric furnaces, electric process cooling and electric generic processes

process emissions from feedstock in the chemical industry represent 0.369 t\co / t ethylene eq.
- we consider process emissions for all the material output
- conservative, because it assumes that all plastic-embedded \co will eventually be released into the atmosphere
- plastic disposal in landfilling will avoid, or at least delay, associated \co emissions

\subsection{Non-metallic Mineral Products}
\label{sec:si:industry:nmmp}

Cement

\citeS{fennellDecarbonizingCement2021}

waste and solid biomass; capture of \co emissions

cement manufacturing involves large process and energy emissions

calcination of limestone to chemically reactive calcium oxide (known as lime)
involves process emissions of 0.54 t \ce{\co} / t cement
\begin{equation}
    \ce{CaCO3 -> CaO + \co}
\end{equation}

cement is used in construction to make concrete

Co2 is emitted from fossil fuels to provide process heat and from the calcination reaction ofr fossil limestone.

This is the biggest source of process emissions in industry in Europe.

Unless alternatives can be found for cement, can do CCS or CCU

mitigation strategies (new raw materials, recovering unused cement from concrete at end of life \url{https://ec.europa.eu/clima/news/commission-calls-climate-neutral-europe-2050_en})
are at a very early development stage and have not been considered

Keep current share of biomass; replace fossil fuels for process heat with
synthetic methane (exact share determined by the model)

assume that FEC (except electricity, low-temperature heat and biomass consumption)
is supplied by methane which can deliver the required high-temperature heat

NB: CH4 burns differently in kiln, not straightforward, see IEEE paper
\citeS{akhtarCoalNatural2013}

process emissions are captured and, for net-zero emission scenarios,
they need to be compensated by negative emissions

Do post-combustion carbon capture on emissions and account for heat and
electricity for capture with aqueous amine solution. The classic amine process
based on monoethanolamine (MEA) Some suggest need CHP for this. DEA input
assumptions in 2030 are 0.72 MWh/t\co heat and 0.022 MWh/t\co electricity
per outputed tonne of \co assuming a capture rate of 90\%. Additional
electricity for \co compression to 150 bar and dehydration is 0.085
MWh/t\co.

Alternative: use oxyfuel so you get concentrated \co. No heat requirement,
electricity only used for O$_2$ air separation unit (ASU) of 0.08 MWh/t\co
and 0.17 MWh/t\co for post treatment (unlike post-combustion capture, output
is impure, so have to remove O$_2$ and N$_2$ by cryogenic distillation). Can
avoid ASU by using O$_2$ from electrolysis. Electricity use higher than
post-combustion, but don't have big heat requirement.

Beware biomass share in IDEES-Industry is higher for EU28 (2920 ktoe in 2015)
than in IDEES-EnergyBalance (906 ktoe). This is because IDEES is putting waste
(including non-renewable waste) together with biomass under biomass.

\citeS{kuramochiComparativeAssessment2012}: The major difference between centralized power plants and
industrial plants such as cement plants is that the former have large quantities
of low-grade heat that can be used for solvent regeneration, whereas the latter
generally do not [42].


Can use oxyfuel for cement to make CCS easier. Already being done, can retrofit
to older plant
\url{https://engineered.thyssenkrupp.com/en/oxyfuel-a-climate-neutral-cement-production-is-getting-closer/}

NEED oxyfuel, otherwise CCS requirements are too high.

TODO: How much O$_2$ from electrolysis? How much O$_2$ required by cement per
ton? For current combustion materials, around 0.17 tO$_2$/tCement => 29
MtO$_2$/a needed, but have 672 MtO$_2$/a from electrolysis; H2 for steel and
ammonia probably is enough to provide O$_2$ for oxyfuel cement (see private.org
- have plenty of O2 from electrolysis)

NB: for assessments of electricity for oxyfuel cement, deduct unneeded ASU (air
separation unit via cryogenic distillation) since don't need 200-300 kWh/tO$_2$
(consistent with 50-60 EUR/tO$_2$ price).

NB: cement production is used to using solid fuels. Natural gas needs some
aadjustment of firing mechanisms
\citeS{akhtarCoalNatural2013}.

Some people push electric kilns for heat, but considered less mature than gas
and certainly less than solid

CemZero project
\url{https://group.vattenfall.com/what-we-do/roadmap-to-fossil-freedom/industry-decarbonisation/cementa},
apparently poo-pooed here \url{https://www.cementa.se/sv/cemzero}.


\citeS{lechtenbohmerDecarbonisingEnergy2016} has electrification of cement with 0.9~MWh\el/tCLinker
(12\% efficiency improvement in thermal demand compared to 2010)



Ceramics

complete electrification because many already electrified processes:
- microwave drying and sintering of raw materials,
- electric kilns for primary production processes
- electric furnaces for the product finishing
- FEC 0.44 MWh/t ceramics
- process emissions: 0.03 t\co/t ceramic

\citeS{furszyferdelrioDecarbonizingCeramics2022a}

Glass

Electrify everything.
- electric melting tanks
- electric annealing
- electricity demand 2.07 MWh/t of glass \citeS{lechtenbohmerDecarbonisingEnergy2016}

\citeS{furszyferdelrioDecarbonizingGlass2022}


\citeS{lechtenbohmerDecarbonisingEnergy2016} also has big efficiency improvement with 0.85~MWh\el/tGlass.

\subsection{Non-ferrous Metals}
\label{sec:si:industry:nfm}

includes
- base metals (aluminium, copper, lead, zink)
- precious metals (gold, silver)
- technology metals (molybdenum, cobalt, silicon)

Aluminium

80\% recycling, for rest: methane for high-enthalpy heat (bauxite to alumina) followed by electrolysis (process emissions 1.5 t \co / t Al)

more than half of the FEC of this sector

Two alternative production routes today to manufacture aluminium in Europe today
- primary route: 40\%
- secondary route: 60\%
- exogenous: increase by 2050 to 80\% \citeS{Friedrichsen_2018} and \url{https://ec.europa.eu/clima/news/commission-calls-climate-neutral-europe-2050_en}

Primary route:
- two energy-intensive processes
- production of alumina from bauxite (aluminium ore)
- electrolysis to transfrom alumina to aluminium via the Hall-H\'{e}roult process
\begin{equation}
    \ce{2Al2O3 + 3C -> 4Al + 3\co}
\end{equation}
- primary route requires high-enthalpy heat to produce alumina - supplied by methane
- 1.5 t\co/t aluminium
- inert anodes might be commercially available in 2030 avoiding processe emissions \citeS{Friedrichsen_2018} but not considered

Secondary route:
- scrap aluminium is remelted
- energy demand is 10\% of primary route and process emissions are avoided
- assuming all subprocesses in this route electrified: 1.7 MWh elec / t aluminium

Other non-ferrous metals, electrification of entire manufacturing process is assumed
- 3.2 MWh / t lead equivalent

\subsection{Other Industry Subsectors}
\label{sec:si:industry:other}

energy demands and \co emissions for the agriculture, forestry and fishing
sector

pulp, paper, printing:

Already high share of biomass. Keep and add biomass for paper production, since
temperatures required are low.

food, beverages, tobacco \citeS{sovacoolDecarbonizingFood2021}

textiles and leather

machinery equipment


transport equipment

wood and wood products

otherwise

low- and mid-temperature process heat in these industries is assumed to be supplied by biomass
while the remaining processes are electrified

Comparison

\begin{itemize}
    \item synergies paper (still see benefit of transmission, but MUCH bigger electrolysis with industry/aviation/shipping) \citeS{brownSynergiesSector2018}
    \item JRC papers (Herib Blanco etc.) \citeS{blancoPotentialHydrogen2018,blancoPotentialPowertoMethane2018}
    \item FZJ steel paper
    \item PAC (uses solid biomass for non-energy requirements in chemicals industry, only 270 TWh in 2050, because of circular economy; phases out waste incineration) \citeS{caneurope/eebBuildingParis}
    \item LTS from commission \citeS{in-depth_2018}
    \item Material Economics reports \citeS{circular_economy,me2019}
\end{itemize}