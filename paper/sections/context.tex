
Many different combinations of infrastructure could make Europe carbon-neutral
by mid-century, but not all solutions meet the same level of acceptance. For
example, power grid reinforcements have faced many delays, despite their value
for integrating renewables. A hydrogen network reusing gas pipelines could
substitute for moving cheap but remote renewables across the continent to where
demand is.

We study trade-offs between new transmission lines and a hydrogen network in the
European energy system with net-zero \co emissions. We find that a hydrogen
network consistently reduces system costs and that large parts could use
repurposed gas pipelines. Energy transport as electrons and molecules offer
complementary strengths, achieving highest cost savings together. However,
neither is essential as long as the system can be coordinated around the
resulting bottlenecks. This reveals many affordable ways to achieve net-zero
emissions in Europe, giving policymakers different options to choose from.
