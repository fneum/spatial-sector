
% TODO 1000 characters in 1-2 paragraphs

Many combinations of infrastructure could make Europe climate-neutral by
mid-century. But not all solutions meet the same level of acceptance. For
example, power transmission reinforcements experience many delays, despite their
value for integrating renewable electricity. A hydrogen network which can reuse
gas pipelines could offer a substitute for moving cheap but remote renewable
energy across the continent to where demand is.

We study such trade-offs between building new transmission lines and developing
a hydrogen network in the European energy system with all sectors represented
and net-zero \co emissions. We find that a hydrogen network is consistently
beneficial infrastructure and that a large part could repurpose unused gas
pipelines. Energy transport as electrons and molecules offer complementary
strengths, achieving highest cost savings together. But neither is truly
essential for a low-cost system. This means that there are many ways to achieve
net-zero emissions in Europe that are affordable, giving policymakers different
options to choose from.
