
Many different combinations of infrastructure could make Europe climate-neutral
by mid-century, but not all solutions meet the same level of acceptance. For
example, power transmission reinforcements have experienced many delays, despite their
value for integrating renewable electricity. A hydrogen network which can reuse
gas pipelines could be a substitute for moving cheap but remote renewable
energy across the continent to where demand is.

We study trade-offs between new transmission lines and a hydrogen network in the
European energy system with all sectors represented and net-zero \co emissions.
We find that a hydrogen network consistently reduces system costs and that large
parts could use repurposed gas pipelines. Energy transport as electrons and
molecules offer complementary strengths, achieving highest cost savings
together. However, neither is essential as long as the system can be coordinated
around the resulting bottlenecks. This means that there are many affordable ways
to achieve net-zero emissions in Europe, giving policymakers different options
to choose from.
