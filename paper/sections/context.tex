
%1000 characters in 1-2 paragraphs

Many different combinations of infrastructure would allow Europe to become
climate-neutral by 2050. But not all solutions meet the same level of
acceptance. For example, power transmission reinforcements have suffered many
delays in recent decades, despite their importance for integrating renewable
electricity. A hydrogen network which can reuse natural gas pipelines could
offer a substitute for moving cheap but remote renewable energy across the
continent to where demand is.

We study such trade-offs between building new electricity transmission lines and
developing a new network of hydrogen pipelines in the European energy system
with all sectors represented and net-zero \co emissions. We find that a hydrogen
backbone is consistently beneficial infrastructure and that a large part could
repurpose unused gas pipelines. Energy transport as electrons and molecules
offer complementary strengths, achieving highest cost savings together.
%But
%neither is essential for a low-cost system.
