\section{Model Overview}
\label{sec:si:model-overview}


PyPSA-Eur-Sec is an open model dataset of the European energy system at the
transmission network level that covers the electricity, heating, transport and
industry sectors. PyPSA-Eur-Sec builds a linear optimisation problem to plan
energy system infrastructure from various open data sources using the workflow
management tool Snakemake,\citeS{snakemake} which is then solved with the
commercial solver Gurobi.\citeS{gurobi} The overall circulation of energy and
carbon is shown in \cref{fig:multisector}. The modelling approaches for the
items listed there are described in detail in the following sections
\crefrange{sec:si:electricity}{sec:si:carbon-management}. A mathematical
formulation of the model is provided in \cref{sec:si:math}. The clustered model
resolution is shown in \cref{fig:clustered-networks} together with the existing
electricity and gas grid capacities. The carriers electricity, hydrogen, gas, heat
and biomass are nodally resolved, whereas other carriers like oil and
carbon dioxide are copperplated in the current version to reduce the problem's
computational burden.

% TODO decide if gas nodally resolved or not