\section{Model Overview}
\label{sec:si:model-overview}

\begin{figure}
    \centering
    \includegraphics[]{../graphics/multisector_figure.pdf}
\end{figure}

\begin{figure}
    \centering
\begin{subfigure}[t]{0.49\textwidth}
    \centering
    % \caption{clustered electricity network}
    \includegraphics[width=\textwidth]{electricity-network-today-map.pdf}
\end{subfigure}
\begin{subfigure}[t]{0.49\textwidth}
    \centering
    % \caption{gas network}
    \includegraphics[width=\textwidth]{gas-network-today-map.pdf}
\end{subfigure}
\caption{Clustered electricity and gas transmission networks.}
\label{fig:clustered-networks}
\end{figure}

\begin{figure}
    \centering
    \includegraphics[height=0.25\textheight]{total-annual-demand.pdf}
    \includegraphics[height=0.25\textheight]{ts-demand.pdf}
    \caption{Final energy demand temporal diversity.}
    \label{fig:demand-time}
\end{figure}

\newgeometry{top=0.5cm, bottom=1.5cm}
\begin{figure}
    \centering
    \begin{subfigure}[t]{0.49\textwidth}
        \centering
        \caption{electricity demand}
        \label{fig:demand-space:electricity}
        \includegraphics[width=\textwidth]{demand-map-electricity.pdf}
    \end{subfigure}
    \begin{subfigure}[t]{0.49\textwidth}
        \centering
        \caption{hydrogen demand}
        \label{fig:demand-space:hydrogen}
        \includegraphics[width=\textwidth]{demand-map-H2.pdf}
    \end{subfigure}
    \begin{subfigure}[t]{0.49\textwidth}
        \centering
        \caption{methane demand}
        \label{fig:demand-space:methane}
        \includegraphics[width=\textwidth]{demand-map-gas.pdf}
    \end{subfigure}
    \begin{subfigure}[t]{0.49\textwidth}
        \centering
        \caption{heat demand}
        \label{fig:demand-space:heat}
        \includegraphics[width=\textwidth]{demand-map-heat.pdf}
    \end{subfigure}
    \begin{subfigure}[t]{0.49\textwidth}
        \centering
        \caption{oil demand}
        \label{fig:demand-space:oil}
        \includegraphics[width=\textwidth]{demand-map-oil.pdf}
    \end{subfigure}
    \begin{subfigure}[t]{0.49\textwidth}
        \centering
        \caption{solid biomass demand}
        \label{fig:demand-space:biomass}
        \includegraphics[width=\textwidth]{demand-map-solid biomass.pdf}
    \end{subfigure}
    \caption{Final energy demand spatial diversity.}
    \label{fig:demand-space}
\end{figure}
\restoregeometry



Not all of the sectors are at the full nodal resolution, and some demand for
some sectors is distributed to nodes using heuristics that need to be corrected.
Some networks are copper-plated to reduce computational times.

overnight scenario

no pathway

weather year 2013

High resolution multi-sectoral approach: Developing the most advanced open source modeling system with regard to spatial, sectoral, technological and temporal resolution
- All energy infrastructures in one optimisation problem (electricity, gas, heat, industry)
- Detailed transmission grid representation in EU-wide model
- Detailed representation of demand sectors: Industry, buildings, transport
- so that the variability of demand and variable renewable supply can be represented, and so that existing grid bottlenecks are visible.

Energy sector coupling, storage and conversion is modelled to connect
electricity, heating (individual buildings, district heating and industry),
transport and gas (methane, hydrogen and carbon dioxide) in the different
sectors (buildings, transport and industry).

The spatial resolution of the model can be customised by the user. It can be set
at electricity substation level, based on administrative boundaries such as
NUTS2 or country-level, or to a custom number of nodes to enhance computational
performance. The spatial resolution of the input data varies: conventional power
plant locations are known exactly, as are large industrial facilities; wind and
solar time series are regionalised based on the underlying ERA5 reanalysis data
(around 20km by 20km); heat and transport demand at NUTS3; electricity demand
time series are at TSO control region level.

Electricity can be converted to heat via heat pumps or resistive heaters; to
hydrogen gas, or further to methane and liquid hydrocarbons; or to work in
various demand devices. Methane can be reformed to hydrogen, and most fuels can
be used for electricity generation in turbines or fuel cells.

Biomass can be used in electricity generation with and
without CCS, as well as to provide low- to medium-temperature process heat in
industry.

Energy/material storage can be optimised in PyPSA-Eur-Sec including conventional
pumped hydro storage, electrochemical storage like Lithium ion batteries,
storage of gases including methane, hydrogen and carbon dioxide, storage of
liquid fuels, as well as thermal energy storage in the form of hot water both in
individual buildings and in district heating networks.

Carbon capture is
needed in the model both to capture and sequester process emissions with a
fossil origin, such as those from calcination of fossil limestone in the cement
industry, as well as to provide carbon for the production of hydrocarbons for
dense transport fuels and as a chemical feedstock, for example for the plastics
industry. Therefore, carbon cycles are carefully tracked in PyPSA-Eur-Sec.

compared to electricity, heating and transport are strongly peaked
- heating is strongly seasonal but also with synoptic variations
- transport has strong daily periodicity