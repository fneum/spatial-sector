Electricity transmission expansion has suffered many delays in Europe in recent
decades, despite its significance for integrating renewable electricity into the
energy system. A hydrogen network which reuses the existing fossil gas network
could not only help to supply demand for low-emission fuels, but could also to
balance variations in wind and solar energy across the continent and thus avoid
power grid expansion. We pursue this idea by varying the allowed expansion of
electricity and hydrogen grids in net-zero \co scenarios for a sector-coupled
and self-sufficient European energy system with high shares of renewables. We
cover the electricity, buildings, transport, agriculture, and industry sectors
across 181 regions and model every third hour of a year. With this high
spatio-temporal resolution, the model can capture bottlenecks in transmission
networks, the variability of demand and renewable supply, as well as regional
opportunities for the retrofitting of legacy gas infastructure and development
of geological hydrogen storage. Our results show consistent system cost
reductions with a pan-continental hydrogen network that connects regions with
low-cost and abundant renewable potentials to demand centers, synthetic fuel
production and cavern storage sites. Developing a hydrogen network reduces
system costs by up to \maxhybenefitabs~bn\euro/a (\maxhybenefitrel\%), with
highest benefits when electricity grid reinforcements cannot be realised.
Between 64\% and 69\% of this network could be built from repurposed natural gas
pipelines. However, we find that hydrogen networks can only partially substitute
for power grid expansion. While the expansion of both networks together can
achieve the largest cost savings of \gridbenefitrel\%, the expansion of neither
is truly essential as long as higher costs can be accepted and regulatory
changes are made to manage grid bottlenecks.

