Europe's electricity transmission expansion suffers many delays, despite its
significance for integrating renewable electricity. A hydrogen network reusing
the existing gas network could not only help to supply demand for low-emission
fuels, but could also balance variations in wind and solar energy across the
continent and thus avoid power grid expansion. Our investigation varies the
allowed expansion of electricity and hydrogen grids in net-zero \co scenarios
for a sector-coupled European energy system, capturing transmission bottlenecks,
renewable supply and demand variability, and pipeline retrofitting and
geological storage potentials. We find that a hydrogen network connecting
regions with low-cost and abundant renewable potentials to demand centers,
electrofuel production and cavern storage sites reduces system costs by up to
\maxhybenefitabs~bn\euro/a (\maxhybenefitrel\%). While expanding both networks
together can achieve the largest cost reductions by \gridbenefitrel\%, the
expansion of neither is essential in a net-zero system as long as higher costs
can be accepted, and flexibility options allow managing transmission
bottlenecks.
