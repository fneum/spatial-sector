Electricity transmission expansion has suffered many delays in Europe in recent
decades, despite its importance for integrating renewable electricity into the
energy system. A hydrogen network which reuses the existing fossil gas network
would not only help supply demand for low-emission fuels, but could also help to
balance variations in wind and solar energy across the continent and thus avoid
power grid expansion. We pursue this idea by varying the allowed expansion of
electricity and hydrogen grids in net-zero \co scenarios for a sector-coupled
European energy system with high shares of renewables and self-sufficient
supply. We cover the electricity, buildings, transport, agriculture, and
industry sectors across 181 regions and model every third hour of a year. With
this high spatio-temporal resolution, the model can capture bottlenecks in
transmission networks, the variability of demand and renewable supply, as well
as regional opportunities for the reuse of legacy gas infastructure and
geological hydrogen storage. Our results show a consistent benefit of a
pan-continental hydrogen backbone that connects high-yield regions with demand
centers, synthetic fuel production and cavern storage sites. Developing a
hydrogen network reduces system costs by up to 6\%, with highest benefits when
electricity grid reinforcements cannot be realised. Between 58\% and 66\% of
this backbone could be built from repurposed natural gas pipelines. However, we
find that hydrogen networks can only partially substitute for power grid
expansion, and that both can achieve strongest cost savings of 12\% together.