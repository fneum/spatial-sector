Can we substitute for the electricity grid by producing eleoctrolytic hydrogen
and transport it through a hydrogen pipeline network?


cross-sectoral approaches are important to reduce CO2 emissions cost-effectively and for flexibility

there are many trade-offs between unpopular infrastructure and system cost

One of the biggest changes seen in the energy system is the built-out of
hydrogen infastructure: huge new electrolyzer capacities, underground storage in
salt caverns as well as a new hydrogen pipeline network.

through limiting power grid expansion, hydrogen network infrastructure

Thus, both ways of transporting energy are important to achieve highest
cost savings.

e.g. limiting power grid expansion costs 47-62 billion per year more

If onshore wind expansion is restricted too, costs rise by further 104 billion per year

but many near-optimal compromise energy systems with equally low cost but higher acceptance

hydrogen networks can partially substitute for power grid expansion, but system costs are
2\% higher

can also get away with neither power grid expansion nor hydrogen network

all results depend strongly on assumptions and modelling approach
- therefore openness and transparency are critical, guaranteed by open licenses for data and code