In this work, we have investigated the potential role of a hydrogen network in net-zero
\co scenarios for Europe with high shares of renewables. Analysis was performed
using the open sector-coupled energy system model PyPSA-Eur-Sec featuring high
spatio-temporal coverage of all energy sectors (electricity, buildings,
transport, agriculture and industry across 181 regions and 3-hourly resolution
for a year). With this level of spatial, temporal, technological and sectoral
resolution, it is possible to represent grid bottlenecks as well as the
variability and regional distribution of demand and renewable supply. Thereby,
the system's infrastructure needs regarding generation, storage, transmission
and conversion can be assessed. This includes in particular trade-offs between
electricity grid reinforcement, which has limited public support, and developing
a hydrogen network, for which increasingly unused gas pipelines can be
repurposed.

Besides large-scale renewables expansion, the build-out of hydrogen
infrastructure is one of the biggest changes seen in our scenarios of the future
European energy system. Huge new electrolyser capacities enter the system and
operate flexibly to aid renewables integration. Furthermore, underground storage
in salt caverns is developed for seasonal balancing and a new continent-spanning
hydrogen pipeline network is built to connect cheap supply and storage
potentials with demand centres. This new hydrogen network is found to be
supported by considerable amounts of gas pipeline retrofitting: between \minretroshare\% and
\maxretroshare\% of the network uses repurposed pipelines.

Our analysis reveals that a hydrogen network can reduce system costs by up to
\maxhybenefitrel\%. Cost reductions are shown to be highest when the expansion of the power grid is
restricted. However, hydrogen networks can only partially substitute for grid
expansion. We found that in fact both ways of transporting energy and balancing
renewable generation complement each other and achieve the highest cost savings
of up to \gridbenefitrel\% together. At the same time, these findings also support the
interpretation that neither electricity nor hydrogen network expansion are
essential for achieving a cost-effective system design if such a cost premium
can be accepted to achieve alternative goals.

In conclusion, there appear to be many infrastructure trade-offs regarding how
energy is transported across Europe with limited impacts on total system cost
such that policymakers could choose from a wide range of near-optimal compromise
energy system designs with equally low cost but possibly higher acceptance.

% cross-sectoral approaches are important to reduce CO2 emissions cost-effectively and for flexibility
% If onshore wind expansion is restricted too, costs rise by further 104 billion per year
% all results depend strongly on uncertain assumptions and model simplifications
% - openness and transparency are critical
% Electricity grid reinforcements could reduce system costs by up to 8\%.