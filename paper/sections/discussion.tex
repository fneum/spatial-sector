To put our results into a broader perspective, for the discussion we compare
them to related literature and proposals presented in the gas industry's
European Hydrogen Backbone reports. This is followed by an appraisal of the
limitations of our study and a derivation of policy implications based on
spatial and operational insights.

\subsection*{Comparison to Related Literature}

Compared to the net-zero scenarios from the European
Commission,\cite{in-depth_2018} we see much larger wind and solar electricity
generation reaching beyond 8600~TWh compared to approximately 5700~TWh. This is
also reflected in the capacities built that exceed 2000~GW in our scenarios
compared to 1200~GW for wind and 3500~GW compared to 1000~GW for
solar.\cite{in-depth_2018} In terms of total electricity produced, our results
approximately show a tripling of today's generation compared to an increase by
145\% in the Commission's net-zero scenarios.\cite{in-depth_2018} Roughly one
third goes to regular electricity demand, one third goes to newly electrified
sectors in heating, transport and industry, and another third goes to hydrogen
production (dominated by demand for liquid hydrocarbons). The major difference
to the Commission's scenarios\cite{in-depth_2018} is caused by their lower
electrification rates, a 15\% share of nuclear power in the electricity mix,
higher biomass usage across all sectors (2900~TWh/a versus our 1400~TWh/a), and
a strong reliance on fossil fuels imports (2900~TWh/a) for non-energy uses (e.g.
plastics and other high-value chemicals). By considering landfill of plastics as
long-term carbon sequestration option, the Commission's scenarios see little
need to produce synthetic hydrocarbons for non-energy feedstocks. On the
contrary, our modelling, which assumes that all carbon in waste will be
incinerated or eventually decay into the atmosphere and limited sequestration
potentials, requires sustainable carbon sources for green electrofuels and
precludes the wide-ranging use of fossil oil.

Using pathway optimisation, Victoria et
al.~\cite{victoriaSpeedTechnological2022} investigate the timing of when certain
technologies become important for the European energy transition, and find a
hydrogen network consistently appearing after 2035. However, owing to a
one-node-per-country resolution in that study, little can be said about
subnational network infrastructure needs, retrofitting opportunities for gas
pipelines or regional geological storage potentials. Compared to our findings,
limited network expansion options affect total energy system costs less in
Victoria et al.~\cite{victoriaSpeedTechnological2022} A doubling of today's
transmission volume reduces cumulative system costs between 2020 and 2050 by 2\%
in Victoria et al.,~\cite{victoriaSpeedTechnological2022} compared to
\maxacbenefitrel\% in this study. Disabling hydrogen network expansion increases
cumulative costs by 0.5\% in Victoria et
al.,\cite{victoriaSpeedTechnological2022} compared to \maxhybenefitrel\% in this
study. This discrepancy arises because country-internal transmission bottlenecks
are not captured, whereby the integration costs of remote resources like
offshore wind within the countries are neglected.

Caglayan et al.~\cite{Caglayan2019} also consider European decarbonisation
scenarios with both electricity and hydrogen networks, but at lower spatial
resolution (96 regions) and without the industry, shipping, aviation,
agriculture and non-electrified heating sectors. A similar pattern of hydrogen
pipeline expansion towards the British Isles and North Sea is seen, but lower
overall electrolyser capacities (258~GW compared to our
\SIrange{\minelectrolysis}{\maxelectrolysis}{\giga\watt}) because not all
sectors are included. Caglayan et al.~\cite{Caglayan2019} also find cost-optimal
hydrogen storage of 130~TWh, whereas our scenarios involve just between
\hydrogenstorageacnhyn~and \hydrogenstorageacnhyy~TWh owing to the larger
flexible hydrogen demand diminishing the need for weekly and monthly balancing.

A large number of cost-effective designs for a climate-neutral European energy
system was also presented by Pickering et
al.~\cite{pickeringDiversityOptions2022} Their 98-region model with 2-hourly
resolution likewise includes all energy sectors including non-energy feedstocks
and also assumes energy self-sufficiency for Europe. However, hydrogen transport
options were not considered such that hydrogen must be produced locally.
Moreover, geological potentials for low-cost underground hydrogen storage and
the option to retrofit gas pipelines are not included. Owing to higher storage
cost in steel tanks and fewer assumed end-uses of hydrogen and its derivatives,
the scenarios involve less hydrogen storage (\SIrange{0}{6}{\twh} versus
\SIrange{\hydrogenstorageacnhyn}{\hydrogenstorageacnhyy}{\twh}) and lower
electrolyser capacities (\SIrange{290}{855}{\giga\watt} versus
\SIrange{\minelectrolysis}{\maxelectrolysis}{\giga\watt}) in our results.
Furthermore, whereas our model allows limited use of fossil fuels and with
options for carbon capture and sequestration, Pickering et
al.~\cite{pickeringDiversityOptions2022} eliminate the use of fossil energy and
only consider direct air capture as a carbon source. Overall, total energy
system costs lie in a similar range between 730 and 866 bn\euro/a compared to
costs between \minsystemcost~and \maxsystemcost~bn\euro/a in our study.

\subsection*{Comparison to the European Hydrogen Backbone}

Our results are aligned with the European Hydrogen Backbone
(EHB).\cite{gasforclimateEuropeanHydrogen2020,gasforclimateExtendingEuropean2021,gasforclimateEuropeanHydrogen2021,gasforclimateEuropeanHydrogen2022}
Whereas no detailed modelling lies behind the visions in the EHB reports, we
present analysis based on temporally resolved spatial co-planning of energy
infrastructures. We see cost-optimal hydrogen network investments in the range
of \minhycost-\maxhycost~bn\euro/a, while the EHB report covering 21 countries
finds slightly higher costs between
4-10~bn\euro/a.\cite{gasforclimateExtendingEuropean2021}\footnote{To calculate
the annuity of the overnight hydrogen network costs listed in the EHB reports, a
lifetime of 50 years and a discount rate of 7\% are assumed.} The extension to
28 countries reports costs between 7-14
bn\euro/a.\cite{gasforclimateEuropeanHydrogen2022} Compared to the hydrogen
backbone vision presented in the EHB from April
2021,\cite{gasforclimateExtendingEuropean2021}\footnote{The newer EHB report
from April 2022 \cite{gasforclimateEuropeanHydrogen2022} lacks sufficient data
to calculate length-weighted network capacities.} our scenarios show a
similarly sized hydrogen network with comparable retrofitting shares. Measured
by the length-weighted sum of pipeline capacities (TWkm), the 309 TWkm indicated
in the EHB report match the upper end of the range of 204-307~TWkm observed in
our scenarios. Likewise, the 69\% share of repurposed natural gas pipelines
\cite{gasforclimateExtendingEuropean2021} roughly agrees with our findings where
between \minretroshare\% and \maxretroshare\% of hydrogen pipelines are
retrofitted gas pipelines. In contrast to the EHB reports, we also explore
solutions without a hydrogen network, which we find to be feasible as well.

\subsection*{Limitations of the study and scope for future investigations}
\label{sec:limitations}

In our main scenarios, Europe is largely energy self-sufficient. While limited
amounts of fossil gas and oil imports are allowed, no imports of renewable
electricity, chemical energy carriers or commodities from outside of Europe are
considered. However, including green imports may change system needs for
electricity and hydrogen transmission infrastructure substantially. New hydrogen
import hubs might require different bulk transmission routes. The import of
large amounts of carbon-based fuels and ammonia would furthermore diminish the
demand for hydrogen overall, and hence also the need to transport it. This
effect of wide-ranging imports of liquid hydrocarbon demand on infrastructure
needs is demonstrated in a sensitivity analysis in
\cref{sec:si:sensitivity-imports} and should be explored in more detail in
future work.
\cite{fasihiTechnoeconomicAssessment2019,heuserTechnoeconomicAnalysis2019,hamppImportOptions2023}

Additionally, the very uneven distribution of energy supply in our results may
interfere with the level of social acceptance for new infrastructure to an
extent that may block a swift energy transition.
\cite{sasseDistributionalTradeoffs2019,sasseRegionalImpacts2020,sasseLowcarbonElectricity2023} Hence, future
investigations should weigh the cost surcharge of increased regionally
self-sufficient energy supply against the potential benefit of higher public
acceptance and increased resilience.

Previous research has shown that the system design can be changed in many ways
with only a small change in total
costs.\cite{Neumann2019,lombardiPolicyDecision2020,pedersenModelingAll2021,pickeringDiversityOptions2022}
This breadth of options makes robust statements about specific locational
infrastructure needs more vague. While we present selected design trade-offs
regarding transmission networks and some further sensitivities
(\cref{sec:si:sensitivity}), a more comprehensive exploration of near-optimal
solutions would be prudent, especially in the directions of carbon management
infrastructure, biomass usage, the level of energy imports, industry transition
and relocation options, more regionally balanced infrastructure, and increased
system resilience.

Owing to the absence of pathway optimisation, our results cannot offer insights
into the required transition steps and how the gradual transformation may
restrict certain options towards the final climate-neutral state. For example,
our results do not show which parts of gas network could be repurposed first or
where the benefit of a hydrogen network might be the highest initially. In the
context of multi-horizon planning, we also neglect the dynamics of technological
learning by
doing.\cite{heubergerPowerCapacity2017,fellingMultihorizonPlanning2022,zeyenEndogenousLearning2022}
The transformation to net-zero emissions requires vast and timely growth rates
of power-to-X and carbon dioxide removal technologies to realise anticipated
cost reductions,\cite{odenwellerProbabilisticFeasibility2022a} which we assume to
be given by assuming fixed technology cost.

The need for high spatial resolution to evaluate the competition between
electricity and hydrogen networks required a compromise on the temporal
resolution, owing to computational constraints. Even though hourly resolution
would have been more desirable and generally viable with the available raw data,
our analysis employs a 3-hourly resolution to be able to focus on spatial
detail, which is important to resolve transmission bottlenecks and to examine
what infrastructure options can cost-effectively integrate high-yield generation
sites with demand clusters, synthetic fuel production and geological storage
potentials. As quantified in the supplemental sensitivity analysis in
\cref{sec:si:sensitivity-time}, the temporal aggregation however tends to
underestimate demand peaks, as well as the expansion of offshore wind and
battery storage for short-term balancing. Conversely, it will overestimate the
development solar photovoltaics to some extent, as solar feed-in fluctuations
are smoothed. Nevertheless, the coarser resolution still captures the dominant
intraday, daily, weekly, and seasonal patterns, as well as the investment
patterns and interactions of electricity and hydrogen transport which are focus
of this study. Moreover, the sector-coupled system's numerous demand
flexibilities (such as electrolysers, heat pumps, and electric vehicle
batteries) provide a wide range of options for managing hour-to-hour variations
partially.

Further limitations include that heat demands and the availability of renewables
vary considerably year by year such that our restriction to a single year may
limit the robustness to interannual weather variability; we do not consider new
nuclear power plants; we do not spatially resolve the \co resource and
infrastructure; we do not consider secondary benefits of grid expansion for the
provision of ancillary services; and for the transport and industry sectors we
make some exogenous assumptions about process switching, drive trains,
alternative fuels for industry heat and recycling rates which may have turned
out differently if they were endogenously optimised.

\subsection*{Derivation of policy implications from regional and operational insights}
\label{sec:policy}

Regardless of the energy carrier transported, our results highlight that
cooperation between European countries is important to reach net-zero \co
emissions most cost-effectively. This is because there are significant
differences in renewable resources across Europe. The cost differential between
supply in Europe's demand centers and periphery outweigh the cost of building
new transmission infrastructure. Thus, we see both substantial net importers
(e.g.~the industry clusters in Ruhr valley and Rotterdam area) and strong net
exporters of energy (e.g.~Denmark, Ireland, Spain, Greece). The option to
transport energy around Europe also counteracts incentives for industry
relocation. Expanding energy transport infrastructure may be less controversial
since it would affect regional development less than the migration of
industries.

Regarding hydrogen production, we see both solar-based hubs in Southern Europe
and wind-based hubs in Northern Europe using water electrolysis. The regional
and technological diversity in electrolytic hydrogen production is the preferred
solution, but the impetus for Southern solar-based hubs is greatly affected by
the evolution of other system components. Difficulties to install sufficient
onshore wind capacities around the North Sea would reinforce their relevance,
whilst the import of most liquid hydrocarbons from outside of Europe would
weaken the case for solar-based hubs. Our results also highlight that compared
to the amount of electrolytic hydrogen, blue hydrogen from steam methane
reforming with carbon capture only plays a marginal role and was only used in
our scenarios when no hydrogen network could be developed.

As the general hydrogen network benefit is not dependent on electricity grid
reinforcements, both networks could be developed in parallel. Thus, policymaking
could focus on options that are most easily achieved and widely accepted. While
the hydrogen network benefit is not affected by alternate technology cost
developments or import policies, the network topology is. Lower costs of solar
photovoltaics raise the appeal of hydrogen production hubs in Southern Europe,
altering the suitable hydrogen network layout. Likewise, wide-ranging hydrogen
imports from the MENA region would need to be supported with transmission
infrastructure in Southern Europe.

The flexible operation of electrolysers has several advantages for system
stability and integrating wind and solar generation cost-effectively and should
be incentivised. Fluctuating renewable generation is buffered in geological
hydrogen storage primarily in the UK, Denmark, Spain and Greece to achieve more
continuous production in capital-intensive fuel synthesis plants in accordance
with their operational restrictions. This leads to low curtailment rates of
renewables and a lower requirement for firm capacity, outlining the benefit of
cross-sectoral approaches for reducing \co emissions cost-effectively. Fuel cell
CHP plants in Germany can further support grid operation when the power grid
cannot be expanded. However, energetically the re-electrification of hydrogen
only plays a minor role in this sector-coupled system.

To reach the net-zero energy systems we have modelled with new transmission
networks and leveraging of various sector-coupling flexibilities, many changes
are needed in policy and regulation. Tight coordination between countries and
energy sectors is required to achieve low-cost solutions, similar to how the
process for the Ten Year Network Development Plan (TYNDP) has moved towards
joint planning.\cite{entso-eTYNDP20222022} To achieve the coordination of
dispatch and capacity expansion at the local level around grid bottlenecks,
particularly if electricity and hydrogen network expansion is limited, local
price signals are required corresponding to our 181 bidding zones
(\cref{fig:si:lmp-ac,fig:si:lmp-h2}). In our model, electric vehicles and heat
pumps operate flexibly, which requires the deployment of smart meters and
dynamic electricity tariffs to incentivise grid-supporting   behaviour. And
finally, a sustained rise in the price of \co emission certificates is needed.
The results we show are also contingent on adjusted regulations and rules for
building infrastructure and developing competitive markets for hydrogen and
carbon dioxide.
