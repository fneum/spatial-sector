\subsection*{Comparison to Related Literature}

Compared to the net-zero scenarios from the European Commission
released in 2018 \cite{in-depth_2018}, we see much larger renewable
generation, reaching beyond 10000~TWh/a. This represents a
tripling of today's electricity generation (compared to at most a
145\% increase by 2050 in \cite{in-depth_2018}), with one third going
to regular electricity demand, one third going to newly-electrified
sectors in heating, transport and industry, and one third going to
hydrogen production (dominated by Fischer-Tropsch fuels). The major
difference is caused by lower electrification rates in other sectors
in \cite{in-depth_2018}, the higher biomass potentials
in \cite{in-depth_2018}, and the fact that \cite{in-depth_2018} relies
on imports of fossil oil for non-energy uses such as plastics (and
does not count them towards net emissions like we do). We are also
conservative on energy efficiency; building renovations and
circular economy concepts would reduce the demand in our model.

In \cite{brownSynergiesSector2018} an optimal grid expansion brought a benefit
of \euro~64~billion per year compared to the case with no transmission
between European countries, which is higher than the \euro~44~billion
per year benefit found here. There are at least four causes for this
difference: the model here has higher resolution (181 versus 30 nodes)
which allows better placement of wind at good sites; here we start
from today's grid, which has international transmission;
in \cite{brownSynergiesSector2018} there was no hydrogen pipeline network and no
underground hydrogen storage (just steel tanks); and finally we have
higher demand for hydrogen from industry and synthetic fuels, which
provides a large flexible load that helps to integrate wind and solar.

In \cite{schlachtbergerCostOptimal2018} only a small change
(8.8-12.2\%) in system costs was found in a model where onshore wind potentials
were restricted, with the biggest change when the electricity grid was
restricted. Onshore wind was largely replaced with offshore wind in
that model. Unlike that model, here we have a higher grid resolution
(181 versus 30 nodes) which allows us to better assess the grid
integration costs of offshore wind.

Caglayan and co-authors \cite{Caglayan2019} also consider European
decarbonisation scenarios with both electricity transmission and new
hydrogen pipelines, but at a lower spatial resolution (96 nodes). A
similar pattern of hydrogen pipeline expansion towards the British
Isles and North Sea is seen, but lower overall hydrogen capacities
because industry, shipping, aviation and non-electrified heating are
not included.

\subsection*{Broad ranges of options with similar costs}

The flatness of the total system costs as we vary grid expansion and
onshore wind potentials is a general feature of energy system models:
there are many directions in the feasible space where we can change
the system composition with only a small change in total system
costs. This flatness can be explored systematically using techniques similar to Modelling to
Generate Alternatives (MGA), and was investigated for an
electricity-only version of this model in \cite{Neumann2019}.

\subsection*{Reliance on hydrogen infrastructure}

One of the biggest changes seen in this model compared to today's
system is the huge built-out of hydrogen infastructure: huge new
electrolyzer capacities, underground storage in salt caverns as well
as a new hydrogen pipeline network. It is not clear that a new
hydrogen network will have any higher public acceptance than the power
grid. However, if existing natural gas pipelines can be reused for hydrogen,
or if at least the pipeline routes can be used, and are always discreetly buried underground, the disturbance to
the public is minimised.


\subsection*{Direct versus indirect costs}

Although our model includes a constraint on \co emissions, we have
not examined the indirect costs to environment, climate and health of
the energy system. The German Environmental Agency (UBA) calculated
the global environmental and health damages of greenhouse gas
emissions in Germany in 2016 to be \euro~164~billion \cite{UBA2019}
(these costs are dominated by climate-related damages amounting to
\SI{180}{\sieuro\per\tco}). This is comparable to the direct costs of the
German energy system, using our assumptions. Many of these indirect
costs would be avoided in the net-zero-emission systems presented
here.


\subsection*{Policy prerequisites}

A net-zero-emission energy system will require policy support.
Smaller bidding zones and dynamic pricing for flexible loads would
ease the management of grid congestion. From our model we see a need
for high, increasing and transparent price for \co pollution
(renewable support schemes alone are not sufficient to decarbonise
transport, heating and industry \cite{zhuImpactCO22018}). The model requires
a \co price of at least \SI{435}{\sieuro\per\tco} to achieve \co
neutrality. As discussed in \cite{brownSynergiesSector2018}, these high abatement
prices arise in building heating from the large price difference
between low-carbon heat and natural gas (\SI{22}{\sieuro\per\mwh} in the model);
if existing taxes and surcharges were included, the \co price would
be lower.



Add biomass supply and industry demand, aviation, shipping, agriculture, and non-energy
feedstocks for chemicals industry.

Examine different levels of: material efficiency, recycling, carbon
sequestration, biomass and import of low-carbon synthetic fuels.

Conclusions: efficiency is important, sequestration below \SI{200}{\mega\tco\per\year} is hard,
costs reduced with lots of biomass and sequestration, but this is challenging.

Without CCS/biomass/nuclear/efficiency/recycling/import need ~10,000 (update!) TWh/a of
wind+solar - very hard without acceptance problems!