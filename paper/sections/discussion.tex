To put our results into a broader perspective, for the discussion we compare
them to related literature and proposals presented in the gas industry's
European Hydrogen Backbone reports. This is followed by an appraisal of the
limitations of our study.

\subsection*{Comparison to Related Literature}

Compared to the net-zero scenarios from the European
Commission,\cite{in-depth_2018} we see much larger wind and solar capacities,
reaching beyond 2000~GW compared to 1200~GW for wind and 3500~GW compared to
1000~GW for solar.\cite{in-depth_2018} In terms of total electricity produced,
our results approximately show a tripling of today's generation compared to
150\% in the Commission's net-zero zenarios.\cite{in-depth_2018} Roughly one
third goes to regular electricity demand, one third goes to newly electrified
sectors in heating, transport and industry, and another third goes to hydrogen
production (dominated by demand for liquid hydrocarbons). The major difference
to the Commission's scenarios\cite{in-depth_2018} is caused by their lower
electrification rates, a 15\% share of nuclear power in the electricity mix,
higher biomass usage (2900~TWh/a versus 1400~TWh/a), and a strong reliance on
fossil fuels imports (2900~TWh/a) for non-energy uses (e.g. plastics and other
high-value chemicals). By considering landfill of plastics as long-term carbon
sequestration option, the Commission's scenarios see little need to produce
synthetic hydrocarbons for non-energy feedstocks. On the contrary, our
modelling, which assumes that all carbon in landfill will eventually decay and
dissipate into the atmosphere and limited sequestration potentials, requires
green electrofuels for feedstocks and precludes the wide-ranging use of fossil oil.

% In Brown et al.~\cite{brownSynergiesSector2018}, a cost-optimal power grid expansion
% brought a benefit of 64~bn\euro/a compared to no
% power grid reinforcements, which is similar to the
% \maxacbenefitabs~bn\euro/a benefit found here. There are at least four causes for
% this difference: the model here has higher resolution (181 compared to 30
% regions) which allows better placement of wind at good sites; here we start from
% today's grid; in Brown et al.~\cite{brownSynergiesSector2018} there was no
% hydrogen pipeline network and no underground hydrogen storage; and finally we
% have higher hydrogen demand for industry and synthetic fuel production, which
% provides a large flexible load that helps to integrate wind and solar.

Victoria et al.~\cite{victoriaSpeedTechnological2022} investigate the timing of
when certain technologies %like carbon capture, electrolysis and a hydrogen network
become important for the European energy transition, and find a hydrogen
network consistently appearing after 2035. However, owing to a
one-node-per-country resolution, little can be said about subnational network
infrastructure needs, retrofitting opportunities for gas pipelines or regional
geological storage potentials. Compared to our findings, limited network
expansion options affect total energy system costs less
in~\cite{victoriaSpeedTechnological2022}. A doubling of today's transmission
volume reduces costs by 2\% in~\cite{victoriaSpeedTechnological2022}, compared to
\maxacbenefitrel\% in this study. Disabling hydrogen network expansion increases
costs by 0.5\% in ~\cite{victoriaSpeedTechnological2022}, compared to
\maxhybenefitrel\% in this study. This discrepancy arises because
country-internal transmission bottlenecks are not captured, whereby the
integration costs of remote resources like offshore wind within the countries
are neglected.
% By boosting the spatial resolution to subnational scales, more
% can be said about the competition between hydrogen and electricity networks.

Caglayan et al.~\cite{Caglayan2019} also consider European decarbonisation
scenarios with both electricity and hydrogen networks, but at lower spatial
resolution (96 regions). A similar pattern of hydrogen pipeline expansion
towards the British Isles and North Sea is seen, but lower overall electrolyser
capacities (258~GW compared to our
\SIrange{\minelectrolysis}{\maxelectrolysis}{\giga\watt}) because industry,
shipping, aviation, agriculture and non-electrified heating are not included.
Caglayan et al.~\cite{Caglayan2019} also find cost-optimal hydrogen
storage of 130~TWh, whereas our scenarios involve just between
\hydrogenstorageacnhyn~and \hydrogenstorageacnhyy~TWh owing to the larger
flexible hydrogen demand diminishing the need for weekly and monthly balancing.

A large number of cost-effective designs for a climate-neutral European energy
system was also presented by Pickering et
al.~\cite{pickeringDiversityOptions2022}. Their 98-region model with 2-hourly
resolution likewise includes all energy sectors including non-energy feedstocks
and also assumes energy self-sufficiency for Europe. However, hydrogen transport
options were not considered such that hydrogen must be produced locally.
Moreover, geological potentials for low-cost hydrogen storage and the option to
retrofit gas pipelines are disregarded. Owing to higher storage cost in steel
tanks and fewer assumed end-uses of hydrogen and its derivatives, the scenarios
involve less hydrogen storage (\SIrange{0}{6}{\twh} versus
\SIrange{\hydrogenstorageacnhyn}{\hydrogenstorageacnhyy}{\twh}) and lower
electrolyser capacities (\SIrange{290}{855}{\giga\watt} versus
\SIrange{\minelectrolysis}{\maxelectrolysis}{\giga\watt}) in our results.
Furthermore, whereas our model allows limited use of fossil fuels and with
options for carbon capture and sequestration, Pickering et
al.~\cite{pickeringDiversityOptions2022} eliminate the
use of fossil energy and only consider direct air capture as carbon
source. Overall, total energy system costs lie in a similar range between 730
and 866 bn\euro/a compared to costs between \minsystemcost~and
\maxsystemcost~bn\euro/a in our study.

% However, the composition differs.
% Pickering et al.~\cite{pickeringDiversityOptions2022} have higher spendings on
% power-to-liquids owing to cost assumptions about conversion plants, higher costs
% for power-to-heat technologies due to higher heat demands, lower expenditures
% for generation due to lower PV cost assumptions, and neglected costs for
% electricity distribution grids.

% 914 bn\euro/a if near-optimal 10\% are excluded, and only least-cost solutions across weather years

% -39 bn\euro/a to euro-calliope

% what speaks for higher cost in euro-calliope
% - no hydrogen grid
% - no cavern storage of hydrogen
% - all carbon from DAC
% - no fossil fuels
% - higher electrolysis cost (609 vs 450 EUR/kW)
% - higher offshore wind cost (1680 vs 1523 EUR/kW)
% - higher energy demands (4000 TWh building heat vos 3000 TWh, assumptions on building retrofitting?)
% recycling
% - steel 50% vs 30% primary in PES
% - plastics 33% vs 45% primary in PES
% - alu 100% primary vs 20% primary in PES
% higher spendings on biofuels, transmission, hydro, heat pumps, onshore, FT in euro-calliope
% lower spendings on offshore, PV, distribution grid, hydrogen liquefaction

% what speaks for lower cost in euro-calliope
% - no process emissions, no sequestration
% - lower PV cost for open-field and rooftop (240 vs 350 EUR/kW)
% - no distribution grid cost
%  hydropower capacities: PHS, reservoir, run-of-river

\subsection*{Comparison to the European Hydrogen Backbone}

% \begin{table}
%   \caption{Comparison of new and retrofitted hydrogen network built between our scenarios and the European Hydrogen Backbone (April 2021) \cite{gasforclimateExtendingEuropean2021}. Spatial coverage uses \href{https://en.wikipedia.org/wiki/ISO_3166-1_alpha-2}{ISO 3166-1 alpha-2} country codes.}
%   \label{tab:ehb}
%   \centering
%   \footnotesize
%   \begin{tabular}{lrrr}
%       \toprule
%        & Repurposed & New &  \\
%        Scenario& [TWkm] & [TWkm] & Spatial Coverage \\
%       \midrule
%       European Hydrogen Backbone \cite{gasforclimateExtendingEuropean2021} & 208 & 101 & EU27+UK+CH\\
%       && & -PT-LT-LV-HR-BG-RO-MT-CY\\
%       PyPSA-Eur-Sec (with grid expansion) & 199 & 143 & EU27+UK+CH+NO+Balkan-MT-CY \\
%       PyPSA-Eur-Sec (no grid expansion) & 277 & 145 & EU27+UK+CH+NO+Balkan-MT-CY \\
%       \bottomrule
%     \end{tabular}
% \end{table}

Our results align with the European Hydrogen Backbone
(EHB).\cite{gasforclimateEuropeanHydrogen2020,gasforclimateExtendingEuropean2021,gasforclimateEuropeanHydrogen2021,gasforclimateEuropeanHydrogen2022}
But other than the visions laid out in the reports, we present analysis based on
temporally resolved spatial co-planning of energy infrastructures. We see
cost-optimal hydrogen network investments between
\minhycost-\maxhycost~bn\euro/a, while the EHB report covering 21 countries
finds slightly higher costs between
4-10~bn\euro/a.\cite{gasforclimateExtendingEuropean2021}\footnote{To calculate
the annuity of the overnight hydrogen network costs listed in the EHB reports, a
lifetime of 50 years and a discount rate of 7\% are assumed.} The extension to
28 countries reports costs between 7-14
bn\euro/a.\cite{gasforclimateEuropeanHydrogen2022} Compared to the hydrogen
backbone vision presented in the EHB from April
2021,\cite{gasforclimateExtendingEuropean2021}\footnote{The newer EHB report
from April 2022 \cite{gasforclimateEuropeanHydrogen2022} lacks sufficient data
to calculate length-weighted network capacities.}, our scenarios show a
similarly sized hydrogen network with comparable retrofitting shares. Measured
by the length-weighted sum of pipeline capacities (TWkm), the 309 TWkm indicated
in the EHB report match the upper end of the range of
204-307~TWkm observed in our scenarios. Likewise, the
69\% share of repurposed natural gas pipelines
\cite{gasforclimateExtendingEuropean2021} roughly agrees with our findings where between
\minretroshare\% and \maxretroshare\% of hydrogen pipelines are retrofitted gas
pipelines.

% In total, one EHB report \cite{gasforclimateEuropeanHydrogen2021} assumes a hydrogen
% demand between 2100 and 2700 TWh for the year 2050, which is similar to the
% range between 2400 and 3100 TWh in our analysis. However, the uses and supply
% routes are slightly different. In our model, we assume higher hydrogen demands
% for for shipping and the production of Fischer-Tropsch fuels as aviation fuel
% and feedstock for the plastics industry. Unlike we do, this EHB report considers energy
% imports into Europe and foresees a larger demand for hydrogen re-electrification
% of 650 TWh$_{\text{H}_2}$ \cite{gasforclimateEuropeanHydrogen2021}, which is lower than
% the 200 TWh$_{\text{H}_2}$ in our analysis.


% With electricity grid expansion, our results show a hydrogen
% network that is 11\% larger; without grid expansion this difference rises to
% 37\%.

\subsection*{Limitations of the study and scope for future investigations}
\label{sec:limitations}

In the presented scenarios, Europe is largely energy self-sufficient. While
limited amounts of fossil gas and oil imports are allowed, no imports of
renewable electricity, chemical energy carriers or commodities from outside of
Europe are considered. However, with imports system needs for electricity and
hydrogen transmission infrastructure may change substantially. New hydrogen
import hubs might require different bulk transmission routes. The import of
large amounts of carbon-based fuels and ammonia would furthermore diminish the
demand for hydrogen overall, and hence also the need to transport it. This
effect of wide-ranging imports of liquid hydrocarbon demand on infrastructure
needs is demonstrated in \cref{sec:si:sensitivity-imports} and should be
explored in more detail in future work.
\cite{fasihiTechnoeconomicAssessment2019,heuserTechnoeconomicAnalysis2019,hamppImportOptions2021}

Additionally, the very uneven distribution of energy supply in our results may
interfere with the level of social acceptance for new infrastructure to an
extent that may block a swift energy transition.
\cite{sasseDistributionalTradeoffs2019,sasseRegionalImpacts2020} Hence, future
investigations should weigh the cost surcharge of increased regionally
self-sufficient energy supply against the potential benefit of higher public
acceptance and increased resilience.

Previous research has shown that the system design can be changed in many ways
with only a small change in total
costs.\cite{Neumann2019,lombardiPolicyDecision2020,pedersenModelingAll2021,pickeringDiversityOptions2022}
This breadth of options makes robust statements about specific infrastructure
needs more vague. While we present selected design trade-offs regarding
transmission networks and some further sensitivities
(\cref{sec:si:sensitivity}), a more comprehensive exploration of near-optimal
solutions would be prudent, especially in the directions of carbon management
infrastructure, biomass usage, the level of energy imports, industry transition
and relocation options, more regionally balanced infrastructure, and increased
system resilience.

Owing to the absence of pathway optimisation in our model setup, our results
cannot offer insights into the transition steps of network development or the
energy system overall and how the gradual transformation may restrict certain
options towards the final climate-neutral state. For example, our results do not
show which parts of gas network could be repurposed first or where the benefit
of a hydrogen network might be the highest initially. In the context of
multi-horizon planning, we also neglect the dynamics of technological learning
by
doing.\cite{heubergerPowerCapacity2017,fellingMultiHorizonPlanning2021,zeyenEndogenousLearning2022}
The transformation to net-zero emissions requires vast and timely growth rates
of power-to-X and carbon dioxide removal technologies to realise anticipated
cost reductions, which we assume to be given by assuming fixed cost projections.

Further limitations include that heat demands and the availability of renewables
vary considerably year by year such that our stricture to a single year may
limit the robustness to interannual weather variability; we do not consider new
nuclear power plants: and for the transport and industry sectors we make some
exogenous assumptions about process switching, drive trains, alternative fuels
for industry heat and recycling rates which may have turned out differently if
they were endogenously optimised.

% For computational reasons, we also need to
% limit the model's resolution to 181 regions with 3-hourly time steps (see
% \cref{sec:si:sensitivity-time,sec:si:sensitivity-space}) and we ignore the
% nonlinearity of electricity and gaseous flow physics such that a single run
% completes within 2 days and 130 GB of memory. Nevertheless, we believe our
% results can demonstrate the conditions under which a hydrogen network in Europe
% would be beneficial.

\subsection*{Policy implications, regional and operational insights}
\label{sec:policy}

Regardless of the energy carrier transported, our results highlight that
cooperation between European countries is important to reach net-zero \co
emissions most cost-effectively. This is because there are significant
differences in renewable resources across Europe. The cost differential between
supply in Europe's demand centres and periphery outweigh the cost of building
new transmission infrastructure. Thus, we see both substantial net importers
(e.g.~the industry clusters in Ruhr valley and Rotterdam area) and strong net
exporters of energy (e.g.~Denmark, Ireland, Spain, Greece). The option to
transport energy around Europe also counteracts the case for industry
relocation. It may be less controversial as transmission infrastructure would
affect regional development less than the migration of industries. Regarding
hydrogen production, we see both solar-based hubs in Southern Europe and
wind-based hubs in Northern Europe. The regional and technological diversity in
hydrogen production appears as the preferred solution.

As the general hydrogen network benefit is not dependent on infeasible
electricity grid reinforcements, both networks could be developed in parallel.
Thus, policymaking could focus on options that are easiest achieved and most
widely accepted. While the hydrogen network benefit is not affected by alternate
technology cost developments or import policies, the network topology is. Lower
costs of solar photovoltaics raise the appeal of hydrogen production hubs in
Southern Europe, altering the suitable hydrogen network layout. Likewise,
wide-ranging hydrogen imports from the MENA region must be supported with
transmission infrastructure through Southern Europe. The flexible operation of
electrolysers has several advantages for system stability and integrating wind
and solar generation cost-effectively.

Fluctuating renewable generation is buffered in geological hydrogen storage
primarily in the UK, Denmark, Spain and Greece to achieve more continuous
production in capital-intensive fuel synthesis plants. This leads to low
curtailment rates of renewables and little requirements for firm capacity,
outlining the benefit of cross-sectoral approaches for reducing \co emissions
cost-effectively. Fuel cell CHP plants in Germany can further support grid
operation when the power grid cannot be expanded. However, energetically the
re-electrification of hydrogen only plays a minor role in this sector-coupled
system.