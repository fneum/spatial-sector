To put our results into a broader perspective, for the discussion we compare our
results to related literature and proposals presented in the European Hydrogen
Backbone reports by the gas industry. This is followed by a short discussion of
the public acceptance for hydrogen infrastructure and a critical apprasal of our
assumption of a self-sufficient Europe without energy imports and other
limitations of the study.

\subsection*{Comparison to Related Literature}

Compared to the net-zero scenarios from the European Commission released in 2018
\cite{in-depth_2018}, we see much larger renewable electricity generation, reaching beyond
8781~TWh/a. This represents approximately a tripling
of today's electricity generation (compared to at most a 145\% increase by 2050
in \cite{in-depth_2018}), with roughly one third going to regular electricity
demand, one third going to newly-electrified sectors in heating, transport and
industry, and one third going to hydrogen production (dominated by
Fischer-Tropsch fuels). The major difference is caused by lower electrification
rates in other sectors in \cite{in-depth_2018}, the higher biomass potentials in
\cite{in-depth_2018}, and the fact that \cite{in-depth_2018} relies on imports
of fossil oil for non-energy uses such as plastics and does not count them
towards net emissions like we do.

In \cite{brownSynergiesSector2018} an optimal grid expansion brought a benefit
of \euro~64~billion per year compared to the case with no transmission between
European countries, which is higher than the \euro~47~billion per year benefit
found here. There are at least four causes for this difference: the model here
has higher resolution (181 versus 30 nodes) which allows better placement of
wind at good sites; here we start from today's grid, which already has some
international transmission; in \cite{brownSynergiesSector2018} there was no
hydrogen pipeline network and no underground hydrogen storage (just steel
tanks); and finally we have higher demand for hydrogen from industry and
synthetic fuels, which provides a large flexible load that helps to integrate
wind and solar.

Caglayan et al.~\cite{Caglayan2019} also consider European decarbonisation
scenarios with both electricity transmission and new hydrogen pipelines, but at
a lower spatial resolution (96 nodes). A similar pattern of hydrogen pipeline
expansion towards the British Isles and North Sea is seen, but lower overall
hydrogen capacities (258~GW compared to more than 1000~GW in our scenarios)
because industry, shipping, aviation and non-electrified heating are not
included. While \cite{Caglayan2019} see 130~TWh of cavern storage, our scenarios
are in a lower range between 32 and 66~TWh.

\subsection*{Comparison to the European Hydrogen Backbone}

compare annualised investment costs and volume to EHB

Cost of hydrogen network
- our models \euro 5-8 bn/a
- EHB: \euro 4-8 bn/a

\begin{tabular}{lrr}
    \toprule
     & Repurposed & New \\
     Scenario& [TWkm] & [TWkm] \\
    \midrule
    European Hydrogen Backbone & 208 & 101 \\
    PyPSA-Eur-Sec (with grid expansion) & 199 & 143 \\
    PyPSA-Eur-Sec (no grid expansion) & 277 & 145 \\
    \bottomrule
  \end{tabular}

EHB (June 2021)
- made for hydrogen demand between 2100 and 2700 TWh for 2050 compared to 2437 TWh in our scenarios
- however, large re-electrification (650 TWh) which is only seen with restricted electricity transmission (200~TWh H2)
- considers imports
- 60\% repurposed pipelines
- levelised transport cost (0.11-0.21 \euro/kg/1000km or 3.3-6.3 \euro/MWh/1000km)

EHB (April 2022)
- 80-143 bn\euro + 1.6-3.2 bn\euro/a approx. 7.5-14 bn\euro/a with 50 year lifetime and 7\% discount rate
- without Balkan

\cite{gasforclimateExtendingEuropean2021}

\subsection*{Public acceptance for hydrogen infrastructure}

One of the biggest changes seen is the built-out of hydrogen infastructure: huge new
electrolyzer capacities, underground storage in salt caverns as well
as a new hydrogen pipeline network. It is not fully clear that a new
hydrogen network will have any higher public acceptance than the power
grid. However, if existing natural gas pipelines can be reused for hydrogen,
or if at least the pipeline routes can be used, and are always discreetly buried underground, the disturbance to
the public is minimised.

\subsection*{Energy imports and European self-sufficiency}

limitation no imports

in these scenarios, Europe is largely self-sufficient (fossil gas imports allowed)

very uneven infrastructure distribution

with imports
- strong point sources
- higher role of hydrogen network

future: increase self-sufficiency constraints of individual regions

\subsection*{Further limitations of the study}

Broad ranges of options with similar costs. The flatness of the total system
costs as we vary grid expansion and onshore wind potentials is a general feature
of energy system models: there are many directions in the feasible space where
we can change the system composition with only a small change in total system
costs. This flatness can be explored systematically using techniques similar to
Modelling to Generate Alternatives (MGA), and was investigated for an
electricity-only version of this model in \cite{Neumann2019}.
