To put our results into a broader perspective, for the discussion we compare our
results to related literature and proposals presented in the gas industry's
European Hydrogen Backbone reports. This is followed by a short discussion of
the public acceptance for hydrogen infrastructure and a critical apprasal of our
assumption of a self-sufficient Europe without energy imports and other
limitations of the study.

\subsection*{Comparison to Related Literature}

Compared to the net-zero scenarios from the European Commission released in 2018
\cite{in-depth_2018}, we see much larger renewable electricity generation,
reaching beyond 8781~TWh/a. This represents approximately a tripling of today's
electricity generation (compared to at most a 150\% increase by 2050 in
\cite{in-depth_2018}), with roughly one third going to regular electricity
demand, one third going to newly electrified sectors in heating, transport and
industry, and one third going to hydrogen production (dominated by
Fischer-Tropsch fuels). The major difference is caused by lower electrification
rates in other sectors in \cite{in-depth_2018}, the higher biomass potentials in
\cite{in-depth_2018}, and the fact that \cite{in-depth_2018} relies on imports
of fossil oil for non-energy uses such as plastics and does not count them
towards net emissions as assumed in this study.

In Brown et al.~\cite{brownSynergiesSector2018}, an optimal grid expansion
brought a benefit of \euro~64~billion per year compared to the case with no
transmission between European countries, which is higher than the
\euro~47~billion per year benefit found here. There are at least four causes for
this difference: the model here has higher resolution (181 versus 30 nodes)
which allows better placement of wind at good sites; here we start from today's
grid, which already has some international transmission; in Brown et
al.~\cite{brownSynergiesSector2018} there was no hydrogen pipeline network and
no underground hydrogen storage (just steel tanks); and finally we have higher
demand for hydrogen from industry and synthetic fuels, which provides a large
flexible load that helps to integrate wind and solar.

Caglayan et al.~\cite{Caglayan2019} also consider European decarbonisation
scenarios with both electricity transmission and new hydrogen pipelines, but at
a lower spatial resolution (96 nodes). A similar pattern of hydrogen pipeline
expansion towards the British Isles and North Sea is seen, but lower overall
hydrogen capacities (258~GW compared to more than 1000~GW in our scenarios)
because industry, shipping, aviation and non-electrified heating are not
included. Caglayan et al.~\cite{Caglayan2019} also find cost-optimal hydrogen
underground storage of 130~TWh, whereas our scenarios show less cavern storage
between 32 and 66~TWh.

\subsection*{Comparison to the European Hydrogen Backbone}

\begin{table}
  \caption{Comparison of new and retrofitted hydrogen network built between our scenarios and the European Hydrogen Backbone (April 2021) \cite{gasforclimateExtendingEuropean2021}.}
  \label{tab:ehb}
  \centering
  \begin{tabular}{lrr}
      \toprule
       & Repurposed & New \\
       Scenario& [TWkm] & [TWkm] \\
      \midrule
      European Hydrogen Backbone \cite{gasforclimateExtendingEuropean2021} & 208 & 101 \\
      PyPSA-Eur-Sec (with grid expansion) & 199 & 143  \\
      PyPSA-Eur-Sec (no grid expansion) & 277 & 145 \\
      \bottomrule
    \end{tabular}
\end{table}

Our results align well with the proposals for a European Hydrogen Backbone (EHB)
from the gas industry
\cite{gasforclimateEuropeanHydrogen2020,gasforclimateExtendingEuropean2021,gasforclimateEuropeanHydrogen2021,gasforclimateEuropeanHydrogen2022}.
While the reports are presented as a vision rather than a proposal based on
detailed network planning, here, we present supporting modelling results that
are backed by temporally resolved spatial co-planning of energy infrastructures
and operation.

While we see cost-optimal hydrogen network investments between 5-8 bn\euro/a,
the EHB report from April 2021 \cite{gasforclimateExtendingEuropean2021} finds
similar costs between 4-10 bn\euro/a for a hydrogen backbone spanning 21
European countries\footnote{To calculate the annuity of the overnight hydrogen
network costs listed in the EHB reports, a lifetime of 50 years and a discount
rate of 7\% are assumed like for our own modelling.}. The extension to 28
countries from April 2022 \cite{gasforclimateEuropeanHydrogen2022} reports costs
between 7-14 bn\euro/a. The latest report only misses a few countries in the
Balkan to match the geographic scope of PyPSA-Eur-Sec.

In total, the EHB \cite{gasforclimateEuropeanHydrogen2021} assumes a hydrogen
demand between 2100 and 2700 TWh for the year 2050, which is similar to the
range between 2400 and 3100 TWh in our analysis. However, the uses and supply
routes are slightly different. In our model, we assume higher hydrogen demands
for for shipping and the production of Fischer-Tropsch fuels as aviation fuel
and feedstock for the plastics industry. Unlike we do, the EHB considers energy
imports into Europe and foresees a larger demand for hydrogen re-electrification
of 650 TWh$_{H_2}$ \cite{gasforclimateEuropeanHydrogen2021}, which is lower than
the 200 TWh$_{H_2}$ in our analysis.

\cref{tab:ehb} compares the hydrogen network volume of the EHB from April 2021
to our analysis \cite{gasforclimateExtendingEuropean2021}\footnote{The newer
report from April 2022 \cite{gasforclimateEuropeanHydrogen2022} lacks sufficient
data to calculate length-weighted network capacities.}. The volume is measured
as the length-weighted sum of pipeline capacities (TWkm) and distinguishes
between repurposed and new pipeline capacities. Both analyses build a similarly
sized hydrogen network. With electricity grid expansion, our results show a
hydrogen network that is 11\% larger; without grid expansion this difference
rises to 37\%. Furthermore, the share of retrofitted gas pipelines in the
hydrogen backbone is comparable. The 69\% volume share of repurposed gas
pipelines \cite{gasforclimateExtendingEuropean2021} agrees with our findings
where between 58\% and 66\% of hydrogen pipelines are retrofitted gas pipelines.

\subsection*{Energy imports and European self-sufficiency}

limitation no imports
in these scenarios, Europe is largely self-sufficient (fossil gas imports allowed)
with imports
- strong point sources
- different role of hydrogen network

Imports from regions with good renewable
resources should be considered in future work
\cite{fasihiTechnoeconomicAssessment2019,heuserTechnoeconomicAnalysis2019}.

very uneven infrastructure distribution
future: increase self-sufficiency constraints of individual regions

\subsection*{Further limitations of the study}

Broad ranges of options with similar costs. The flatness of the total system
costs as we vary grid expansion and onshore wind potentials is a general feature
of energy system models: there are many directions in the feasible space where
we can change the system composition with only a small change in total system
costs. This flatness can be explored systematically using techniques similar to
Modelling to Generate Alternatives (MGA), and was investigated for an
electricity-only version of this model in \cite{Neumann2019}.
