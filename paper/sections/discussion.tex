To put our results into a broader perspective, for the discussion we compare our
results to related literature and proposals presented in the gas industry's
European Hydrogen Backbone reports. This is followed by an appraisal of the
limitations of our study.

\subsection*{Comparison to Related Literature}

Compared to the net-zero scenarios from the European
Commission\cite{in-depth_2018}, we see much larger renewable electricity
generation, reaching beyond 8750~TWh/a. This represents almost a tripling of
today's total electricity generation (270\% compared to at most a 150\% increase
by 2050\cite{in-depth_2018}), with roughly one third going to regular
electricity demand, one third going to newly electrified sectors in heating,
transport and industry, and one third going to hydrogen production (dominated by
demand for Fischer-Tropsch fuels). The major difference to the Commission's
scenarios\cite{in-depth_2018} is caused by their lower electrification rates in
other sectors, higher biomass potentials, and a strong reliance on imports of
fossil oil for non-energy uses such as plastics that were not counted towards
net emissions as we do in this study.

In Brown et al.~\cite{brownSynergiesSector2018}, an optimal grid expansion
brought a benefit of \euro~64~billion per year compared to the case with no
transmission between European countries, which is higher than the
\euro~47~billion per year benefit found here. There are at least four causes for
this difference: the model here has higher resolution (181 compared to 30
regions) which allows better placement of wind at good sites; here we start from
today's grid; in Brown et al.~\cite{brownSynergiesSector2018} there was no
hydrogen pipeline network and no underground hydrogen storage; and finally we
have higher demand for hydrogen from industry and synthetic fuels, which
provides a large flexible load that helps to integrate wind and solar in time.

Caglayan et al.~\cite{Caglayan2019} also consider European decarbonisation
scenarios with both electricity transmission and new hydrogen pipelines, but at
a lower spatial resolution (96 regions). A similar pattern of hydrogen pipeline
expansion towards the British Isles and North Sea is seen, but lower overall
hydrogen capacities (258~GW compared to more than 1000~GW in our scenarios)
because industry, shipping, aviation, agriculture and non-electrified heating
are not included. Caglayan et al.~\cite{Caglayan2019} also find cost-optimal
hydrogen underground storage of 130~TWh, whereas our scenarios involve less
cavern storage between 32 and 66~TWh.

A large number of feasible and cost-effective designs for a climate-neutral
European energy system was also recently presented by Pickering et
al.~\cite{pickeringDiversityOptions}. Their 98-region model with 2-hourly
resolution likewise includes all energy sectors including non-energy feedstocks
and also optimises for fully self-sufficient energy supply chains within Europe.
However, their analysis contains no detailed account of hydrogen infrastructure.
Hydrogen transport is not considered such that hydrogen must be
produced locally, whereby power grid expansion gains importance. Moreover,
their scenarios disregard geological potentials for cheap underground hydrogen
storage as well as the potential to retrofit gas pipelines for hydrogen
transport. Owing to high cost of hydrogen storage in steel tanks and fewer
assumed direct uses of hydrogen, their results show less hydrogen storage
(\SIrange{0}{6}{\twh} versus \SIrange{32}{66}{\twh}) and lower electrolyser
capacities (\SIrange{290}{855}{\giga\watt} versus
\SIrange{1057}{1297}{\giga\watt}) compared to our results. Furthermore, whereas
our model allows limited use of fossil fuels and with options for carbon capture
and sequestration, Pickering et al.~\cite{pickeringDiversityOptions} are more
restrictive by eliminating the use of fossil energy sources and only considering
direct air capture as carbon source. Overall, total system costs lie in a range
between 811 and 954 bn\euro/a compared to costs between 764 and 857 bn\euro/a in
our study.

\subsection*{Comparison to the European Hydrogen Backbone}

\begin{table}
  \caption{Comparison of new and retrofitted hydrogen network built between our scenarios and the European Hydrogen Backbone (April 2021) \cite{gasforclimateExtendingEuropean2021}.}
  \label{tab:ehb}
  \centering
  \footnotesize
  \begin{tabular}{lrrr}
      \toprule
       & Repurposed & New &  \\
       Scenario& [TWkm] & [TWkm] & Spatial Coverage \\
      \midrule
      European Hydrogen Backbone \cite{gasforclimateExtendingEuropean2021} & 208 & 101 & EU27+UK+CH\\
      && & -PT-LT-LV-HR-BG-RO-MT-CY\\
      PyPSA-Eur-Sec (with grid expansion) & 199 & 143 & EU27+UK+CH+NO+Balkan-MT-CY \\
      PyPSA-Eur-Sec (no grid expansion) & 277 & 145 & EU27+UK+CH+NO+Balkan-MT-CY \\
      \bottomrule
    \end{tabular}
\end{table}

Our results align well with the proposals for a European Hydrogen Backbone (EHB)
from the gas industry
\cite{gasforclimateEuropeanHydrogen2020,gasforclimateExtendingEuropean2021,gasforclimateEuropeanHydrogen2021,gasforclimateEuropeanHydrogen2022}.
Whereas the reports are presented as visions rather than proposals based on
detailed network planning, here, we present supporting modelling results based
on temporally resolved spatial co-planning of energy infrastructures. We see
cost-optimal hydrogen network investments between 5-8 bn\euro/a, while the EHB
report from April 2021 \cite{gasforclimateExtendingEuropean2021} finds similar
costs between 4-10 bn\euro/a across 21 countries\footnote{To calculate the
annuity of the overnight hydrogen network costs listed in the EHB reports, a
lifetime of 50 years and a discount rate of 7\% are assumed like for our own
modelling.}. The extension to 28 countries from April 2022
\cite{gasforclimateEuropeanHydrogen2022} reports costs between 7-14 bn\euro/a.

% In total, one EHB report \cite{gasforclimateEuropeanHydrogen2021} assumes a hydrogen
% demand between 2100 and 2700 TWh for the year 2050, which is similar to the
% range between 2400 and 3100 TWh in our analysis. However, the uses and supply
% routes are slightly different. In our model, we assume higher hydrogen demands
% for for shipping and the production of Fischer-Tropsch fuels as aviation fuel
% and feedstock for the plastics industry. Unlike we do, this EHB report considers energy
% imports into Europe and foresees a larger demand for hydrogen re-electrification
% of 650 TWh$_{\text{H}_2}$ \cite{gasforclimateEuropeanHydrogen2021}, which is lower than
% the 200 TWh$_{\text{H}_2}$ in our analysis.

\cref{tab:ehb} compares the hydrogen network volume of the EHB from April 2021 with 21 European countries
to our analysis \cite{gasforclimateExtendingEuropean2021}\footnote{The newer
EHB report from April 2022 \cite{gasforclimateEuropeanHydrogen2022} lacks sufficient
data to calculate length-weighted network capacities.}. The volume is measured
as the length-weighted sum of pipeline capacities (TWkm) and distinguishes
between repurposed and new pipeline capacities. Both analyses build a similarly
sized hydrogen network. With electricity grid expansion, our results show a
hydrogen network that is 11\% larger; without grid expansion this difference
rises to 37\%. Furthermore, the share of retrofitted gas pipelines in the
hydrogen backbone is comparable. The 69\% volume share of repurposed gas
pipelines \cite{gasforclimateExtendingEuropean2021} agrees with our findings
where between 58\% and 66\% of hydrogen pipelines are retrofitted gas pipelines.

\subsection*{Selected limitations of the study and scope for future investigations}

In the presented scenarios, Europe is largely energy self-sufficient. While
limited amounts of fossil gas and oil imports are allowed, no imports of
renewable electricity, chemical energy carriers or commodities from outside of
Europe are considered. However, with imports system needs for electricity and
hydrogen transmission infrastructure may change substantially. New hydrogen
import hubs would require different bulk transmission routes. The import of
large amounts of carbon-based fuels and ammonia would furthermore diminish the
demand for hydrogen overall, and hence also the need to transport it. This
effect of imports on infrastructure needs should be explored in future work
\cite{fasihiTechnoeconomicAssessment2019,heuserTechnoeconomicAnalysis2019,hamppImportOptions2021}.

Additionally, the very uneven distribution of energy supply in our results may
interfere with the level of social acceptance for new infrastructure to an
extent that may block a swift energy transition. Hence, future investigations
should weigh the cost surcharge of increased regionally self-sufficient energy
supply against the potential benefit of higher public acceptance and increased
resiliency.

Moreover, previous research has shown that there are many directions in the
feasible space where the system composition can be changed with only a small
change in system costs. This makes robust statements about specific
infrastructure needs hard and more vague. While we present selected design
trade-offs regarding transmission networks, a more comprehensive exploration of
near-optimal solutions in sector-coupled systems using techniques like
Modelling-to-Generate-Alternatives (MGA)
\cite{Neumann2019,lombardiPolicyDecision2020,pedersenModelingAll2020,pickeringDiversityOptions},
would likely reveal a broader range of alternative system layouts.

\todo[noline]{Wxpand by a few more sentences.}

Besides, there are many more limitations pertaining to the quality and
availability of open data on demands and infrastructure, uncertainty in future
technology assumptions, limited robustness to interannual weather variation,
neglect of demand sufficiency, and exogenous assumptions about fuel and process
switching in transport and industry. %Nevertheless, we believe our results can
% demonstrate the conditions under which a hydrogen backbone in Europe would be beneficial.

% no nuclear