To put our results into a broader perspective, for the discussion we compare
them to related literature and proposals presented in the gas industry's
European Hydrogen Backbone reports. This is followed by an appraisal of the
limitations of our study.

\subsection*{Comparison to Related Literature}

Compared to the net-zero scenarios from the European
Commission\cite{in-depth_2018}, we see much larger renewable electricity
generation, reaching beyond 8750~TWh/a compared to 2500~TWh/a.\cite{in-depth_2018} In terms of total electricity produced, our results show
almost a tripling of today's geeneration. By 2050, it rises to 270\% compared to 150\%
in the Commission's report.\cite{in-depth_2018} Roughly one third goes to regular
electricity demand, one third goes to newly electrified sectors in heating,
transport and industry, and another third goes to hydrogen production (dominated by
demand for Fischer-Tropsch fuels). The major difference to the Commission's
scenarios\cite{in-depth_2018} is caused by their lower electrification rates in
other sectors, higher biomass potentials, and a strong reliance on imports of
fossil oil for non-energy uses such as plastics that were not counted towards
net emissions as we do in this study.

In Brown et al.~\cite{brownSynergiesSector2018}, an optimal grid expansion
brought a benefit of 64~bn\euro/a compared to the case with no
transmission between European countries, which is higher than the
47~~bn\euro/a benefit found here. There are at least four causes for
this difference: the model here has higher resolution (181 compared to 30
regions) which allows better placement of wind at good sites; here we start from
today's grid; in Brown et al.~\cite{brownSynergiesSector2018} there was no
hydrogen pipeline network and no underground hydrogen storage; and finally we
have higher demand for hydrogen from industry and synthetic fuels, which
provides a large flexible load that helps to integrate wind and solar in time.

Victoria et al.~\cite{victoriaSpeedTechnological2022} investigate the timing of
when technologies like carbon capture, electrolysis and a hydrogen network
become important for the European energy transition. The authors find a hydrogen
network consistently appearing after 2035 in all scenarios. However, owing to a
focus on transition pathways, the presented scenarios only resolve a single
region per country. Thereby, little can be said about network infrastructure
needs beyond cross-border transmission capacities, retrofitting opportunities
for legacy gas pipelines or the availability of geological hydrogen storage
potentials within countries. Compared to our findings, limited network expansion
options affect system costs much less. A doubling of today's transmission volume
reduces cost by 2\%, compared to 6.2\% in this study. Disabling hydrogen network
expansion increases system cost by 0.5\%, compared to 5.7\% in this study. This
substantial discrepancy arises because the one-node-per-country model version in
Victoria et al.~\cite{victoriaSpeedTechnological2022} does not capture
country-internal transmission bottlenecks of either electricity or hydrogen
networks. As the system integration costs of remote resources like offshore wind
are thereby neglected within the countries, the benefit of network
infrastructure is degraded. By boosting the spatial resolution to subnational
scales in this study, more can be said about the competition between hydrogen
and electricity networks.

Caglayan et al.~\cite{Caglayan2019} also consider European decarbonisation
scenarios with both electricity transmission and new hydrogen pipelines, but at
a lower spatial resolution (96 regions). A similar pattern of hydrogen pipeline
expansion towards the British Isles and North Sea is seen, but lower overall
hydrogen capacities (258~GW compared to our \SIrange{1057}{1297}{\giga\watt})
because industry, shipping, aviation, agriculture and non-electrified heating
are not included. Caglayan et al.~\cite{Caglayan2019} also find cost-optimal
hydrogen underground storage of 130~TWh, whereas our scenarios involve less
cavern storage between 32 and 66~TWh because the larger demand for hydrogen
means there is both more wind and solar generation as well as more flexible demand.
Therefore, less storage is needed for weekly and monthly balancing.

A large number of feasible and cost-effective designs for a climate-neutral
European energy system was also recently presented by Pickering et
al.~\cite{pickeringDiversityOptions2022}. Their 98-region model with 2-hourly
resolution likewise includes all energy sectors including non-energy feedstocks
and also optimises for fully self-sufficient energy supply chains within Europe.
However, they do not focus on hydrogen infrastructure. Hydrogen transport is not
considered so that hydrogen must be produced locally, whereby power grid
expansion gains importance. Moreover, their scenarios disregard geological
potentials for low-cost storage in salt aquifers and the option to retrofit gas
pipelines. Owing to high storage cost in steel tanks and fewer assumed end-uses
of hydrogen (e.g.~in shipping and heavy-duty transport), their scenarios involve
less hydrogen storage (\SIrange{0}{6}{\twh} versus \SIrange{32}{66}{\twh}) and
lower electrolyser capacities (\SIrange{290}{855}{\giga\watt} versus
\SIrange{1057}{1297}{\giga\watt}) compared to our results. Furthermore, whereas
our model allows limited use of fossil fuels and with options for carbon capture
and sequestration, Pickering et al.~\cite{pickeringDiversityOptions2022} are more
restrictive by eliminating the use of fossil energy sources and only considering
direct air capture as carbon source for synthetic hydrocarbons. Overall, total
system costs lie in a similar range between 730 and 866 bn\euro/a compared to
costs between 746 and 839 bn\euro/a in our study (both excluding existing
hydroelectric facilities). However, the composition differs. Pickering et
al.~\cite{pickeringDiversityOptions2022} have higher spendings on power-to-liquids
owing to cost assumptions about conversion plants, higher costs for
power-to-heat technologies due to higher heat demands, lower expenditures for
generation due to lower PV cost assumptions, and neglected costs for electricity
distribution grids.

% 914 bn\euro/a if near-optimal 10\% are excluded, and only least-cost solutions across weather years

% -39 bn\euro/a to euro-calliope

% what speaks for higher cost in euro-calliope
% - no hydrogen grid
% - no cavern storage of hydrogen
% - all carbon from DAC
% - no fossil fuels
% - higher electrolysis cost (609 vs 450 EUR/kW)
% - higher offshore wind cost (1680 vs 1523 EUR/kW)
% - higher energy demands (4000 TWh building heat vos 3000 TWh, assumptions on building retrofitting?)
% recycling
% - steel 50% vs 30% primary in PES
% - plastics 33% vs 45% primary in PES
% - alu 100% primary vs 20% primary in PES
% higher spendings on biofuels, transmission, hydro, heat pumps, onshore, FT in euro-calliope
% lower spendings on offshore, PV, distribution grid, hydrogen liquefaction

% what speaks for lower cost in euro-calliope
% - no process emissions, no sequestration
% - lower PV cost for open-field and rooftop (240 vs 350 EUR/kW)
% - no distribution grid cost
%  hydropower capacities: PHS, reservoir, run-of-river

\subsection*{Comparison to the European Hydrogen Backbone}

\begin{table}
  \caption{Comparison of new and retrofitted hydrogen network built between our scenarios and the European Hydrogen Backbone (April 2021) \cite{gasforclimateExtendingEuropean2021}. Spatial coverage uses \href{https://en.wikipedia.org/wiki/ISO_3166-1_alpha-2}{ISO 3166-1 alpha-2} country codes.}
  \label{tab:ehb}
  \centering
  \footnotesize
  \begin{tabular}{lrrr}
      \toprule
       & Repurposed & New &  \\
       Scenario& [TWkm] & [TWkm] & Spatial Coverage \\
      \midrule
      European Hydrogen Backbone \cite{gasforclimateExtendingEuropean2021} & 208 & 101 & EU27+UK+CH\\
      && & -PT-LT-LV-HR-BG-RO-MT-CY\\
      PyPSA-Eur-Sec (with grid expansion) & 199 & 143 & EU27+UK+CH+NO+Balkan-MT-CY \\
      PyPSA-Eur-Sec (no grid expansion) & 277 & 145 & EU27+UK+CH+NO+Balkan-MT-CY \\
      \bottomrule
    \end{tabular}
\end{table}

Our results align well with the proposals for a European Hydrogen Backbone (EHB)
from the gas industry.\cite{gasforclimateEuropeanHydrogen2020,gasforclimateExtendingEuropean2021,gasforclimateEuropeanHydrogen2021,gasforclimateEuropeanHydrogen2022}
Whereas the reports are presented as visions rather than proposals based on
detailed network planning, here, we present supporting modelling results based
on temporally resolved spatial co-planning of energy infrastructures. We see
cost-optimal hydrogen network investments between 5-8 bn\euro/a, while the EHB
report from April 2021 \cite{gasforclimateExtendingEuropean2021} finds similar
costs between 4-10 bn\euro/a across 21 countries\footnote{To calculate the
annuity of the overnight hydrogen network costs listed in the EHB reports, a
lifetime of 50 years and a discount rate of 7\% are assumed like for this paper's
modelling.}. The extension to 28 countries from April 2022
\cite{gasforclimateEuropeanHydrogen2022} reports costs between 7-14 bn\euro/a.

% In total, one EHB report \cite{gasforclimateEuropeanHydrogen2021} assumes a hydrogen
% demand between 2100 and 2700 TWh for the year 2050, which is similar to the
% range between 2400 and 3100 TWh in our analysis. However, the uses and supply
% routes are slightly different. In our model, we assume higher hydrogen demands
% for for shipping and the production of Fischer-Tropsch fuels as aviation fuel
% and feedstock for the plastics industry. Unlike we do, this EHB report considers energy
% imports into Europe and foresees a larger demand for hydrogen re-electrification
% of 650 TWh$_{\text{H}_2}$ \cite{gasforclimateEuropeanHydrogen2021}, which is lower than
% the 200 TWh$_{\text{H}_2}$ in our analysis.

\cref{tab:ehb} compares the hydrogen network volume of the EHB from April 2021 with 21 European countries
to our analysis. \cite{gasforclimateExtendingEuropean2021}\footnote{The newer
EHB report from April 2022 \cite{gasforclimateEuropeanHydrogen2022} lacks sufficient
data to calculate length-weighted network capacities.} The volume is measured
as the length-weighted sum of pipeline capacities (TWkm) and distinguishes
between repurposed and new pipeline capacities. Both analyses build a similarly
sized hydrogen network. With electricity grid expansion, our results show a
hydrogen network that is 11\% larger; without grid expansion this difference
rises to 37\%. Furthermore, the share of retrofitted gas pipelines in the
hydrogen backbone is comparable. The 69\% volume share of repurposed natural gas
pipelines \cite{gasforclimateExtendingEuropean2021} agrees with our findings
where between 58\% and 66\% of hydrogen pipelines are retrofitted gas pipelines.

\subsection*{Selected limitations of the study and scope for future investigations}

In the presented scenarios, Europe is largely energy self-sufficient. While
limited amounts of fossil gas and oil imports are allowed, no imports of
renewable electricity, chemical energy carriers or commodities from outside of
Europe are considered. However, with imports system needs for electricity and
hydrogen transmission infrastructure may change substantially. New hydrogen
import hubs would require different bulk transmission routes. The import of
large amounts of carbon-based fuels and ammonia would furthermore diminish the
demand for hydrogen overall, and hence also the need to transport it. This
effect of imports on infrastructure needs should be explored in future work.
\cite{fasihiTechnoeconomicAssessment2019,heuserTechnoeconomicAnalysis2019,hamppImportOptions2021}

Additionally, the very uneven distribution of energy supply in our results may
interfere with the level of social acceptance for new infrastructure to an
extent that may block a swift energy transition.
\cite{sasseDistributionalTradeoffs2019,sasseRegionalImpacts2020} Hence, future
investigations should weigh the cost surcharge of increased regionally
self-sufficient energy supply against the potential benefit of higher public
acceptance and increased resilience.

Moreover, previous research has shown that there are many directions in the
feasible space where the system composition can be changed with only a small
change in system costs. This breadth of options makes robust statements about
specific infrastructure needs more vague. While we present selected design
trade-offs regarding transmission networks, a more comprehensive exploration of
near-optimal solutions in sector-coupled systems using techniques like
Modelling-to-Generate-Alternatives (MGA),
\cite{Neumann2019,lombardiPolicyDecision2020,pedersenModelingAll2021,pickeringDiversityOptions2022}
would reveal an even broader range of alternative system layouts.

Besides, there are many more limitations. For instance, high-quality data on
demands and infrastructure is not always publicly available; projections for
future technology costs and assumptions are uncertain and we neglect the
dynamics of technological learning by doing; we also do not consider new nuclear
power plants; heat demands and the availability of renewables vary considerably
year by year such that our stricture to a single year may limit the solutions'
robustness to interannual weather variability; and for the transport and
industry sectors we make some exogenous assumptions about process switching,
drive trains, alternative fuels for industry heat and recycling rates which may
have turned out differently if they were endogenously optimised. For
computational reasons, we also need to limit the model's spatial resolution to
181 regions and we ignore the nonlinearity of electricity and gaseous flow
physics. Nevertheless, we believe our results can demonstrate the conditions
under which a hydrogen backbone in Europe would be beneficial.
