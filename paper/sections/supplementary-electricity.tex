\section{Electricity Sector}
\label{sec:si:electricity}

Modelling of electricity supply and demand in Europe largely follows the open
electricity generation and transmission model PyPSA-Eur
\citeS{horschPyPSAEurOpen2018}. PyPSA-Eur processes publicly available data on
the topology of the power transmission network, historical time series of
weather observations and electricity consumption, conventional power plants, and
renewable potentials.

\subsection{Electricity Demand}
\label{sec:si:electricity:demand}

Hourly electricity demand at country-level for the reference year 2013 published
by ENTSO-E is retrieved via the interface of the Open Power System Data (OPSD)
initiative.\citeS{muehlenpfordtTimeSeries2020} Existing electrified heating is subtracted from this
demand, so that power-to-heat options can be optimised separately. Furthermore,
current industry electricity demand is subtracted and handled separately
considering further electrification in the industry sector (see \cref{sec:si:industry}).

For the distribution of electricity demand for industry we leverage geographical
data from the industrial database developed within the Hotmaps project.
\citeS{manzGeoreferencedIndustrial2018} The remaining
electricity demand for households and services is heuristically distributed
inside each country to 40\% proportional to population density and to 60\%
proportional to gross domestic product based on a regression performed by Hörsch
et al.~\citeS{horschPyPSAEurOpen2018}. The total spatial distribution of
electricity demands is shown in \cref{fig:demand-space:electricity}.

\subsection{Electricity Supply}

\begin{SCfigure}
    \caption{Existing conventional power plant capacities in Europe by technology. Marker size is proportional to nominal capacity.}
    \label{fig:powerplants}
    \includegraphics[width=0.7\textwidth]{powerplants.pdf}
\end{SCfigure}

For conventional electricity generators, PyPSA-Eur-Sec uses the open
\textit{powerplantmatching} tool, which merges datasets from a variety of
sources.\citeS{gotzensPerformingEnergy2019} As shown in \cref{fig:powerplants}, it
provides data on the power plants about their location, technology and fuel
type, age, and capacity, inlcuding hard coal, lignite, oil, open and combined
cycle gas turbines (OCGT and CCGT), and nuclear generators. Furthermore,
existing run-of-river, pumped-hydro storage plants, and hydro-electric dams, are
also part of the dataset, for which inflow is modelled based on runoff data from
reanalysis weather data and and scaled hydropower generation statistics (see
\cref{sec:si:renewable-ts}). In general, we suppose these to be non-extendable
due to assumed geographical constraints. The overnight scenarios in this study
only take into account existing hydro-electricity plants.

Expandable renewable generators include onshore and offshore wind, utility-scale
and rooftop solar photovoltaics, biomass from multiple feedstocks. The model
decides to build new capacities based on available land and on the weather
resource (see \cref{sec:si:renewable-potentials} and
\cref{sec:si:renewable-ts}). Because the continent-wide availability of data on
the locations of wind and solar installations is fragmentary, we disregard
already existing wind and solar capacities. Moreover, new OCGT and CCGT as well
as gas or biomass-fueled combined heat and power (CHP) generators may be built.
For CHP generators we assume back-pressure operation with heat production
proportional to electricity output. Specific techno-economic assumptions, like
costs, lifetimes and efficiencies are included in \cref{sec:si:costs}.

\subsection{Electricity Storage}
\label{sec:si:electricity:storage}

Electric energy can be stored in batteries (home, utility-scale, electric
vehicles), existing pumped-hydro storage (PHS), hydrogen storage and other
synthetically produced energy carriers (like methane, methanol and oil). For
stationary batteries we distinguish costs for inverters and for storage at home
or utility-scale. With these assumptions, home battery storage is about 40\%
more expensive than utility-scale battery storage (see \cref{sec:si:costs}). The
batteries' energy and power capacities can be independently sized.

To store electricity, hydrogen may be produced by water electrolysis (see
\cref{sec:si:h2:supply}), stored in overground steel tanks or underground salt
caverns (see see \cref{sec:si:h2:storage}), and re-electrified in a
utility-scale fuel cell. Synthetic methane can be re-electrified through an open
cycle gas turbine (OCGT) or a combined heat and power (CHP) plant.

\subsection{Electricity Transport}
\label{sec:si:electricity:transport}

\begin{figure}
    \includegraphics[width=\textwidth, center]{network.pdf}
    \caption{Unclustered European electricity transmission network by voltage level including planned TYNDP projects. Network data was retrieved from \href{https://www.entsoe.eu/data/map/}{entsoe.eu/data/map} and \href{https://tyndp.entsoe.eu/}{tyndp.entsoe.eu}.}
    \label{fig:base-network}
\end{figure}

\begin{SCfigure}
    \caption{Exemplary Voronoi cells of the transmission network's substations.}
    \includegraphics[width=0.55\textwidth]{voronoi.pdf}
    \label{fig:voronoi}
\end{SCfigure}

% topology

The topology of the European electricity transmission network is represented at
substation level based on maps released in the interactive \mbox{ENTSO-E} map\citeS{ENTSOE}
using a modified version of the GridKit tool.\citeS{gridkit} As displayed in
\cref{fig:base-network}, the dataset includes HVAC lines at and above 220 kV
across the mulitple synchronous zones of the \mbox{ENTSO-E} area, but excludes Turkey
and North-African countries which are also synchronised to the continental
European grid, interconnections to Russia, Belarus and Ukraine as well as small
island networks with less than four nodes at transmission level, such as Cyprus,
Crete and Malta. In total, the network encompasses around 3000 substations, 6600
HVAC lines and around 70 HVDC links, some of which are planned projects from the
Ten Year Network Development Plant (TYNDP) that are not yet in operation.
\citeS{tyndp2018}

% unclustered model regions

The transmission network topology determines the basic regions of the
PyPSA-Eur-Sec model. Each substation has an associated Voronoi cell that
describes the region that is closer to the substation than to any other
substation except for country borders, which are kept to retain the integrity of
country totals. Exemplary Voronoi cells are illustrated in \cref{fig:voronoi}.
We use these as geographical catchment area for demands, renewable resource
potentials, and power plants, assuming that supply and demand always connect to
the closest substation. The Voronoi cells are also computed for offshore regions
based on the countries' Exclusive Economic Zones (EEZs) and the adjacent onshore
substations.

% physical modelling

Capacities and electrical characteristics of transmission lines and substations,
such as impedances and thermal ratings, are
inferred from standard types for each voltage level from Oeding and Oswald.\citeS{oedingElektrischeKraftwerke2011} For each HVAC
line, we further restrict line loading to 70\% of the nominal rating to
approximate $N-1$ security, which protects the system against overloading if any
one transmission line fails. This conservative security margin is commonly
applied in the industry.\citeS{ackermannOptimisingEuropean2016} Dynamic line rating is not considered. Power
flow is modelled through lossless linearised power flow equations using an
efficient cycle-based formulation of Kirchhoff's voltage law.\citeS{horschLinearOptimal2018}

% clustering

Solving the capacity expansion optimisation for the whole European energy system
at full network resolution is to large to be solved in reasonable time.
Therefore, we simplify the network topology by lowering the spatial resolution.
We initially remove the network's radial paths, i.e.~nodes with only one
connection, by linking remote resources to adjacent nodes and transforming the
network to a uniform voltage level of \SI{380}{\kilo\volt}. We also aggregate
generators of the same kind that connect to the same substation. Based on these
initial simplification, the network resolution is further reduced to a variable
number of nodes, in this case to 181 regions, by using a \textit{k-means}
clustering algorithm, which uses regional electricity consumption as weights.
\citeS{frysztackiStrongEffect2021, Hoersch2017} Only substations within the
same country can be aggregated. The equivalent lines connecting the clustered
regions are determined by the aggregated electro-technical characteristics of
original transmission lines. Their weighted cost takes into consideration the
underwater fraction of the lines and adds 25\% to the crow-fly distance to
approximate routing constraints. The clustered electricity network resolution
and associated model regions, as shown in \cref{fig:clustered-networks}, are
applied uniformly to the other nodally resolved energy carriers as well.

% distribution grid

Contrary to the transmission level, the grid topology at the distribition level
(at and below \SI{110}{\kilo\volt}) is not included. Only the total power
exchange capacity between transmission and distribution level is co-optimised.
Costs of \SI{500}{\sieuro\per\kilo\watt} are assumed as well as lossless
distribution. Rooftop PV, heat pumps, resistive heaters, home batteries,
electric vehicles and electricity demands are connected to the low-voltage
level. All other remaining technologies connect directly to the transmission
grid. In this way, distribution grid capacity is developed if it is beneficial
to balance the local mismatch between supply and demand.